\chapter{Chi tiết hệ thống}
\section{Tiền xử lý}
Trong quá trình rút trích đặc trưng, một số khái niệm được miêu tả cụ thể làm cho việc so trùng chuỗi hoặc tìm kiếm từ các nguồn tri thức nhân loại thiếu chính xác \cite{YanXu2012}. Ví dụ như khái niệm "her CT scan" và khái niệm "a CT scan". Mặc dù hai khái niệm này cùng chỉ một thủ tục y tế nhưng không trùng chuỗi. Ngoài ra các mạo từ "her", "a" làm việc tìm kiếm tri thức nhân loại từ các nguồn tri thức như Wikipedia, WordNet không được chính xác hoặc không thể tìm được kết quả. Vì vậy trước khi rút trích đặc trưng, các khái niệm được tiền xử lý để loại bỏ mạo từ và các thông tin ngữ cảnh. Tuy nhiên, quá trình tiền xử lý chỉ được áp dụng cho các đặc trưng liên quan so trùng chuỗi và tìm kiếm tri thức nhân loại, các đặc trưng khác không cần qua quá trình tiền xử lý mà nhận vào nguyên gốc khái niệm được xác định.

Quá trình tiền xử lý gồm hai bước. Đầu tiên khái niệm sẽ được loại bỏ tất cả mạo từ. Sau đó, nếu khái niệm có bao gồm giới từ thì giới từ đó và toàn bộ nội dung theo sau sẽ được lược bỏ. Ví dụ như khái niệm “an MRI of the knee” sau quá trình tiền xử lý sẽ trở thành “MRI”. Danh sách mạo từ được xây dựng từ tập dữ liệu và các mạo từ thông dụng của tiếng Anh.

Đặc biệt các khái niệm thuộc lớp Problem/Treatment/Test thường được kèm thêm thông tin về định lượng như 10mg, 5 lit và các thông tin về vị trí giải phẫu học như "upper", "left", "right". Để tăng khả năng tìm kiếm tri thức nhân loại, chúng tôi đề xuất loại bỏ các thông tin ngữ cảnh về số, định lượng và vị trí giải phẫu khỏi khái niệm. Các thông tin ngữ cảnh được loại bỏ bằng cách sử dụng biểu thức chính quy và các từ vựng được xây dựng từ tập dữ liệu. Các đặc trưng liên quan so trùng chuỗi không áp dụng bước tiền xử lý loại bỏ thông tin ngữ cảnh này.
\section{Xây dựng các cặp khái niệm}
\section{Rút trích đặc trưng}
Từ các phân tích được đề cập ở Phần 3, ngoài các thuộc tính chung về mặt ngôn ngữ (như ngữ pháp hay từ vựng), từng lớp khái niệm ở BAĐT còn mang những đặc tính khác nhau. Việc này đòi hỏi chúng tôi phải thiết kế ba hệ thống rút trích đặc trưng và phân loại tương ứng khác nhau cho lớp Person, lớp Problem/Treatment/Test và lớp Pronoun. Hình \ref{fig:TongquanPhangiai} mô tả tổng quan ba hệ thống này, trong đó các khối "Đồng tham chiếu lớp X" bao hàm cả Hệ thống rút trích đặc trưng và Hệ thống phân loại cho lớp tương ứng.

\begin{figure}[ht]
\centering
\begin{tikzpicture}[%
	>=angle 60,
	start chain=going below,
	node distance=1.5cm and 4cm,
	every join/.style={->, draw},
	font=\tiny\sffamily]
	\tikzset{
		wproc/.style = {proc, text width=4.5em},
		wdoc/.style = {doc, text width=3.5em}
	};
	
	\node[io](pati){Các khái niệm Person};
	\node[io](perp){Các cặp khái niệm Person};
	\node[io](prop){Các cặp khái niệm Problem};
	\node[io](tstp){Các cặp khái niệm Test};
	\node[io](trep){Các cặp khái niệm Treatment};
	\node[io](pron){Các khái niệm Pronoun};
	
	\node[wdoc,left=2cm of $(prop)!0.5!(tstp)$](emr){EMR + Concepts};
	\node[altproc,right=2.5cm of pati](patic){Phân loại bệnh nhân};
	\node[wproc,right=4cm of perp](perr){Phân giải đồng tham chiếu lớp Person};
	\node[wproc](pror){Phân giải đồng tham chiếu lớp Problem};
	\node[wproc](tstr){Phân giải đồng tham chiếu lớp Test};
	\node[wproc](trer){Phân giải đồng tham chiếu lớp Treatment};
	\node[wproc](pronr){Phân giải đồng tham chiếu lớp Pronoun};

	\node[io,right=6cm of perr](perch){Các chuỗi Person};
	\node[io](proch){Các chuỗi Problem};
	\node[io](tstch){Các chuỗi Test};
	\node[io](trech){Các chuỗi Treatment};
	
	\coordinate[left=0.5cm of perch.west](perch-left);
	\coordinate[left=1cm of proch.west](proch-left);
	\coordinate[left=1.5cm of tstch.west](tstch-left);
	\coordinate[left=2cm of trech.west](trech-left);
	
	\draw[->] (emr) -- ++(right:1.3cm) |- (pati);
	\draw[->] (emr) -- ++(right:1.3cm) |- (perp);
	\draw[->] (emr) -- ++(right:1.3cm) |- (prop);
	\draw[->] (emr) -- ++(right:1.3cm) |- (tstp);
	\draw[->] (emr) -- ++(right:1.3cm) |- (trep);
	\draw[->] (emr) -- ++(right:1.3cm) |- (pron);
	\draw[->] (pati) -> (patic);
	\draw[->] (patic) -| (perr);
	\draw[->] (perp) -> (perr);
	\draw[->] (prop) -> (pror);
	\draw[->] (tstp) -> (tstr);
	\draw[->] (trep) -> (trer);
	\draw[->] (pron) -> (pronr);
	\draw[->] (perr) -> (perch);
	\draw[->] (pror) -> (proch);
	\draw[->] (tstr) -> (tstch);
	\draw[->] (trer) -> (trech);
	\draw[->] (pronr) -| (perch-left);
	\draw[->] (pronr) -| (proch-left);
	\draw[->] (pronr) -| (tstch-left);
	\draw[->] (pronr) -| (trech-left);
\end{tikzpicture}
\caption{Tổng quan hệ thống phân giải đồng tham chiếu \label{fig:TongquanPhangiai}}
\end{figure}

\subsection*{Nhóm Person}
Tổng quát, các khái niệm thuộc lớp Person có thể là các đại từ nhân xưng (he, she, it, they, ...), đại từ sở hữu (his, her, its, their, ...), đại từ phản thân (himself, herself, itself, themselves, ...) hoặc tên người (Stephanie I Sept, Mr. Anders, Heidi Laura Md, ...). Việc phân giải đồng tham chiếu cho tên người và đại từ là công việc khó, vì thông tin có được từ các đại từ và tên người là rất ít. Ngoài ra trong một văn bản thường đề cập đến nhiều hơn một người, khiến cho việc phát hiện chính xác chuỗi đồng tham chiếu cho các khái niệm này là một thách thức lớn.

Dựa vào hệ thống I, việc giới hạn vấn đề lại trong phạm vi BAĐT giúp công việc này trở nên đơn giản hơn. Trong BAĐT, các khái niệm thuộc lớp Person thường được chia vào ba nhóm chính: bệnh nhân, người thân của bệnh nhân hoặc nhân sự của bệnh viện. Trong đó bệnh nhân là nhóm có số lượng khái niệm được đề cập nhiều nhất và chiếm phần lớn tổng số khái niệm lớp Person. Do vậy việc xác định một khái niệm thuộc vào nhóm nào đóng vai trò quan trọng trong việc phân giải chính xác chuỗi đồng tham chiếu cho khái niệm đó \cite{YanXu2012}. Từ lí do trên, đặc trưng có phải là bệnh nhân hay không được thêm vào hệ thống. Đặc trưng lớp Patient được xác định bằng phương pháp phân loại nhị phân SVM. Hai nhóm người thân của bệnh nhân và nhân sự của bệnh viện được xác định bằng các đặc trưng từ vựng. Bảng \ref{tab:PersonFeatures} trình bày đầy đủ các đặc trưng dùng cho lớp Person.

\begin{table}[th]
\centering\ra{1.2}
\caption{Tập đặc trưng cho lớp Person \label{tab:PersonFeatures}}
\footnotesize\sffamily

\begin{tabularx}{\textwidth}{@{}P{\raggedright}{0.3}lL@{}}
\toprule 
\textbf{Đặc Trưng} & \textbf{Giá trị} & \textbf{Giải thích}\\
\midrule
Patient-class & 0, 1, 2 & Không có khái niệm nào là bệnh nhân (0), cả hai khái niệm đều là bệnh nhân (1), trường hợp khác (2)\\
Distance between sentences & 0, 1, 2, 3, ... & Số câu xuất hiện giữa hai khái niệm được xét\\
Distance between mentions & 0, 1, 2, 3, ... & Số khái niệm xuất hiện giữa hai khái niệm được xét\\
String match & 0, 1 & Trùng chuỗi hoàn toàn (1), ngược lại (0)\\
Levenshtein distance between two mentions & 0, 1, 2, 3, ... & Khoảng cách Levenshtein giữa hai khái niệm\\
Number & 0, 1, 2 & Cả hai đều là số ít hoặc nhiều (1), ngược lại (0), không xác định (2)\\
Gender & 0, 1, 2 & Cùng giới tính (1), khác giới tính (0), không xác định (2)\\
Apposition & 0, 1 & Là đồng vị ngữ (1), ngược lại (0)\\
Alias & 0, 1 & Là từ viết tắt hoặc cùng nghĩa (1), ngược lại (0)\\
Who & 0, 1 & Nếu hai khái niệm liền kề nhau và được phân cách bởi dấu ``:''\\
Name match & 0, 1 & Loại bỏ các	``stop word'' (dr, dr., mr, ...), so trùng chuỗi con, trùng (1), không trùng (0)\\
Relative match & 0, 1 & Cả hai đều cùng chỉ đến một thân nhân (1), ngược lại (0)\\
Department match & 0, 1 & Cả hai cùng chỉ đến một lĩnh vực y học (1), ngược lại (0)\\
Doctor title match & 0, 1 & Cả hai có cùng một chức vụ bác sĩ (1), ngược lại (0)\\
Doctor general match & 0, 1 & Cả hai cùng đề cập đến bác sĩ nói chung (1), ngược lại (0)\\
Twin/triplet & 0, 1 & Cả hai đều chỉ về cùng cặp sinh đôi/sinh ba (1), ngược lại (0)\\
We & 0, 1 & Cả hai đều chứa thông tin về ``chúng tôi'' (1), ngược lại (0)\\
You & 0, 1 & Cả hai đều chứa thông tin về ``tôi'' (1), ngược lại (0)\\
I & 0, 1 & Cả hai đều chứa thông tin về ``bạn'' (1), ngược lại (0)\\
Pronoun match & 0, 1 & Cả hai đều là đại từ chỉ người (1), ngược lại (0)\\
\bottomrule
\end{tabularx}
\end{table}

Với các đặc trưng ``Name match'', ``Relative match'', ``Department match'', ``Doctor title match'', ``Doctor general match'', ``Twin/Triplet'', ``We'', ``You'', ``I'', ``Pronoun match'', chúng tôi hiện thực bằng cách xây dựng tập từ điển tương ứng với từng đặc trưng dựa trên việc khảo sát tập dữ liệu và sử dụng các biểu thức chính quy.

Đặc trưng về Giới tính được chúng tôi xác định dựa trên ba bước phân loại \cite{WeeSoon2001}. Bước thứ nhất: kiểm tra khái niệm có chứa các đại từ xác định giới tính như ``Mr'', ``Ms'', ``she'', ``he'', ... hay không. Nếu có, xác định giới tính dựa trên đại từ xuất hiện. Nếu không thực hiện bước thứ hai: kiểm tra khái niệm có xuất hiện nhiều hơn một lần hay không. Nếu xuất hiện nhiều hơn một lần thì các lần xuất hiện có chứa đại từ xác định giới tính hay không. Ví dụ khái niệm ``Peter H. Diller'' có thể xuất hiện nhiều lần, trong đó có xuất hiện dưới hình thức ``Mr. Diller''. Nếu không thể xác định giới tính qua hai bước kiểm tra, khái niệm sẽ được phân loại bằng cách sử dụng cơ sở dữ liệu về tên tiếng Anh của hệ thống Apache OpenNLP.

\subsection*{Nhóm Patient-class}
Từ nhận định trong việc rút trích đặc trưng của lớp Person, chúng tôi xây dựng một hệ thống SVM nhị phân để phân loại khái niệm thuộc lớp Person có phải là bệnh nhân hay không. Trong BAĐT thường chỉ có một bệnh nhân đóng vai trò là chủ thể của bệnh án.Vì vậy, các khái niệm nếu được xác định là bệnh nhân, thì sẽ được đưa vào một chuỗi đồng tham chiếu duy nhất về bệnh nhân đó. Thông qua phân tích tập dữ liệu, chúng tôi nhận thấy việc xác định một khái niệm thuộc lớp Person hay không có thể đạt được bằng cách xác định tập từ khóa chỉ về bệnh nhân như ``patient'', ``pt'', ... và tập từ khóa chỉ về nhóm người không phải bệnh nhân như ``doctor'', ``dr'', ``wife'', ...

Vì tập dữ liệu không có thông tin xác định một khái niệm thuộc lớp Person có phải là bệnh nhân hay không, dựa theo hệ thống I chúng tôi xác định bằng cách chọn chuỗi đồng tham chiếu có nhiều khái niệm nhất trong tập kết quả làm chuỗi đồng tham chiếu chỉ bệnh nhân. Các khái niệm thuộc chuỗi đồng tham chiếu này sẽ được xem là khái niệm chỉ bệnh nhân và được chọn làm mẫu dương trong quá trình huấn luyện. Các khái niệm thuộc lớp Person còn lại không thuộc vào chuỗi đồng tham chiếu này sẽ được chọn làm mẫu âm trong quá trình huấn luyện. Tuy nhiên, chúng tôi nhận thấy phương pháp xác định bệnh nhân này có một nhược điểm là các BAĐT nhỏ, có nội dung ngắn sẽ tồn tại nhiều chuỗi đồng tham chiếu lớp Person có kích thước tương tự nhau. Trong đó chuỗi đồng tham chiếu chỉ bệnh nhân không chắc chắn là chuỗi đồng tham chiếu có kích thước lớn nhất.

Bảng \ref{tab:PatientFeatures} trình bày đầy đủ các đặc trưng được sử dụng cho việc xác định khái niệm có phải là bệnh nhân hay không. Kết quả của việc phân loại này sẽ được sử dụng làm giá trị cho đặc trưng ``Patient-class'' khi rút trích đặc trưng cho lớp Person.

\begin{table}[th]
\centering\ra{1.2}
\caption{Tập đặc trưng cho lớp Patient \label{tab:PatientFeatures}}
\footnotesize\sffamily

\begin{tabularx}{\textwidth}{@{}P{\raggedright}{0.3}lL@{}}
\toprule 
\textbf{Đặc Trưng} & \textbf{Giá trị} & \textbf{Giải thích}\\
\midrule
Keyword of patient & 0, 1 & Các từ khóa về bệnh nhân (như mr., mr, ms., ms, yo-, y.o., y/o, year-old, ...)\\
Keyword of doctor & 0, 1 & Các từ khóa về bác sĩ (dr, dr., md, m.d., m.d,…)\\
Key word of doctor title & 0, 1 & Các từ khóa về chức vụ của bác sĩ (dentist, orthodontist, …)\\
Key word of department  & 0, 1 & Các từ khóa về chuyên ngành bác sĩ (electrophysiology, …)\\
Key word of general deparment & 0, 1 & Các từ khóa chung về phòng ban (team, service)\\
Key word of general doctor & 0, 1 & Các từ khóa chung về bác sĩ (doctor, dict, author, pcp, attend, provider)\\
Key word of relative & 0, 1 & Các từ khóa về người thân (wife, brother, sibling, nephew)\\
Name & 0, 1 & Có phải là tên riêng hay không\\
Last n line doctor & 0, 1 & Là tên bác sĩ ở n dòng cuối cùng\\
Twin or triplet information & 0, 1 & Thông tin về cặp sinh đôi, sinh ba (baby 1, 2, 3,…)\\
Preceded by non-patient & 0, 1 & Khái niệm đứng trước không phải là bệnh nhân.\\
Signed information  & 0, 1 & Có liên quan đến việc kí/xác nhận bệnh án\\
Previous sentence &  & Câu hoàn chỉnh liền trước khái niệm\\
Next sentence &  & Câu hoàn chỉnh liền sau khái niệm\\
Pronouns we & 0, 1 & Là đại từ chỉ chúng tôi (we, us, our, ourselves)\\
Pronouns I & 0, 1 & Là đại từ chỉ tôi (I, my, me, myself)\\
Pronouns you & 0, 1 & Là đại từ chỉ bạn (you, your, yourself)\\
Pronouns they & 0, 1 & Là đại từ chỉ họ (they, them, their, themselves)\\
Pronouns he/she most & 0, 1 & Thuộc phần đa số của đại từ chỉ cô ấy/anh ấy (he, his, her)\\
Who & 0, 1 & Là đại từ “who” hoặc liền kề với khái niệm đứng trước\\
Appositive & 0, 1 & Là đồng vị ngữ\\
\bottomrule
\end{tabularx}
\end{table}

Các đặc trưng về từ khóa được chúng tôi hiện thực bằng cách khảo sát tập dữ liệu và xây dựng bộ từ điển thích hợp cho từng đặc trưng.

Các đặc trưng ``Previous sentence'' và ``Next sentence'' được hiện thực bằng cách khảo sát toàn bộ các khái niệm thuộc lớp Person, sau đó xây dựng bộ từ điển các câu có thể đứng trước hoặc đứng sau khái niệm đang xét. Giá trị của đặc trưng được lấy bằng chỉ mục của câu đứng trước (hoặc đứng sau) trong bộ từ điển các câu.

Đặc trưng ``Pronouns he/she most'' mang ý nghĩa giới tính chiếm đa số trong BAĐT được xét. Việc xác định giới tính chiếm đa số trong BAĐT được hiện thực bằng cách xác định giới tính cho từng khái niệm thuộc lớp Person, sau đó chọn giới tính có số lượng khái niệm lớn hơn. Phương pháp xác định giới tính được thực hiện theo miêu tả trong đặc trưng của nhóm Person. Nếu trong BAĐT có giới tính Nam chiếm đa số thì những khái niệm là đại từ chỉ về giới tính Nam như ``he'', ``him'', ``himself'', ... sẽ có đặc trưng ``Pronouns he/she most'' mang giá trị là 1. Tương tự cho BAĐT có giới tính Nữ chiếm đa số.

\subsection*{Nhóm Pronoun}

\subsection*{Nhóm Problem/Test/Treatment}

\section{Gom cụm và xây dựng chuỗi đồng tham chiếu}
Ở mô hình cặp thực thể, hệ thống phân loại không có khả năng xây dựng chuỗi đồng tham chiếu mà nó chỉ có thể xác định một cặp khái niệm là có đồng tham chiếu hay không. Mặt khác, đối với một văn bản HSXV, số cặp khái niệm được sinh ra rất nhiều và trong số đó có nhiều cặp có chung khái niệm đứng sau, ví dụ hai cặp “Dr. John”-“his” và “Mr. Brown”-“his” có chung khái niệm đứng sau là “his” mà hai cặp này đều được hệ thống phân loại xác định là đồng tham chiếu, tuy nhiên chỉ một trong hai khái niệm “Dr. John” và “Mr. Brown” được chọn làm tiền đề cho khái niệm “his” này. Như vậy cần thiết phải có một giải thuật lựa chọn các cặp đồng tham chiếu và xây dựng các chuỗi đồng tham chiếu từ chúng.

Như đã được đề cập ở mục, có hai giải thuật được đề xuất là: \emph{gom cụm gần nhất trước} và \emph{gom cụm tốt nhất trước}. Chúng tôi lựa chọn thực hiện giải thuật gom cụm tốt nhất trước cho hệ thống của mình vì hai lý do:
\begin{enumerate}
\item Theo [x], giải thuật gom cụm tốt nhất trước cho kết quả tốt hơn giải thuật gom cụm gần nhất trước.
\item Các tác giả hệ thống [I] cũng hiện thực giải thuật này cho hệ thống của họ.
\end{enumerate}



\section{Đánh giá hiệu năng}

\subsection*{Hệ đo MUC}

\subsection*{Hệ đo B-CUBED}

\subsection*{Hệ đo CEAF}
