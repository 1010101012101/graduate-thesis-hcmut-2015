\chapter{Hiện thực hệ thống} \label{hienthuchethong}
\section{Nội dung bài toán}
Bài toán mà chúng tôi giải quyết là bài toán: ``Phân giải đồng tham chiếu cho các hồ sơ xuất viện tiếng Anh với các khái niệm đã được trích xuất và gán nhãn''. Đầu vào của bài toán bao gồm hai phần:

\begin{enumerate}[leftmargin=\the\parindent]
\item \emph{Tập các hồ sơ xuất viện: }Đây là những văn bản lâm sàng được viết tay bằng ngôn ngữ tự nhiên bởi các bác sĩ, y tá. Chúng mô tả lại toàn bộ thông tin của bệnh nhân trong một lần điều trị, bao gồm các thông tin về tên bệnh mà bệnh nhân mắc phải, các thủ tục y tế được thực hiện và các phương pháp điều trị được áp dụng lên bệnh nhân. Để thuận tiện, từ đây trở đi chúng tôi dùng từ viết tắt HSXV để nói về các hồ sơ này.
\item \emph{Tập các khái niệm đã được trích xuất và gán nhãn:} Mỗi hồ sơ xuất viện đi kèm với một văn bản chứa toàn bộ các khái niệm được đề cập trong hồ sơ đó. Các khái niệm này đã được gán nhãn cho phù hợp với loại thực thể mà nó đề cập tới. Có tất cả năm nhãn là Problem, Treatment, Test, Person và Pronoun được i2b2 định nghĩa. Bảng \ref{tab:EntityLabels} mô tả chi tiết ý nghĩa của năm nhãn này.
\end{enumerate}

Mục tiêu của chúng tôi là phân giải đồng tham chiếu cho các khái niệm trong tập các khái niệm ứng với mỗi hồ sơ xuất viện. Cụ thể kết quả đầu ra là danh sách các chuỗi đồng tham chiếu của khái niệm đó, ví dụ \texttt{c="the patient" 13:0 13:1||c="he" 14:0 14:0||c="his" 14:7 14:7||t="coref person"} mô tả một chuỗi đồng tham chiếu bao gồm các khái niệm \texttt{"the patient"} (xuất hiện ở dòng thứ 13, từ vị trí 0 đến 1), \texttt{"he"} và \texttt{"his"}. Các khái niệm này đồng tham chiếu tới cùng một người (\texttt{t="coref person"}).

\begin{table}[th]
\centering\ra{1.3}
\caption{Ý nghĩa các lớp thực thể được đề xuất bởi i2b2\label{tab:EntityLabels}}
\footnotesize\sffamily

\begin{tabularx}{\textwidth}{@{}lLP{\colleft}{0.3}@{}}
\toprule
\textbf{Lớp} & \textbf{Định nghĩa} & \textbf{Ví dụ}\\
\midrule
\emph{Person} & Những chủ thể người hoặc một nhóm người được đề cập trong bệnh án và các đại từ nhân xưng & Dr.Lightman, the patient, cardiology, he, she, ...\\
\emph{Problem} & Những bất thường về sức khỏe thân thể hoặc tinh thần của bệnh nhân, được mô tả bởi bệnh nhân hoặc quan sát của bác sĩ & Heart attack, blood pressure, cancer, ...\\
\emph{Test} & Những thủ tục y tế như xét nghiệm, đo đạc, kiểm tra trên cơ thể bệnh nhân để cung cấp thêm thông tin cho ``Problem'' & CT scan, Temperature, ...\\
\emph{Treatment} & Những thủ tục y tế hoặc quy trình áp dụng để chữa trị cho ``Problem'', bao gồm thuốc, phẫu thuật hoặc phương pháp điều trị & Surgery, ice pack, Tylenol, ...\\
\emph{Pronoun} & Những đại từ có thể tham chiếu đến bất kì lớp nào trong bốn lớp kể trên nhưng không phải là đại từ nhân xưng & Which, it, that, ...\\
\bottomrule
\end{tabularx}
\end{table}

\section{Ý tưởng hiện thực\label{ytuonghienthuc}}
Dựa vào hệ thống có hiệu năng tốt nhất của thử thách i2b2 năm 2011 (hệ thống \cite{YanXu2012}), mô hình phân giải đồng tham chiếu mà chúng tôi sử dụng để hiện thực hệ thống là mô hình cặp thực thể. Tư tưởng cơ bản của mô hình này là xác định xem hai khái niệm bất kì có đồng tham chiếu với nhau hay không, sau đó gom nhóm các cặp đồng tham chiếu có một khái niệm chung lại để tạo thành các chuỗi đồng tham chiếu. Như vậy kiến trúc tổng quát của hệ thống chúng tôi hiện thực gồm hai quy trình: \emph{quy trình huấn luyện hệ thống phân loại} và \emph{quy trình phân giải đồng tham chiếu}. Trong đó quy trình huấn luyện là bước huấn luyện các model phân loại dựa trên dữ liệu mẫu đã được phân giải đồng tham chiếu. Quy trình phân giải sử dụng các model phân loại đã được huấn luyện để xác định tính đồng tham chiếu của các cặp khái niệm, từ đó sử dụng một giải thuật gom nhóm các cặp đồng tham chiếu lại để tạo thành các chuỗi đồng tham chiếu.

\subsection*{Quy trình huấn luyện}
Để xác định tính đồng tham chiếu giữa hai khái niệm bất kì, ta cần huấn luyện một mô hình phân loại dựa trên dữ liệu mẫu. Vì đầu vào của quy trình là các văn bản HSXV và danh sách các khái niệm đã được gán nhãn, hệ thống cần trích xuất đặc trưng của các dữ liệu thô này rồi mới có thể đưa vào để huấn luyện. Bên cạnh đó, các khái niệm đã được phân loại vào bốn nhóm chính là Person, Problem, Test và Treatment, còn các đại từ được phân vào nhóm Pronoun nên để giảm bớt số cặp khái niệm được sinh ra, chúng tôi huấn luyện bốn model để xác định tính đồng tham chiếu của riêng các cặp Person-Person, Problem-Problem, Test-Test và Treatment-Treatment (vì hai khái niệm thuộc hai lớp khác nhau thì nghiễm nhiên không đồng tham chiếu với nhau). Các đại từ đa phần chỉ tới các khái niệm ở trước đó, nên việc xác định xem một đại từ thực chất mang ý nghĩa của lớp nào trong bốn lớp chính Person, Problem, Test, Treatment là một việc quan trọng. Sau khi xác định được lớp chính của đại từ, chúng tôi chọn khái niệm thuộc lớp tương ứng ở gần nhất trước đó làm tiền đề cho nó. Các ý này đều là của các tác giả hệ thống \cite{YanXu2012}.

Ngoài ra cũng theo các tác giả này, thông tin một khái niệm lớp Person có chỉ về bệnh nhân hay không góp một phần quan trọng trong việc phân loại đúng tính đồng tham chiếu của các cặp khái niệm lớp này. Trong miền văn bản HSXV, các khái niệm chỉ người thường chỉ đề cập đến một trong ba loại: bệnh nhân, người thân của bệnh nhân và nhân sự của bệnh viện. Một BAĐT, mà cụ thể là một HSXV, thông thường chỉ đề cập đến một bệnh nhân nên những khái niệm nào chỉ về bệnh nhân thì chắc chắn nằm trong cùng chuỗi đồng tham chiếu lớn nhất và duy nhất chỉ về bệnh nhân đó. Từ nhận định này, nhóm tác giả của hệ thống \cite{YanXu2012} thêm vào đặc trưng lớp Patient (Patient-class) cho cặp hai khái niệm lớp Person, nó mang giá trị ``có'' khi hai khái niệm đều chỉ về bệnh nhân và ``không'' trong các trường hợp khác. Ở bước huấn luyện, thông tin ``một khái niệm Person có chỉ về bệnh nhân hay không'' được lấy từ tập chuỗi kết quả, còn ở bước phân giải đồng tham chiếu thông tin này được xác định nhờ một model phân loại đã được huấn luyện.

Như vậy mục đích của quy trình huấn luyện là xây dựng tổng cộng sáu mô hình SVM, trong đó bốn mô hình SVM nhằm mục đích phân loại và đánh giá độ tin cậy đồng tham chiếu của các cặp khái niệm Person-Person, Problem-Problem, Test-Test và Treatment-Treatment; một mô hình SVM để xác định các khái niệm Person có là bệnh nhân hay không (Patient-class) và một mô hình SVM để phân loại các đại từ (các khái niệm lớp Pronoun) vào một trong bốn lớp Person, Problem, Test và Treatment. Đầu vào của quy trình này là toàn bộ các văn bản HSXV với các khái niệm đã được trích xuất và gán nhãn. Sau khi tiền xử lý, hệ thống xây dựng các mẫu huấn luyện bao gồm: Person, Person-Person, Problem-Problem, Test-Test, Treatment-Treatment và Pronoun từ danh sách các khái niệm. Sáu tập mẫu này được trích xuất thuộc tính và đưa vào để huấn luyện sáu SVM model (Hình \ref{fig:SDHL}). Thư viện SVM được nhóm sử dụng là LibSVM. 

\begin{figure}[th]
\centering
\begin{tikzpicture}[%
	>=angle 60,
	start chain=going below,
	node distance=1.5cm and 2.3cm,
	every join/.style={->, draw},
	font=\scriptsize\sffamily]

	\node[multidoc](emr){HSXV};
	\node[multidoc, right=of emr, text width=3.5em](con){Danh sách khái niệm};
	\node[proc,below=1.3cm of $(emr)!0.5!(con)$](prep){Tiền xử lý};
	\node[io,join]{Các khái niệm/cặp khái niệm};
	\node[proc,join](ex){Rút trích đặc trưng};
	\node[io, right=3cm of con](perp){Đặc trưng cặp Person};
	\node[io](pati){Đặc trưng Person};
	\node[io](prop){Đặc trưng cặp Problem};
	\node[io](tstp){Đặc trưng cặp Test};
	\node[io](trep){Đặc trưng cặp Treatment};
	\node[io](pron){Đặc trưng Pronoun};	
	\node[proc, right=2.6cm of $(prop)!0.5!(tstp)$](train){Huấn luyện};
	\node[io,right=7cm of perp](perm){Mô hình SVM cặp Person};
	\node[io](patim){Mô hình SVM lớp Patient};
	\node[io](prom){Mô hình SVM cặp Problem};
	\node[io](tstm){Mô hình SVM cặp Test};
	\node[io](trem){Mô hình SVM cặp Treatment};
	\node[io](pronm){Mô hình SVM Pronoun};

	\draw[->] (emr) -- ++(down:0.9cm) -| (prep.130);
	\draw[->] (con) -- ++(down:0.9cm) -| (prep.50);
	\draw[->] (ex) -- ++(right:2.3cm) |- (perp);
	\draw[->] (ex) -- ++(right:2.3cm) |- (pati);
	\draw[->] (ex) -- ++(right:2.3cm) |- (prop);
	\draw[->] (ex) -- ++(right:2.3cm) |- (tstp);
	\draw[->] (ex) -- ++(right:2.3cm) |- (trep);
	\draw[->] (ex) -- ++(right:2.3cm) |- (pron);
	\draw[->] (perp) -- ++(right:1.8cm) |- (train);
	\draw[->] (pati) -- ++(right:1.8cm) |- (train);
	\draw[->] (prop) -- ++(right:1.8cm) |- (train);
	\draw[->] (tstp) -- ++(right:1.8cm) |- (train);
	\draw[->] (trep) -- ++(right:1.8cm) |- (train);
	\draw[->] (pron) -- ++(right:1.8cm) |- (train);
	\draw[->] (train) -- ++(right:1.5cm) |- (perm);
	\draw[->] (train) -- ++(right:1.5cm) |- (patim);
	\draw[->] (train) -- ++(right:1.5cm) |- (prom);
	\draw[->] (train) -- ++(right:1.5cm) |- (tstm);
	\draw[->] (train) -- ++(right:1.5cm) |- (trem);
	\draw[->] (train) -- ++(right:1.5cm) |- (pronm);	
\end{tikzpicture}
\caption{Sơ đồ huấn luyện\label{fig:SDHL}}
\end{figure}

\subsection*{Quy trình phân giải}
Quy trình phân giải đồng tham chiếu sử dụng sáu mô hình SVM đã được huấn luyện ở bước trên, cùng với đó là một giải thuật gom nhóm các cặp khái niệm đã được phân loại là đồng tham chiếu với nhau lại để cuối cùng tạo thành các chuỗi đồng tham chiếu. Có thể xem đây là quy trình mang đi ứng dụng thực tế để phân giải cho những văn bản HSXV mới. Dựa vào hệ thống \cite{YanXu2012}, chúng tôi sử dụng giải thuật gom cụm tốt nhất trước để lựa chọn các cặp đồng tham chiếu có độ tin cậy cao nhất, sau đó xây dựng các chuỗi đồng tham chiếu bằng cách nối các cặp có một khái niệm chung. Đối với lớp Pronoun, sau khi đã xác định được lớp chính của một đại từ bất kì, chúng tôi tạo một cặp đồng tham chiếu giữa đại từ đó và khái niệm thuộc lớp chính tương ứng ở gần nhất trước đó trong văn bản. Theo nhận định của các tác giả hệ thống \cite{YanXu2012}, tuy cách làm này đơn giản nhưng lại tỏ ra rất hiệu quả.

Hình \ref{fig:SDPG} mô tả trực quan quy trình phân giải đồng tham chiếu. Ở bước trích xuất thuộc tính của các cặp Person, chúng tôi sử dụng model phân loại bệnh nhân để xác định giá trị cho đặc trưng Patient-class. Theo kết quả đánh giá các hệ thống dự thi thử thách i2b2 2011, ba hệ đo được sử dụng để đánh giá hiệu năng là: MUC, B-CUBED và CEAF. Chúng tôi cũng hiện thực các hệ đo này để đánh giá các chuỗi đồng tham chiếu được xuất ra bởi hệ thống của mình và so sánh kết quả với hệ thống \cite{YanXu2012}. 

Ngoài ra Hình \ref{fig:DongKhaiNiem} mô tả các luồng khái niệm bắt đầu từ danh sách các khái niệm thô đến các chuỗi đồng tham chiếu, trong đó các khối \emph{Phân giải đồng tham chiếu lớp X} mang ý nghĩa trừu tượng về hệ thống phân giải đồng tham chiếu bao gồm các module rút trích đặc trưng, module phân loại và module gom cụm tương ứng cho lớp $X$.

\begin{figure}[th]
\centering
\begin{tikzpicture}[%
	>=angle 60,
	start chain=going below,
	node distance=1.7cm and 2.7cm,
	every join/.style={->, draw},
	font=\scriptsize\sffamily]

	\node[doc](emr){HSXV};
	\node[doc, right=of emr, text width=3.5em](con){Danh sách khái niệm};
	\node[proc, below=1.3cm of $(emr)!0.5!(con)$](prep){Tiền xử lý};
	\node[io,join](ins){Các khái niệm/cặp khái niệm};
	\node[proc,join](ex){Rút trích đặc trưng};
	\node[io, right=of ex, join](feat){Tập vector đặc trưng};
	\node[proc, right=of feat, join](clas){Phân loại};
	\node[multidoc, above=of clas, text width=3.5em](model){6 mô hình SVM};
	\node[io,below=of clas](conf){Độ tin cậy đồng tham chiếu};
	\node[proc,join,below=of ex](clus){Gom cụm};
	\node[io,join](sysc){Các chuỗi đồng tham chiếu};
	\node[proc,join,right=of sysc](eval){Đánh giá hiệu năng};
	\node[doc,right=of eval, text width=2cm](gt){Các chuỗi đồng tham chiếu kết quả (ground truth)};
	\node[io,below=of eval](res){Các số liệu độ đo};

	\draw[->] (emr) -- ++(down:0.9cm) -| (prep.130);
	\draw[->] (con) -- ++(down:0.9cm) -| (prep.50);
	\draw[->] (model) -> (clas);
	\draw[->] (clas) -> (conf);
	\draw[->] (ins) -- ++(left:1.8cm) |- (clus);
	\draw[->] (gt) -> (eval);
	\draw[->] (eval) -> (res);
	\draw[->] (model.200) -- ++(left:3cm) node[midway,above,sloped]{SVM Patient-class} -> (ex.north east);
\end{tikzpicture}
\caption{Sơ đồ phân giải đồng tham chiếu\label{fig:SDPG}}
\end{figure}

\begin{figure}[ht]
\centering
\begin{tikzpicture}[%
	>=angle 60,
	start chain=going below,
	node distance=1.5cm and 4cm,
	every join/.style={->, draw},
	font=\scriptsize\sffamily]
	\tikzset{
		wproc/.style = {proc, text width=5.2em},
		wdoc/.style = {doc, text width=3.7em}
	};
	
	\node[io](pati){Các khái niệm Person};
	\node[io](perp){Các cặp khái niệm Person};
	\node[io](prop){Các cặp khái niệm Problem};
	\node[io](tstp){Các cặp khái niệm Test};
	\node[io](trep){Các cặp khái niệm Treatment};
	\node[io](pron){Các khái niệm Pronoun};
	
	\node[wdoc,left=2.3cm of $(prop)!0.5!(tstp)$](emr){HSXV + DS khái niệm};
	\node[altproc,right=2.5cm of pati](patic){Phân loại bệnh nhân};
	\node[wproc,right=4.2cm of perp](perr){Phân giải đồng tham chiếu lớp Person};
	\node[wproc](pror){Phân giải đồng tham chiếu lớp Problem};
	\node[wproc](tstr){Phân giải đồng tham chiếu lớp Test};
	\node[wproc](trer){Phân giải đồng tham chiếu lớp Treatment};
	\node[wproc](pronr){Phân giải đồng tham chiếu lớp Pronoun};

	\node[io,right=5.2cm of perr](perch){Các chuỗi Person};
	\node[io](proch){Các chuỗi Problem};
	\node[io](tstch){Các chuỗi Test};
	\node[io](trech){Các chuỗi Treatment};
	
	\coordinate[left=0.5cm of perch.west](perch-left);
	\coordinate[left=1cm of proch.west](proch-left);
	\coordinate[left=1.5cm of tstch.west](tstch-left);
	\coordinate[left=2cm of trech.west](trech-left);
	
	\draw[->] (emr) -- ++(right:1.3cm) |- (pati);
	\draw[->] (emr) -- ++(right:1.3cm) |- (perp);
	\draw[->] (emr) -- ++(right:1.3cm) |- (prop);
	\draw[->] (emr) -- ++(right:1.3cm) |- (tstp);
	\draw[->] (emr) -- ++(right:1.3cm) |- (trep);
	\draw[->] (emr) -- ++(right:1.3cm) |- (pron);
	\draw[->] (pati) -> (patic);
	\draw[->] (patic) -| (perr);
	\draw[->] (perp) -> (perr);
	\draw[->] (prop) -> (pror);
	\draw[->] (tstp) -> (tstr);
	\draw[->] (trep) -> (trer);
	\draw[->] (pron) -> (pronr);
	\draw[->] (perr) -> (perch);
	\draw[->] (pror) -> (proch);
	\draw[->] (tstr) -> (tstch);
	\draw[->] (trer) -> (trech);
	\draw[->] (pronr) -| (perch-left);
	\draw[->] (pronr) -| (proch-left);
	\draw[->] (pronr) -| (tstch-left);
	\draw[->] (pronr) -| (trech-left);
\end{tikzpicture}
\caption{Các luồng khái niệm tạo thành chuỗi đồng tham chiếu \label{fig:DongKhaiNiem}}
\end{figure}

\section{Tiền xử lý}
Trong quá trình rút trích đặc trưng, một số khái niệm được miêu tả cụ thể làm cho việc so trùng chuỗi hoặc tìm kiếm từ các nguồn tri thức nhân loại thiếu chính xác \cite{YanXu2012}. Ví dụ như khái niệm ``her CT scan'' và khái niệm ``a CT scan'', mặc dù hai khái niệm này cùng chỉ một thủ tục y tế nhưng không trùng chuỗi. Ngoài ra các mạo từ ``her'', ``a'' làm việc tìm kiếm tri thức nhân loại từ các nguồn như Wikipedia, WordNet không được chính xác hoặc không thể tìm được kết quả, vì vậy trước khi rút trích đặc trưng, các khái niệm cần được tiền xử lý để loại bỏ mạo từ. Nếu khái niệm có giới từ, chúng tôi cũng lược bỏ giới từ và đoạn văn theo sau giới từ đó. Tuy nhiên, quá trình tiền xử lý chỉ được áp dụng cho các đặc trưng liên quan so trùng chuỗi và tìm kiếm tri thức nhân loại, các đặc trưng khác không cần qua quá trình tiền xử lý mà nhận vào nguyên gốc khái niệm được xác định.

Đặc biệt các khái niệm thuộc lớp Problem/Treatment/Test thường được kèm thêm thông tin về định lượng như 10mg, 5 lit và các thông tin về vị trí giải phẫu học như ``upper'', ``left'', ``right''. Để tăng khả năng tìm kiếm tri thức nhân loại, chúng tôi đề xuất loại bỏ các thông tin ngữ cảnh về số, định lượng và vị trí giải phẫu khỏi khái niệm. Các thông tin ngữ cảnh được loại bỏ bằng cách sử dụng biểu thức chính quy và các từ vựng được xây dựng từ tập dữ liệu. Các đặc trưng liên quan so trùng chuỗi không áp dụng bước tiền xử lý loại bỏ thông tin ngữ cảnh này.

Tổng quan, quá trình tiền xử lý gồm hai bước: đầu tiên khái niệm sẽ được loại bỏ tất cả mạo từ, sau đó, nếu khái niệm có bao gồm giới từ thì giới từ đó và toàn bộ nội dung theo sau sẽ được lược bỏ. Ví dụ như khái niệm “an MRI of the knee” sau quá trình tiền xử lý sẽ trở thành “MRI”. Danh sách mạo từ được xây dựng từ tập dữ liệu và các mạo từ thông dụng của tiếng Anh. Cuối cùng, chúng tôi lược bỏ các thông tin về số, định lượng và vị trí.

\section{Xây dựng các cặp khái niệm}
Sau khi tiền xử lý, hệ thống xây dựng các cặp khái niệm với mục đích xác định xem hai khái niệm bất kì có đồng tham chiếu với nhau hay không. Vì SVM không có khả năng học tăng cường nên đối với \emph{quy trình huấn luyện}, chúng tôi cần xây dựng các cặp khái niệm từ toàn bộ danh sách khái niệm trong tập huấn luyện, sau đó rút trích đặc trưng cho toàn bộ các cặp khái niệm này và đưa vào huấn luyện các mô hình SVM. Một khó khăn của cách làm này là số cặp khái niệm phát sinh rất nhiều nếu tất cả các cặp được xây dựng cho một văn bản HSXV. Cụ thể với tập huấn luyện gồm 200 HSXV, mỗi hồ sơ chứa trung bình 150 khái niệm thì số các cặp khái niệm được sinh ra là $200\times \binom{150}{2}=2.235.000$ cặp. Dựa theo thực nghiệm của chúng tôi, sử dụng một tập vector đặc trưng chứa khoảng 300.000 vector để huấn luyện một mô hình SVM với kernel phi tuyến tính mất từ 4 đến 5 tiếng đồng hồ. Với một tập gồm hơn 2 triệu vector đặc trưng thì việc huấn luyện như vậy là bất khả thi.

Bên cạnh số lượng các vector đặc trưng, một vấn đề khác mà chúng tôi gặp phải trong quá trình huấn luyện là sự mất cân bằng trong phân bố lớp: các cặp không đồng tham chiếu (các mẫu âm) có số lượng quá nhiều so với các cặp đồng tham chiếu (các mẫu dương). Một tập vector đặc trưng bị mất cân bằng lớp sẽ khiến cho mô hình SVM được huấn luyện dựa trên tập đó có xu hướng phân loại về phía lớp có số lượng nhiều hơn. 

Để giải quyết vấn đề về số lượng cặp khái niệm, ở phần trước chúng tôi có nêu lên ý tưởng hiện thực là huấn luyện các mô hình SVM để phân loại tính đồng tham chiếu cho riêng các cặp Person-Person, Problem-Problem, Treatment-Treatment và Test-Test. Điều này không chỉ giúp cho các đặc trưng được rút trích chuyên biệt hơn cho mỗi loại cặp khái niệm, mà còn làm giảm đi số vector đặc trưng sinh ra để huấn luyện vì hệ thống chỉ xây dựng các cặp mà hai khái niệm có chung lớp chính.

Mặc dù cách làm trên giúp giảm bớt số lượng các mẫu huấn luyện, trong quá trình thực nghiệm, chúng tôi nhận thấy vấn đề về mất cân bằng mẫu vẫn xuất hiện ở các tập vector đặc trưng của các cặp Problem, Treatment và Test. Ở tập vector đặc trưng của cặp Test, số mẫu âm nhiều hơn số mẫu dương đến 381 lần. Vì vậy để giải quyết việc này, chúng tôi quyết định thử nghiệm với hai phương pháp chọn lọc mẫu sau:
\begin{enumerate}
\item Đối với mẫu dương, chỉ xây dựng các cặp đồng tham chiếu khi hai khái niệm nằm sát nhau theo thứ tự xuất hiện. Các mẫu âm được xây dựng giữa các khái niệm nằm giữa một cặp đồng tham chiếu và khái niệm đứng sau của cặp đó \cite{Soon2001}. Ví dụ Hình \ref{fig:insgen-eg} mô tả một đoạn văn bản và các chuỗi đồng tham chiếu của nó (giả định rằng tất cả các khái niệm ở ví dụ này thuộc cùng lớp). Có hai chuỗi đồng tham chiếu là $(C_1-C_2-C_3)$ và $(D_1-D_2)$, hai khái niệm $a$ và $b$ là duy nhất. Dựa theo phương pháp chọn lọc mẫu này, các mẫu dương bao gồm các cặp $(C_1,\,C_2)$, $(C_2,\,C_3)$ và $(D_1,\,D_2)$. Các mẫu âm bao gồm các cặp $(D_1,\,C_2)$, $(D_2,\,C_2)$, $(a,C_2)$ và $(b,C_2)$.

\begin{figure}[ht]
\centering
\begin{tikzpicture}[%
	>=angle 60,
	start chain=going right,
	node distance=1.5cm and 2cm,
	out=60, in=120,
	font=\sffamily]	
	\tikzset{
		con/.style={base, rectangle, inner sep=1mm, minimum height=0.8cm, minimum width=0.8cm},
		sgt/.style={base, circle, inner sep=1mm, minimum height=0.8cm}
	};

	\node[con](C1){$C_1$};
	\node[con](D1){$D_1$};
	\node[con](D2){$D_2$};
	\node[sgt](a){$a$};
	\node[sgt](b){$b$};
	\node[con](C2){$C_2$};
	\node[con](C3){$C_3$};
	
	\path[->, looseness=0.3] (C1.north) edge (C2.north);
	\path[->, looseness=0.7] (D1.north) edge (D2.north);
	\path[->, looseness=0.7] (C2.north) edge (C3.north);
	
\end{tikzpicture}
\caption{Ví dụ cho việc sinh các mẫu huấn luyện\label{fig:insgen-eg}}
\end{figure}

\item Phương pháp này xây dựng các mẫu huấn luyện theo các bước sau: với mỗi hồi chỉ $C_j$, xác định tiền đề của nó ở xa nhất theo thứ tự xuất hiện, gọi là $C_k$. Xây dựng các cặp với khái niệm đứng sau là $C_j$, khái niệm đứng trước là các khái niệm bắt đầu từ $C_k$ đến $C_{j-1}$ (chỉ xét những cặp hai khái niệm cùng lớp) \cite{VincentNg2002b}. Như vậy đối với ví dụ trên Hình \ref{fig:insgen-eg}, các mẫu  huấn luyện được sinh ra theo các bước của phương pháp này là: đối với $C_3$, tiền đề xa nhất của nó là $C_1$, như vậy các cặp $(C_1,\,C_3)$, $(D_1,\,C_3)$, $(D_2,\,C_3)$, $(a,\,C_3)$, $(b,\,C_3)$ và $(C_2,\,C_3)$ được xây dựng. Làm tương tự cho các khái niệm từ $C_2$ về $C_1$ để xây dựng các mẫu huấn luyện còn lại.
\end{enumerate}

Trong quá trình thử nghiệm hai phương pháp chọn lọc mẫu trên, chúng tôi nhận thấy số lượng cặp khái niệm của mỗi loại giảm đi đáng kể khiến cho thời gian huấn luyện được cải thiện. Tuy nhiên kết quả đánh giá hiệu năng không được như mong đợi, cụ thể là độ đầy đủ tăng tên rất cao trong khi độ chính xác lại rất thấp. Điều này có thể được dễ dàng giải thích là:
\begin{itemize}
\item Đa phần các cặp đồng tham chiếu ở ba lớp Problem, Test và Treatment là các cặp có hai khái niệm đồng thuận về một tính chất nào đó dựa trên chuỗi kí tự biểu diễn.
\item Do có nhiều cặp không đồng tham chiếu (các mẫu âm) bị loại bỏ, các thông tin về việc phân lớp cho các trường hợp dạng này không được đưa vào để huấn luyện nên mô hình phân loại có xu hướng xác định đồng tham chiếu cho các cặp mà hai khái niệm có liên quan về chuỗi kí tự biểu diễn.
\end{itemize}

Ngoài ra, chúng tôi nhận thấy đa phần các mẫu âm quan trọng bị loại bỏ là các cặp gồm hai khái niệm duy nhất nhưng có liên quan về chuỗi kí tự biểu diễn. Vì vậy để cải thiện độ chính xác, chúng tôi quyết định sử dụng cách xây dựng mẫu thứ 2 và đưa thêm vào đó các cặp gồm hai khái niệm duy nhất mà có sự trùng lắp chuỗi con giữa hai khái niệm này.

\section{Rút trích đặc trưng lớp Person và lớp Pronoun}
Các lớp Person và Pronoun là các khái niệm tổng quát, có thể được bắt gặp trong nhiều lĩnh vực khác nhau không chỉ trong riêng lĩnh vực y tế và HSXV. Hai lớp khái niệm này có thể được phân biệt nhờ vào các thuộc tính chung về mặt ngôn ngữ như ngữ pháp hay từ vựng. Tuy nhiên, khi đặt trong lĩnh vực y tế, ngoài các đặc tính chung đó, từng lớp khái niệm còn mang những đặc tính riêng khác nhau. Việc này đòi hỏi chúng tôi phải thiết kế các hệ thống rút trích đặc trưng và phân loại tương ứng khác nhau. Đối với lớp Person, đầu vào của hệ thống rút trích đặc trưng là một cặp gồm hai khái niệm chỉ con người, tuy nhiên đầu vào của hệ thống rút trích đặc trưng lớp Pronoun chỉ là một khái niệm duy nhất thuộc lớp tương ứng.

\subsection*{Lớp Person}
\begin{table}[t!]
\centering\ra{1.2}
\caption{Tập đặc trưng cho lớp Person \label{tab:PersonFeatures}}
\footnotesize\sffamily

\begin{tabularx}{\textwidth}{@{}P{\raggedright}{0.3}lL@{}}
\toprule 
\textbf{Đặc Trưng} & \textbf{Giá trị} & \textbf{Giải thích}\\
\midrule
Patient-class & 0, 1, 2 & Không có khái niệm nào là bệnh nhân (0), cả hai khái niệm đều là bệnh nhân (1), trường hợp khác (2)\\
Distance between sentences & 0, 1, 2, 3, ... & Số câu xuất hiện giữa hai khái niệm được xét\\
Distance between mentions & 0, 1, 2, 3, ... & Số khái niệm xuất hiện giữa hai khái niệm được xét\\
String match & 0, 1 & Trùng chuỗi hoàn toàn (1), ngược lại (0)\\
Levenshtein distance between two mentions & 0, 1, 2, 3, ... & Khoảng cách Levenshtein giữa hai khái niệm\\
Number & 0, 1, 2 & Cả hai đều là số ít hoặc nhiều (1), ngược lại (0), không xác định (2)\\
Gender & 0, 1, 2 & Cùng giới tính (1), khác giới tính (0), không xác định (2)\\
Apposition & 0, 1 & Là đồng vị ngữ (1), ngược lại (0)\\
Alias & 0, 1 & Là từ viết tắt hoặc cùng nghĩa (1), ngược lại (0)\\
Who & 0, 1 & Nếu hai khái niệm liền kề nhau và được phân cách bởi dấu ``:''\\
Name match & 0, 1 & Loại bỏ các	``stop word'' (dr, dr., mr, ...), so trùng chuỗi con, trùng (1), không trùng (0)\\
Relative match & 0, 1 & Cả hai đều cùng chỉ đến một thân nhân (1), ngược lại (0)\\
Department match & 0, 1 & Cả hai cùng chỉ đến một lĩnh vực y học (1), ngược lại (0)\\
Doctor title match & 0, 1 & Cả hai có cùng một chức vụ bác sĩ (1), ngược lại (0)\\
Doctor general match & 0, 1 & Cả hai cùng đề cập đến bác sĩ nói chung (1), ngược lại (0)\\
Twin/triplet & 0, 1 & Cả hai đều chỉ về cùng cặp sinh đôi/sinh ba (1), ngược lại (0)\\
We & 0, 1 & Cả hai đều chứa thông tin về ``chúng tôi'' (1), ngược lại (0)\\
You & 0, 1 & Cả hai đều chứa thông tin về ``tôi'' (1), ngược lại (0)\\
I & 0, 1 & Cả hai đều chứa thông tin về ``bạn'' (1), ngược lại (0)\\
Pronoun match & 0, 1 & Cả hai đều là đại từ chỉ người (1), ngược lại (0)\\
\bottomrule
\end{tabularx}
\end{table}

Như đã đề cập trong phần \ref{ytuonghienthuc}, trong HSXV, các khái niệm thuộc lớp Person thường được chia vào ba nhóm chính: bệnh nhân, người thân của bệnh nhân hoặc nhân sự của bệnh viện. Trong đó bệnh nhân là nhóm có số lượng khái niệm được đề cập nhiều nhất và chiếm phần lớn tổng số khái niệm lớp Person. Do vậy việc xác định một khái niệm thuộc vào nhóm nào đóng vai trò quan trọng trong việc phân giải chính xác chuỗi đồng tham chiếu cho khái niệm đó \cite{YanXu2012}. Đặc trưng lớp Patient được xác định bằng phương pháp phân loại nhị phân SVM. Hai nhóm người thân của bệnh nhân và nhân sự của bệnh viện được xác định bằng các đặc trưng từ vựng. Bảng \ref{tab:PersonFeatures} trình bày đầy đủ các đặc trưng dùng cho lớp Person.

Đặc trưng ``Alias'' được chúng tôi hiện thực theo các bước như sau. Bước một, loại bỏ các mạo từ hoặc đại từ trong khái niệm. Bước hai, kiểm tra các chữ cái đầu tiên của mỗi từ trong khái niệm có được viết hoa hay không. Nếu có thì ghép các chữ cái đầu tiên của mỗi từ để tạo thành từ viết tắt của khái niệm. Cuối cùng, so sánh hai từ đang xét với từ viết tắt của chính hai từ đó, nếu trùng chuỗi thì xác định đặc trưng ``Alias'' là 1, ngược lại là 0.

Đặc trưng về Giới tính được chúng tôi xác định dựa trên ba bước phân loại \cite{Soon2001}. Bước thứ nhất: kiểm tra khái niệm có chứa các đại từ xác định giới tính như ``Mr'', ``Ms'', ``she'', ``he'', ... hay không. Nếu có, xác định giới tính dựa trên đại từ xuất hiện. Nếu không thực hiện bước thứ hai: kiểm tra khái niệm có xuất hiện nhiều hơn một lần hay không. Nếu xuất hiện nhiều hơn một lần thì các lần xuất hiện có chứa đại từ xác định giới tính hay không. Ví dụ khái niệm ``Peter H. Diller'' có thể xuất hiện nhiều lần, trong đó có xuất hiện dưới hình thức ``Mr. Diller''. Nếu không thể xác định giới tính qua hai bước kiểm tra, khái niệm sẽ được phân loại bằng cách sử dụng cơ sở dữ liệu về tên tiếng Anh của hệ thống Apache OpenNLP.

Với các đặc trưng ``Name match'', ``Relative match'', ``Department match'', ``Doctor title match'', ``Doctor general match'', ``Twin/Triplet'', ``We'', ``You'', ``I'', ``Pronoun match'', chúng tôi hiện thực bằng cách xây dựng tập từ điển tương ứng với từng đặc trưng dựa trên việc khảo sát tập dữ liệu và sử dụng các biểu thức chính quy. Các đặc trưng còn lại, chúng tôi hiện thực như miêu tả trong bảng \ref{tab:PersonFeatures}.

\subsection*{Đặc trưng Patient-class}
\begin{table}[t!]
\centering\ra{1.2}
\caption{Tập đặc trưng cho lớp Patient \label{tab:PatientFeatures}}
\footnotesize\sffamily

\begin{tabularx}{\textwidth}{@{\hspace{1em}}P{\raggedright}{0.3}lL@{}}
\toprule 
\rowgroup{\textbf{Đặc Trưng}} & \textbf{Giá trị} & \textbf{Giải thích}\\
\midrule
\rowgroup{\emph{Ngữ nghĩa}}\\
Keyword of patient & 0, 1 & Các từ khóa về bệnh nhân (như mr., mr, ms., ms, yo-, y.o., y/o, year-old, ...)\\
Keyword of doctor & 0, 1 & Các từ khóa về bác sĩ (dr, dr., md, m.d., m.d,…)\\
Keyword of doctor title & 0, 1 & Các từ khóa về chức vụ của bác sĩ (dentist, orthodontist, …)\\
Keyword of department  & 0, 1 & Các từ khóa về chuyên ngành bác sĩ (electrophysiology, …)\\
Keyword of general deparment & 0, 1 & Các từ khóa chung về phòng ban (team, service)\\
Keyword of general doctor & 0, 1 & Các từ khóa chung về bác sĩ (doctor, dict, author, pcp, attend, provider)\\
Keyword of relative & 0, 1 & Các từ khóa về người thân (wife, brother, sibling, nephew)\\
Name & 0, 1 & Có phải là tên riêng hay không\\
Last $n$ line doctor & 0, 1 & Là tên bác sĩ ở $n$ dòng cuối cùng\\
Twin or triplet information & 0, 1 & Thông tin về cặp sinh đôi, sinh ba (baby 1, 2, 3,…)\\
Preceded by non-patient & 0, 1 & Khái niệm đứng trước không phải là bệnh nhân.\\
Signed information  & 0, 1 & Có liên quan đến việc kí/xác nhận bệnh án\\
Previous sentence &  & Câu hoàn chỉnh liền trước khái niệm\\
Next sentence &  & Câu hoàn chỉnh liền sau khái niệm\\
\rowgroup{\emph{Ngữ pháp}}\\
Pronouns we & 0, 1 & Là đại từ chỉ chúng tôi (we, us, our, ourselves)\\
Pronouns I & 0, 1 & Là đại từ chỉ tôi (I, my, me, myself)\\
Pronouns you & 0, 1 & Là đại từ chỉ bạn (you, your, yourself)\\
Pronouns they & 0, 1 & Là đại từ chỉ họ (they, them, their, themselves)\\
Pronouns he/she most & 0, 1 & Thuộc phần đa số của đại từ chỉ cô ấy/anh ấy (he, his, her)\\
Who & 0, 1 & Là đại từ “who” hoặc liền kề với khái niệm đứng trước\\
Appositive & 0, 1 & Là đồng vị ngữ\\
\rowgroup{\emph{Từ vựng}}\\
\multicolumn{3}{l}{First three words before mention}\\
\multicolumn{3}{l}{First three words after mentiont}\\
\multicolumn{3}{l}{First three words in between mention, Problem}\\
\multicolumn{3}{l}{Last three words in between mention, Problem}\\
\multicolumn{3}{l}{First three words in between mention, Treatment}\\
\multicolumn{3}{l}{Last three words in between mention, Treatment}\\
\multicolumn{3}{l}{First three words in between mention, Test}\\
\multicolumn{3}{l}{Last three words in between mention, Test}\\
\bottomrule
\end{tabularx}
\end{table}

Từ nhận định trong việc rút trích đặc trưng của lớp Person, chúng tôi huấn luyện một mô hình SVM nhị phân để phân loại khái niệm thuộc lớp Person có phải là bệnh nhân hay không. Trong HSXV thường chỉ có một bệnh nhân đóng vai trò là chủ thể của bệnh án.Vì vậy, các khái niệm nếu được xác định là bệnh nhân, thì sẽ được đưa vào một chuỗi đồng tham chiếu duy nhất về bệnh nhân đó. Thông qua phân tích tập dữ liệu, chúng tôi nhận thấy việc xác định một khái niệm thuộc lớp Person hay không có thể đạt được bằng cách xác định tập từ khóa chỉ về bệnh nhân như ``patient'', ``pt'', ... và tập từ khóa chỉ về nhóm người không phải bệnh nhân như ``doctor'', ``dr'', ``wife'', ...

Vì tập dữ liệu không có thông tin xác định một khái niệm thuộc lớp Person có phải là bệnh nhân hay không, dựa theo hệ thống \cite{YanXu2012} chúng tôi xác định bằng cách chọn chuỗi đồng tham chiếu có nhiều khái niệm nhất trong tập kết quả làm chuỗi đồng tham chiếu chỉ bệnh nhân. Các khái niệm thuộc chuỗi đồng tham chiếu này sẽ được xem là khái niệm chỉ bệnh nhân và được chọn làm mẫu dương trong quá trình huấn luyện. Các khái niệm thuộc lớp Person còn lại không thuộc vào chuỗi đồng tham chiếu này sẽ được chọn làm mẫu âm trong quá trình huấn luyện. Tuy nhiên, chúng tôi nhận thấy phương pháp xác định bệnh nhân này có một nhược điểm là các HSXV nhỏ, có nội dung ngắn sẽ tồn tại nhiều chuỗi đồng tham chiếu lớp Person có kích thước tương tự nhau. Trong đó chuỗi đồng tham chiếu chỉ bệnh nhân không chắc chắn là chuỗi đồng tham chiếu có kích thước lớn nhất.

Bảng \ref{tab:PatientFeatures} trình bày đầy đủ các đặc trưng được sử dụng cho việc xác định khái niệm có phải là bệnh nhân hay không. Kết quả của việc phân loại này sẽ được sử dụng làm giá trị cho đặc trưng ``Patient-class'' khi rút trích đặc trưng cho lớp Person.

Các đặc trưng ``Previous sentence'' và ``Next sentence'' được hiện thực bằng cách khảo sát toàn bộ các khái niệm thuộc lớp Person, sau đó xây dựng bộ từ điển các câu có thể đứng trước hoặc đứng sau khái niệm đang xét. Giá trị của đặc trưng được lấy bằng chỉ mục của câu đứng trước (hoặc đứng sau) trong bộ từ điển các câu.

Đặc trưng ``Pronouns he/she most'' mang ý nghĩa giới tính chiếm đa số trong HSXV được xét. Việc xác định giới tính chiếm đa số trong HSXV được hiện thực bằng cách xác định giới tính cho từng khái niệm thuộc lớp Person, sau đó chọn giới tính có số lượng khái niệm lớn hơn. Phương pháp xác định giới tính được thực hiện theo miêu tả trong đặc trưng của nhóm Person. Nếu trong HSXV có giới tính Nam chiếm đa số thì những khái niệm là đại từ chỉ về giới tính Nam như ``he'', ``him'', ``himself'', ... sẽ có đặc trưng ``Pronouns he/she most'' mang giá trị là 1. Tương tự cho HSXV có giới tính Nữ chiếm đa số.

Qua việc khảo sát tập dữ liệu, chúng tôi nhận thấy trong HSXV thường được kết thúc bằng các thông tin hướng dẫn liên lạc với bác sĩ nếu bệnh nhân gặp vấn đề sau khi xuất viện, thông tin về ngày tháng và mã hồ sơ, thông tin về hướng dẫn sau khi xuất viện và ký tên của bác sĩ. Để xác định phần HSXV chứa các thông tin này, chúng tôi khảo sát từng câu trong HSXV từ dưới lên đến khi gặp các từ khóa như ``Follow-up'', ``Dictated By'', ``Signed By'', ... Các đặc trưng ``Last n line doctor'' và ``Signed information'' được xác định bằng các thông tin trong phân đoạn HSXV này.

Các đặc trưng liên quan đến từ khóa được chúng tôi hiện thực bằng cách khảo sát tập dữ liệu và xây dựng bộ từ điển thích hợp cho từng đặc trưng. Các đặc trưng còn lại được chúng tôi hiện thực theo như miêu tả trong bảng \ref{tab:PatientFeatures}.

\subsection*{Lớp Pronoun}
\begin{table}[t!]
\centering\ra{1.2}
\caption{Tập đặc trưng cho lớp Pronoun \label{tab:PronounFeatures}}
\footnotesize\sffamily

\begin{tabularx}{\textwidth}{@{\hspace{1em}}P{\raggedright}{0.3}lL@{}}
\toprule 
\rowgroup{\textbf{Đặc Trưng}} & \textbf{Giá trị} & \textbf{Giải thích}\\
\midrule
\rowgroup{\emph{Quan hệ}}\\
First previous mention type & 0, 1, 2, 3, 4 & Khái niệm đứng liền trước thuộc lớp Person (0) hoặc Problem (1) hoặc Treatment (2) hoặc Test (3) hoặc không thuộc lớp nào (4)\\
Second previous mention type & 0, 1, 2, 3, 4 & Khái niệm đứng liền trước thứ 2 thuộc lớp Person (0) hoặc Problem (1) hoặc Treatment (2) hoặc Test (3) hoặc không thuộc lớp nào (4)\\
First next mention type & 0, 1, 2, 3, 4 & Khái niệm đứng liền sau thuộc lớp Person (0) hoặc Problem (1) hoặc Treatment (2) hoặc Test (3) hoặc không thuộc lớp nào (4)\\
\rowgroup{\emph{Khoảng cách}}\\
Sentence distance & 0, 1, 2, ... & Số câu xuất hiện giữa hai khái niệm được xét\\
\rowgroup{\emph{Ngữ pháp}}\\
Pronoun & 0, 1, 2, ..., 14 & Chỉ mục của đại từ được xét trong bảng tra 15 đại từ\\
\rowgroup{\emph{Cú pháp}}\\
Part of speech & 0, 1, 2 & DT (0), WDT (1), PRP (2)\\
First next verb after mention & & Động từ đứng liền sau khái niệm\\
First word before mention is preposition & 0, 1 & Từ liền trước là giới từ (1), ngược lại (0)\\
First one/two/three words before mention & & 1/2/3 từ liền trước khái niệm được xét\\
First one/two/three words after mention & & 1/2/3 từ liền sau khái niệm được xét\\
An adjacent mention after pronoun + VP  & 0, 1 & Khái niệm được xét liền sau đại từ + cụm động từ\\
\rowgroup{\emph{Ngữ nghĩa}}\\
And, as well as, in addition to & 0, 1 & Khái niệm được xét liền sau ``and'', ``as well as'', ``in addition to''\\
\bottomrule
\end{tabularx}
\end{table}

Khác với các lớp khái niệm khác trong BAĐT, lớp khái niệm Pronoun là lớp khái niệm trừu tượng, có thể chỉ về bất kì khái niệm thuộc bốn lớp Person, Problem, Treatment, Test hoặc là khái niệm duy nhất (singleton). Để giải quyết vấn đề đồng tham chiếu cho lớp Pronoun, tác giả của hệ thống I đề xuất xây dựng hệ thống phân loại SVM nhiều lớp (multi-class SVM). Hệ thống SVM này dùng để phân loại khái niệm thuộc lớp Pronoun đang xét đồng tham chiếu đến lớp khái niệm nào trong bốn lớp Person, Problem, Treatment, Test. Sau đó ta chọn khái niệm thuộc lớp tương ứng ở gần nhất trước đó để làm tiền đề cho đại từ đang xét. Ví dụ ``\underline{Hepatitis C cirrhosis} for \textit{which} the patient was on the liver transplant list'' có ``which'' là đại từ đang xét và ``Hepatitis C cirrhosis'' là khái niệm thuộc lớp Problem. Nếu xác định được đại từ ``which'' thuộc lớp Problem, ta có thể kết luận ``which'' và ``Hepatitis C cirrhosis'' đồng tham chiếu do khái niệm ``Hepatitis C cirrhosis'' cùng thuộc lớp Problem và ở gần nhất trước đó. Bảng \ref{tab:PronounFeatures} mô tả đầy đủ các đặc trưng được sử dụng cho lớp Pronoun.

Các đặc trưng ``First next verb after mention'', ``First one/two/three words before mention'', ``First one/two/three words after mention'' được hiện thực bằng cách khảo sát toàn bộ các khái niệm thuộc lớp Pronoun, sau đó xây dựng bộ từ điển các từ có thể đứng trước hoặc đứng sau khái niệm đang xét. Giá trị của đặc trưng được lấy bằng chỉ mục của từ đứng trước (hoặc đứng sau) trong bộ từ điển các từ.

Các đặc trưng ``Part of speech'', ``First word after mention is preposition'', ``An adjacent mention after pronoun + VP'' được hiện thực bằng cách tìm câu văn chứa khái niệm đang xét trong BAĐT. Sau đó, chúng tôi tiến hành phân đoạn câu (tokenize) và đánh dấu thông tin từ loại (POS tag) đối với câu văn chứa khái niệm đang xét. Từ danh sách thông tin từ loại có được, các đặc trưng nêu trên được tính giá trị theo mô tả. Chúng tôi sử dụng hệ thống xử lý ngôn ngữ tự nhiên Apache OpenNLP cho các tác vụ phân đoạn câu và đánh dấu thông tin từ loại. Các đặc trưng còn lại được chúng tôi hiện thực theo như mô tả trong bảng \ref{tab:PronounFeatures}.

\section{Rút trích đặc trưng lớp Problem, Treatment và Test}
Nhóm lớp Problem/Treatment/Test là nhóm lớp đặc biệt thuộc riêng lĩnh vực y khoa. Trong lĩnh vực này, cùng một khái niệm có thể được biểu diễn dưới nhiều hình thức khác nhau. Ví dụ để chỉ nồng độ bạch cầu trong máu, trong HSXV có thể được diễn đạt là ``WBC'', ``white blood cell count'' hoặc ``white blood count''. Việc xác định được các cụm từ có cách diễn đạt khác nhau nhưng cùng chỉ một khái niệm có thể giúp tăng độ chính xác và giải sai sót trong quá trình phân loại. Để làm được điều này, tác giả của hệ thống \cite{YanXu2012} đề xuất sử dụng các nguồn tri thức nhân loại bên ngoài như Wikipedia, UMLS hoặc WordNet.

\subsection*{Wikipedia}
Wikipedia là hệ thống bách khoa toàn thư miễn phí đa ngôn ngữ, nền Web dựa trên mô hình cho phép người dùng chỉnh sửa nội dung. Các tri thức nhân loại từ Wikipedia có thể được sử dụng để xác định tên giả (alias), tên viết tắt hoặc các từ đồng nghĩa hay có liên quan với nhau. Ví dụ hai khái niệm ``head trauma'' và ``head injury'' có thể được xác định là đồng nghĩa dựa trên thông tin từ Wikipedia.

Dựa trên thiết kế của hệ thống \cite{YanXu2012}, chúng tôi xây dựng bộ trích xuất các thông tin về tiêu đề bài viết (title hoặc redirected link), từ in đậm (bold name) trong bài viết và các khái niệm liên quan (anchor link) được đề cập trong bài viết. Ngôn ngữ của Wikipeda được chúng tôi sử dụng là tiếng Anh . Để trích xuất được các thông tin này, chúng tôi sử dụng công cụ Wikipedia-miner hỗ trợ khai thác thông tin từ Wikipedia dưới dạng dữ liệu cục bộ lưu tại máy. Do số lượng khái niệm cần được rút trích thông tin từ Wikipedia rất lớn, vì vậy để cải thiện hiệu năng thời gian của hệ thống, chúng tôi đề xuất trước khi thực hiện rút trích đặc trưng của nhóm lớp Problem/Treatment/Test, các khái niệm thuộc nhóm lớp này cần được rút trích riêng thông tin từ Wikipedia và ghi xuống file lưu trữ. Wikipedia được sử dụng cho các đặc trưng so trùng tiêu đề bài viết (Title match), so trùng từ in đậm (Bold name match) và so trùng khái niệm liên quan (Anchor link match) như đã miêu tả trong phần rút trích đặc trưng nhóm lớp Problem/Treatment/Test.

\subsection*{WordNet}
WordNet là bộ từ điển được xây dựng tiếng Anh bao gồm các từ vựng đã được phân loại vai trò ngữ pháp như danh từ, tính từ, trạng từ, động từ. Các từ xuất hiện trong WordNet đã được nhóm lại với nhau thành các nhóm từ có cùng ý nghĩa. WordNet có vai trò giống như Wikipedia trong việc xác định tên giả, viết tắt và từ đồng nghĩa.

Cụ thể khi sử dụng công cụ WordNet, từ một khái niệm đầu vào, chúng tôi xác định được danh sách các từ đồng nghĩa (cognitive synonyms / synsets) với khái niệm đó kèm theo vai trò ngữ pháp của các từ đồng nghĩa. Danh sách các từ đồng nghĩa này được sử dụng cho đặc trưng ``WordNet match'' như đã miêu tả trong phần rút trích đặc trưng nhóm lớp Problem/Treatment/Test. Việc xác định hai khái niệm có đồng nghĩa hay không được chúng tôi thực hiện bằng cách kiểm tra một khái niệm có tồn tại trong danh sách từ đồng nghĩa của khái niệm còn lại hay không.

\subsection*{UMLS}
Trong quá trình phát triển của y học, nhu cầu có một bộ từ điển chung, chứa các thông tin đã được chuẩn hóa về các loại bệnh, thuốc, nguyên nhân hoặc các thuật ngữ trong y tế ngày càng cao. Vì lí do đó, bộ từ điển UMLS - Unified Medical Language System đã được Mỹ tiến hành xây dựng từ năm 1986 bởi Thư viện Y học Quốc gia Hoa Kì (US National Library of Medicine).

Trong mô tả của hệ thống \cite{YanXu2012}, một số đặc trưng về ngữ nghĩa trong văn bản cần khai thác thông tin từ bộ từ điển UMLS. Cụ thể bộ trích xuất ngữ nghĩa trong văn bản cần xác định một khái niệm có xuất hiện trong bộ từ điển UMLS hay không, nếu có thì khái niệm đó thuộc phân nhóm nào trong từ điển UMLS. Ví dụ như khái niệm ``lung cancer'' xuất hiện trong từ điển UMLS dưới phân nhóm Các tiến trình liên quan u, bướu (Neoplastic Process). Để trích xuất các thông tin này, chúng tôi sử dụng công cụ MetaMap được đề cập trong phần \ref{tools}. Tương tự như việc rút trích thông tin từ Wikipedia, chúng tôi đề xuất các khái niệm cần được rút trích thông tin từ UMLS và ghi xuống file lưu trữ trước khi tiến hành rút trích đặc trưng cho nhóm lớp Problem/Treatment/Test.

Các thông tin được rút trích từ bộ từ điển UMLS không phục vụ cho đặc trưng so trùng thông tin như hai nguồn thông tin nhân loại Wikipedia và WordNet, các thông tin này được sử dụng để hỗ trợ cho việc hiện thực bộ trích xuất ngữ nghĩa văn bản như trích xuất thông tin cơ quan trên cơ thể (anatomy), trích xuất thông tin chỉ định (indicator), trích xuất thông tin thiết bị y tế (equipment) và trích xuất thông tin phẫu thuật y tế (operation).

Mặt khác, đối với lớp Problem/Test/Treatment, cùng một sự kiện y khoa có thể xảy ra nhiều lần nhưng các sự kiện y khoa đó có thể không đồng tham chiếu mà mang nhiều ý nghĩa khác nhau. Nguyên nhân vì trong các sự kiện y khoa thường bị ảnh hưởng bởi ngữ cảnh mà chúng được đề cập đến. Ví dụ ``the right \textit{leg}'' và ``the left \textit{leg}'', khái niệm được đề cập đều là ``leg'' tuy nhiên ngữ cảnh trong văn bản cho thấy đó là hai khái niệm chỉ hai chân khác nhau. Vì vậy để xây dựng chính xác chuỗi đồng tham chiếu của nhóm lớp Problem/Treatment/Test, dựa theo hệ thống \cite{YanXu2012}, cần phải có một hệ thống trích xuất thông tin ngữ cảnh trong văn bản.

\subsection*{Trích xuất thông tin cơ quan trên cơ thể (Anatomy)}
Trong HSXV, hai triệu chứng bệnh giống nhau có thể không đồng tham chiếu nếu chúng xuất hiện tại các cơ quan cơ thể khác nhau. Ví dụ trong hai khái niệm sau ``a \textit{thrombosis} of the left subclavian vein'' và ``\textit{thrombosis} of the left internal jugular vein''. Khái niệm đang được xét cùng là ``thrombosis'' (chứng huyết khối) tuy nhiên, triệu chứng này lại xuất hiện tại hai cơ quan khác nhau là ``subclavian vein'' (tĩnh mạch dưới đòn) và ``jugular vein'' (tĩnh mạch cổ). Vì vậy dựa theo tập kết quả, hai khái niệm trên không đồng tham chiếu với nhau.

Để phân biệt các khái niệm như ví dụ trên, chúng tôi hiện thực bộ rút trích thông tin cơ quan trên cơ thể dựa trên bộ từ điển UMLS. Trong UMLS, các khái niệm được chia vào nhiều phân nhóm khác nhau dựa theo ý nghĩa của chúng. Cụ thể, chúng tôi sử dụng MetaMap để giới hạn việc nhận dạng các khái niệm thuộc các phân nhóm sau ``Anatomical Structure'', ``Body Location or Region'', ``Body Part, Organ, or Organ Component'', ``Body Space or Junction'' và ``Body System''. Các phân nhóm trên đều là các phân nhóm được định nghĩa sẵn trong từ điển UMLS. Sau khi nhận dạng được các thông tin cơ quan trên cơ thể, chúng tôi so trùng chuỗi của cơ quan nhận dạng được.

\subsection*{Trích xuất thông tin vị trí (Position)}
Một số cơ quan trên cơ thể tuy giống nhau về mặt ngữ nghĩa nhưng lại được phân biệt bằng các tính từ chỉ vị trí đi kèm. Ví dụ như ``\textit{burning sensation} in the \underline{upper right leg}'' và ``\textit{burning sensation} in the \underline{lower right leg}''. Cùng một triệu chứng là cảm giác bỏng rát (burning sensation) ở chân phải (right leg), nhưng lại xuất hiện ở hai vị trí khác nhau là phía trên của chân (upper) và phía dưới của chân (lower). Vì vậy hai khái niệm không đồng tham chiếu với nhau.

Từ việc khảo sát tập dữ liệu, chúng tôi xây dựng tập từ điển gồm các từ chỉ vị trí như ``left'', ``right'', ``upper'', ``lower'', ``back'', ``front'', ... Các thông tin vị trí được rút trích thông qua việc tìm kiếm sự xuất hiện của từ trong bộ từ điển. Tuy nhiên, chúng tôi nhận thấy rằng việc so trùng thông tin vị trí chỉ mang tính chất tương đối, một số khái niệm như ``left upper leg'' và ``left leg'' vẫn có thể đồng tham chiếu với nhau mặc dù vị trí ``left upper'' và ``left'' là khác nhau.

\subsection*{Trích xuất thông tin thuốc y tế (Medical information)}
Các thông tin về thuốc y tế là đặc trưng quan trọng đối với lớp Treatment. Thông tin về thuốc y tế bao gồm tên thuốc (drug name, ví dụ như Morphine, Anti-biotic,...), đường hấp thụ (mode, ví dụ như qua đường miệng, tiêm, qua hậu môn,...), liều lượng (dosage, ví dụ như 10mg, 2 viên,...), tần số sử dụng (frequency, ví dụ như 2 lần một ngày, mỗi ngày,...), thời gian sử dụng (duration, ví dụ như 3 tuần, 1 tháng,...). Khi các thông tin trên có giá trị khác nhau, hai khái niệm sẽ không đồng tham chiếu với nhau. Ví dụ ``Modifinal 100 mg po qd'' có tên thuốc được đề cập là Modifinal, liều lượng là 100 mg, đường hấp thụ là uống qua miệng (po) và tần số sử dụng là 1 lần một ngày (qd). Để rút trích thông tin thuốc y tế, chúng tôi sử dụng công cụ MedEx được đề cập trong phần \ref{tools}.

\subsection*{Trích xuất thông tin chỉ định của thủ tục y tế (Indicator)}
Một thủ tục y tế (Test) thường đi kèm với một hoặc một vài thông tin chỉ định mô tả chi tiết thủ tục đó, ví dụ nồng độ bạch cầu (WBC), nồng độ hồng cầu (RBC), nồng độ sắt trong máu (HCT),... Khi hai khái niệm lớp Test chứa thông tin chỉ định khác nhau sẽ không đồng tham chiếu với nhau. Ví dụ khái niệm ``100 WBC'' và ``130 WBC'' tuy cùng thông tin chỉ định là đo nồng độ bạch cầu trong máu, nhưng có kết quả xét nghiệm khác nhau nên hai khái niệm trên không đồng đám chiếu. Một ví dụ khác là ``100 WBC'' và ``100 RBC'', tuy hai khái niệm có cùng kết quả xét nghiệm nhưng thông tin chỉ định là khác nhau thì hai khái niệm trên cũng không đồng tham chiếu. Như vậy cần thiết phải xác định rõ thông tin chỉ định cũng như giá trị kết quả khi phân giải đồng tham chiếu cho các khái niệm thuộc lớp Test.

Qua việc khảo sát tập dữ liệu, chúng tôi nhận thấy các thông tin chỉ định đa số là xác định nồng độ các hợp chất, tế bào, thành phần cơ thể,... hoặc các giá trị đo lường. Để xác định các thông tin chỉ định này, chúng tôi sử dụng MetaMap để nhận dạng khái niệm thuộc các phân nhóm liên quan như ``Quantitative Concept'', ``Pharmacologic Substance'', ``Element, Ion, or Isotope'',... Sau xác định được thông tin chỉ định, chúng tôi kết hợp với biểu thức chính quy để tìm kết quả xét nghiệm.

\subsection*{Trích xuất thông tin thời gian (Temporal)}
Thông tin thời gian là một thông tin mạnh cho phân giải đồng tham chiếu ở ba lớp Problem/Treatment/Test. Một thủ tục y tế được tiến hành ở hai thời điểm khác nhau thì không đồng tham chiếu hoặc cùng một loại thuốc nhưng được kê khai ở các thời điểm khác nhau thì độc lập. Thông tin thời gian này được chia làm hai loại. Thời gian tường minh, ví dụ 03/06/2015. Thời gian suy diễn là các mốc thời gian được trích xuất từ một số từ khóa đứng gần khái niệm đang xét, như ``Admission date'' (ngày nhập viện) hay ``Post-op day 2'' (ngày thứ 2 sau khi phẫu thuật).

Để hiện thực, chúng tôi thực hiện các bước sau: xác định câu văn chứa khái niệm đang được xét, sau đó sử dụng công cụ MedTime được đề cập trong phần \ref{tools} để nhận dạng thời gian xuất hiện trong câu văn, thời gian của khái niệm đang xét sẽ được tính bằng giá trị thời gian nhận diện được. Ví dụ trong câu ``The patient 's \textit{right thigh hematoma} was performed on \underline{2010-06-28}'', thời gian của khái niệm ``right thigh hentoma'' được xác định là ``2010-06-28''. Nếu trong câu văn xuất hiện hai giá trị thời gian khác nhau, chúng tôi sẽ sử dụng giá trị thời gian nằm gần khái niệm đang xét nhất. Ngoài ra, chúng tôi nhận thấy một số thông tin thời gian được tách riêng và nằm trong câu văn độc lập. Ví dụ trong trích đoạn HSXV sau:

\begin{quote}
``\underline{2014-04-23 07:56 AM}\\
\textit{WBC} - 9.4\\
\textit{RBC} - 4.75''
\end{quote}

Khái niệm ``WBC'' và ``RBC'' có thời gian xác định là 2014-04-23, tuy nhiên giá trị thời gian này không nằm trên cùng câu văn với khái niệm đang được xét. Để giải quyết vấn đề ngày, chúng tôi đề xuất nếu câu văn chứa khái niệm không có thông tin về thời gian, hệ thống sẽ xem xét 3 câu văn liền trước để cố gắng tìm kiếm thông tin thời gian liên quan.

\subsection*{Trích xuất thông tin phân đoạn bệnh án (Section)}
Một hồ sơ xuất viện thường được chia làm các phân đoạn (section) như: tiền sử bệnh, tiền sử dùng thuốc, tiền sử nhập viện,... Hai khái niệm xuất hiện ở hai phần khác nhau thì thường không đồng tham chiếu với nhau cho dù chúng có cùng cách viết. Ví dụ cụm “CT scan” xuất hiện ở phần lịch sử bệnh và phần xét nghiệm thể chất thì hai khái niệm “CT scan” này là độc lập với nhau. Một số tên phân đoạn phổ biến là xét nghiệm thể chất (physical examination), lịch sử bệnh (history of present illness), tiền sử thuốc (past medical history), dị ứng và kháng thuốc (allergies and adverse reactions), ...

Mỗi HSXV thường được viết dựa trên cấu trúc của từng khoa trong bệnh viện. Tùy vào chức năng, đặc trưng của từng khoa mà HSXV của các khoa sẽ bao gồm các phân đoạn khác nhau. Vì vậy tập hợp các phân đoạn trong HSXV rất đa dạng và không có sẵn đầy đủ. Để chọn ra danh sách tên các phân đoạn có trong tập dữ liệu, chúng tôi dựa trên một số luật rút ra từ việc khảo sát dữ liệu \cite{RandolphMiller2008}. Những luật đó bao gồm: các đầu mục phải kết thúc bằng dấu hai chấm (:), các tên phân đoạn thường được viết hoa toàn bộ hoặc viết hoa các chữ cái đầu tiên. Từ các luật được nêu ra, chúng tôi tạo danh sách toàn bộ các cụm từ có khả năng là tên của phân đoạn trong HSXV. Sau đó, chúng tôi chọn 15 tên phân đoạn có tần số xuất hiện nhiều nhất.

\subsection*{Trích xuất thông tin bổ từ (Modifier)}
Tính đồng tham chiếu của một số khái niệm thuộc lớp Test có thể được xác định bởi các từ bổ nghĩa đi kèm theo chúng. Ví dụ các từ “recent”, “prior” hay “initial”. Tập từ khóa của thông tin bổ từ được chúng tôi xây dựng thông qua việc khảo sát dữ liệu.

\subsection*{Trích xuất thông tin thiết bị y tế (Equipment)}
Thiết bị y tế được sử dụng cũng là một gợi ý ngữ nghĩa cho việc xác định phân giải đồng tham chiếu. Lí do vì các thủ tục y tế thường được đặt tên theo thiết bị sử dụng. Để hiện thực đặc trưng này, chúng tôi sử dụng MetaMap để nhận diện khái niệm có xuất hiện trong UMLS hay không. Nếu có xuất hiện thì khái niệm có kết thúc bằng các tiếp vĩ ngữ như ``-graphy'', ``-gram'', ``-metry'' hoặc ``-scopy'' hay không. Nếu hai điều kiện trên được thỏa thì khái niệm đang xét được coi là một thiết bị y tế.

\subsection*{Trích xuất thông tin phẫu thuật y tế (Operation)}
Phần lớn các khái niệm thuộc lớp Treatment được đề cập là các phẫu thuật y tế. Vì vậy đặc trưng thông tin phẫu thuật y tế có thể xem là một gợi ý cho việc phân giải đồng tham chiếu lớp Treatment. Các phẫu thuật y tế được xác định bằng cách: khái niệm có tồn tại trong bộ từ điển UMLS và kết thúc bằng các tiếp vĩ ngữ ``-tomy'', ``-plasty''.

Tuy nhiên từng lớp Problem/Treatment/Test không sử dụng toàn bộ các đặc trưng về ngữ nghĩa trong văn bản. Ví dụ đặc trưng về thuốc y tế chỉ được sử dụng cho lớp Treatment mà không sử dụng cho lớp Problem và Test. Bảng \ref{tab:SemanticFeatures} mô tả chi tiết các đặc trưng ngữ nghĩa được sử dụng cho từng lớp khái niệm. Các đặc trưng được mô tả đầy đủ trong phần Các nguồn tri thức nhân loại và phần Bộ trích xuất ngữ nghĩa trong văn bản. Bảng \ref{tab:ProbTreatTestFeatures} liệt kê đầy đủ các đặc trưng được chúng tôi sử dụng trong việc huấn luyện mô hình SVM cho nhóm Problem/Treatment/Test, trong đó đầu vào của hệ thống rút trích đặc trưng lớp Problem/Treatment/Test là một cặp hai khái niệm thuộc lớp tương ứng.

\begin{table}[ht]
\centering\ra{1.2}
\caption{Phân chia đặc trưng được sử dụng trong ba lớp Problem, Treatment và Test \label{tab:SemanticFeatures}}
\footnotesize\sffamily

\begin{tabularx}{0.8\textwidth}{@{}l *5{>{\centering\arraybackslash}X}@{}}
\toprule 
& \textbf{Problem} & \textbf{Treatment} & \textbf{Test}\\
\midrule
Anatomy & + & & +\\
Position & + & + & +\\
Medication & & + & \\
Indicator & & & +\\
Temporal & & + & +\\
Section & + & + & +\\
Modifier & & & +\\
Equipment & & & +\\
Operation & & + & \\
\bottomrule
\end{tabularx}
\end{table}

\section{Gom cụm và xây dựng chuỗi đồng tham chiếu}
Ở mô hình cặp thực thể, hệ thống phân loại không có khả năng xây dựng chuỗi đồng tham chiếu mà nó chỉ có thể xác định một cặp khái niệm là có đồng tham chiếu hay không. Mặt khác, đối với một văn bản HSXV, số cặp khái niệm được sinh ra rất nhiều và trong số đó có nhiều cặp có chung khái niệm đứng sau, ví dụ hai cặp (“Dr. John”, “his”) và (“Mr. Brown”, “his”) có chung khái niệm đứng sau là “his” mà hai cặp này đều được hệ thống phân loại xác định là đồng tham chiếu, tuy nhiên chỉ một trong hai khái niệm “Dr. John” và “Mr. Brown” được chọn làm tiền đề cho khái niệm “his” này. Như vậy cần thiết phải có một giải thuật lựa chọn các cặp đồng tham chiếu và xây dựng các chuỗi đồng tham chiếu từ chúng.

Như đã được đề cập ở mục \ref{coref-model}, có hai giải thuật được đề xuất là: \emph{gom cụm gần nhất trước} và \emph{gom cụm tốt nhất trước}. Chúng tôi lựa chọn thực hiện giải thuật gom cụm tốt nhất trước cho hệ thống của mình vì hai lý do:
\begin{enumerate}
\item Theo các tác giả của giải thuật, giải thuật gom cụm tốt nhất trước cho kết quả tốt hơn giải thuật gom cụm gần nhất trước \cite{VincentNg2002a}.
\item Các tác giả hệ thống \cite{YanXu2012} cũng hiện thực giải thuật này cho hệ thống của họ.
\end{enumerate}

Về cơ bản, giải thuật gom cụm tốt nhất trước lựa chọn các cặp khái niệm được xác định là đồng tham chiếu và có độ tin cậy cao nhất ứng với mỗi hồi chỉ; đối với đại từ, giải thuật sử dụng module phân loại xác định lớp chính của đại từ đó và tạo một cặp đồng tham chiếu giữa nó với khái niệm thuộc lớp tương ứng gần nhất trước đó (theo thứ tự xuất hiện) trong văn bản HSXV. Sau khi có được tập các cặp đồng tham chiếu, giải thuật nối các cặp có một khái niệm chung lại để tạo thành các chuỗi. Đây chính là kết quả cuối cùng của hệ thống phân giải.

Như vậy giải thuật gom cụm cụm tốt nhất trước bao gồm hai bước chính: chọn lọc các cặp đồng tham chiếu tốt nhất và ghép nối các cặp được chọn để tạo thành danh sách các chuỗi đồng tham chiếu. Bước đầu tiên làm việc như sau:
\begin{enumerate}
\item Nhận đầu vào là văn bản $E$ và danh sách các khái niệm $C$ được sắp xếp theo thứ tự xuất hiện. Gọi $M$ là số khái niệm trong $C$. $P$ là danh sách các cặp khái niệm, khởi tạo $P=\emptyset$.
\item Duyệt toàn bộ danh sách $C$ từ đầu đến cuối theo thứ tự xuất hiện ($j=1\rightarrow M$), với mỗi khái niệm $C_j$, nếu $C_j$ thuộc một trong 4 lớp Person, Problem, Test hoặc Treatment thì tới bước 3, nếu $C_j$ thuộc lớp Pronoun thì tới bước 4. Nếu đã duyệt xong, đi tới bước 6.
\item Với mỗi cặp $(C_i,\,C_j)$ mà $C_i$ đứng trước $C_j$ ($i<j$), chọn ra cặp $(C_k,\,C_j)$ mà hệ thống phân loại xác định là đồng tham chiếu và có độ tin cậy cao nhất, đưa cặp $(C_k,\,C_j)$ tới bước 5.
\item Sử dụng hệ thống phân loại xác định lớp chính $T$ của $C_j$. Duyệt từ khái niệm ngay trước $C_j$ về đầu ($i=j-1\rightarrow 1$), nếu $C_i$ thuộc lớp $T$, ngừng duyệt và đưa cặp $(C_i,\,C_j$) tới bước 5.
\item Nhận cặp $(C_i,\,C_j)$ được chọn ở một trong hai bước trên, đưa cặp vào $P$ và quay trở lại bước 2.
\item Trả $P$ về làm kết quả.
\end{enumerate} 

Sau khi đã có được danh sách các cặp đồng tham chiếu tốt nhất từ bước trên, hệ thống ghép nối các cặp trong danh sách lại để tạo thành các chuỗi đồng tham chiếu:
\begin{enumerate}
\item Nhận đầu vào là danh sách cặp khái niệm $P$.
\item Khởi tạo danh sách chuỗi $H=\emptyset$.
\item Với mỗi cặp $(C_i,\,C_j)$ trong $P$, duyệt tất cả các chuỗi $K$ trong $H$, nếu tìm được một chuỗi $K^{\prime}$ mà chứa một trong hai khái niệm $C_i$ hoặc $C_j$ của cặp thì đưa $C_i$, $C_j$ vào $K^{\prime}$ và quay lại bước 3. Nếu không tìm được chuỗi nào thỏa điều kiện trên, đưa cặp $(C_i,\,C_j)$ tới bước 4. Nếu đã duyệt xong, tới bước 5.
\item Khởi tạo một chuỗi $K^{\prime}$ mới gồm hai khái niệm của cặp, $K^{\prime}=\{C_i,\,C_j\}$, đưa chuỗi vào $H$ và quay lại bước 3.
\item Trả $H$ về làm kết quả.
\end{enumerate}

\begin{table}[th]
\centering
\caption{Tập đặc trưng cho ba lớp Problem, Treatment và Test \label{tab:ProbTreatTestFeatures}}
\footnotesize\sffamily

\begin{tabularx}{\textwidth}{@{}>{\hspace{1em}}P{\raggedright}{0.3}lL@{}}
\toprule 
\rowgroup{\textbf{Đặc Trưng}} & \textbf{Giá trị} & \textbf{Giải thích}\\
\midrule
\rowgroup{\textit{Tri thức nhân loại}}\\
Wiki page match & 0, 1 & Hai khái niệm cùng dẫn đến một trang Wiki\\
Wiki anchor match & 0, 1 & Trang Wiki của một khái niệm chứa liên kết đến trang wiki của khái niệm còn lại\\
Wiki bold name match & 0, 1 & Trang Wiki của một khái niệm chứa từ in đậm chỉ khái niệm còn lại\\
WordNet match & 0, 1 & Hai khái niệm được xác định đồng nghĩa trên WordNet\\
\rowgroup{\textit{Trích xuất thông tin ngữ nghĩa trong văn bản}}\\
Anatomy & 0, 1, 2 & Không cùng cơ quan cơ thể (0), cùng cơ quan (1), không xác định (2)\\
Position & 0, 1, 2 & Không cùng vị trí (0), cùng vị trí (1), không xác định (2)\\
Indicator & 0, 1, 2 & Không cùng thông tin chỉ định (0), cùng thông tin chỉ định (1), không xác định (2)\\
Temporal & 0, 1, 2 & Không cùng thời gian (0), cùng thời gian (1), không xác định (2)\\
Section & 1, 2,..., $n^{2}$ & Hai khái niệm thuộc về phân đoạn i và j của văn bản\\
Equipment & 0, 1, 2 & Không cùng thiết bị (0), cùng thiết bị (1), không xác định (2)\\
Operation & 0, 1, 2 & Không cùng phẫu thuật (0), cùng phẫu thuật (1), không xác định (2)\\
\rowgroup{\textit{Trích xuất thông tin thuốc y tế}}\\
Drug & 0, 1 & Cùng tên thuốc (1), ngược lại (0)\\
Mode & 0, 1, 2,..., 29 & Chỉ mục của 29 đường hấp thụ hoặc không xác định (29)\\
Dosage & 0, 1 & Cùng liều lượng (1), ngược lại (1)\\
Duration & 0, 1 & Cùng khoảng thời gian sử dụng (1), ngược lại (0)\\
Frequency & 0, 1 & Cùng tần suất sử dụng (1), ngược lại (0)\\
\textit{Khoảng cách}\\
Sentence distance & 0, 1, 2,... & Số câu xuất hiện giữa hai khái niệm được xét\\
\rowgroup{\textit{Ngữ pháp}}\\
Article & 1, 2,..., $3^{2}$ & Trạng thái các từ hạn định (determiner) đứng trước hai khái niệm, bao gồm 3 loại: (a|an), (the|his|her…) hoặc không có (NULL)\\
\rowgroup{\textit{So trùng chuỗi}}\\
Head noun match & 0, 1 & Cùng danh từ trung tâm (1), ngược lại (0)\\
Contains & 0, 1 & Một khái niệm chứa toàn bộ chuỗi của khái niệm còn lại\\
Capital match & 0, 1 & Các kí tự đầu tiên trùng nhau (1), ngược lại (0)\\
Substring match & 0, 1 & Có cùng chuỗi con (1), ngược lại (0)\\
Cos distance & (0, 1) & Khoảng cách cos (góc) giữa hai khái niệm\\
\rowgroup{\textit{Ngữ nghĩa}}\\
Word match & 0, 1 & Hai khái niệm trùng chuỗi hoàn toàn\\
Procedure match & 0, 1 & Có chứa từ ``Procedure'' (1), ngược lại (0)\\
\bottomrule
\end{tabularx}
\end{table}
