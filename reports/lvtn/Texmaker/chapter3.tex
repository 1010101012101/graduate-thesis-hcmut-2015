\chapter{Kiến thức nền tảng}
\section{Các định nghĩa và thuật ngữ}
Trong các công trình nghiên cứu về phân giải đồng tham chiếu, các tác giả thường sử dụng từ \emph{markable} để chỉ tới những từ/cụm từ cần được phân giải đồng chiếu, hoặc từ \emph{noun phrase} hay NP vì đa phần các công trình trước đây chỉ xem xét tới các danh từ/cụm danh từ. Một số tài liệu khác sử dụng từ \emph{mention} để chỉ tới những ``đề cập'' trong văn bản vì bản chất của phân giải đồng tham chiếu là xác định xem các từ/cụm từ có đề cập tới cùng một thực thể hay không. Để thuận tiện trong việc diễn đạt bằng tiếng Việt, chúng tôi sử dụng từ \emph{khái niệm} để chỉ tới những thực thể cần được phân giải đồng tham chiếu. Một lý do khác mà chúng tôi sử dụng từ này bắt nguồn từ việc Thách thức I2B2 năm 2011 gọi các tập tin chứa những thực thể đã được gán nhãn là ``concept files''.

Các khái niệm đa phần là danh từ hay cụm danh từ. Một khái niệm có thể được lồng trong khái niệm khác. Thông thường sự lồng nhau này xuất hiện ở những cụm danh từ mang ý nghĩa sở hữu, ví dụ cụm ``ngôi nhà của anh ta'' chứa hai khái niệm khác nhau là (ngôi nhà của anh ta) và (anh ta). Một số tài liệu gọi các khái niệm lồng nhau là \emph{khái niệm đầy đủ}. Một hệ thống phân giải đồng tham chiếu có xem xét đến sự lồng nhau này hay không phụ thuộc vào bước trích xuất và gán nhãn thực thể trước đó. Một số công trình nghiên cứu đề xuất các giải pháp phân giải riêng biệt cho các khái niệm lồng nhau, tuy nhiên phương pháp mà chúng tôi hiện thực ở đây không xem sự lồng nhau là một trường hợp đặc biệt.

Hai khái niệm được xem là \emph{đồng tham chiếu} nếu cả hai cùng chỉ về một thực thể trong thế giới thực, ví dụ ``thủ tướng Việt Nam'' và ``Nguyễn Tấn Dũng''. Một đặc điểm cần lưu ý của tính đồng tham chiếu là nó phụ thuộc vào ngữ cảnh và thời điểm mà các khái niệm được đề cập đến. Như ở ví dụ trên, hai khái niệm ``thủ tướng Việt Nam'' và ``Nguyễn Tấn Dũng'' chỉ đồng tham chiếu nếu thời điểm mà chúng được đề cập trong văn bản nằm trong khoảng thời gian mà ông Nguyễn Tấn Dũng đang là thủ tướng Việt Nam.

Có thể xem đồng tham chiếu là một mối quan hệ giữa hai hay nhiều khái niệm khi chúng cùng đề cập tới một thực thể. Dễ dàng nhận thấy đây là mối quan hệ tương đương vì nó có những tính chất sau:
\begin{itemize}[noitemsep]
\item \emph{Tính phản xạ}: một khái niệm bất kì thì luôn đồng tham chiếu với chính nó.
\item \emph{Tính đối xứng}: nếu khái niệm $C_1$ đồng tham chiếu với $C_2$ thì $C_2$ cũng đồng tham chiếu với $C_1$.
\item \emph{Tính bắc cầu}: nếu khái niệm $C_1$ đồng tham chiếu với $C_2$, $C_2$ đồng tham chiếu với $C_3$ thì $C_1$ cũng đồng tham chiếu với $C_3$.
\end{itemize}

Một \emph{cặp khái niệm} gồm hai khái niệm có thể có hoặc không đồng tham chiếu với nhau. Đối với một cặp đồng tham chiếu, khái niệm đứng trước được gọi là \emph{tiền đề} (antecedent), khái niệm đứng sau được gọi là \emph{hồi chỉ} (anaphora). Những đặc trưng của một cặp khái niệm đa phần là những \emph{đặc trưng đồng thuận} (agreement feature), chúng mang giá trị là ``có'' khi cả hai khái niệm của cặp cùng đồng thuận về một đặc tính nào đó, ví dụ như sự đồng thuận về giới tính hay số lượng. Trong một văn bản, nhiều khái niệm có thể cùng tham chiếu tới cùng một thực thể, khi đó chúng tạo thành một chuỗi đồng tham chiếu.

\emph{Phân giải đồng tham chiếu} là công việc xác định những chuỗi đồng tham chiếu trong văn bản. Xuyên suốt này luận văn này, chúng tôi gọi các chuỗi đồng tham chiếu được phân giải bởi con người là các \emph{chuỗi kết quả}, các chuỗi được xuất ra bởi hệ thống phân giải đồng tham chiếu là các \emph{chuỗi hệ thống}.

\section{Support Vector Machine}
\section{Các mô hình học máy phân giải đồng tham chiếu}
Phân giải đồng tham chiếu, một trong những tác vụ cơ bản của xử lý ngôn ngữ tự nhiên, là công việc xác định xem những khái niệm nào trong văn bản cùng chỉ đến một thực thể trong thế giới thực. Mặc dù nhiều phương pháp giải quyết đã được nghiên cứu phát triển từ những năm 60 của thế kỉ 20, các hệ thống dựa trên luật hay theo hướng tiếp cận heuristic trong thời gian này đòi hỏi nhiều kiến thức phức tạp và không thực sự hiệu quả (một trong những nền tảng của các hệ thống này là \emph{lý thuyết trung tâm} (centering theory) \cite{Grosz1983}).

Bắt đầu từ những năm 1990, khi các mô hình xác suất trở nên phổ biến vì tính hiệu quả của chúng, các phương pháp học máy phân giải đồng tham chiếu ra đời dần thay thế các phương pháp heuristic thủ công. Cùng với xu thế đó là sự xuất hiện của các tập dữ liệu được gán nhãn bắt nguồn từ hội nghị MUC-6 (1996) và MUC-7 (1997), các nghiên cứu về phân giải đồng tham chiếu dựa trên học máy càng được phát triển không chỉ cho các miền văn bản chung mà còn đi sâu vào các miền văn bản cụ thể (như Thách thức i2b2 năm 2011 về phân giải đồng tham chiếu cho bệnh án điện tử \cite{OzlemUzuner2012}).

Mặc dù tính bổ biến và sự hiệu quả của các phương pháp học máy có giám sát, sự khó khăn trong việc xây dựng các tập dữ liệu gán nhãn nhất là đối với các ngôn ngữ khác tiếng Anh đã làm động lực cho sự ra đời của các phương pháp học máy bán giám sát hoặc không giám sát. Ưu điểm của các phương pháp này là không cần đòi hỏi các tập dữ liệu đã gán nhãn hoặc chỉ cần gán nhãn một phần \cite{CardieWagstaff1999}.

Có ba mô hình học máy có giám sát được đề xuất để phân giải đồng tham chiếu là: mô hình \emph{cặp khái niệm}, mô hình \emph{đề cập thực thể} và mô hình \emph{xếp hạng}. Mô hình cặp khái niệm ra đời đầu tiên và có ý tưởng đơn giản nhất, tuy nhiên cũng chính vì thế mà nó có một số nhược điểm khiến cho việc hiện thực không thực sự dễ dàng. Hai mô hình đề cập thực thể và xếp hạng được đề xuất sau đó nhằm khắc phục các nhược điểm của mô hình cặp khái niệm bằng cách đưa vào các ý tưởng thực tế hơn như \emph{đặc trưng ở mức cụm} (cluster-level feature) \cite{} hay huấn luyện một mô hình xếp hạng lựa chọn tiền đề \cite{}.

Mặc dù có nhiều nhược điểm, mô hình cặp thực thể vẫn rất phổ biến nhờ ý tưởng đơn giản của nó, đặc biệt là trong các miền văn bản chuyển môn khi mà các đặc trưng chuyên sâu về một hay một vài lĩnh vực mang nhiều ý nghĩa hơn cho việc xác định tính đồng tham chiếu của các khái niệm. Điển hình là hệ thống có kết quả cao nhất trong Thử thách i2b2 năm 2011 sử dụng mô hình cặp khái niệm và tận dụng các thông tin về sự cập đến bệnh nhân, các kiến thức nền (Wikipedia, WordNet, v.v...) và các thông tin ngữ cảnh trong bệnh án \cite{YanXu2012}. Trong phần này chúng tôi chỉ trình bày cụ thể về mô hình cặp khái niệm vì đây cũng là mô hình được chúng tôi hiện thực trong hệ thống của mình.

\subsection*{Mô hình cặp khái niệm}
Đây là mô hình học máy phân giải đồng tham chiếu đầu tiên và được giới thiệu vào năm 1995 \cite{}. Ý tưởng chính của mô hình này là xác định tính đồng tham chiếu cho từng cặp hai khái niệm bất kì trong văn bản. Mặc dù mô hình cặp khái niệm phổ biến nhờ ý tưởng đơn giản của nó, tính bắc cầu của quan hệ đồng tham chiếu bị bỏ qua ở mô hình này vì trường hợp như sau có thể xảy ra: hai khái niệm $A$ và $B$ được xác định là đồng tham chiếu, $B$ và $C$ cũng được xác định là đồng tham chiếu trong khi $A$ và $C$ lại không được xác định là đồng tham chiếu. Điều này đòi hỏi phải có một cơ chế gom cụm các khái niệm và xây dựng các chuỗi sau khi đã xác định tính đồng tham chiếu cho các cặp hai khái niệm.

Một nhược điểm khác của mô hình này là để huấn luyện mô hình phân loại, ứng với mỗi cặp khái niệm hệ thống cần trích xuất các đặc trưng đại diện cho cặp, sau đó tổng hợp lại thành một tập dùng để huấn luyện. Mặt khác trong một văn bản, tổng số các cặp khái niệm có thể rất lớn trong khi số các cặp đồng tham chiếu lại rất nhỏ, như vậy nếu tất cả các cặp được rút trích đặc trưng sẽ dẫn đến tình trạng mất cân bằng lớp: các mẫu âm (các cặp không đồng tham chiếu) có số lượng lấn át các mẫu dương (các cặp đồng tham chiếu).



\section{Các công cụ hỗ trợ rút trích đặc trưng} \label{tools}

\subsection*{OpenNLP}

\subsection*{Wikipedia-miner}

\subsection*{MetaMap}

\subsection*{cTakes và các công cụ dựa trên cTakes}
