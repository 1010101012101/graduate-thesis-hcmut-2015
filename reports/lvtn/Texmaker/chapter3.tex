\chapter{Kiến thức nền tảng}
\section{Các định nghĩa và thuật ngữ}
Trong các công trình nghiên cứu về phân giải đồng tham chiếu, các tác giả thường sử dụng từ \emph{markable} để chỉ tới những từ/cụm từ cần được phân giải đồng chiếu, hoặc từ \emph{noun phrase} hay NP vì đa phần các công trình trước đây chỉ xem xét tới các danh từ/cụm danh từ. Một số tài liệu khác sử dụng từ \emph{mention} để chỉ tới những ``đề cập'' trong văn bản vì bản chất của phân giải đồng tham chiếu là xác định xem các từ/cụm từ có đề cập tới cùng một thực thể hay không. Để thuận tiện trong việc diễn đạt bằng tiếng Việt, chúng tôi sử dụng từ \emph{khái niệm} để chỉ tới những thực thể cần được phân giải đồng tham chiếu. Một lý do khác mà chúng tôi sử dụng từ này bắt nguồn từ việc Thách thức I2B2 năm 2011 gọi các tập tin chứa những thực thể đã được gán nhãn là ``concept files''.

Các khái niệm đa phần là danh từ hay cụm danh từ. Một khái niệm có thể được lồng trong khái niệm khác. Thông thường sự lồng nhau này xuất hiện ở những cụm danh từ mang ý nghĩa sở hữu, ví dụ cụm ``ngôi nhà của anh ta'' chứa hai khái niệm khác nhau là (ngôi nhà của anh ta) và (anh ta). Một số tài liệu gọi các khái niệm lồng nhau là \emph{khái niệm đầy đủ}. Một hệ thống phân giải đồng tham chiếu có xem xét đến sự lồng nhau này hay không phụ thuộc vào bước trích xuất và gán nhãn thực thể trước đó. Một số công trình nghiên cứu đề xuất các giải pháp phân giải riêng biệt cho các khái niệm lồng nhau, tuy nhiên phương pháp mà chúng tôi hiện thực ở đây không xem sự lồng nhau là một trường hợp đặc biệt.

Hai khái niệm được xem là \emph{đồng tham chiếu} nếu cả hai cùng chỉ về một thực thể trong thế giới thực, ví dụ ``thủ tướng Việt Nam'' và ``Nguyễn Tấn Dũng''. Một đặc điểm cần lưu ý của tính đồng tham chiếu là nó phụ thuộc vào ngữ cảnh và thời điểm mà các khái niệm được đề cập đến. Như ở ví dụ trên, hai khái niệm ``thủ tướng Việt Nam'' và ``Nguyễn Tấn Dũng'' chỉ đồng tham chiếu nếu thời điểm mà chúng được đề cập trong văn bản nằm trong khoảng thời gian mà ông Nguyễn Tấn Dũng đang là thủ tướng Việt Nam.

Có thể xem đồng tham chiếu là một mối quan hệ giữa hai hay nhiều khái niệm khi chúng cùng đề cập tới một thực thể. Dễ dàng nhận thấy đây là mối quan hệ tương đương vì nó có những tính chất sau:
\begin{itemize}[noitemsep]
\item \emph{Tính phản xạ}: một khái niệm bất kì thì luôn đồng tham chiếu với chính nó.
\item \emph{Tính đối xứng}: nếu khái niệm $C_1$ đồng tham chiếu với $C_2$ thì $C_2$ cũng đồng tham chiếu với $C_1$.
\item \emph{Tính bắc cầu}: nếu khái niệm $C_1$ đồng tham chiếu với $C_2$, $C_2$ đồng tham chiếu với $C_3$ thì $C_1$ cũng đồng tham chiếu với $C_3$.
\end{itemize}

Một \emph{cặp khái niệm} gồm hai khái niệm có thể có hoặc không đồng tham chiếu với nhau. Đối với một cặp đồng tham chiếu, khái niệm đứng trước được gọi là \emph{tiền đề} (antecedent), khái niệm đứng sau được gọi là \emph{hồi chỉ} (anaphora). Những đặc trưng của một cặp khái niệm đa phần là những \emph{đặc trưng đồng thuận} (agreement feature), chúng mang giá trị là ``có'' khi cả hai khái niệm của cặp cùng đồng thuận về một đặc tính nào đó, ví dụ như sự đồng thuận về giới tính hay số lượng. Trong một văn bản, nhiều khái niệm có thể cùng tham chiếu tới cùng một thực thể, khi đó chúng tạo thành một chuỗi đồng tham chiếu.

\emph{Phân giải đồng tham chiếu} là công việc xác định những chuỗi đồng tham chiếu trong văn bản. Xuyên suốt này luận văn này, chúng tôi gọi các chuỗi đồng tham chiếu được phân giải bởi con người là các \emph{chuỗi kết quả}, các chuỗi được xuất ra bởi hệ thống phân giải đồng tham chiếu là các \emph{chuỗi hệ thống}.

\section{Support Vector Machine}
\section{Các mô hình học máy phân giải đồng tham chiếu}
Phân giải đồng tham chiếu, một trong những tác vụ cơ bản của xử lý ngôn ngữ tự nhiên, là công việc xác định xem những khái niệm nào trong văn bản cùng chỉ đến một thực thể trong thế giới thực. Mặc dù nhiều phương pháp giải quyết đã được nghiên cứu phát triển từ những năm 60 của thế kỉ 20, các hệ thống dựa trên luật hay theo hướng tiếp cận heuristic trong thời gian này đòi hỏi nhiều kiến thức phức tạp và không thực sự hiệu quả (một trong những nền tảng của các hệ thống này là \emph{lý thuyết trung tâm} \cite{Grosz1983}).

Bắt đầu từ những năm 1990, khi các mô hình xác suất trở nên phổ biến vì tính hiệu quả của chúng, các phương pháp học máy phân giải đồng tham chiếu ra đời dần thay thế các phương pháp heuristic thủ công. Cùng với xu thế đó là sự xuất hiện của các tập dữ liệu được gán nhãn bắt nguồn từ hội nghị MUC-6 (1996) và MUC-7 (1997), các nghiên cứu về phân giải đồng tham chiếu dựa trên học máy càng được phát triển không chỉ cho các miền văn bản chung mà còn đi sâu vào các miền văn bản cụ thể (như Thách thức i2b2 năm 2011 về phân giải đồng tham chiếu cho bệnh án điện tử \cite{OzlemUzuner2012}).

Mặc dù tính bổ biến và sự hiệu quả của các phương pháp học máy có giám sát, sự khó khăn trong việc xây dựng các tập dữ liệu gán nhãn nhất là đối với các ngôn ngữ khác tiếng Anh đã làm động lực cho sự ra đời của các phương pháp học máy bán giám sát hoặc không giám sát. Ưu điểm của các phương pháp này là không cần đòi hỏi các tập dữ liệu đã gán nhãn hoặc chỉ cần gán nhãn một phần \cite{CardieWagstaff1999}.

Có ba mô hình học máy có giám sát được đề xuất để phân giải đồng tham chiếu là: mô hình \emph{cặp khái niệm}, mô hình \emph{đề cập thực thể} và mô hình \emph{xếp hạng}. Mô hình cặp khái niệm ra đời đầu tiên và có ý tưởng đơn giản nhất, tuy nhiên cũng chính vì thế mà nó có một số nhược điểm khiến cho việc hiện thực không thực sự dễ dàng. Hai mô hình đề cập thực thể và xếp hạng được đề xuất sau đó nhằm khắc phục các nhược điểm của mô hình cặp khái niệm bằng cách đưa vào các ý tưởng thực tế hơn như \emph{đặc trưng ở mức cụm} \cite{Yang2004} hay huấn luyện một mô hình xếp hạng lựa chọn tiền đề \cite{Yang2003}.

Mặc dù có nhiều nhược điểm, mô hình cặp thực thể vẫn rất phổ biến nhờ ý tưởng đơn giản của nó, đặc biệt là trong các miền văn bản chuyển môn khi mà các đặc trưng chuyên sâu về một hay một vài lĩnh vực mang nhiều ý nghĩa hơn cho việc xác định tính đồng tham chiếu của các khái niệm. Điển hình là hệ thống có kết quả cao nhất trong Thử thách i2b2 năm 2011 sử dụng mô hình cặp khái niệm và tận dụng các thông tin về sự cập đến bệnh nhân, các kiến thức nền (Wikipedia, WordNet, v.v...) và các thông tin ngữ cảnh trong bệnh án \cite{YanXu2012}. Trong phần này chúng tôi chỉ trình bày cụ thể về mô hình cặp khái niệm vì đây cũng là mô hình được chúng tôi hiện thực trong hệ thống của mình.

\subsection*{Mô hình cặp khái niệm}
Đây là mô hình học máy phân giải đồng tham chiếu đầu tiên và được giới thiệu vào năm 1995 \cite{Aone&Bennett1995}. Ý tưởng chính của mô hình này là xác định tính đồng tham chiếu cho từng cặp hai khái niệm bất kì trong văn bản. Mặc dù mô hình cặp khái niệm phổ biến nhờ ý tưởng đơn giản của nó, tính bắc cầu của quan hệ đồng tham chiếu bị bỏ qua ở mô hình này vì trường hợp như sau có thể xảy ra: hai khái niệm $A$ và $B$ được xác định là đồng tham chiếu, $B$ và $C$ cũng được xác định là đồng tham chiếu trong khi $A$ và $C$ lại không được xác định là đồng tham chiếu. Điều này đòi hỏi phải có một cơ chế gom cụm các khái niệm và xây dựng các chuỗi sau khi đã xác định tính đồng tham chiếu cho các cặp hai khái niệm.

Một nhược điểm khác của mô hình này là để huấn luyện mô hình phân loại, ứng với mỗi cặp khái niệm hệ thống cần trích xuất các đặc trưng đại diện cho cặp, sau đó tổng hợp lại thành một tập dùng để huấn luyện. Mặt khác trong một văn bản, tổng số các cặp khái niệm có thể rất lớn trong khi số các cặp đồng tham chiếu lại rất nhỏ, như vậy nếu tất cả các cặp được rút trích đặc trưng sẽ dẫn đến tình trạng mất cân bằng lớp: các mẫu âm (các cặp không đồng tham chiếu) có số lượng lấn át các mẫu dương (các cặp đồng tham chiếu).

Một số công trình nghiên cứu đề xuất giải pháp cho vấn đề mất cân bằng lớp theo hướng tiếp cận heuristic. Chẳng hạn như chỉ xem xét sinh các mẫu dương từ các cặp đồng tham chiếu mà hai khái niệm gần kề nhau theo thứ tự xuất hiện, còn các mẫu âm được sinh theo quy tắc: với mỗi cặp đồng tham chiếu được sinh làm mẫu dương $(C_i,\,C_j)$, sinh các mẫu âm từ các cặp $(C_k,\,C_j)$ mà $i<k<j$ \cite{Soon2001}. Một chỉnh sửa nhỏ của cách này là loại bỏ các mẫu mà khái niệm đứng trước là một đại từ \cite{VincentNg2002a}. Hay một cách lọc mẫu khác được trình bày ở \cite{VincentNg2002b}: với mỗi hồi chỉ $C_j$ mà tiền xa nhất của nó là $C_i$, sinh tất cả các mẫu âm được tạo bởi $C_k$ và $C_j$ mà $i<k<j$. Ngoài ra, hệ thống ở \cite{VincentNg2002b} còn sử dụng luật quy nạp (rule induction) để lọc bỏ các mẫu dương khó làm ảnh hưởng đến việc huấn luyện.

Để xác định tính đồng tham chiếu cho hai khái niệm bất kì trong văn bản, hệ thống sử dụng mô hình cặp khái niệm trước tiên cần huấn luyện một mô hình phân loại. Một số giải thuật như cây quyết định hay luật quy nạp, v.v... được sử dụng ở thời kì đầu của ứng dụng học máy. Sau khi các mô hình thống kê trở nên phổ biến, một số giải thuật được sử dụng vào lĩnh vực này gồm có mô hình entropy cực đại \cite{Berger1996}, mạng neuron bầu cử \cite{Freund1999} và support vector machine. Các giải thuật học máy dựa trên mô hình thống kê có một đặc điểm lợi thế là chúng có thể xuất ra độ tin cậy đồng tham chiếu của các cặp khái niệm. 

Mô hình phân loại sau khi đã huấn luyện có thể được sử dụng để xác định tính đồng tham chiếu cho các cặp hai khái niệm. Tuy nhiên để có thể xây dựng các chuỗi đồng tham chiếu, cần thiết phải sử dụng một giải thuật gom cụm. Có hai giải thuật gom cụm được sử dụng phổ biến trong phân giải đồng tham chiếu:

\begin{enumerate}
\item \emph{Gom cụm gần nhất trước (closest-first clustering):}

Giải thuật này trước tiên chọn ra tiền đề ở gần nhất trước đó cho một hồi chỉ. Cụ thể với mỗi khái niệm $C_j$, giải thuật sử dụng mô hình phân loại đã được huấn luyện ở bước trước để xác định tính đồng tham chiếu cho từng cặp $(C_i,\,C_j)$ mà $i<j$. Sau khi gặp một cặp $(C_k,\,C_j)$ đầu tiên được xác định là đồng tham chiếu, giải thuật đưa cặp vào danh sách $L$. Sau khi đã chọn tiền cho toàn bộ hồi chỉ, từ danh sách $L$ các cặp có một chung khái niệm được gom lại với nhau thành từng cụm, mỗi cụm ứng với một chuỗi đồng tham chiếu \cite{Soon2001}.
\item \emph{Gom cụm tốt nhất trước (best-first clustering):}

Giải thuật này chỉ khác giải thuật trên ở bước lựa chọn tiền đề cho các hồi chỉ. Thay chỉ chọn tiền đề ở gần nhất, giải thuật gom cụm tốt nhất trước lựa chọn tiền đề có độ tin cậy đồng tham chiếu cao nhất. Điều này có thể được thực hiện nhờ sử dụng các mô hình phân loại thống kê, điển hình là SVM. Theo các tác giả của giải thuật này, nhờ tận dụng được độ tin cậy đồng tham chiếu của mô hình phân loại, giải thuật gom cụm tốt nhất trước cho kết quả tốt hơn giải thuật gom cụm gần nhất trước \cite{VincentNg2002a}.
\end{enumerate}

\section{Các công cụ hỗ trợ rút trích đặc trưng} \label{tools}

\subsection*{OpenNLP}
Xử lý ngôn ngữ tự nhiên (Natural language processing) là một lĩnh vực của khoa học máy tính, trong đó vấn đề được quan tâm là cách thức tương tác giữa máy tính và ngôn ngữ tự nhiên của con người. Nhiều thách thức được đặt ra trong xử lý ngôn ngữ tự nhiên để giúp máy tính có thể hiểu được những suy nghĩ, ý nghĩa được hàm chứa trong ngôn ngữ tự nhiên.

Thư viện Apache OpenNLP bao gồm các công cụ dựa trên học máy dùng để xử lý, phân tích các chuỗi văn bản được viết dưới dạng ngôn ngữ tự nhiên. Thư viện OpenNLP hỗ trợ các tác vụ cơ bản trong việc xử lý ngôn ngữ tự nhiên như tách từ (Tokenize), phân giải đoạn văn thành các câu văn (Sentence segmentation), đánh dấu vai trò ngữ pháp (Part-of-speech tagging), phân đoạn câu văn (Chunking), đánh dấu danh từ trung tâm (Head nouns finding),... Trong phạm vi luận văn, chúng tôi sử dụng chức năng phân giải câu văn thành các từ, đánh dấu vai trò ngữ pháp, phân đoạn câu văn và đánh dấu danh từ trung tâm của bộ thư viện Apache OpenNLP trong quá trình rút trích đặc trưng lớp Person và Pronoun.

Trong việc phân tích từ vựng (lexical analysis), tác vụ tách từ là quá trình chia nhỏ câu văn thành các thành phần nhỏ hơn như từ, cụm từ, ký hiệu, hoặc các yếu tố có ý nghĩa khác gọi là Token. Các Token này là đầu vào cho các tác vụ xử lý ngôn ngữ tự nhiên khác như đánh dấu vai trò ngữ pháp, phân đoạn câu văn, ... Ví dụ câu văn ``Female patient is 65 years old .'' sau khi được phân giải sẽ gồm các Token là ``Female'', ``patient'', ``is'', ``65'', ``years'', ``old'' and ``.''. Lưu ý, các dấu câu hoặc các ký tự cũng được xem như Token. Tác vụ tách từ được thư viện OpenNLP hỗ trợ bằng phương thức nhận vào câu văn và trả về một mảng các Token tương ứng.

Đánh dấu vai trò ngữ pháp là quá trình phân loại các từ dựa vào từ loại hoặc vai trò ngữ pháp của chúng trong câu. Danh sách các từ loại hoặc vai trò ngữ pháp trong tiếng Anh được định nghĩa sẵn dựa theo dự án Penn TreeBank \cite{Santorini1990}. Ví dụ một số nhãn nằm trong tập Penn TreeBank là: liên từ (Conjunction) được gán nhãn là CC, tính từ (Adjective) được gán nhãn là JJ, danh từ số ít (Singular noun) được gán nhãn là NN, ... Tác vụ đánh dấu vai trò ngữ pháp được thư viện OpenNLP hỗ trợ bằng phương thức nhận vào một mảng các Token và trả về một mảng các vai trò ngữ pháp tương ứng với các Token đó. Ví dụ câu văn ``They refuse to permit us'' được tách từ thành [\textit{They, refuse, to, permit, us}] và được đánh dấu vai trò ngữ pháp như sau [\textit{They/PRP, refuse/VBP, to/TO, permit/VB}]. Trong đó các nhãn mang ý nghĩa là: đại từ PRP, động từ ngôi 3 VBP, thể vô định TO, động từ nguyên mẫu VB.

Phân đoạn câu văn là quá trình phân tích câu văn thành các cụm động từ, cụm giới từ, cụm danh từ, cụm trạng từ, ... cấu thành nên câu văn đó. Ví dụ trong câu văn ``The patient is a 47 years old male'' được phân đoạn thành ba cụm: cụm danh từ ``The patient'', cụm động từ ``is'', cụm danh từ ``a 47 years old male''. Vai trò cũng như loại của cụm từ được định nghĩa trong tập nhãn Penn TreeBank \cite{Santorini1990}. Các cụm từ được mô tả theo định dạng IBO giúp xác định vị trí của từ  đang xét trong cụm từ. B mang ý nghĩa từ đang xét nằm đầu cụm từ, I mang ý nghĩa từ đang xét nằm trong cụm từ nhưng không phải là từ đầu tiên, O mang ý nghĩa từ đang xét nằm ngoài cụm từ. Ví dụ cụm danh từ ``a 47 years old'' được cấu thành từ ``a|B'', ``47|I'', ``years|I'', ``old|I''.

Trong ngôn ngữ học, từ trung tâm của cụm từ là từ giúp xác định vai trò ngữ pháp cụm từ đó. Ví dụ từ trung tâm của \textit{cụm danh từ} ``boiling hot water'' là \textit{danh từ} ``water''. Thư viện OpenNLP hỗ trợ tác vụ xác định danh từ trung tâm bằng phương thức nhận vào một cụm từ theo định dạng IBO và trả về từ trung tâm của cụm từ đó.

\subsection*{MetaMap}
UMLS là kho từ vựng y sinh do Thư viện Y học Quốc gia Hoa Kì tiến hành xây dựng từ năm 1986 nhằm phục vụ cho nhu cầu tra cứu thông tin đã được chuẩn hóa về các loại bệnh, thuốc, nguyên nhân hoặc các thuật ngữ trong y tế. Bộ từ điển UMLS tích hợp hơn hai triệu tên gọi cho khoảng 900000 khái niệm y sinh và khoảng 12 triệu tên gọi cho quan hệ giữa các khái niệm này \cite{Olivier2004}. Hiện nay, bộ từ điển UMLS vẫn đang được tiếp tục cập nhật và cho phép sử dụng miễn phí phục vụ mục đích nghiên cứu khoa học.

MetaMap là công cụ hỗ trợ nhận dạng các khái niệm trong bộ từ điển UMLS từ văn bản y sinh. Đầu vào của MetaMap là văn bản thuần không có cấu trúc, kết quả là một chuỗi định dạng XML hoặc định dạng con người đọc được chứa các khái niệm trong từ điển UMLS nhận dạng được. Các thông tin từ UMLS rút trích được bao gồm: mã khái niệm trong bộ từ điển UMLS, các từ đồng nghĩa và lớp ngữ nghĩa của khái niệm trong UMLS. Hình \ref{metamapstructure} miêu tả kiến trúc tổng quát của MetaMap. Quá trình xử lý dữ liệu của MetaMap có thể được tóm tắt qua hai bước. Bước đầu tiên, dữ liệu dạng văn bản sẽ được thực hiện các tác vụ xử lý ngôn ngữ tự nhiên cơ bản như: tách câu văn (Sentence segmentation), tách từ (Tokenize), đánh dấu vai trò ngữ pháp (POS tag), tra cứu từ vựng (lexical lookup) và phân tích cú pháp. Sau bước đầu tiên, từ văn bản ngôn ngữ tự nhiênm MetaMap sẽ cho kết quả là các cụm từ (Phrase). Bước thứ hai, MetaMap sẽ tiến hành sinh các biến thể (variant) cho các cụm từ đã tìm được, xác định ứng cử viên (candidate) từ các khái niệm trong UMLS khớp với các biến thể được sinh ra và tính độ tin cậy cho các ứng cử viên đó. Đầu ra của bước thứ hai là các cụm từ được phân giải từ bước một, kèm theo các ứng cử viên trong bộ từ điển UMLS cho cụm từ đó. Trong phạm vi Luận văn, chúng tôi sử dụng kết quả tên khái niệm và lớp ngữ nghĩa của khái niệm cho quá trình rút trích đặc trưng lớp Problem, Treatment và Test.

\begin{figure}
\centering
\includegraphics[width=\textwidth]{image/metamap.png}
\caption{Kiến trúc tổng quát của MetaMap\label{metamapstructure}}
\end{figure}

\begin{figure}[ht]
\centering
\lstset{
	keywords={},
	tabsize=3,
	%frame=lines,
	frame=single,
	xleftmargin=20pt,
	framexleftmargin=15pt,
	numbers=left,
	numberstyle=\tiny,
	numbersep=5pt,
	breaklines=true,
	showstringspaces=false,
	basicstyle=\footnotesize}
\lstinputlisting{sample_code/metamap_output.txt}
\caption{Mẫu kết quả của công cụ MetaMap\label{metamapoutput}}
\end{figure}

Hình \ref{metamapoutput} là mẫu kết quả thu được từ MetaMap khi phân tích câu văn ``The patient has a right upper lobe mass''. Sau bước một, câu văn trên được phân giải thành ba cụm từ ``The patient'', ``has'' và ``a right upper lobe mass''. Trong đó kết quả ánh xạ các cụm từ vào bộ từ điển UMLS như sau:

\begin{itemize}[noitemsep]
\item \emph{Cụm từ ``The patient''}: có duy nhất một ứng cử viên có độ tin cậy 1000 ứng với khái niệm ``patient'' mang mã số khái niệm là C0030705. Khái niệm ``patient'' thuộc lớp ngữ nghĩa bệnh nhân hoặc nhóm khuyết tật (Patient or Disabled Group) trong bộ từ điển UMLS.
\item \emph{Cụm từ ``has''}: không tìm được ứng cử viên nào ứng với cụm từ.
\item \emph{Cụm từ ``a right upper lobe mass''}: có bốn ứng cử viên đồng số điểm 916. Ứng cử viên đầu tiên bao gồm hai khái niệm mang mã số C1261074 và C0577573. Các ứng cử viên sau được ghi rõ trong Hình \ref{metamapoutput}. 
\end{itemize}

\subsection*{cTakes và các công cụ dựa trên cTakes}
Apache cTakes (clinical Text Analysis and Knowledge Extraction System) là hệ thống xử lý ngôn ngữ tự nhiên giúp rút trích thông tin từ các bệnh án điện tử dưới dạng văn bản thuần không có cấu trúc. Hệ thống cTakes có khả năng nhận diện các thực thể trong y tế như thuốc, bất thường sức khỏe, triệu chứng, rối loạn, ... Hệ thống cTakes bao gồm nhiều bộ phận nhỏ như phát hiện câu văn, đánh dấu vai trò ngữ pháp, tra cứu dựa trên các bộ từ điển y tế, chuẩn hóa thông tin y tế, ... Mỗi bộ phận của cTakes đã được huấn luyện đặc biệt  cho riêng lĩnh vực y tế, các thông tin rút trích được có thể được sử dụng làm đầu vào cho các hệ thống hỗ trợ ra quyết định y tế hoặc dùng để nghiên cứu.

cTakes được xây dựng dựa trên framework UIMA giúp ứng dụng được phân tách thành các bộ phận con. Mỗi bộ phận con hiện thực một nhiệm vụ nhất định được và cung cấp một tập tin XML đóng vai trò là siêu dữ liệu mô tả bộ phận con đó. Hình \ref{ctakesdesc} là tập tin XML mô tả bộ phận POS Tagger của cTakes, bộ phận này được hiện thực bởi lớp \textit{org.apache.ctakes.postagger.POSTagger} và nhận vào tham số dạng chuỗi là đường dẫn tới tập tin mô hình đã được huấn luyện. Ngoài ra framework UIMA còn hỗ trợ cTakes trong việc gói bộ phận con lại như một mạng lưới các dịch vụ, trong đó bộ phận con này có thể sử dụng kết quả của bộ phận con khác bằng các miêu tả trong tập tin XML, tương tự như mô hình đường ống (pipeline). Trong phạm vi Luận văn, chúng tôi sử dụng các công cụ MedEx, MedTime xây dựng dựa trên cTakes và được cung cấp bởi hội đồng Open Health Natural Language Processing trong quá trình rút trích đặc trưng lớp Problem, Treatment và Test.

\begin{figure}[ht]
\centering
\lstset{
	language=xml,
	keywords={},
	tabsize=3,
	%frame=lines,
	frame=single,
	xleftmargin=20pt,
	framexleftmargin=15pt,
	numbers=left,
	numberstyle=\tiny,
	numbersep=5pt,
	breaklines=true,
	showstringspaces=false,
	basicstyle=\footnotesize}
\lstinputlisting{sample_code/ctakes_desc.xml}
\caption{Tập tin XML mô tả bộ phận con cTakes\label{ctakesdesc}}
\end{figure}

MedEx là công cụ hỗ trợ trích xuất các thông tin về thuốc y tế như tên thuốc (Drug name), liều lượng (strength), đường hấp thu (route), tần số (frequency), ... \cite{HuaXu2009}. Đầu vào công cụ là văn bản thuần không có cấu trúc, kết quả là một tập tin chứa các thông tin thuốc y tế nhận dạng được. Mỗi dòng trong tập tin kết quả là một thông tin thuốc nhận dạng được theo định dạng:

\begin{center}
\textit{<Chỉ số câu>	Nội dung câu|Thông tin thuốc|Tên thuốc được chuẩn hóa}
\end{center}

Trong đó thông tin về thuốc bao gồm:

\begin{itemize}[noitemsep]
\item Tên thuốc (Drug name) như kháng sinh, Vancomycin , ...
\item Nhà sản xuất (Brand name) như Zocor, ...
\item Dạng thuốc (Drug form) như dạng viên nén, dạng viên nhộng, ...
\item Mức độ (Strength) như 10mg, 5ml, ...
\item Liều lượng (Dose amount) như 2 viên, ...
\item Đường hấp thụ (Route) như qua đường miệng, qua đường tiêm, ...
\item Tần số (Frequency) như 2 lần mỗi ngày, mỗi sáng, ...
\item Thời gian (Duration) như trong 10 ngày, trong 1 tháng, ...
\end{itemize}

MedTime là công cụ hỗ trợ trích xuất các thông tin về thời gian trong văn bản thuần không có cấu trúc, bao gồm cả hai loại là thời gian tường minh và thời gian suy diễn \cite{Sohn2013}. Thời gian tường minh là các dạng thời gian được ghi chú rõ ràng, cụ thể trong văn bản như 03/06/2015. Loại thời gian tường minh có thể bao gồm ngày, tháng, năm và thời gian hoặc chỉ bao gồm ngày, tháng, năm. Thời gian suy diễn là các mốc thời gian được trích xuất từ một số từ khóa đứng gần khái niệm đang xét hoặc các khái niệm chỉ về sự kiện trong y tế, ví dụ như ``Admission date'' (ngày nhập viện) hay ``Post-op day 2'' (ngày thứ 2 sau khi phẫu thuật). Giống như công cụ MedEx, đầu vào của công cụ là văn bản thuần không có cấu trúc, kết quảlà một tập tin chứa các thông tin thời gian nhận dạng được. Mỗi dòng trong tập tin kết quả là một thông tin thời gian nhận dạng được theo định dạng:

\begin{center}
\textit{<TIMEX3 id=``chỉ số'' start=``vị trí bắt đầu'' end=``vị trí kết thúc'' text=``chuỗi giá trị nhận dạng'' type=``loại'' val=``giá trị chuẩn hóa''/>}\\
\end{center}
