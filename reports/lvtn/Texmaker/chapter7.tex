\chapter{Tổng kết}

\section{Kết quả đạt được}
Sau quá trình nghiên cứu và tìm hiểu các phương pháp đồng tham chiếu cho các văn bản nói chung và bệnh án điện tử nói riêng, chúng tôi đã hoàn thành hệ thống phân giải đồng tham chiếu cho các HSXV tiếng Anh với độ F là 81.5\%. Các kiến thức chúng tôi thu thập được trong quá trình hiện thực luận án có thể làm nền tảng cho những phát triển sâu hơn của việc khai thác tri thức trong bệnh án điện tử trong tương lai, cụ thể là:
\begin{enumerate}
\item Đối với các văn bản nói chung, có ba mô hình được đề xuất để giải quyết bài toán phân giải đồng tham chiếu: mô hình cặp khái niệm, mô hình đề cập thực thể và mô hình xếp hạng. Mô hình cặp khái niệm có tư tưởng đơn giản nhất nhưng vì thế nó cũng có nhiều nhược điểm, vì vậy một số vấn đề cần phải được cân nhắc trong quá trình hiện thực như sự mất cân bằng lớp, giải thuật học máy, các đặc trưng được rút trích và các phương pháp gom cụm. Mô hình đề cập thực thể được đề xuất nhằm giải quyết vấn đề về sự độc lập của việc xác định tính đồng tham chiếu của các cặp khái niệm, bằng cách đưa vào các đặc trưng ở mức cụm. Trong khi đó mô hình xếp hạng lại nhắm tới vấn đề lựa chọn tiền đề, bằng cách xếp hạng các ứng cử viên để chọn ra tiền đề phù hợp nhất. 
\item Về vấn đề mất cân bằng lớp, có hai cách khác nhau để giải quyết việc này: chọn lọc mẫu (undersampling) hoặc tạo thêm mẫu (oversampling). Việc tạo thêm mẫu giúp cho các mẫu dương ngang bằng với các mẫu âm làm cho số lượng mẫu tăng lên, trong một bài toán có nhiều mẫu được sinh ra thì cách làm này khiến cho thời gian huấn luyện mô hình phân loại tăng lên rất nhiều, nhất là đối với SVM. Trong khi đó việc chọn lọc mẫu một cách tự động lại khiến cho nhiều mẫu chứa các thông tin quan trọng bị loại bỏ, dẫn tới việc mô hình phân loại có xu hướng phân loại về dương nhiều hơn mà bỏ quá sự chính xác của chúng.
\item Đối với bệnh án điện tử, nhiều đặc điểm chuyên biệt trong miền văn bản này giúp cho việc phân giải đồng tham chiếu được dễ dàng. Điển hình là thông tin về sự đề cập đến bệnh nhân, sự giống nhau về chuỗi kí tự của các khái niệm đồng tham chiếu chỉ về bệnh, thủ tục y tế và phương pháp điều trị hay các thông tin về ngữ cảnh y tế trong bệnh án. Ngoài ra đặc điểm một bệnh án chỉ đề cập tới một bệnh nhân cũng là nhân tố quan trọng trong việc phân giải đồng tham chiếu cho các khái niệm chỉ người ở miền văn bản này.
\end{enumerate}

Về mặt thực tiễn, tuy hệ thống của chúng tôi không được hiện thực cho các văn bản tiếng Việt, chúng tôi hy vọng rằng những kết quả đạt được ở luận án này sẽ đóng góp tích cực vào việc nghiên cứu và phát triển BAĐT ở Việt Nam.

\section{Hạn chế và hướng phát triển}
\subsection*{Hạn chế}
Ngoài những kết quả đạt được, hệ thống của chúng tôi vẫn còn một số hạn chế nhất định. Thứ nhất là thời gian chạy thử nghiệm tốn nhiều thời gian. Hạn chế này dẫn đến việc chúng tôi không có đủ thời gian để điều chỉnh hệ thống và thực hiện các thí nghiệm nhiều lần. Để cải thiện vấn đề này, chúng tôi đề xuất sử dụng các máy tính có khả năng tính toán mạnh trong quá trình huấn luyện hoặc áp dụng các kĩ thuật xử lý song song và hệ phân bố cho việc tìm kiếm lưới.

Thứ hai là hạn chế trong việc trích xuất các thông tin từ điển UMLS, Wikipedia và các thông tin ngữ cảnh cho các cặp khái niệm lớp Problem, Test và Treatment. Để rút ngắn thời gian huấn luyện cũng như chạy thử nghiệm trên tập kiểm tra, chúng tôi đã tiến hành rút trích các đặc trưng được nêu thành các tập tin được lưu trữ và sử dụng lại nhiều lần. Tuy nhiên cách hiện thực này làm giảm sự linh động của quá trình phân giải vì đối với các HSXV mới không có sẵn các tập tin này, chúng cần được trích xuất trước khi tiến hành phân giải.

\subsection*{Hướng phát triển}
Vì giới hạn thời gian hiện thực Luận Văn cùng với những hạn chế nêu trên, nhiều ý tưởng và hướng phát triển không được chúng tôi hiện thực trong phạm vi đề tài. Một trong số đó là phát triển để hệ thống hoạt động được với HSXV bằng tiếng Việt. Để làm được điều này chúng tôi cần thay thế các công cụ xử lý ngôn ngữ tự nhiên từ tiếng Anh qua tiếng Việt, đồng thời xây dựng lại các bộ từ điển cho các đặc trưng. Thông qua tìm hiểu, chúng tôi biết được hiện nay đã có một số công cụ hỗ trợ xử lý ngôn ngữ tự nhiên cho tiếng Việt như: công cụ JVnTextPro và Framework VNLP. Vì vậy chúng tôi tin rằng nếu tìm được tập dữ liệu tiếng Việt phù hợp, việc phát triển để hệ thống hoạt động được cho HSXV tiếng Việt là hoàn toàn khả thi.

Ngoài ra, chúng tôi cũng dự định kết hợp hệ thống hiện tại với các hệ thống nhận diện thực thể để tạo thành một công cụ hoàn chỉnh giúp phân giải cho các văn bản BAĐT thô mà không cần danh sách các thực thể cho trước. Hiện nay đã tồn tại nhiều công cụ giúp thực hiện bước nhận diện thực thể như: \textit{CliNER} của nhóm Text Machine Lab cho các văn bản tiếng Anh hoặc kết quả từ luận văn \textit{Nhận dạng tên thực thể trong bệnh án điện tử}, được thực hiện bởi Đào Trọng Điệp và Lê Khắc Sinh, cho các văn bản tiếng Việt. Việc kết hợp yêu cầu thời gian tìm hiểu mã nguồn và cách sử dụng các công cụ nêu trên.
