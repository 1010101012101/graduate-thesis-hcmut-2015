\chapter{Tổng kết}

\section{Kết quả đạt được}
Sau quá trình thực hiện Luận Văn, chúng tôi đã hoàn thành hệ thống phân giải đồng tham chiếu HSXV cho tiếng Anh độ chính xác là 81.5\%. Hệ thống đã đóng góp tích cực vào việc nghiên cứu và phát triển BAĐT tại Việt Nam.

\section{Khó khăn và hạn chế}
Ngoài những kết quả đạt được, hệ thống của chúng tôi vẫn còn một số khó khăn và hạn chế nhất định. Thứ nhất là thời gian huấn luyện mô hình tốn nhiều thời gian. Khó khăn này dẫn đến việc chúng tôi không có đủ thời gian để điều chỉnh hệ thống và thực hiện các thí nghiệm. Nguyên nhân của khó khăn này là số lượng các đặc trưng được sử dụng nhiều cũng như số cặp khái niệm được sinh ra rất lớn (khoảng cặp). Ngoài ra một số công cụ như OpenNLP, MetaMap, cTakes có thời gian xử lý chậm cũng làm ảnh hưởng tới tổng thời gian. Để cải thiện vấn đề này, chúng tôi đề xuất sử dụng các máy tính có khả năng tính toán mạnh trong quá trình huấn luyện hoặc áp dụng các kĩ thuật xử lý song song và hệ phân bố.

Thứ hai là hạn chế trong việc trích xuất các thông tin từ điển UMLS, Wikipedia và các thông tin ngữ cảnh như thuốc y tế và thời gian thành các tập tin lưu trữ. Để rút ngắn thời gian huấn luyện, chúng tôi đã tiến hành rút trích các đặc trưng được nêu thành các tập tin lưu trữ và sử dụng lại nhiều lần. Tuy nhiên cách hiện thực này làm giảm sự linh động của quá trình phân giải vì HSXV cần được duyệt qua để sinh ra các tập tin cần thiết trước khi tiến hành phân giải.

Cuối cùng là hạn chế của mô hình cặp khái niệm khi xây dựng chuỗi phân giải đồng tham chiếu. Mô hình cặp khái niệm xem các ứng cử viên cho vai trò tiền tố của khái niệm đang xét là độc lập với nhau, vì vậy mô hình này chỉ quan tâm đến quan hệ đồng tham chiếu giữa hai khái niệm được xét chứ không quan tâm đến mối quan hệ giữa hai khái niệm đó với các khái niệm khác \cite{VincentNg2010}. Ví dụ cặp khái niệm A-B và cặp khái niệm B-C được xác định là đồng tham chiếu, tuy nhiên nếu cặp khái niệm A-C được xác định là không đồng tham chiếu thì mô hình cặp khái niệm vẫn xây dựng chuỗi đồng tham chiếu là A-B-C.

\section{Hướng phát triển}
Vì giới hạn thời gian hiện thực Luận Văn, nhiều ý tưởng và hướng phát triển không được chúng tôi hiện thực trong phạm vi đề tài. Một trong số đó là phát triển để hệ thống hoạt động được với HSXV bằng tiếng Việt. Để làm được điều này chúng tôi cần thay thế các công cụ xử lý ngôn ngữ tự nhiên từ tiếng Anh qua tiếng Việt, đồng thời xây dựng lại các bộ từ điển cho các đặc trưng. Thông qua tìm hiểu, chúng tôi biết được hiện nay đã có một số công cụ hỗ trợ xử lý ngôn ngữ tự nhiên cho tiếng Việt như: công cụ JVnTextPro và Framework VNLP. Vì vậy chúng tôi tin rằng nếu tìm được tập dữ liệu tiếng Việt phù hợp, việc phát triển để hệ thống hoạt động được cho HSXV tiếng Việt là hoàn toàn khả thi.

Ngoài ra, chúng tôi cũng dự định hệ thống hiện tại với các hệ thống nhận diện thực thể để tạo thành một công cụ hoàn chỉnh giúp phân giải HSXV mà không cần danh sách các thực thể cho trước. Hiện nay đã tồn tại nhiều công cụ giúp thực hiện bước nhận diện thực thể như: \textit{CliNER} của nhóm Text Machine Lab cho các văn bản tiếng Anh hoặc kết quả từ luận văn \textit{Nhận dạng tên thực thể trong bệnh án điện tử}, được thực hiện bởi Đào Trọng Điệp và Lê Khắc Sinh, cho các văn bản tiếng Việt. Việc kết hợp yêu cầu thời gian tìm hiểu sử dụng và mã nguồn từ các công cụ nêu trên.