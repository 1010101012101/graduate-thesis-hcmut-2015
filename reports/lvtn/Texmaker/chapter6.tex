\chapter{Thí nghiệm đánh giá}
\section{Tập dữ liệu}
Tập dữ liệu chúng tôi sử dụng để đánh giá hiệu năng hệ thống được cung cấp bởi Partners Healthcare và Beth Israel Deaconess Medical Center. Đây cũng chính là tập được sử dụng ở tác vụ 1C của Thách thức i2b2 2011 để đánh giá các hệ thống tham gia. Như vậy để có được tập dữ liệu này, chúng tôi đã liên hệ với tổ chức i2b2 và cam kết về các điều khoản sử dụng dữ liệu (Data Usage Agreement) bao gồm việc chỉ sử dụng cho mục đích nghiên cứu. Bản cam kết cần được kí và gửi lại cho i2b2 qua email hoặc fax.

Tập dữ liệu chúng tôi nhận được bao gồm \emph{251 hồ sơ xuất viện} dùng cho huấn luyện và \emph{173 hồ sơ} dùng cho kiểm tra đánh giá. Trong đó, ngoài các hồ sơ xuất viện là các văn bản thuần được ghi chép lại bởi các bác sĩ, y tá (Hình \ref{hsxv-eg}) thì mỗi hồ sơ còn đi kèm với một tập tin chứa danh sách các khái niệm đã được gán nhãn bởi các chuyên gia y tế theo mẫu: \texttt{c="<khái niệm>" <bắt đầu> <kết thúc>||t="<lớp>"}. Ví dụ \texttt{c="the patient" 20:5 20:6||t="person"} mô tả khái niệm ``the patient'' xuất hiện bắt đầu từ dòng 20 từ thứ 5, kết thúc ở dòng 20 từ thứ 6 và thuộc lớp Person (Hình \ref{con-eg}).

\begin{figure}[ht]
\begin{mdframed}
\ttfamily\footnotesize
\protect\raggedright
Discharge Date :\\
06/22/1991\\
DISCHARGE DIAGNOSIS :\\
METASTATIC CERVICAL CARCINOMA .\\
SECONDARY DIAGNOSES INCLUDE PERICARDIAL EFFUSION , PULMONARY LYMPHANGITIC SPREAD , AND LIVER AND SPLEEN METASTASES .\\
OPERATIONS AND PROCEDURES :\\
Echocardiogram times two .\\
HISTORY OF PRESENT ILLNESS :\\
Patient is a 28 year old gravida IV , para 2 with metastatic cervical cancer admitted with a question of malignant pericardial effusion .\\
Patient underwent a total abdominal hysterectomy in 02/90 for a 4x3.6x2 cm cervical mass felt to be a fibroid at Vanor .
\end{mdframed}
\caption{Ví dụ nội dung của một tập tin hồ sơ xuất viện\label{hsxv-eg}}
\end{figure}

\begin{figure}[ht]
\begin{mdframed}
\ttfamily\footnotesize
\protect\raggedright
c="hallucinations" 94:9 94:9||t="problem"\\
c="patient" 22:0 22:0||t="person"\\
c="which" 29:21 29:21||t="pronoun"\\
c="which" 65:9 65:9||t="pronoun"
\end{mdframed}
\caption{Ví dụ nội dung của một tập tin chứa các khái niệm\label{con-eg}}
\end{figure}

Ngoài danh sách khái niệm, mỗi hồ sơ còn đi kèm với một tập tin chứa danh sách các chuỗi đồng tham chiếu đã được phân giải bởi các chuyên gia (các \emph{chuỗi kết quả}) nhằm huấn luyện hệ thống phân loại có giám sát cũng như để đánh giá hiệu năng của hệ thống phân giải (Hình \ref{chains-eg}).

\begin{figure}[ht]
\begin{mdframed}
\ttfamily\footnotesize
\protect\raggedright
c="pulmonary lymphangitic spread" 17:6 17:8||c="extensive lymphatic invasion" 23:25 23:27||c="a malignant lymphangitic spread" 72:6 72:9||c="lymphangitic spread" 83:2 83:3||t="coref problem"\\
c="patient" 21:0 21:0||c="patient" 22:0 22:0||c="patient" 24:0 24:0||c="she" 24:11 24:11||c="the patient" 25:5 25:6||c="she" 26:0 26:0||c="the patient" 28:0 28:1||c="she" 29:0 29:0||c="her" 29:12 29:12||t="coref person"\\
c="dr. mielke" 29:4 29:5||c="who" 29:10 29:10||c="dr. mielke" 101:5 101:6||c="earllamarg s. mielke , m.d" 104:1 104:5||t="coref person"\\
c="a follow-up echocardiogram" 78:3 78:5||c="echocardiogram" 79:0 79:0||t="coref test"\\
c="cisplatin" 86:3 86:3||c="cisplatin chemotherapy" 99:25 99:26||t="coref treatment"
\end{mdframed}
\caption{Ví dụ nội dung của một tập tin chứa các chuỗi đồng tham chiếu\label{chains-eg}}
\end{figure}

\section{Các phương pháp đánh giá}
Hiệu năng hệ thống được đánh giá qua ba độ đo: \emph{độ đúng đắn} (precision), \emph{độ đầy đủ} (recall) và \emph{độ F} (F-measure). Bài báo đánh giá các hệ thống dự thi thử thách i2b2 sử dụng ba phương pháp khác nhau để tính toán các độ này, bao gồm: MUC, B-CUBED và CEAF.

\subsection*{Hệ đo MUC}
Hệ đo MUC \cite{MarcVilain1995} xem chuỗi đồng tham chiếu là một danh sách các liên kết giữa các cặp khái niệm tạo nên chuỗi, từ đó đánh giá hệ thống dựa trên số lượng ít nhất các liên kết cần được thêm vào và loại bỏ để các chuỗi đồng tham chiếu xuất ra bởi hệ thống trùng với các chuỗi kết quả. Có thể hiểu số liên kết cần được loại bỏ là độ thiếu chính xác (precision \emph{error}) và số liên kết cần được thêm vào là độ thiếu đầy đủ (recall \emph{error}). 

Với mỗi văn bản $d$, gọi $G$ là tập các chuỗi kết quả, $S$ là tập các chuỗi hệ thống. Các độ đúng đắn ($P$) và độ đầy đủ ($R$) được tính như sau:
\[P_d=\frac{\sum_{s\in S} \left(|s| - m(s, G)\right)}{\sum_{s\in S}\left(|s| - 1\right)}\]
\[R_d=\frac{\sum_{g\in G}(|g|-m(g,S))}{\sum_{g\in G}(|g|-1)}\]

\noindent trong đó, $|s|$ là số khái niệm tạo thành chuỗi $s$, $m(s,G)$ được tính là tổng số chuỗi trong $G$ có giao nhau với $s$ cộng với số khái niệm trong $s$ không xuất hiện trong tất cả các chuỗi trong $G$. Độ $F$ của hệ MUC là trung bình điều hòa của độ chính xác và độ đầy đủ:
\[F_d=\frac{2\times P_d\times R_d}{P_d + R_d}\]

\subsection*{Hệ đo B-CUBED}
Khác với hệ đo MUC đánh giá dựa trên sự thiếu và thừa các liên kết trong chuỗi hệ thống, hệ đo B-CUBED đánh giá hiệu năng trên mỗi khái niệm trong văn bản. Theo nhận định của các tác giả hệ đo B-CUBED, cách đánh giá dựa trên các liên kết có hai nhược điểm \cite{AmitBagga1998}:

\begin{enumerate}[leftmargin=\parindent]
\item Không xét tới các \emph{khái niệm duy nhất} vì các liên kết chỉ tồn tại khi có ít nhất hai khái niệm. Mặt khác, theo quy ước, các chuỗi chỉ chứa một khái niệm không được đưa vào tập kết quả và các hệ thống tuân theo quy ước cũng không xuất những chuỗi như vậy ra.

\item Các lỗi của hệ thống được xem là như nhau, tức hệ đo MUC đánh giá cùng một mức phạt như nhau cho tất cả các sai sót của hệ thống. Tuy nhiên có thể nhận thấy là có vài loại sai sót làm cho chuỗi hệ thống bị sai lệch nhiều hơn so với các loại khác.
\end{enumerate}

Hệ đo B-CUBED được thiết kể để khắc phục hai nhược điểm trên bằng cách tính toán độ chính xác và độ đầy đủ cho từng khái niệm trong văn bản, sau đó kết hợp chúng lại để ra kết quả cuối cùng. Như vậy theo \cite{AmitBagga1998}, với khái niệm thứ $i$, độ chính xác và độ đầy đủ của nó được định nghĩa như sau:
\[P_i=\frac{\text{số khái niệm đúng trong chuỗi hệ thống chứa khái niệm thứ $i$}}{\text{số khái niệm trong chuỗi hệ thống chứa khái niệm thứ $i$}}\]
\[R_i=\frac{\text{số khái niệm đúng trong chuỗi hệ thống chứa khái niệm thứ $i$}}{\text{số khái niệm trong chuỗi kết quả chứa khái niệm thứ $i$}}\]

Độ chính xác và đầy đủ trên toàn văn bản được tính theo công thức:
\[P=\sum_{i=1}^{M} w_i \times P_i,\quad R=\sum_{i=1}^{M} w_i\times R_i\]

\noindent trong đó, $w_i$ là trọng số của khái niệm thứ $i$, $M$ là số khái niệm của văn bản đang xét. Bài báo đánh giá các hệ thông dự thi thử thách i2b2 năm 2011 sử dụng chung một trọng số cho tất cả các khái niệm, $w_i=1/M$. Để tiện cho việc so sánh với hệ thống \cite{YanXu2012}, chúng tôi cũng dùng cách gán trọng số như trên.

Như vậy để thuận tiện trong tính toán, nếu gọi $m$ là một khái niệm trong văn bản $d$ có chứa $M$ khái niệm, $G_m$ là chuỗi kết quả có chứa $m$, $S_m$ là chuỗi hệ thống có chứa $m$, $O_m$ là chuỗi giao nhau giữa $G_m$ và $S_m$ (tức $O_m=G_m\cap S_m$) thì độ đúng đắn và độ đầy đủ của hệ B-CUBED cho $d$ được tính như sau:
\[P_d=\frac{1}{M}\sum_{m\in d}\frac{|O_m|}{|S_m|},\quad R_d=\frac{1}{M}\sum_{m\in d}\frac{|O_m|}{|G_m|}\]

Độ F của hệ B-CUBED được tính tương tự như hệ MUC.

\subsection*{Hệ đo CEAF}
Hệ đo CEAF được tác giả Xiaoqiang Lou giới thiệu vào năm 2005 như một cách đánh giá khác để khắc phục các nhược điểm của hệ MUC \cite{XiaoquangLuo2005}. Thay vì tính toán các độ đo trên từng khái niệm như hệ B-CUBED, hệ CEAF đánh giá hệ thống bằng cách tối ưu hóa sự \emph{sắp xếp đầy đủ} giữa các chuỗi kết quả và chuỗi hệ thống dựa trên tổng các độ tương tự của từng cặp chuỗi, với điều kiện một chuỗi kết quả chỉ được ghép cặp với nhiều nhất một chuỗi hệ thống. Theo nhận định của tác giả hệ đo CEAF, sắp xếp tối ưu giúp ngăn ngừa việc ``ăn gian'' của các hệ thống phân giải: một hệ thống xuất ra quá nhiều chuỗi đồng tham chiếu sẽ bị đánh giá độ chính xác thấp, trong khi xuất ra quá ít chuỗi thì sẽ bị đánh giá độ đầy đủ thấp.

Với mỗi văn bản $d$, gọi $G=\{g_i:i=1,2,\dots,|G|\}$ là tập các chuỗi kết quả của $d$, $S=\{s_i:i=1,2,\dots,|S|\}$ là tập các chuỗi hệ thống xuất ra cho $d$. Vì vai trò như nhau của $S$ và $G$ trong hệ đo này, chúng tôi giả định rằng $|S|<|G|$. Theo định nghĩa của \cite{XiaoquangLuo2005}, một sự sắp xếp đầy đủ có thể được xem là một ánh xạ một-một đi từ $S$ vào $G$, $H=\{h:S\mapsto G\}$, thỏa hai điều kiện:
\begin{enumerate}
\item $\forall s \in S,\forall s^{\prime} \in S: s\neq s^{\prime} \Leftrightarrow h(s)\neq h(s^{\prime})$
\item $|H|=|S|$
\end{enumerate}

%\vspace{6pt}
Đặt $m=|S|,\, M=|G|$, $\mathcal{H}$ là toàn bộ những sắp xếp đầy đủ có thể giữa $S$ và $G$, dễ dàng tính được $|\mathcal{H}|=\binom{M}{m}m!$. Gọi $\phi(s,g)$ là độ tương tự giữa hai chuỗi $s$ và $g$ bất kì. Độ tương tự cho một sự sắp xếp đầy đủ $H\in\mathcal{H}$ giữa $S$ và $G$ được định nghĩa là tổng các độ tương tự giữa mỗi cặp $(s,h(s))$ trong $H$: $\Phi(H)=\sum_{s\in S} \phi(s,h(s))$. Như vậy một sự sắp xếp đầy đủ tối ưu giữa $S$ và $G$ chính là sự sắp xếp $H^*$ thỏa:
\begin{align*}
\Phi(H^*)&=\max_{H\in\mathcal{H}} \Phi(H)\\
&=\max_{H\in\mathcal{H}} \sum_{s\in S} \phi(s,h(s))
\end{align*}

Trong trường hợp $|S|>|G|$, vai trò của $S$ và $G$ được hoán đổi cho nhau ở các định nghĩa trên. Để tính độ tương tự giữa hai chuỗi đồng tham chiếu, tác giả hệ CEAF đề xuất bốn cách tính khác nhau:
\begin{align*}
\phi_1(s,g)=
\begin{cases}
	1 & \text{nếu } s = g\\
	0 & \text{nếu } s \neq g
\end{cases}&,\quad
\phi_2(s,g)=
\begin{cases}
	1 & \text{nếu } s\cap g\neq\emptyset\\
	0 & \text{các trường hợp khác}
\end{cases},\\
\phi_3(s,g)=|s\cap g|&,\quad\phi_4(s,g)=\frac{2|s\cap g|}{|s|+|g|}
\end{align*}

Theo nhận định của chính tác giả, $\phi_3$ và $\phi_4$ tỏ ra hiệu quả hơn trong việc đánh giá các chuỗi đồng tham chiếu. Mặt khác, thử thách i2b2 năm 2011 sử dụng $\phi_4$ để đánh giá các hệ thống dự thi nên để cho tiện trong việc so sánh với hệ thống \cite{YanXu2012}, chúng tôi cũng sử dụng $\phi_4$ để đánh giá hệ thống của mình.

Ngoài ra, việc tìm kiếm một sự sắp xếp tối ưu giữa hai tập chuỗi không thể được thực hiện bằng cách duyệt toàn bộ các sự sắp xếp đầy đủ có thể vì số các cách sắp xếp đầy đủ là rất lớn, $\binom{M}{m}m!$. Về mặt giải thuật, cách làm này được xem là có độ phức tạp hàm giai thừa. Tuy nhiên theo nhận định của tác giả CEAF, bài toán sắp xếp tối ưu này chính là bài toán giao công việc (\emph{assignment problem}) hay bài toán vận chuyển (\emph{transportation problem}) đã được giải quyết bởi Harold W. Kuhn vào năm 1955 với giải thuật có tên là \emph{phương pháp Hungary} \cite{HungarianMethod} có độ phức tạp là $\BigO(n^4)$.

Từ đây, hệ CEAF tính độ đúng đắn và độ đầy đủ cho một văn bản $d$ theo công thức:
\[P_d=\frac{\Phi(H^*)}{\sum_{s\in S} \phi(s,s)},\quad R_d=\frac{\Phi(H^*)}{\sum_{g\in G} \phi(g,g)}\]

Độ F của hệ CEAF được tính tương tự như hệ MUC.

\section{Kết quả}
Vì mỗi hệ đo nêu trên đánh giá các chuỗi đồng tham chiếu cho một văn bản HSXV duy nhất nên ứng với mỗi hệ đo, trung bình không trọng số của mỗi độ đo tính trên toàn tập kiểm tra được lấy làm kết quả của hệ đo đó. Gọi $D$ là tập gồm $N$ HSXV dùng để kiểm tra hệ thống, kết quả đo trung bình của hệ MUC được tính như sau:
\[P^{\text{MUC}}=\frac{\sum_{d\in D}P_d^{\text{MUC}}}{N},\quad R^{\text{MUC}}=\frac{\sum_{d\in D}R_d^{\text{MUC}}}{N},\quad F^{\text{MUC}}=\frac{\sum_{d\in D}F_d^{\text{MUC}}}{N}\]

Các hệ B-CUBED và CEAF được tính tương tự. Cuối cùng, trung bình không trọng số của các hệ được lấy làm kết quả chung cuộc để đánh giá hệ thống:
\[P=\frac{P^{\text{MUC}}+P^{\text{B-CUBED}}+P^{\text{CEAF}}}{3}\]
\[R=\frac{R^{\text{MUC}}+R^{\text{B-CUBED}}+R^{\text{CEAF}}}{3}\]
\[F=\frac{F^{\text{MUC}}+F^{\text{B-CUBED}}+F^{\text{CEAF}}}{3}\]

Các kết quả đánh giá hệ thống của chúng tôi với đầy đủ các đặc trưng được trình bày trên Bảng \ref{final-result} với kết quả chung cuộc là: độ chính xác 0.847, độ đầy đủ 0.795 và độ F là 0.815. Kết quả phân giải cho các chuỗi thuộc lớp Person rất khả quan với độ F là 0.890. Điều này cho thấy sự khác biệt giữa miền văn bản BAĐT với các miền văn bản khác, trong đó thông tin về bệnh nhân đóng góp một phần không nhỏ.

\begin{rtable}{Kết quả đánh giá hệ thống với đầy đủ đặc trưng\label{final-result}}
Tất cả & 0.882 & 0.817 & 0.845 && 0.921 & 0.873 & 0.895 && 0.737 & 0.695 & 0.707 && \underline{0.847} & \underline{0.795} & \underline{0.815} \\
Person & 0.965 & 0.928 & 0.944 && 0.939 & 0.895 & 0.910 && 0.872 & 0.795 & 0.816 && 0.925 & 0.873 & 0.890 \\
Problem & 0.742 & 0.633 & 0.666 && 0.814 & 0.674 & 0.689 && 0.685 & 0.653 & 0.654 && 0.747 & 0.653 & 0.670 \\
Test & 0.631 & 0.558 & 0.568 && 0.812 & 0.676 & 0.637 && 0.634 & 0.580 & 0.589 && 0.692 & 0.604 & 0.598 \\
Treatment & 0.795 & 0.763 & 0.759 && 0.821 & 0.805 & 0.772 && 0.779 & 0.772 & 0.761 && 0.798 & 0.780 & 0.764 \\
\end{rtable} 

Bảng .. cho thấy hiệu năng của mô hình phân loại bệnh nhân dựa trên kiểm chứng chéo 10 mẫu (10-fold cross validation) với độ F là .. Bảng .. so sánh kết quả đánh giá phân giải đồng tham chiếu cho lớp Person với việc sử dụng và không sử dụng đặc trưng Patient-class. Hệ thống với đặc trưng Patient-class có hiệu năng được cải thiện lên đến xx\%, điều này cho thấy thông tin về bệnh nhân trong miền văn bản BAĐT là một thông tin rất hữu ích.

Ngoài ra, các kết quả đánh giá cho các lớp Problem, Test và Treatment có độ F lần lượt là 0.670, 0.598 và 0.764. Chúng tôi có một số nhận xét sau cho kết quả của ba lớp này:
\begin{enumerate}
\item Kết quả của lớp Test là thấp nhất vì theo quan sát của chúng tôi, tập vector đặc trưng dùng để huấn luyện của lớp Test chịu ảnh hưởng nhiều nhất bởi sự mất cân bằng mẫu (khi chưa qua chọn lọc mẫu, tỉ lệ mẫu âm trên mẫu dương là 381:1). Mặt khác, nhiều thủ tục y tế trong một hồ sơ xuất viện trùng tên hoàn toàn hoặc một phần với nhau nhưng lại là những khái niệm duy nhất không đồng tham chiếu với bất kì khái niệm nào.
\end{enumerate}
