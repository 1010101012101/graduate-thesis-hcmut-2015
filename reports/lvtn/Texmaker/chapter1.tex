\chapter{Tổng quan}
\section{Giới thiệu đề tài}\label{gioithieudetai}
Trong hơn mười năm trở lại đây, với sự bùng nổ của kỉ nguyên công nghệ thông tin, việc số hóa dữ liệu trở nên phổ biến hơn bao giờ hết, và bệnh án cũng không phải là ngoại lệ. Bệnh án điện tử (Electronic Medical Record) đã và đang dần thay thế cho phương pháp ghi chép và lưu trữ truyền thống thông tin của bệnh nhân trong quá trình khám và chữa bệnh. Hầu hết bệnh viện ở những nước phát triển đã triển khai các hệ thông tin bệnh viện (HTTBV) để phục vụ cho việc số hóa loại tài liệu này.

Bên cạnh việc xây dựng bệnh án điện tử (BAĐT) thì việc khai thác nguồn dữ liệu lớn này cũng là một lĩnh vực đang rất được quan tâm trong những năm gần đây. Năm 2004, Viện y tế Quốc gia Hoa Kì (NIH: National Institute of Health) đã kêu gọi thành lập mạng lưới nghiên cứu cấp quốc gia về y sinh. Để đáp lại lời kêu gọi đó, bảy Trung tâm nghiên cứu công nghệ tính toán y sinh (NBCB: National Center for Biomedical Computing) đã được thành lập dưới sự tài trợ của NIH với nhiệm vụ xây dựng cơ sở hạ tầng phục vụ cho việc áp dụng khoa học máy tính vào lĩnh vực y sinh, hỗ trợ cho công việc nghiên cứu. Trong đó, i2b2 (Informatics for Integrating Biology and the Bedside), một NBCB được thành lập bởi sự hợp tác giữa hai trường đại học nổi tiếng Havard và MIT, bắt đầu từ năm 2006 đã tổ chức các cuộc thi hàng năm nhằm tìm kiếm các phương pháp phân tích và rút trích kiến thức trên dữ liệu BAĐT, gọi tắt là các Thách thức (Challenges). Mỗi Thách thức đưa ra một vấn đề phân tích và một tập dữ liệu BAĐT được cung cấp bởi các bệnh viện trong và ngoài nước Mỹ. Hàng năm có trên dưới 100 nhóm nghiên cứu tham gia đề xuất giải pháp và gửi kết quả phân tích, những giải pháp tốt được chọn lọc để công bố ở một hội thảo quốc tế và được áp dụng rộng rãi vào các dịch vụ chăm sóc sức khỏe.

Tại Việt Nam, các HTTBV cũng đang dần được triển khai, tiêu biểu là Bệnh viện đa khoa Vân Đồn tỉnh Quảng Ninh – cơ quan y tế đầu tiên có trang bị hệ thống bệnh án điện tử hiện đại và hoàn chỉnh với giải pháp MEDI SOLUTIONS của công ty phần mềm Hoa Sen. Cùng với việc xây dựng, tập thể nghiên cứu ``Học máy và ứng dụng'' của viện John von Neumann thuộc đại học Quốc Gia TP Hồ Chí Minh đã tiến hành phát triển các phương pháp và phần mềm phục vụ cho khai thác bệnh án điện tử tiếng Việt. Một trong những vấn đề của việc khai thác dữ liệu BAĐT đó là phân giải đồng tham chiếu. Thách thức lần thứ 5 (năm 2011) của i2b2 đã đưa ra một cái nhìn có hệ thống về vấn đề này. Một cách tổng quát, việc phân giải đồng tham chiếu các khái niệm trong văn bản là xác định liệu hai sự đề cập trong cùng văn bản có ám chỉ tới cùng một sự vật hoặc hiện tượng hay không, từ đó xây dựng các chuỗi đồng tham chiếu. Khi mà đa phần các văn bản được viết tay bằng ngôn ngữ tự nhiên, chứa đựng rất nhiều các khái niệm phụ thuộc vào ngữ cảnh thì việc phân giải đồng tham chiếu này giúp cho máy tính có một cái nhìn mang tính cấu trúc hơn về văn bản, từ đó làm nền tảng cho việc rút trích các kiến thức sâu từ những hiểu biết này.

Tuy vấn đề về phân giải đồng tham chiếu trong những năm gần đây đã được quan tâm nghiên cứu rất nhiều cho các loại văn bản khác (ví dụ các bài báo) thì ở phạm vi BAĐT vấn đề này vẫn còn ít được sự quan tâm. Đứng trước nhu cầu đó, nhóm quyết định bắt tay vào phát triển một hệ thống phân giải đồng tham chiếu cho dữ liệu BAĐT.

\section{Mục tiêu và phạm vi đề tài}
Những ích lợi BADT mang lại như đề cập trong Phần \ref{gioithieudetai} đã tạo động lực cho chúng tôi tiến hành huấn luyện hệ thống phân giải đồng tham chiếu trong BAĐT. Việc xây dựng thành công hệ thống phân giải đồng tham chiếu góp phần hỗ trợ cho các công trình nghiên cứu sâu hơn về BAĐT sau này cũng như được dùng để xây dựng các công cụ thống kê hoặc các hệ thống truy xuất thông tin trong y tế.

Vì giới hạn thời gian, chúng tôi quyết định chỉ huấn luyện hệ thống phân giải đồng tham chiếu cho các hồ sơ xuất viện được viết bằng tiếng Anh với danh sách các thực thể đã được xác định trước. Thách thức I2B2 2011 cung cấp sẵn tập dữ liệu là các báo cáo xuất viện được phân giải đồng tham chiếu sẵn bởi các chuyên gia y tế phù hợp cho quá trình huấn luyện hệ thống. Tập dữ liệu này được tổ chức I2B2 cung cấp miễn phí kèm theo một số cam kết sử dụng dữ liệu (Data agreement), đây là tập dữ liệu vô cùng giá trị cho viết huấn luyện hệ thống. Dựa trên kết quả Thách thức 2011, vấn đề phân giải đồng tham chiếu trong hồ sơ xuất viện có hai hướng giải quyết đạt kết quả tốt là: hướng tiếp cận dựa trên các luật (Rule-based hoặc còn gọi là hướng tiếp cận về ngôn ngữ học) và hướng tiếp cận dựa trên Học máy. Chúng tôi quyết định hiện thực hệ thống dựa trên bài báo có kết quả tốt nhất trong Thách thức 2011 của tác giả Yan Xu \cite{YanXu2012}. Trong đó, giải thuật học máy được áp dụng là Support Vector Machine và mô hình phân giải đồng tham chiếu được áp dụng là mô hình cặp khái niệm (Mention-pair)

\section{Cấu trúc luận văn}
Nội dung bài viết được chúng tôi chia thành bảy chương. Từng chương trình bày một quá trình để xây dựng hoàn thiện hệ thống phân giải đồng tham chiếu trong HSXV cũng như kỹ thuật và kết quả các quá trình hiện thực hệ thống. Chương cuối cùng dùng để tóm tắt, nêu các đánh giá để tìm ra hạn chế và các hướng mở rộng, cải tiến trong tương lai. Nội dung sơ lược của từng chương được tóm tắt như sau.

Chương một nêu lên mục tiêu, động cơ và phạm vi của Luận văn. Toàn bộ chương một giúp người đọc có được cái nhìn toàn cảnh về lí do chúng tôi tiến hành thực hiện đề tài, vai trò và vị trí của đề tài trong việc phát triển BAĐT ở Việt Nam cũng như phạm vi hiện thực của đề tài.

Chương hai giới thiệu các công trình liên quan đến hệ thống. Toàn bộ chương hai giúp người đọc có được kiến thức khái quát về các lĩnh vực nền tảng cho việc phân giải đồng tham chiếu trong BAĐT, đây là các kiến thức nền giúp người đọc dễ dàng nắm bắt quá trình hiện thực hệ thống. Các lĩnh vực này bao gồm khái niệm BAĐT là gì và đầu vào cần thiết trước khi tiến hành phân giải đồng tham chiếu.

Chương ba giới thiệu các kiến thức nền tảng được sử dụng trực tiếp trong việc xây dựng hệ thống. Toàn bộ chương ba giúp người đọc hiểu được các thuật ngữ, kĩ thuật và công nghệ được áp dụng khi tiến hành hiện thực hệ thống. Các kiến thức được nêu chỉ mang tính giới thiệu, nội dung sâu hơn của các kĩ thuật có thể được tìm thấy trong nguồn tham khảo.

Chương bốn giới thiệu chi tiết hệ thống chúng tôi đã xây dựng. Qua chương bốn, người đọc có thể hiểu được kiến trúc tổng quát toàn bộ hệ thống cũng như chi tiết hiện thực từng bước như tiền xử lý, rút trích đặc trưng, huấn luyện mô hình SVM và gom cụm xây dựng chuỗi đồng tham chiếu. Chương bốn giúp người đọc có thể tự hiện thực lại toàn bộ hệ thống với mục đích nghiên cứu, cải tiến hoặc sử dụng.

Chương năm giới thiệu về tập dữ liệu được sử dụng trong quá trình huấn luyện cũng như các thí nghiệm đánh giá hệ thống. Trong chương năm, chúng tôi đề cập chi tiết và các văn bản được cung cấp trong tập dữ liệu I2B2 2011 và các bước để có được tập dữ liệu này. Ngoài ra, các độ đo được sử dụng và kết quả thí nghiệm để đánh giá hiệu năng hệ thống cũng được nêu chi tiết.

Nội dung tổng kết được chúng tôi đề cập trong chương bảy, bao gồm tóm tắt nội dung luận văn, các nhận xét và hướng phát triển trong tương lai.