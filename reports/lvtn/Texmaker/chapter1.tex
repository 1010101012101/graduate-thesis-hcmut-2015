\chapter{Tổng quan}
\section{Giới thiệu đề tài\label{gioithieudetai}}
Trong hơn mười năm trở lại đây, với sự bùng nổ của kỉ nguyên công nghệ thông tin, việc số hóa dữ liệu trở nên phổ biến hơn bao giờ hết, và bệnh án cũng không phải là ngoại lệ. Bệnh án điện tử (Electronic Medical Record) đã và đang dần thay thế cho phương pháp ghi chép và lưu trữ truyền thống thông tin của bệnh nhân trong quá trình khám và chữa bệnh. Hầu hết bệnh viện ở những nước phát triển đã triển khai các hệ thông tin bệnh viện (HTTBV) để phục vụ cho việc số hóa loại tài liệu này.

Bên cạnh việc xây dựng bệnh án điện tử (BAĐT) thì việc khai thác nguồn dữ liệu lớn này cũng là một lĩnh vực đang rất được quan tâm trong những năm gần đây. Năm 2004, Viện y tế Quốc gia Hoa Kì (NIH: National Institute of Health) đã kêu gọi thành lập mạng lưới nghiên cứu cấp quốc gia về y sinh. Để đáp lại lời kêu gọi đó, bảy Trung tâm nghiên cứu công nghệ tính toán y sinh (NBCB: National Center for Biomedical Computing) đã được thành lập dưới sự tài trợ của NIH với nhiệm vụ xây dựng cơ sở hạ tầng phục vụ cho việc áp dụng khoa học máy tính vào lĩnh vực y sinh, hỗ trợ cho công việc nghiên cứu. Trong đó, I2B2 (Informatics for Integrating Biology and the Bedside), một NBCB được thành lập bởi sự hợp tác giữa hai trường đại học nổi tiếng Havard và MIT, bắt đầu từ năm 2006 đã tổ chức các cuộc thi hàng năm nhằm tìm kiếm các phương pháp phân tích và rút trích kiến thức trên dữ liệu BAĐT, gọi tắt là các Thách thức (Challenges). Mỗi Thách thức đưa ra một vấn đề phân tích và một tập dữ liệu BAĐT được cung cấp bởi các bệnh viện trong và ngoài nước Mỹ. Hàng năm có trên dưới 100 nhóm nghiên cứu tham gia đề xuất giải pháp và gửi kết quả phân tích, những giải pháp tốt được chọn lọc để công bố ở một hội thảo quốc tế và được áp dụng rộng rãi vào các dịch vụ chăm sóc sức khỏe.

Tại Việt Nam, các HTTBV cũng đang dần được triển khai, tiêu biểu là Bệnh viện đa khoa Vân Đồn tỉnh Quảng Ninh--cơ quan y tế đầu tiên có trang bị hệ thống bệnh án điện tử hiện đại và hoàn chỉnh với giải pháp MEDI SOLUTIONS của công ty phần mềm Hoa Sen. Cùng với việc xây dựng, tập thể nghiên cứu ``Học máy và ứng dụng'' của viện John von Neumann thuộc đại học Quốc Gia TP Hồ Chí Minh đã tiến hành phát triển các phương pháp và phần mềm phục vụ cho khai thác bệnh án điện tử tiếng Việt. Một trong những vấn đề của việc khai thác dữ liệu BAĐT đó là phân giải đồng tham chiếu. Thách thức lần thứ 5 (năm 2011) của I2B2 đã đưa ra một cái nhìn có hệ thống về vấn đề này. Một cách tổng quát, việc phân giải đồng tham chiếu các khái niệm trong văn bản là xác định liệu hai sự đề cập trong cùng văn bản có ám chỉ tới cùng một sự vật hoặc hiện tượng hay không, từ đó xây dựng các chuỗi đồng tham chiếu. Khi mà đa phần các văn bản được viết tay bằng ngôn ngữ tự nhiên, chứa đựng rất nhiều các khái niệm phụ thuộc vào ngữ cảnh thì việc phân giải đồng tham chiếu này giúp cho máy tính có một cái nhìn mang tính cấu trúc hơn về văn bản, từ đó làm nền tảng cho việc rút trích các kiến thức sâu từ những hiểu biết này.

Tuy vấn đề về phân giải đồng tham chiếu trong những năm gần đây đã được quan tâm nghiên cứu rất nhiều cho các loại văn bản khác (ví dụ các bài báo) thì ở phạm vi BAĐT vấn đề này vẫn còn ít được sự quan tâm, đặc biệt là ở Việt Nam. Đứng trước nhu cầu đó, chúng tôi quyết định bắt tay vào phát triển một hệ thống phân giải đồng tham chiếu cho dữ liệu BAĐT tiếng Anh, làm nền tảng cho các nghiên cứu trên dữ liệu tiếng Việt sau này.

\section{Mục tiêu và phạm vi đề tài}
Những ích lợi BADT mang lại như đề cập ở trên đã tạo động lực cho chúng tôi tiến hành hiện thực một hệ thống phân giải đồng tham chiếu cho BAĐT. Việc xây dựng thành công hệ thống phân giải đồng tham chiếu góp phần hỗ trợ cho các công trình nghiên cứu sâu hơn về BAĐT sau này cũng như được dùng để xây dựng các công cụ thống kê hoặc các hệ thống truy xuất thông tin trong y tế.

Vì giới hạn thời gian, chúng tôi quyết định chỉ hiện thực hệ thống phân giải đồng tham chiếu cho các hồ sơ xuất viện được viết bằng tiếng Anh với danh sách các thực thể đã được xác định trước. Thách thức I2B2 2011 cung cấp sẵn tập dữ liệu là các báo cáo xuất viện được phân giải đồng tham chiếu sẵn bởi các chuyên gia y tế phù hợp cho quá trình huấn luyện hệ thống. Tập dữ liệu này được tổ chức I2B2 cung cấp miễn phí kèm theo một số cam kết sử dụng dữ liệu (Data Usage Agreement), đây là tập dữ liệu vô cùng giá trị cho viết huấn luyện hệ thống. Dựa trên kết quả Thách thức 2011, vấn đề phân giải đồng tham chiếu trong hồ sơ xuất viện có hai hướng giải quyết đạt kết quả tốt là: hướng tiếp cận \emph{dựa trên luật} (rule-based, hay còn gọi là hướng tiếp cận về ngôn ngữ học) và hướng tiếp cận dựa trên \emph{học máy} (machine learning). Chúng tôi quyết định hiện thực hệ thống dựa trên hệ thống sử dụng học máy có kết quả tốt nhất trong Thách thức 2011 của các tác giả Yan Xu et al. \cite{YanXu2012}. Trong đó, giải thuật học máy được áp dụng là Support Vector Machine và mô hình phân giải đồng tham chiếu được áp dụng là mô hình \emph{cặp khái niệm}.

\section{Cấu trúc luận văn}
Toàn bộ nội dung luận văn được chúng tôi trình bày thành sáu chương. Các chương này nêu lên những kiến thức cần thiết, chi tiết cách thức hiện thực để xây dựng và hoàn thiện hệ thống phân giải đồng tham chiếu trong BAĐT cũng như các kết quả đánh giá. Ở chương cuối, chúng tôi đưa ra các tổng kết về kết quả đạt được của luận văn, các hạn chế và các hướng phát triển trong tương lai. Sau đây là nội dung chính của mỗi chương:

\begin{description}[style=nextline,leftmargin=0cm]
\item[Chương 1: Tổng quan] Chương đầu tiên chúng tôi nêu lên mục tiêu, động cơ và phạm vi của luận án. Toàn bộ chương này giúp người đọc có được cái nhìn toàn cảnh về lí do chúng tôi tiến hành thực hiện đề tài, vai trò và vị trí của đề tài trong việc phát triển BAĐT ở Việt Nam cũng như phạm vi hiện thực của đề tài.
\item[Chương 2: Các công trình liên quan] Chương này nêu lên một số công trình liên quan đến phân giải đồng tham chiếu trong bệnh án điện tử, bao gồm giới thiệu chung về bệnh án, các loại văn bản bệnh án, lịch sử số hóa bệnh án và các chuẩn chung cho việc định dạng cấu trúc của bệnh án điện tử. Ngoài ra, chúng tôi cũng trình bày về vấn đề nhận dạng thực thể và một số giải pháp cho nó, vì kết quả của nhận dạng thực thể là đầu vào của hệ thống phân giải đồng tham chiếu mà chúng tôi hiện thực.
\item[Chương 3: Kiến thức nền tảng] Chương này cung cấp các cơ sở lý thuyết về các mô hình, giải thuật và các công cụ được chúng tôi sử dụng để hiện thực hệ thống, bao gồm các mô hình phân giải đồng tham chiếu mà đặc biệt là mô hình cặp khái niệm, giải thuật học Support Vector Machine dùng cho phân loại tính đồng tham chiếu và các công cụ rút trích đặc trưng cho mục đích huấn luyện hay phân loại trong quá trình phân giải.
\item[Chương 4: Hiện thực hệ thống] Trong chương này, chúng tôi trình bày về các chi tiết hiện thực của hệ thống, bao gồm quy trình huấn luyện hệ thống phân loại và quy trình phân giải đồng tham chiếu. Ngoài mô hình phân giải đồng tham chiếu, nội dung của chương còn tập trung vào vấn đề rút trích đặc trưng cho các thực thể trong văn bản thô, bao gồm những đặc trưng cần thiết và cách thức rút trích chúng.
\item[Chương 5: Thí nghiệm đánh giá] Các kết quả so sánh thực nghiệm của hệ thống được chúng tôi trình bày trong chương này. Tập dữ liệu được sử dụng để đánh giá là tập các bệnh án điện tử được cung cấp bởi I2B2 và các phương pháp đánh giá bao gồm MUC, B-CUBED và CEAF, tất cả được trình bày chi tiết trong nội dung của chương. Ngoài ra, cuối chương chúng tôi nêu lên một số giải thích và nhận xét được rút ra từ kết quả đánh giá hệ thống.
\item[Chương 6: Tổng kết] 
\end{description}
