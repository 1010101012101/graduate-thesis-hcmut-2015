\documentclass[11pt,a4paper,twoside]{report}
\usepackage[T1]{fontenc}
\usepackage[utf8]{inputenc}
\usepackage[vietnam]{babel}
\usepackage{amsmath}
\usepackage{amsfonts}
\usepackage{amssymb}
\usepackage{graphicx}
\usepackage[tmargin=1in,bmargin=1in,lmargin=1.4in,rmargin=1in]{geometry}
\usepackage{setspace}
\usepackage{array}
\usepackage{booktabs}
\usepackage{tabularx}
%\usepackage[ruled,lined,linesnumbered]{algorithm2e}
\usepackage{titlesec}
\usepackage{caption}
\usepackage{tikz}
\usepackage{fancyhdr}
\usepackage[unicode,hidelinks]{hyperref}
\usepackage{enumitem}
%\usepackage{hanging}
\usepackage{etoolbox}
\usepackage[linewidth=0.8pt,skipbelow=0pt]{mdframed}
%\usepackage{layout}
%\usepackage{showframe}
\usepackage{listings}
\usepackage{color}
\usepackage{subcaption}

\makeatletter

\pagestyle{fancy}
\onehalfspacing
\usetikzlibrary{
	shapes,arrows,shapes.symbols,shapes.geometric,shadows,
	chains,calc,positioning,shapes.misc
}

\renewcommand{\headrulewidth}{0pt}
\renewcommand{\chaptermark}[1]{\markboth{#1}{}}
%\renewcommand{\sectionmark}[1]{\markright{#1}{}}

\fancyhf{}
\fancyhead[RO]{\sffamily Phân giải đồng tham chiếu trong bệnh án điện tử}
\fancyhead[LE]{\sffamily\leftmark}
\fancyfoot[RO,LE]{\sffamily\thepage}
\fancypagestyle{plain}{
   \fancyhf{}
}

\captionsetup{format=hang,labelsep=period,labelfont=bf,font=sf}
\captionsetup[table]{position=top,justification=raggedright}

\setlength\parindent{3em}
\newlength\titleindent
\setlength\titleindent{\parindent}
\titleformat{\section}
  {\normalfont\Large\bfseries}
  {\makebox[\titleindent][l]{\thesection}}
  {0pt}
  {}
  
\tikzset{
	base/.style = { draw, on grid, align=center, on chain, inner sep=2pt },
	doc/.style = { base, shape=tape, fill=white, tape bend top=none, minimum height=2.5em, text width=3em },
	multidoc/.style = { doc, double copy shadow={shadow xshift=.5ex, shadow yshift=.5ex} },
	proc/.style = { base, rectangle, text width=4em, minimum height=2em },
	io/.style = { base, trapezium, trapezium left angle=60, trapezium right angle=120, text width=4em, minimum height=2em },
	altproc/.style = { base, rounded rectangle, text width=3.5em, minimum height=2em },	
	wpoint/.style = { circle, draw, minimum size=0.25cm, inner sep=0cm, outer sep=0cm },
	bpoint/.style = { wpoint, fill=black }, 
	swpoint/.style={wpoint,minimum size=0.15cm},
	sbpoint/.style={bpoint,minimum size=0.15cm},
}

\setlength{\headheight}{16.5pt}
\setlength{\footskip}{35pt}

\newcommand{\colleft}{\raggedright}
\newcommand{\colright}{\raggedleft}

\newcommand{\ra}[1]{\renewcommand{\arraystretch}{#1}}
\newcolumntype{L}{>{\raggedright\arraybackslash}X}
\newcolumntype{C}{>{\centering\arraybackslash}X}
\newcolumntype{P}[2]{>{#1\arraybackslash}p{#2\textwidth}}

\newcommand{\rowgroup}[1]{\hspace{-1em}#1}

%\renewcommand{\thealgocf}{}

%\newenvironment{titlepara}[1]
%{	
%	%\addvspace{0.5\baselineskip}
%	\par\noindent\emph{#1}
%	\begin{hangparas}{3em}{0}
%	\setlength{\parskip}{0.5\baselineskip}
%	\let\svpar\par
%	\edef\svparskip{\the\parskip}
%	\def\revertpar{\svpar\setlength\parskip{\svparskip}\let\par\svpar}
%	\def\noparskip{\leavevmode\setlength\parskip{0pt}%
%  		\def\par{\svpar\let\par\revertpar}}
%	\noparskip\par
%}
%{
%	\end{hangparas}
%	\addvspace{0.5\baselineskip}
%	\par\@afterindentfalse\@afterheading
%}

\newcommand*\NoIndentAfterEnv[1]{%
  \AfterEndEnvironment{#1}{\par\@afterindentfalse\@afterheading}}
  
\NoIndentAfterEnv{itemize}
\NoIndentAfterEnv{enumerate}

\setlist[itemize]{leftmargin=\parindent,parsep=0pt,partopsep=0pt}
\setlist[enumerate]{leftmargin=\parindent,parsep=0pt,partopsep=0pt}

\DeclareMathOperator{\BigO}{O}
\DeclareMathOperator{\Kernel}{\mathbf{K}}
\renewcommand{\vec}[1]{\mathbf{#1}}

\newenvironment{newtable}[2]
{
	\begin{table}[t!]
	\centering\ra{#1}
	\caption{#2}
	\footnotesize\sffamily
}
{
	\end{table}
}

\addto\captionsvietnam{
	\renewcommand{\listfigurename}{Danh sách hình}
	\renewcommand{\listtablename}{Danh sách bảng}
	\renewcommand{\bibname}{Tài liệu tham khảo}
}

\newcolumntype{+}{>{\global\let\currentrowstyle\relax}}
\newcolumntype{^}{>{\currentrowstyle}}
\newcommand{\rowstyle}[1]{\gdef\currentrowstyle{#1}%
#1\ignorespaces
}

\newenvironment{rtable}[2][ht]
{
	\@float{table}[#1]
	\renewcommand{\tabcolsep}{3.75pt}
	\centering\ra{1.2}
	\caption{#2}
	\footnotesize\sffamily
	
	\tabular{@{}llllclllclllclll@{}}
	\toprule
	&\multicolumn{3}{l}{\textbf{MUC}}&\phantom{a}&\multicolumn{3}{l}{\textbf{B-CUBED}}&\phantom{a}&\multicolumn{3}{l}{\textbf{CEAF}}&\phantom{a}&\multicolumn{3}{l}{\textbf{Trung bình}}\\
	\cmidrule{2-4} \cmidrule{6-8} \cmidrule{10-12} \cmidrule{14-16}
	& \textbf{P} & \textbf{R} & \textbf{F} && \textbf{P} & \textbf{R} & \textbf{F} && \textbf{P} & \textbf{R} & \textbf{F} && \textbf{P} & \textbf{R} & \textbf{F}\\
	\midrule
}
{
	\bottomrule
	\endtabular
	\end@float
}

\makeatother

\begin{document}

%\SetKw{Null}{null}
%\SetKw{KwOr}{or}
%\SetKw{KwAnd}{and}
%\SetKw{KwBreak}{break}
%\SetKw{KwIn}{in}
%\SetKw{False}{false}
%\SetKw{True}{true}
%\SetKwInOut{Input}{Đầu vào}
%\SetKwInOut{Output}{Đầu ra}
%\SetKw{KwDownTo}{downto}
%\SetKw{KwContinue}{continue}
%\SetAlgorithmName{Giải thuật}{giải thuật}{Danh sách giải thuật}

\chapter*{Lời cam đoan}
Chúng tôi xin cam đoan rằng, ngoại trừ các kết quả tham khảo từ các công trình khác như đã ghi rõ trong luận văn, các nội dung trình bày trong luận văn này là do chính chúng tôi thực hiện và chưa có phần nội dung nào của luận văn này được nộp để lấy bằng cấp ở một trường khác.

\begin{flushright}
Tp.HCM ngày 17 tháng 12 năm 2015
\end{flushright}

\chapter*{Lời cảm ơn}
Trước hết, chúng tôi xin gửi lời cám ơn chân thành nhất đến GS.TS Cao Hoàng Trụ, giảng viên hướng dẫn luận văn và là người thầy gắn bó với chúng tôi trong nhóm nghiên cứu khoa học hơn một năm vừa qua. Chính nhờ những tri thức thầy truyền đạt cùng với sự hướng dẫn tận tình, những góp ý khoa học của thầy đã giúp chúng tôi hoàn thành tốt nhất luận văn.

Chúng tôi cũng gửi lời cảm ơn đến anh Đào Trọng Điệp, anh Đinh Quang Tuấn, anh Huỳnh Minh Huy, những người đã góp ý cho chúng tôi hoàn thành luận văn tốt nghiệp này. Những tài liệu tham khảo quý báu của các anh là nhân tố không thể thiếu giúp chúng tôi vượt qua khó khăn trong quá trình hiện thực luận văn.

Ngoài ra, chúng tôi cảm ơn trung tâm i2b2 đã cung cấp chúng tôi tập dữ liệu tiếng Anh để thực hiện luận văn. Đồng thời, chúng tôi cảm ơn quý Thầy Cô khoa Khoa học và Kỹ thuật Máy tính trường Đại học Bách Khoa, TP.HCM đã tận tình dạy dỗ chúng tôi suốt hơn bốn năm qua.

Cuối cùng, chúng tôi xin cảm ơn gia đình, bạn bè, những người đã hỗ trợ chúng tôi về mặt tinh thần và giúp đỡ chúng tôi trong suốt quá trình thực hiện đề tài này.

\begin{flushright}
Nhóm tác giả
\end{flushright}

\chapter*{Tóm tắt}
Bệnh án điện tử là vấn đề đang nhận được nhiều sự quan tâm nghiên cứu trên thế giới. Các nước phát triển đã đạt nhiều thành tựu trong việc xây dựng và sử dụng bệnh án điện tử vào khám chữa bệnh. Tại Việt Nam, các bệnh viện lớn như Chợ Rẫy, Hùng Vương,... đã bắt đầu sử dụng các hệ thống bệnh án điện tử, tuy nhiên các hệ thống này vẫn còn đơn giản và chưa khai thác hoàn toàn thông tin trong bệnh án. Bài toán phân giải đồng tham chiếu hiện đang được nhiều tổ chức nghiên cứu ứng dụng vào bệnh án điện tử, trong đó có trung tâm i2b2. Vì vậy, chúng tôi thực hiện đề tài \emph{Phân giải đồng tham chiếu trong hồ sơ xuất viện tiếng Anh} nhằm tạo tiền đề cho các nghiên cứu khác trong bệnh án điện tử tại Việt Nam.

Về phương pháp thực hiện, chúng tôi xây dựng hệ thống dựa trên bài báo \emph{A classification approach to coreference in discharge summaries: 2011 i2b2 challenge} của các tác giả Yan Xu et al. Hệ thống được hiện thực sử dụng mô hình cặp khái niệm trong phân giải đồng tham chiếu, kết hợp với phương pháp học máy sử dụng mô hình Support Vector Machine. Kết quả được thử nghiệm trên tập dữ liệu của trung tâm i2b2 bao gồm 251 hồ sơ xuất viện cho tập huấn luyện và 173 hồ sơ xuất viện cho tập kiểm tra. Mô hình của chúng tôi đạt F-measure trung bình là 81.5\% khi tính trên ba độ đo MUC, B-CUBED và CEAF. Kết quả nghiên cứu hứa hẹn việc áp dụng hệ thống vào sử dụng thực tiễn.

\clearpage
\tableofcontents{}

\clearpage
\phantomsection
\addcontentsline{toc}{chapter}{\listfigurename}
\listoffigures

\clearpage
\phantomsection
\addcontentsline{toc}{chapter}{\listtablename}
\listoftables

\chapter{Tổng quan}
\section{Giới thiệu đề tài\label{gioithieudetai}}
Trong hơn mười năm trở lại đây, với sự bùng nổ của kỉ nguyên công nghệ thông tin, việc số hóa dữ liệu trở nên phổ biến hơn bao giờ hết, và bệnh án cũng không phải là ngoại lệ. Bệnh án điện tử đã và đang dần thay thế cho phương pháp ghi chép và lưu trữ truyền thống thông tin của bệnh nhân trong quá trình khám và chữa bệnh. Hầu hết bệnh viện ở những nước phát triển đã triển khai các \emph{hệ thông tin bệnh viện} (HTTBV) để phục vụ cho việc số hóa loại tài liệu này.

%Bệnh án điện tử (BAĐT) là bệnh án được số hóa bằng các công cụ hiện đại. BAĐT chứa đầy đủ các thông tin cơ bản của một bệnh án như: dữ liệu quản lý, dữ liệu cận lâm sàng và lâm sàng. So với bệnh án được lưu trữ bằng giấy, BAĐT có nhiều ưu điểm như: lưu trữ chính xác và đầy đủ thông tin bệnh nhân, hỗ trợ quá trinh tìm kiếm và truy xuất thông tin, dữ liệu có thể được chia sẻ hoặc tích hợp. Qua nhiều năm sử dụng, BAĐT có thể thu thập được một lượng lớn dữ liệu liên quan đến bệnh, triệu chứng và cách điều trị. Lượng dữ liệu này vô cùng quý giá cho việc nghiên cứu y tế. Vì vậy, bên cạnh việc xây dựng các hệ thống BAĐT, khai thác dữ liệu chứa trong BAĐT cũng là một vấn đề quan trọng.

Bên cạnh việc xây dựng bệnh án điện tử (BAĐT) thì việc khai thác nguồn dữ liệu lớn
này cũng đang là một lĩnh vực rất được quan tâm trong những năm gần đây. Nền tảng của việc khai thác này là rút trích và cấu trúc hóa thông tin trong các văn bản thô. Một trong những vấn đề của việc rút trích thông tin là phân giải đồng tham chiếu. Một cách tổng quát, việc phân giải đồng tham chiếu cho một văn bản là xác định liệu hai hay nhiều sự đề cập trong cùng văn bản có ám chỉ tới cùng một sự vật hoặc hiện tượng hay không, từ đó xây dựng các chuỗi đồng tham chiếu. Khi mà đa phần các văn bản được viết tay bằng ngôn ngữ tự nhiên, chứa đựng rất nhiều các khái niệm phụ thuộc vào ngữ cảnh thì việc phân giải đồng tham chiếu này giúp cho máy tính có một cái nhìn mang tính cấu trúc hơn về văn bản, từ đó làm nền tảng cho việc rút trích các kiến thức sâu từ những hiểu biết này.

Tuy vấn đề về phân giải đồng tham chiếu trong những năm gần đây đã được quan tâm nghiên cứu rất nhiều cho các loại văn bản khác (ví dụ các bài báo) thì ở phạm vi BAĐT vấn đề này vẫn còn ít được quan tâm, đặc biệt là ở Việt Nam. Đứng trước nhu cầu đó, chúng tôi quyết định bắt tay vào phát triển một hệ thống phân giải đồng tham chiếu cho các văn bản BAĐT tiếng Anh, làm nền tảng cho các nghiên cứu trên dữ liệu tiếng Việt sau này. 

Đầu vào của hệ thống của chúng tôi phụ thuộc vào đầu ra của bước nhận dạng thực thể trước đó, bao gồm: 
\begin{enumerate}
\item Các văn bản BAĐT ở dạng thô chưa qua xử lý.
\item Ứng với mỗi văn bản BAĐT là một tập tin chứa những khái niệm đã được nhận dạng và gán nhãn xuất hiện trong bệnh án.
\end{enumerate}

Trong đó việc nhận dạng và gán nhãn các khái niệm là việc trích xuất các từ/cụm từ trong văn bản và gán nhãn dựa theo ý nghĩa mà chúng đề cập đến, bao gồm con người, các vấn đề về sức khỏe, các thủ tục y tế hay các phương pháp điều trị. Kết quả đầu ra của hệ thống các chuỗi đồng tham chiếu được phân giải cho một văn bản, trong đó mỗi chuỗi chứa ít nhất hai khái niệm chỉ về cùng một người hay một thực thể y tế như đã nêu.

\section{Mục tiêu và phạm vi đề tài}
Những ích lợi BADT mang lại như đề cập ở trên đã tạo động lực cho chúng tôi tiến hành hiện thực một hệ thống phân giải đồng tham chiếu cho BAĐT. Việc xây dựng thành công hệ thống phân giải đồng tham chiếu góp phần hỗ trợ cho các công trình nghiên cứu sâu hơn về BAĐT cũng như được dùng để xây dựng các công cụ thống kê hoặc các hệ thống truy xuất thông tin trong y tế.

Vì giới hạn thời gian, chúng tôi quyết định chỉ hiện thực hệ thống phân giải đồng tham chiếu cho các hồ sơ xuất viện được viết bằng tiếng Anh với danh sách các thực thể đã được xác định trước. Thách thức i2b2 2011 cung cấp sẵn tập dữ liệu là các báo cáo xuất viện được phân giải đồng tham chiếu sẵn bởi các chuyên gia y tế phù hợp cho quá trình huấn luyện hệ thống. Tập dữ liệu này được trung tâm i2b2 cung cấp miễn phí kèm theo một số cam kết sử dụng dữ liệu (Data Usage Agreement), đây là tập dữ liệu vô cùng giá trị cho viết huấn luyện hệ thống. Dựa trên kết quả Thách thức 2011, vấn đề phân giải đồng tham chiếu trong hồ sơ xuất viện có hai hướng giải quyết đạt kết quả tốt là: hướng tiếp cận \emph{dựa trên luật} (rule-based, hay còn gọi là hướng tiếp cận về ngôn ngữ học) và hướng tiếp cận dựa trên \emph{học máy} (machine learning). Chúng tôi quyết định hiện thực hệ thống dựa trên hệ thống sử dụng học máy có kết quả tốt nhất trong Thách thức 2011 của các tác giả Yan Xu et al. \cite{YanXu2012}. Trong đó, giải thuật học máy được áp dụng là Support Vector Machine và mô hình phân giải đồng tham chiếu được áp dụng là mô hình \emph{cặp khái niệm}.

\section{Cấu trúc luận văn}
Toàn bộ nội dung luận văn được chúng tôi trình bày thành sáu chương. Các chương này nêu lên những kiến thức cần thiết, chi tiết cách thức hiện thực để xây dựng và hoàn thiện hệ thống phân giải đồng tham chiếu trong BAĐT cũng như các kết quả đánh giá. Ở chương cuối, chúng tôi đưa ra các tổng kết về kết quả đạt được của luận án, các hạn chế và các hướng phát triển trong tương lai. Sau đây là nội dung chính của mỗi chương:

\begin{description}[style=nextline,leftmargin=0cm]
\item[Chương 1: Tổng quan] Chương đầu tiên chúng tôi nêu lên mục tiêu, động cơ và phạm vi của luận án. Toàn bộ chương này giúp người đọc có được cái nhìn toàn cảnh về lí do chúng tôi tiến hành thực hiện đề tài, vai trò và vị trí của đề tài trong việc phát triển BAĐT ở Việt Nam cũng như phạm vi hiện thực của đề tài.
\item[Chương 2: Các công trình liên quan] Chương này nêu lên một số công trình liên quan đến phân giải đồng tham chiếu trong bệnh án điện tử, bao gồm giới thiệu chung về bệnh án, các loại văn bản bệnh án, lịch sử số hóa bệnh án và các chuẩn chung cho việc định dạng cấu trúc của bệnh án điện tử. Ngoài ra, chúng tôi cũng trình bày về vấn đề nhận dạng thực thể và một số giải pháp cho nó, vì kết quả của nhận dạng thực thể là đầu vào của hệ thống phân giải đồng tham chiếu mà chúng tôi hiện thực.
\item[Chương 3: Kiến thức nền tảng] Chương này cung cấp các cơ sở lý thuyết về các mô hình, giải thuật và các công cụ được chúng tôi sử dụng để hiện thực hệ thống, bao gồm các mô hình phân giải đồng tham chiếu mà đặc biệt là mô hình cặp khái niệm, giải thuật học Support Vector Machine dùng cho phân loại tính đồng tham chiếu và các công cụ rút trích đặc trưng cho mục đích huấn luyện hay phân loại trong quá trình phân giải.
\item[Chương 4: Hiện thực hệ thống] Trong chương này, chúng tôi trình bày về các chi tiết hiện thực của hệ thống, bao gồm quy trình huấn luyện hệ thống phân loại và quy trình phân giải đồng tham chiếu. Ngoài mô hình phân giải đồng tham chiếu, nội dung của chương còn tập trung vào vấn đề rút trích đặc trưng cho các thực thể trong văn bản thô, bao gồm những đặc trưng cần thiết và cách thức rút trích chúng.
\item[Chương 5: Thí nghiệm đánh giá] Các kết quả so sánh thực nghiệm của hệ thống được chúng tôi trình bày trong chương này. Tập dữ liệu được sử dụng để đánh giá là tập các bệnh án điện tử được cung cấp bởi i2b2 và các phương pháp đánh giá bao gồm MUC, B-CUBED và CEAF, tất cả được trình bày chi tiết trong nội dung của chương. Ngoài ra, cuối chương chúng tôi nêu lên một số giải thích và nhận xét được rút ra từ kết quả đánh giá hệ thống.
\item[Chương 6: Tổng kết] 
Trong chương này, chúng tôi tổng kết lại những kết quả đạt được cũng như những hạn chế của hệ thống, từ đó đề xuất những hướng phát triển mở rộng trong tương lai.
\end{description}

\chapter{Các công trình liên quan}
\section{Bệnh án điện tử}
Bệnh án là văn bản ghi chép các thông tin sức khỏe của một cá nhân trong quá trình khám và chữa bệnh. Bệnh án điện tử chính là bệnh án được số hóa bằng HTTBV. BAĐT thông thường chứa những dữ liệu cơ bản cho quản lý, các dữ liệu cận lâm sàng và lâm sàng của người bệnh trong một lần nằm viện \cite{HoTuBao2015}. Dữ liệu lâm sàng là những \textit{văn bản lâm sàng} (clinical text) do bác sĩ và y tá ghi chép hàng ngày về thông tin khám và chữa bệnh của người bệnh. Các văn bản lâm sàng trong bệnh án điện tử chủ yếu gồm ba loại \cite{HoTuBao2015}:

\begin{enumerate}
\item \emph{Phiếu điều trị} (doctor daily notes): ghi chép các chuẩn đoán, nhận định và y lệnh hàng ngày của bác sĩ về bệnh nhân.
\item \emph{Phiếu chăm sóc} (nurse narratives): là những ghi chép trong ngày của y tá trong quá trình chăm sóc và thực hiện y lệnh của bác sĩ.
\item \emph{Hồ sơ xuất viện} (discharge summary): toàn bộ dữ liệu và thông tin cơ bản của bệnh nhân trong một lần điều trị.
\end{enumerate}

Dữ liệu trong BAĐT thường tồn tại dưới dạng tường thuật, ghi chép của bác sĩ hoặc y tá, đây là dạng dữ liệu không có cấu trúc. Một số thông tin hữu ích có trong dữ liệu không cấu trúc của BAĐT là:

\begin{itemize}
\item Lý do nhập viện, lịch sử điều trị, tiền sử thuốc sử dụng.
\item Ghi chép của bác sĩ hoặc y tá trong quá trình điều trị hàng ngày.
\item Các kết quả xét nghiệm.
\item Các ghi chú về quá trình phẫu thuật.
\end{itemize}

Ngoài các văn bản lâm sàng được lưu trữ dưới dạng phi cấu trúc, một số tiêu chuẩn được đưa ra để nhằm giúp cấu trúc hóa các BAĐT như:

\begin{itemize}
\item \emph{IDC} (International Classification of Diseases): bao gồm các loại mã cũng như thông tin về bệnh như tên bệnh, mô tả, triệu chứng, dấu hiệu hay mức độ, v.v...
\item \textit{CPT} (Current Procedural Terminology): bao gồm các mã mang tính thủ tục trong bệnh viện như mã xét nghiệm, gây tê, phẫu thuật, X quang, thuốc hay cấp cứu, v.v...
\end{itemize}

Tại các nước phát triển, BAĐT đã được quan tâm phát triển trong hơn một thập kỉ qua. Năm 2004, Viện y tế Quốc gia Hoa Kì (NIH: National Institute of Health) đã kêu gọi thành lập một mạng lưới nghiên cứu cấp quốc gia về y sinh. Để đáp lại lời kêu gọi đó, bảy Trung tâm nghiên cứu công nghệ tính toán y sinh (NBCB: National Center for Biomedical Computing) đã được thành lập dưới sự tài trợ của NIH với nhiệm vụ xây dựng cơ sở hạ tầng phục vụ cho việc áp dụng khoa học máy tính vào lĩnh vực y sinh, hỗ trợ cho công việc nghiên cứu. Trong đó, i2b2 (Informatics for Integrating Biology and the Bedside), một NBCB được thành lập bởi sự hợp tác giữa hai trường đại học nổi tiếng là Havard và MIT, bắt đầu từ năm 2006 đã tổ chức các cuộc thi hàng năm nhằm tìm kiếm các phương pháp phân tích và rút trích kiến thức trên dữ liệu BAĐT, gọi tắt là các Thách thức (Challenges). Mỗi Thách thức đưa ra một vấn đề phân tích và một tập dữ liệu BAĐT được cung cấp bởi các bệnh viện trong và ngoài nước Mỹ. Hàng năm có trên dưới 100 nhóm nghiên cứu tham gia đề xuất giải pháp và gửi kết quả phân tích, trong đó những giải pháp tốt được chọn lọc để công bố ở một hội thảo quốc tế và được áp dụng rộng rãi vào các dịch vụ chăm sóc sức khỏe. 

Vào năm 2009, ngay sau khi trở thành tổng thống Hoa Kỳ, Barack Obama đã yêu cầu chuẩn hóa và số hóa mọi bệnh án của các bệnh viện trong vòng 5 năm. Ở Nhật Bản, các bệnh viện lớn và vừa cũng được chính phủ tạo điều kiện để xây dựng BAĐT. Tính đến năm 2011, khoảng 34.7\% bệnh viện lớn và vừa tại Nhật đã có hệ thống BAĐT sử dụng được \cite{HoTuBao2015}. Một số ví dụ thực tiễn trong việc ứng dụng BAĐT vào dự đoán, điều trị bệnh trên thế giới là hệ thống dự đoán nguy cơ mắc bệnh đái tháo đường loại 2 từ cấu trúc gen \cite{AbelKho2012} hoặc hệ thống cho phép nghiên cứu diện rộng bệnh tâm thần và cách điều trị chứng phiền muộn \cite{Perlis2012}.

Tại Việt Nam, các HTTBV cũng đang dần được triển khai, tiêu biểu là Bệnh viện đa khoa Vân Đồn tỉnh Quảng Ninh--cơ quan y tế đầu tiên có trang bị hệ thống bệnh án điện tử hiện đại và hoàn chỉnh với giải pháp MEDI SOLUTIONS của công ty phần mềm Hoa Sen. Cùng với việc xây dựng, tập thể nghiên cứu ``Học máy và ứng dụng'' của viện John von Neumann thuộc đại học Quốc Gia TP Hồ Chí Minh đã tiến hành phát triển các phương pháp và phần mềm phục vụ cho khai thác bệnh án điện tử tiếng Việt.

Nguyên nhân BAĐT nhận được nhiều sự quan tâm như vậy là vì BAĐT không những thuận tiện hơn bệnh án giấy trong việc lưu giữ các thông tin và tri thức thu thập được trong quá trình khám chữa bệnh mà còn cho phép chia sẻ nguồn thông tin đó giữa các bệnh viện, các thành phố hoặc giữa các quốc gia với nhau. Thông qua chia sẻ, nhiều BAĐT được đối chiếu và phân tích để phát hiện những tri thức y học mới hoặc kiểm chứng những kiến thức đã có. BAĐT đóng vai trò quan trọng trong sự phát triển của việc khám chữa bệnh cũng như nghiên cứu trong y học.

\section{Nhận dạng thực thể có tên}
Rút trích thông tin (information extraction), một trong những vấn đề của xử lý ngôn ngữ tự nhiên, là công việc tự động rút trích thông tin từ những dữ liệu không có cấu trúc hoặc dữ liệu bán cấu trúc. Dữ liệu có cấu trúc là dữ liệu máy tính hiểu hoàn toàn, thông thường nằm ở dạng bảng hoặc trong các hệ quản trị dữ liệu quan hệ. Dữ liệu không có cấu trúc là dữ liệu máy tính hoàn toàn không hiểu, như ngôn ngữ tự nhiên. Dữ liệu bán cấu trúc là dữ liệu chứa các thẻ hoặc các hình thức đánh dấu khác giúp phân tách bộ phận ngữ cảnh nền ra khỏi dữ liệu, điển hình là các ngôn ngữ đánh dấu như XML, JSON, HTML.

Tác vụ rút trích thông tin gồm hai bước con là nhận dạng thực thể và rút trích quan hệ. Trong đó, nhận dạng thực thể là bước đầu tiên của bài toán rút trích thông tin. Nhận dạng thực thể, hay nhận dạng thực thể có tên, là xác định các thực thể được đề cập trong văn bản và phân loại chúng vào các lớp khái niệm được định nghĩa sẵn, trong đó khái niệm có tên là các cụm từ có chứa tên của con người, tổ chức hay nơi chốn \cite{KimSang2003}. Ví dụ đầu ra của bước nhận dạng thực thể từ câu văn ``Duy Hưng là sinh viên đại học Bách Khoa của thành phố Hồ Chí Minh'' là:

\begin{itemize}
\item ``Duy Hưng'' - Con người
\item ``đại học Bách Khoa'' - Tổ chức
\item ``thành phố Hồ Chí Minh'' - Nơi chốn
\end{itemize}

Bài toán nhận dạng thực thể có tên thường bao gồm 2 bước: xác định thực thể và phân loại thực thể vào các nhóm ngữ nghĩa \cite{KimSang2003}. Trong đó, bước đầu tiên của bài toán thường được xem đơn giản như là một bài toán phân mảnh các từ trong câu thành các ``tên'', với ``tên'' là một chuỗi các từ liên tục có ý nghĩa và chỉ tới một thực thể có thật.

Các hệ thống nhận diện thực thể có tên nếu hoạt động tốt trong một lĩnh vực chuyên biệt (như y tế, địa chất, ký sự) thì sẽ cho kết quả không tốt nếu đem ứng dụng vào lĩnh vực khác. Việc chỉnh sửa cho một hệ thống có sẵn để hoạt động tốt trong một lĩnh vực mới thường tiêu tốn nhiều công sức.

Tùy theo mỗi lĩnh vực quan tâm cụ thể, các loại thực thể sẽ được định nghĩa khác nhau. Với những vấn đề không đặc thù, những nhóm thực thể thường được nhắc đến là: động vật, con người, tổ chức hay nơi chốn v.v... Khi nghiên cứu về nhận dạng thực thể trong bệnh án điện tử, có năm loại thực thể cần được quan tâm là: vấn đề về sức khỏe (Problem), phương pháp điều trị (Treatment), thủ tục y tế (Test), con người (Person) và đại từ (Pronoun).

Năm 2010, trung tâm i2b2 đưa ra Thách thức về vấn đề xử lý ngôn ngữ tự nhiên cho các văn bản y tế lâm sàng bao gồm ba tác vụ:

\begin{enumerate}
\item Trích xuất và nhận dạng các thực thể có tên trong y học.
\item Phân loại bệnh vào một trong các dạng: đang xảy ra ở hiện tại, không xảy ra ở hiện tại, có thể xảy ra trong tương lai, ...
\item Rút trích các quan hệ giữa các bệnh, phương pháp điều trị và thủ tục y tế.
\end{enumerate}

Đối với Thách thức này, i2b2 tập trung vào giải quyết nhóm bài toán rút trích thông tin vì đây là nhóm bài toán nền tảng, tạo tiền đề để nghiên cứu cho các hướng đi khác. Tuy Thách thức i2b2 năm 2010 có đề cập đến việc rút trích các quan hệ giữa các thực thể trong bệnh án (tác vụ thứ 3), nhưng mối quan hệ đồng tham chiếu lại không được bao gồm trong số đó. Chính vì thế, năm 2011, i2b2 tổ chức Thách thức lần thứ 5 dành riêng cho việc giải quyết vấn đề phân giải đồng tham chiếu trên dữ liệu BAĐT với đầu vào là kết quả nhận diện thực thể từ Thách thức năm 2010. Vấn đề được nêu trong Thách thức i2b2 2011 cũng chính là vấn đề được chúng tôi giải quyết trong nội dung luận án và được trình bày chi tiết trong các phần sau.
\chapter{Kiến thức nền tảng}
\section{Các định nghĩa và thuật ngữ}
Trong các công trình nghiên cứu về phân giải đồng tham chiếu, các tác giả thường sử dụng từ \emph{markable} để chỉ tới những từ/cụm từ cần được phân giải đồng chiếu, hoặc từ \emph{noun phrase} hay NP vì đa phần các công trình trước đây chỉ xem xét tới các danh từ/cụm danh từ. Một số tài liệu khác sử dụng từ \emph{mention} để chỉ tới những ``đề cập'' trong văn bản vì bản chất của phân giải đồng tham chiếu là xác định xem các từ/cụm từ có đề cập tới cùng một thực thể hay không. Để thuận tiện trong việc diễn đạt bằng tiếng Việt, chúng tôi sử dụng từ \emph{khái niệm} để chỉ tới những thực thể cần được phân giải đồng tham chiếu. Một lý do khác mà chúng tôi sử dụng từ này bắt nguồn từ việc Thách thức I2B2 năm 2011 gọi các tập tin chứa những thực thể đã được gán nhãn là ``concept files''.

Các khái niệm đa phần là danh từ hay cụm danh từ. Một khái niệm có thể được lồng trong khái niệm khác. Thông thường sự lồng nhau này xuất hiện ở những cụm danh từ mang ý nghĩa sở hữu, ví dụ cụm ``ngôi nhà của anh ta'' chứa hai khái niệm khác nhau là (ngôi nhà của anh ta) và (anh ta). Một số tài liệu gọi các khái niệm lồng nhau là \emph{khái niệm đầy đủ}. Một hệ thống phân giải đồng tham chiếu có xem xét đến sự lồng nhau này hay không phụ thuộc vào bước trích xuất và gán nhãn thực thể trước đó. Một số công trình nghiên cứu đề xuất các giải pháp phân giải riêng biệt cho các khái niệm lồng nhau, tuy nhiên phương pháp mà chúng tôi hiện thực ở đây không xem sự lồng nhau là một trường hợp đặc biệt.

Hai khái niệm được xem là \emph{đồng tham chiếu} nếu cả hai cùng chỉ về một thực thể trong thế giới thực, ví dụ ``thủ tướng Việt Nam'' và ``Nguyễn Tấn Dũng''. Một đặc điểm cần lưu ý của tính đồng tham chiếu là nó phụ thuộc vào ngữ cảnh và thời điểm mà các khái niệm được đề cập đến. Như ở ví dụ trên, hai khái niệm ``thủ tướng Việt Nam'' và ``Nguyễn Tấn Dũng'' chỉ đồng tham chiếu nếu thời điểm mà chúng được đề cập trong văn bản nằm trong khoảng thời gian mà ông Nguyễn Tấn Dũng đang là thủ tướng Việt Nam.

Có thể xem đồng tham chiếu là một mối quan hệ giữa hai hay nhiều khái niệm khi chúng cùng đề cập tới một thực thể. Dễ dàng nhận thấy đây là mối quan hệ tương đương vì nó có những tính chất sau:
\begin{itemize}[noitemsep]
\item \emph{Tính phản xạ}: một khái niệm bất kì thì luôn đồng tham chiếu với chính nó.
\item \emph{Tính đối xứng}: nếu khái niệm $C_1$ đồng tham chiếu với $C_2$ thì $C_2$ cũng đồng tham chiếu với $C_1$.
\item \emph{Tính bắc cầu}: nếu khái niệm $C_1$ đồng tham chiếu với $C_2$, $C_2$ đồng tham chiếu với $C_3$ thì $C_1$ cũng đồng tham chiếu với $C_3$.
\end{itemize}

Một \emph{cặp khái niệm} gồm hai khái niệm có thể có hoặc không đồng tham chiếu với nhau. Đối với một cặp đồng tham chiếu, khái niệm đứng trước được gọi là \emph{tiền đề} (antecedent), khái niệm đứng sau được gọi là \emph{hồi chỉ} (anaphora). Những đặc trưng của một cặp khái niệm đa phần là những \emph{đặc trưng đồng thuận} (agreement feature), chúng mang giá trị là ``có'' khi cả hai khái niệm của cặp cùng đồng thuận về một đặc tính nào đó, ví dụ như sự đồng thuận về giới tính hay số lượng. Trong một văn bản, nhiều khái niệm có thể cùng tham chiếu tới cùng một thực thể, khi đó chúng tạo thành một chuỗi đồng tham chiếu.

\emph{Phân giải đồng tham chiếu} là công việc xác định những chuỗi đồng tham chiếu trong văn bản. Xuyên suốt này luận văn này, chúng tôi gọi các chuỗi đồng tham chiếu được phân giải bởi con người là các \emph{chuỗi kết quả}, các chuỗi được xuất ra bởi hệ thống phân giải đồng tham chiếu là các \emph{chuỗi hệ thống}.

\section{Các mô hình học máy phân giải đồng tham chiếu}
Phân giải đồng tham chiếu, một trong những tác vụ cơ bản của xử lý ngôn ngữ tự nhiên, là công việc xác định xem những khái niệm nào trong văn bản cùng chỉ đến một thực thể trong thế giới thực. Mặc dù nhiều phương pháp giải quyết đã được nghiên cứu phát triển từ những năm 60 của thế kỉ 20, các hệ thống dựa trên luật hay theo hướng tiếp cận heuristic trong thời gian này đòi hỏi nhiều kiến thức phức tạp và không thực sự hiệu quả (một trong những nền tảng của các hệ thống này là \emph{lý thuyết trung tâm} \cite{Grosz1983}).

Bắt đầu từ những năm 1990, khi các mô hình xác suất trở nên phổ biến vì tính hiệu quả của chúng, các phương pháp học máy phân giải đồng tham chiếu ra đời dần thay thế các phương pháp heuristic thủ công. Cùng với xu thế đó là sự xuất hiện của các tập dữ liệu được gán nhãn bắt nguồn từ hội nghị MUC-6 (1996) và MUC-7 (1997), các nghiên cứu về phân giải đồng tham chiếu dựa trên học máy càng được phát triển không chỉ cho các miền văn bản chung mà còn đi sâu vào các miền văn bản cụ thể (như Thách thức i2b2 năm 2011 về phân giải đồng tham chiếu cho bệnh án điện tử \cite{OzlemUzuner2012}).

Mặc dù tính bổ biến và sự hiệu quả của các phương pháp học máy có giám sát, sự khó khăn trong việc xây dựng các tập dữ liệu gán nhãn nhất là đối với các ngôn ngữ khác tiếng Anh đã làm động lực cho sự ra đời của các phương pháp học máy bán giám sát hoặc không giám sát. Ưu điểm của các phương pháp này là không cần đòi hỏi các tập dữ liệu đã gán nhãn hoặc chỉ cần gán nhãn một phần \cite{CardieWagstaff1999}.

Có ba mô hình học máy có giám sát được đề xuất để phân giải đồng tham chiếu: mô hình \emph{cặp khái niệm}, mô hình \emph{đề cập thực thể} và mô hình \emph{xếp hạng}. Mô hình cặp khái niệm ra đời đầu tiên và có ý tưởng đơn giản nhất, tuy nhiên cũng chính vì thế mà nó có một số nhược điểm khiến cho việc hiện thực không thực sự dễ dàng. Hai mô hình đề cập thực thể và xếp hạng được đề xuất sau đó nhằm khắc phục các nhược điểm của mô hình cặp khái niệm bằng cách đưa vào các ý tưởng thực tế hơn như \emph{đặc trưng ở mức cụm} \cite{Yang2004} hay huấn luyện một mô hình xếp hạng lựa chọn tiền đề \cite{Yang2003}.

Mặc dù có nhiều nhược điểm, mô hình cặp thực thể vẫn rất phổ biến nhờ ý tưởng đơn giản của nó, đặc biệt là trong các miền văn bản chuyển môn khi mà các đặc trưng chuyên sâu về một hay một vài lĩnh vực mang nhiều ý nghĩa hơn cho việc xác định tính đồng tham chiếu của các khái niệm. Điển hình là hệ thống có kết quả cao nhất trong Thử thách i2b2 năm 2011 sử dụng mô hình cặp khái niệm và tận dụng các thông tin về sự cập đến bệnh nhân, các kiến thức nền (Wikipedia, WordNet, v.v...) và các thông tin ngữ cảnh trong bệnh án \cite{YanXu2012}. Trong phần này chúng tôi chỉ trình bày cụ thể về mô hình cặp khái niệm vì đây cũng là mô hình được chúng tôi hiện thực trong hệ thống của mình.

\subsection*{Mô hình cặp khái niệm}
Đây là mô hình học máy phân giải đồng tham chiếu đầu tiên và được giới thiệu vào năm 1995 \cite{Aone&Bennett1995}. Ý tưởng chính của mô hình này là xác định tính đồng tham chiếu cho từng cặp hai khái niệm bất kì trong văn bản. Mặc dù mô hình cặp khái niệm phổ biến nhờ ý tưởng đơn giản của nó, tính bắc cầu của quan hệ đồng tham chiếu bị bỏ qua ở mô hình này vì trường hợp như sau có thể xảy ra: hai khái niệm $A$ và $B$ được xác định là đồng tham chiếu, $B$ và $C$ cũng được xác định là đồng tham chiếu trong khi $A$ và $C$ lại không được xác định là đồng tham chiếu. Điều này đòi hỏi phải có một cơ chế gom cụm các khái niệm và xây dựng các chuỗi sau khi đã xác định tính đồng tham chiếu cho các cặp hai khái niệm.

Một nhược điểm khác của mô hình này là để huấn luyện mô hình phân loại, ứng với mỗi cặp khái niệm hệ thống cần trích xuất các đặc trưng đại diện cho cặp, sau đó tổng hợp lại thành một tập dùng để huấn luyện. Mặt khác trong một văn bản, tổng số các cặp khái niệm có thể rất lớn trong khi số các cặp đồng tham chiếu lại rất nhỏ, như vậy nếu tất cả các cặp được rút trích đặc trưng sẽ dẫn đến tình trạng mất cân bằng lớp: các mẫu âm (các cặp không đồng tham chiếu) có số lượng lấn át các mẫu dương (các cặp đồng tham chiếu).

Một số công trình nghiên cứu đề xuất giải pháp cho vấn đề mất cân bằng lớp theo hướng tiếp cận heuristic. Chẳng hạn như chỉ xem xét sinh các mẫu dương từ các cặp đồng tham chiếu mà hai khái niệm gần kề nhau theo thứ tự xuất hiện, còn các mẫu âm được sinh theo quy tắc: với mỗi cặp đồng tham chiếu được sinh làm mẫu dương $(C_i,\,C_j)$, sinh các mẫu âm từ các cặp $(C_k,\,C_j)$ mà $i<k<j$ \cite{Soon2001}. Một chỉnh sửa nhỏ của cách này là loại bỏ đi các mẫu mà khái niệm đứng trước là một đại từ \cite{VincentNg2002a}. Hay một cách lọc mẫu khác được trình bày ở \cite{VincentNg2002b}: với mỗi hồi chỉ $C_j$ mà tiền xa nhất của nó là $C_i$, sinh tất cả các mẫu âm được tạo bởi $C_k$ và $C_j$ mà $i<k<j$. Ngoài ra, hệ thống ở \cite{VincentNg2002b} còn sử dụng luật quy nạp (rule induction) để lọc bỏ các mẫu dương khó làm ảnh hưởng đến việc huấn luyện.

Để xác định tính đồng tham chiếu cho hai khái niệm bất kì trong văn bản, hệ thống sử dụng mô hình cặp khái niệm trước tiên cần huấn luyện một mô hình phân loại. Một số giải thuật như cây quyết định hay luật quy nạp, v.v... được sử dụng ở thời kì đầu của ứng dụng học máy. Sau khi các mô hình thống kê trở nên phổ biến, một số giải thuật được sử dụng vào lĩnh vực này gồm có mô hình entropy cực đại \cite{Berger1996}, mạng neuron bầu cử \cite{Freund1999} và support vector machine. Các giải thuật học máy dựa trên mô hình thống kê có một đặc điểm lợi thế là chúng có thể xuất ra độ tin cậy đồng tham chiếu của các cặp khái niệm. 

Mô hình phân loại sau khi đã huấn luyện có thể được sử dụng để xác định tính đồng tham chiếu cho các cặp hai khái niệm. Tuy nhiên để có thể xây dựng các chuỗi đồng tham chiếu, cần thiết phải sử dụng một giải thuật gom cụm. Có hai giải thuật gom cụm được sử dụng phổ biến trong phân giải đồng tham chiếu:

\begin{enumerate}
\item \emph{Gom cụm gần nhất trước (closest-first clustering):}

Giải thuật này trước tiên chọn ra tiền đề ở gần nhất trước đó cho một hồi chỉ. Cụ thể với mỗi khái niệm $C_j$, giải thuật sử dụng mô hình phân loại đã được huấn luyện ở bước trước để xác định tính đồng tham chiếu cho từng cặp $(C_i,\,C_j)$ mà $i<j$. Sau khi gặp một cặp $(C_k,\,C_j)$ đầu tiên được xác định là đồng tham chiếu, giải thuật đưa cặp vào danh sách $L$. Sau khi đã chọn tiền cho toàn bộ hồi chỉ, từ danh sách $L$ các cặp có một chung khái niệm được gom lại với nhau thành từng cụm, mỗi cụm ứng với một chuỗi đồng tham chiếu \cite{Soon2001}.
\item \emph{Gom cụm tốt nhất trước (best-first clustering):}

Sự khác biệt của giải thuật này so với giải thuật trên là: thay vì chọn tiền đề ở gần nhất, giải thuật gom cụm tốt nhất trước lựa chọn tiền đề có độ tin cậy đồng tham chiếu cao nhất. Điều này có thể được thực hiện nhờ sử dụng các mô hình phân loại thống kê, điển hình là SVM. Theo các tác giả của giải thuật này, nhờ tận dụng được độ tin cậy đồng tham chiếu của mô hình phân loại, giải thuật gom cụm tốt nhất trước cho kết quả tốt hơn giải thuật gom cụm gần nhất trước \cite{VincentNg2002a}.
\end{enumerate}

\section{Support Vector Machine}
Trong lĩnh vực học máy, support vector machine (SVM) là mô hình học có giám với giải thuật huấn luyện có thể phân tích và nhận diện mẫu. Lần đầu tiên được giới thiệu bởi Vladimir N. Vapnik vào năm 1992, SVM là một trong những giải thuật học máy được sử dụng phổ biến nhất trong lĩnh vực học máy hiện đại, đa phần là bởi vì SVM thường tỏ ra hiệu quả hơn nhiều so với các giải thuật học máy khác khi được huấn luyện trên một tập dữ liệu vừa (SVM không làm việc tốt trên các tập dữ liệu quá lớn, vì giải thuật học của SVM có liên quan đến việc nghịch đảo các ma trận, một việc làm rất tốn kém về mặt tính toán). 

Tư tưởng cơ bản của SVM là cố gắng tìm kiếm một hoặc một tập các mặt siêu phẳng (hyperplane) trong một không gian có nhiều chiều để phân chia các điểm đại diện của tập dữ liệu ra hai hay nhiều phần, từ đó có thể thực hiện các tác vụ phân loại hay hồi quy. Một đặc điểm của SVM là giải thuật học của nó cho phép chúng ta phân biệt được giữa một mô hình phân loại tốt và một mô hình không tốt, ngay cả khi cả hai mô hình đều cho cùng một kết quả phân loại trên một tập dữ liệu.

Hình \ref{sep-eg} mô tả ba đường phân loại có thể cho một tập dữ liệu. Mặc dù cả ba đường đều phân loại các điểm thuộc các lớp khác nhau một cách chính xác, có thể cảm thấy rằng hình thứ ba cho một đường phân loại tốt hơn hai hình còn lại. Điều này có thể được giải thích là: khi sử dụng các đường ở Hình \ref{sep-eg} để phân loại cho các điểm dữ liệu mới, đường phân loại thứ ba cho xác suất các điểm này rơi vào phía phân lớp sai thấp hơn so với hai đường còn lại. Lý do là bởi vì khoảng cách từ đường phân loại thứ ba đến ba đến các điểm dữ liệu của hai lớp là lớn nhất và gần như là bằng nhau cho cả hai bên.

\begin{figure}[ht]
\centering
\begin{subfigure}[b]{0.32\textwidth}
\centering
\resizebox{\linewidth}{!}{
	\begin{tikzpicture}
		\draw[->] (-0.1,0) -- (4.1,0) node[below]{$x_1$};
		\draw[->] (0,-0.1) -- (0,4.1) node[left]{$x_2$};
		
		\draw (0.5,2.5) node[bpoint]{};
		\draw (1.5,2.85) node[bpoint]{};
		\draw (2.5,3.2) node[bpoint]{};
		\draw (0.7,3.7) node[bpoint]{};
		\draw (0.8,3.1) node[bpoint]{};
		\draw (1.3,3.8) node[bpoint]{};
		\draw (1.8,3.4) node[bpoint]{};
		
		\draw (1.2,1.5) node[wpoint]{};
		\draw (1.8,1.1) node[wpoint]{};
		\draw (2,1.8) node[wpoint]{};
		\draw (1.6,0.6) node[wpoint]{};
		\draw (2.3,0.8) node[wpoint]{};
		\draw (2.4,1.4) node[wpoint]{};
		\draw (2.8,1.7) node[wpoint]{};
		
		\draw (-0.5375,2.55) -- (3.975,1.6);
	\end{tikzpicture}
}
\end{subfigure}
\begin{subfigure}[b]{0.32\textwidth}
\centering
\resizebox{\linewidth}{!}{
	\begin{tikzpicture}
		\draw[->] (-0.1,0) -- (4.1,0) node[below]{$x_1$};
		\draw[->] (0,-0.1) -- (0,4.1) node[left]{$x_2$};
		
		\draw (0.5,2.5) node[bpoint]{};
		\draw (1.5,2.85) node[bpoint]{};
		\draw (2.5,3.2) node[bpoint]{};
		\draw (0.7,3.7) node[bpoint]{};
		\draw (0.8,3.1) node[bpoint]{};
		\draw (1.3,3.8) node[bpoint]{};
		\draw (1.8,3.4) node[bpoint]{};
		
		\draw (1.2,1.5) node[wpoint]{};
		\draw (1.8,1.1) node[wpoint]{};
		\draw (2,1.8) node[wpoint]{};
		\draw (1.6,0.6) node[wpoint]{};
		\draw (2.3,0.8) node[wpoint]{};
		\draw (2.4,1.4) node[wpoint]{};
		\draw (2.8,1.7) node[wpoint]{};		
		
		\draw (-0.5,0) -- (3.5,4);
	\end{tikzpicture}
}
\end{subfigure}
\begin{subfigure}[b]{0.32\textwidth}
\centering
\resizebox{\linewidth}{!}{
	\begin{tikzpicture}
		\draw[->] (-0.1,0) -- (4.1,0) node[below]{$x_1$};
		\draw[->] (0,-0.1) -- (0,4.1) node[left]{$x_2$};
		
		\draw (0.5,2.5) node[bpoint]{};
		\draw (1.5,2.85) node[bpoint]{};
		\draw (2.5,3.2) node[bpoint]{};
		\draw (0.7,3.7) node[bpoint]{};
		\draw (0.8,3.1) node[bpoint]{};
		\draw (1.3,3.8) node[bpoint]{};
		\draw (1.8,3.4) node[bpoint]{};
		
		\draw (1.2,1.5) node[wpoint]{};
		\draw (1.8,1.1) node[wpoint]{};
		\draw (2,1.8) node[wpoint]{};
		\draw (1.6,0.6) node[wpoint]{};
		\draw (2.3,0.8) node[wpoint]{};
		\draw (2.4,1.4) node[wpoint]{};
		\draw (2.8,1.7) node[wpoint]{};	
		
		\draw (-0.5,1.5375) -- (4,3.1125);
	\end{tikzpicture}
}
\end{subfigure}
\caption{Ví dụ về các đường phân loại có thể cho một tập dữ liệu\label{sep-eg}}
\end{figure}

Như vậy mục đích của giải thuật học SVM chính tìm ra một đường phân loại tối ưu theo tiêu chí đã kể trên, tức là cực đại hóa khoảng cách của đường phân loại tới điểm gần nhất của hai lớp. Các điểm gần với đường phân loại nhất được gọi là các \emph{vector hỗ trợ} (support vector), các điểm này là quan trọng bởi vì chúng là những điểm dễ bị phân loại sai nhất trong quá trình học. Điều này dẫn đến một đặc điểm thú vị của giải thuật học SVM: sau khi huấn luyện, các điểm dữ liệu khác của tập huấn luyện có thể bỏ đi và chỉ cần giữ lại các vector hỗ trợ là đủ.

Giả sử tập dữ diệu huấn luyện được biểu diễn bằng một tập các điểm \[\mathcal{D}=\left\{(\vec{x}_i,y_i)\,|\,\vec{x}_i\in\mathbb{R}^p,\,y_i\in\{-1,1\}\right\}_{i=1}^{n}\] trong đó $y_i$ có thể là $-1$ hoặc $1$ mang ý nghĩa là lớp mà $\vec{x}_i$ thuộc vào, mỗi $\vec{x}_i$ là một vector có $p$ chiều. Một mặt siêu phẳng bất kì trong không gian $p$ chiều có thể được biểu diễn dưới dạng $\vec{w}\cdot\vec{x}+b=0$ với $\vec{w}\cdot\vec{x}$ là tích vô hướng của hai vector $\vec{w}$ và $\vec{x}$, tức $\vec{w}\cdot\vec{x}=\sum_{i=1}^{p}w_ix_i$. Trong đó $\vec{w}$ được gọi là vector pháp tuyến của mặt siêu phẳng $\vec{w}\cdot\vec{x}+b=0$, còn $\frac{b}{||\vec{w}||}$ là khoảng cách từ đường đến góc tọa độ.

Với mỗi đường phân loại bất kì, xét hai đường song song với nó và đi qua các điểm gần nhất của hai lớp (Hình \ref{optimal-sep}), gọi là hai đường biên. Để cho thuận tiện, gọi các điểm đen $\vec{x}^\bullet$ trên Hình \ref{optimal-sep} là các điểm thuộc vào lớp 1, còn các điểm trắng $\vec{x}^\circ$ là các điểm thuộc vào lớp $-1$. Như vậy nếu đường phân loại có dạng $\vec{w}\cdot\vec{x}+b=0$ thì các điểm đen và trắng được phân loại theo tiêu chí $\vec{w}\cdot\vec{x}^\bullet+b\geq1$ và $\vec{w}\cdot\vec{x}^\circ+b\leq-1$. Dễ dàng suy ra phương trình của đường biên đen là $\vec{w}\cdot\vec{x}+b=1$ và của đường biên trắng là $\vec{w}\cdot\vec{x}+b=-1$, từ đó tính được khoảng cách của hai đường biên là $\frac{2}{||\vec{w}||}$.

Như đã nói ở trên, các điểm đen có phân lớp $y=1$, tiêu chí phân loại các điểm này là $\vec{w}\cdot\vec{x}^\bullet+b\geq1$. Trong khi đó các điểm trắng có phân lớp $y=-1$ được phân loại theo tiêu chí $\vec{w}\cdot\vec{x}^\circ+b\leq-1$. Các tiêu chí này có thể được viết lại một cách đơn giản là: $y(\vec{w}\cdot\vec{x}+b)\geq1$. Mặt khác, để cực đại hóa khoảng cách của hai đường biên, tức cực đại hóa $\frac{2}{||\vec{w}||}$, thì $||\vec{w}||$ phải đạt cực tiểu. Như vậy mục đích cuối cùng của giải thuật học SVM là cực tiểu hóa $||\vec{w}||$ với điều kiện $y_i(\vec{w}\cdot\vec{x}_i+b)\geq1$ cho mọi $i=1,\dots,n$.

\begin{figure}[ht]
\centering
\scalebox{1.5}{
	\begin{tikzpicture}[font=\tiny]
		\draw[->] (-0.1, 0) -- (4.1, 0) node[below]{$x_1$};
		\draw[->] (0, -0.1) -- (0, 4.1) node[left]{$x_2$};
			
		\draw[thick,gray!60] (0.5, 2.5) node[bpoint]{};
		\draw[thick,gray!60] (1.5, 2.85) node[bpoint]{};
		\draw[thick,gray!60] (2.5, 3.2) node[bpoint]{};
		\draw (0.7, 3.7) node[bpoint]{};
		\draw (0.8, 3.1) node[bpoint]{};
		\draw (1.3, 3.8) node[bpoint]{};
		\draw (1.8, 3.4) node[bpoint]{};
		
		\draw[thick] (1.2, 1.5) node[wpoint]{};
		\draw (1.8, 1.1) node[wpoint]{};
		\draw[thick] (2, 1.8) node[wpoint]{};
		\draw (1.6, 0.6) node[wpoint]{};
		\draw (2.3, 0.8) node[wpoint]{};
		\draw (2.4, 1.4) node[wpoint]{};
		\draw (2.8, 1.7) node[wpoint]{};	
			
		\draw (-0.55,1.52) -- (3.809,3.0457) node[pos=0.8,sloped,above,outer sep=0,inner sep=0.5mm]{$\vec{w}\cdot\vec{x}+b=0$};
		
		\draw[dashed] (-0.68,2.087) -- (3.618,3.5913) node[pos=1,sloped,above,outer sep=0,inner sep=0.5mm]{$\vec{w}\cdot\vec{x}+b=1$};
		\draw[dashed] (-0.3,0.9825) -- (4.004,2.4889) node[pos=0.8,sloped,above,outer sep=0,inner sep=0.5mm]{$\vec{w}\cdot\vec{x}+b=-1$};
		
		\draw[<->] (4.0707,2.5122) -- (3.6859,3.6147) node[midway,above,sloped]{$\frac{2}{||\vec{w}||}$};
		\draw[<->] (-0.0668,-0.0234) -- (-0.6001,1.5022) node[midway,below,sloped]{$\frac{b}{||\vec{w}||}$};
	\end{tikzpicture}
}
\caption{Tối ưu hóa khoảng cách của đường phân loại\label{optimal-sep}}
\end{figure}

\subsection*{Đường biên mềm}
Như đã đề cập ở trên, bài toán huấn luyện SVM mang bản chất là bài toán tối ưu tuyến tính với hàm mục tiêu (objective function)
\begin{equation}\label{svm-hard-obj-func}
L(\vec{w})=\vec{w}\cdot\vec{w}
\end{equation}
và các ràng buộc
\begin{equation}\label{svm-hard-constraints}
y_i(\vec{w}\cdot\vec{x}_i+b)\geq1\,,\forall i=1,\dots,n
\end{equation}

Tuy nhiên nhiều bài toán phân loại sẽ không giải quyết được với những điều kiện chặt như vậy, tức tìm kiếm một mặt siêu phẳng có thể phân chia các điểm dữ liệu của hai lớp ra hai bên một cách hoàn hảo. Một ví dụ điển hình là khi tập dữ liệu xuất hiện nhiễu (noise).

Năm 1995, Corinna Cortes và Vladimir N. Vapnik đề xuất chỉnh sửa ý tưởng học SVM chặt để cho phép phân loại sai một số điểm dữ liệu, gọi là phương pháp \emph{đường biên mềm} (soft margin). Nếu không thể tìm ra một mặt siêu phẳng hoàn hảo, phương pháp đường biên mềm sẽ cố gắng tìm một mặt siêu phẳng nào đó có thể phân chia các điểm dữ liệu một cách càng gọn càng tốt, trong khi vẫn có thể duy trì khoảng cách cực đại từ nó đến các điểm phân loại đúng gần nhất. Để làm được điều đó, phương pháp này giới thiệu một biến mới $e_i$ vào trong ràng buộc ở (\ref{svm-hard-constraints}), gọi là \emph{biến lỏng} (slack variable):
\begin{equation}
y_i(\vec{w}\cdot\vec{x}_i+b)\geq1-e_i\,,\forall i=i,\dots,n
\end{equation}

Với chỉnh sửa như trên, hàm mục tiêu (\ref{svm-hard-obj-func}) trở thành:
\begin{equation}\label{svm-soft-obj-func}
L(\vec{w},\vec{e})=\vec{w}\cdot\vec{w}+C\sum_{i=1}^{n}e_i
\end{equation}
trong đó, $C$ là tham số đánh đổi (trade-off) giữa khoảng cách biên lớn và số điểm phân loại sai nhỏ. Đa phần các giải thuật SVM được hiện thực hiện nay đều sử dụng phương pháp đường biên mềm vì tính linh hoạt của nó. Để giải quyết bài toán tối ưu cho phương pháp đường biên mềm, khi mà hàm mục tiêu là một hàm hai biến, một kĩ thuật gọi là \emph{quy hoạch toàn phương} (quadratic programming) được giới thiệu để loại bỏ $\vec{w}$ ra khỏi phương trình (\ref{svm-soft-obj-func}) và biến đổi nó trở thành:
\begin{equation}\label{svm-soft-obj-func-modified}
L(\vec{a})=\sum_{i=1}^{n}a_i-\frac{1}{2}\sum_{i=1}^{n}\sum_{j=1}^{n}a_ia_jt_it_j\vec{x}_i\cdot\vec{x}_j
\end{equation}
với các ràng buộc $0\leq a\leq C$ và $\sum_{i=1}^{n}a_i\vec{x}_k=0$.

\subsection*{Kĩ thuật kernel}
Trong nhiều trường hợp, các điểm đại diện của tập dữ liệu không thể nào được phân cách bằng một mặt siêu phẳng trong chiều không gian gốc của nó. Khi này, một kĩ thuật biến đổi dữ liệu được áp dụng để nâng số chiều của không gian dữ liệu lên, với mục đích là tìm kiếm mặt phân loại siêu phẳng tối ưu trong các chiều không gian mới. Hình \ref{kernel-trick} mô tả một tập dữ liệu ở hai chiều được biến đổi sang một không gian ba chiều. Có thể dễ dàng nhận thấy, trong không gian hai chiều, các điểm dữ liệu của hai lớp không thể được phân cách bằng một đường thẳng. Tuy nhiên sau khi biến đổi dữ liệu để số chiều tăng lên, chúng dễ dàng được phân cách bởi một mặt phẳng.

\begin{figure}[ht]
\begin{subfigure}[b]{0.49\textwidth}
\centering
\begin{tikzpicture}
	\draw[->] (-2.5, 0) -- (2.5, 0) node[below]{$x_1$};
	\draw[->] (0, -2.5) -- (0, 2.5) node[left]{$x_2$};
	
	\draw[dashed] (0,0) node[ellipse,draw,minimum width=3cm,minimum height=2cm]{};
	
	\draw (0.1,0.15) node[swpoint]{};
	\draw (0.5,0.65) node[swpoint]{};
	\draw (1,0.25) node[swpoint]{};
	\draw (0.9,-0.45) node[swpoint]{};
	\draw (0.3,-0.25) node[swpoint]{};
	\draw (-0.45,-0.55) node[swpoint]{};
	\draw (-0.45,0.5) node[swpoint]{};
	\draw (-1.2,0.15) node[swpoint]{};
	
	\draw (0.6,1.5) node[sbpoint]{};
	\draw (1.7,0.6875) node[sbpoint]{};
	\draw (1.803,1.845) node[sbpoint]{};
	\draw (1.1,1.1) node[sbpoint]{};
	\draw (0.9,1.92) node[sbpoint]{};
	
	\draw (1.5,-0.7) node[sbpoint]{};
	\draw (2.2,-0.9) node[sbpoint]{};
	\draw (2,-1.85) node[sbpoint]{};
	\draw (1.1,-1.3) node[sbpoint]{};
	\draw (0.7,-2) node[sbpoint]{};
	
	\draw (-0.5,-1.85) node[sbpoint]{};
	\draw (-1.35,-1.9) node[sbpoint]{};
	\draw (-0.7,-1.2) node[sbpoint]{};
	\draw (-1.4,-0.8) node[sbpoint]{};
	\draw (-2,-0.5) node[sbpoint]{};
	
	\draw (-2.2,0.8) node[sbpoint]{};
	\draw (-1.45,0.7) node[sbpoint]{};
	\draw (-1,1.4) node[sbpoint]{};
	\draw (-0.9,2.2) node[sbpoint]{};
	\draw (-2,1.9) node[sbpoint]{};
\end{tikzpicture}
\end{subfigure}
\begin{subfigure}[b]{0.49\textwidth}
\centering
\begin{tikzpicture}
	\draw[-,dashed] (0,0) -- (1.5,0);
	\draw[->] (1.5,0) -- (3.8,0) node[below]{$z_1$};
	\draw[-,dashed] (0,0) -- (0,2.5);
	\draw[->] (0,2.5) -- (0,3.8) node[left]{$z_3$};
	\draw[dashed] (0,0) -- (0,0,0.8);
	\draw[->] (0,0,0.8) -- (0,0,3.8) node[below right]{$z_2$};
	
	\draw (1.5,0) -- (0,2.5);
	\draw (1.5,0) -- (1.5,0,3.5) -- (0,2.5,3.5) -- (0,2.5);
	
	\draw (0.3,1.75) node[swpoint]{};
	\draw (0.35,1.4) node[swpoint]{};
	\draw (0.7,1) node[swpoint]{};
	\draw (0.5,0.7) node[swpoint]{};
	\draw (0.8,0.6) node[swpoint]{};
	\draw (0.7,0.35) node[swpoint]{};
	\draw (0.2,0.35) node[swpoint]{};
	\draw (1.1,0.2) node[swpoint]{};
	
	\draw (0.4,2.2) node[sbpoint]{};
	\draw (0.8,2.55) node[sbpoint]{};
	\draw (0.75,3.2) node[sbpoint]{};
	\draw (1,1.85) node[sbpoint]{};
	\draw (1.15,1.2) node[sbpoint]{};
	\draw (1.45,0.7) node[sbpoint]{};
	\draw (2,0.5) node[sbpoint]{};
	\draw (1.3,2.9) node[sbpoint]{};
	\draw (1.55,2) node[sbpoint]{};
	\draw (1.95,2.45) node[sbpoint]{};
	\draw (2.1,3.3) node[sbpoint]{};
	\draw (2.45,2.25) node[sbpoint]{};
	\draw (2.75,1.35) node[sbpoint]{};
\end{tikzpicture}
\end{subfigure}
\caption{Minh họa kĩ thuật kernel giúp biến đổi không gian dữ liệu\label{kernel-trick}}
\end{figure}

Để giúp SVM có thể tìm kiếm các mặt siêu phẳng trong các chiều không gian cao hơn, một kĩ thuật được giới thiệu có tên là kernel giúp làm tăng số chiều của dữ liệu mà không làm mất đi bản chất của nó. Cơ sở của kĩ thuật này là khi thay $\vec{x}$ trong phương trình (\ref{svm-soft-obj-func-modified}) bằng một vector $\Phi(\vec{x})$ có nhiều số chiều hơn được biến đổi từ $\vec{x}$, tích $\Phi(\vec{x}_j)\cdot\Phi(\vec{x}_j)$ trong phương trình sau một số phép biến đổi sẽ trở thành một hàm kernel đối với $\vec{x}_i$ và $\vec{x}_j$, $\Kernel(\vec{x}_i,\vec{x}_j)$. Vì số chiều của $\Phi(\vec{x})$ có thể là rất lớn, phép biến đổi kernel giúp cho giải thuật có tính khả thi về mặt tính toán. Cụ thể là thay vì sử dụng trực tiếp các $\Phi(\vec{x})$, ta có thể thay thế tích $\vec{x}_i\cdot\vec{x}_j$ bằng ngay chính hàm kernel $\Kernel(\vec{x}_i,\vec{x}_j)$. Sau đây là một số hàm kernel thông dụng:

\begin{itemize}
\item hàm đa thức (polynomial) biến đổi số chiều lên một mức độ $s$ nào đó:
\[\Kernel(\vec{x},\vec{y})=(1+\vec{x}\cdot\vec{y})^s\]
\item hàm xichma (sigmoid function) với tham số $\kappa$ và $\delta$:
\[\Kernel(\vec{x},\vec{y})=\tanh(\kappa \vec{x}\cdot \vec{y}-\delta)\]
\item một số dạng hàm bán kính cơ sở (radial basis function) như:
\begin{itemize}
\item kernel Gauss: $\Kernel(\vec{x},\vec{y})=\exp\left(-\dfrac{||\vec{x}-\vec{y}||^2}{2\sigma^2}\right)$
\item kernel mũ: $\Kernel(\vec{x},\vec{y})=\exp\left(-\dfrac{||\vec{x}-\vec{y}||}{2\sigma^2}\right)$
\item kernel Laplace: $\Kernel(\vec{x},\vec{y})=\exp\left(-\dfrac{||\vec{x}-\vec{y}||}{\sigma}\right)$
\end{itemize}
\end{itemize}

Việc tìm kiếm các bộ tham số tốt nhất cho hàm kernel có thể được thực hiện bằng nhiều cách, trong đó cách tìm kiếm lưới (grid search) được cho là đơn giản nhất và có thể song song hóa được. Ví dụ để huấn luyện một mô hình SVM với kernel Gauss, có hai tham số cần được xác định là $C$ (tham số đánh đổi ở phương trình (\ref{svm-soft-obj-func})) và $\gamma=\frac{1}{2\sigma^2}$ là tham số của kernel. Một cách tìm kiếm lưới thường được thực hiện là duyệt toàn bộ các cặp $(C,\gamma)$ với $C\in\{2^{-5},2^{-3},\dots,2^{13},2^{15}\}$ và $\gamma\in\{2^{-15},2^{-13},\dots,2^1,2^3\}$ để tìm ra bộ $(C,\gamma)$ tối ưu nhất dựa trên một phép kiểm chứng nào đó, ví dụ phép kiểm chứng chéo $k$ mẫu ($k$-fold cross validation).

\subsection*{Mô hình SVM mở rộng}
Mô hình SVM về bản chất chỉ phân loại được cho hai lớp, tuy nhiên trong thực tế có nhiều bài toán mà dữ liệu được phân vào nhiều hơn hai lớp. Để giải quyết các bài toán dạng này, một số phương pháp được đề xuất để mở rộng SVM cho các bài toán phân loại nhiều lớp. Tư tưởng cơ bản của các phương pháp này là thay vì chỉ huấn luyện một mô hình, giải thuật học mở rộng huấn luyện nhiều mô hình SVM phân loại hai lớp, sau đó sử dụng biểu quyết (vote) để quyết định phân lớp của dữ liệu. Có hai phương pháp thông dụng để mở rộng SVM cho phân loại nhiều lớp:
\begin{enumerate}
\item \emph{Một đấu với tất cả (one-against-all):}

Huấn luyện tất cả $n$ mô hình ứng với $n$ lớp. Mô hình thứ $i$ được huấn luyện để phân loại dữ liệu vào lớp $i$ hoặc lớp khác $i$. Như vậy ứng với $n$ mô hình là $n$ phương trình mặt siêu phẳng:
\[
\begin{matrix}
\vec{w}_1\cdot\vec{x}+b_1\\
\vdots\\
\vec{w}_n\cdot\vec{x}+b_n
\end{matrix}
\]

Sau khi huấn luyện, với mỗi điểm dữ liệu cần phân loại $\vec{x}^*$, giải thuật lựa chọn phân lớp $k$ sao cho $|\vec{w}_k\cdot\vec{x}^*+b_k|$ có giá trị lớn nhất.
\item \emph{Một đấu với một (one-against-one):}

Huấn luyện $m$ mô hình cho bài toán có $n$ lớp, mỗi mô hình có nhiệm vụ phân loại dữ liệu vào hai lớp bất kì trong $n$, tức $m$ chính là tổ hợp chập 2 của $n$: $m=\frac{n(n-1)}{2}$:\[(1,2),(1,3),\dots,(1,n),(2,3),(2,4),\dots,(2,n),\dots,(n-1,n)\]

Sau khi huấn luyện, với mỗi điểm dữ liệu cần phân loại $\vec{x}^*$, giải thuật lựa chọn phân lớp mà $\vec{x}^*$ được phần lớn $m$ mô hình phân loại vào.
\end{enumerate}

\section{Các công cụ hỗ trợ rút trích đặc trưng} \label{tools}
Vì đầu vào của hệ thống phân giải đồng tham chiếu là các văn bản thô và danh sách các khái niệm, chúng cần được rút trích đặc trưng và mã hóa thành dạng số để có thể đưa vào huấn luyện SVM. Để thuận tiện, chúng tôi sử dụng một số công cụ để xử lý chung về ngôn ngữ tự nhiên hay hỗ trợ rút trích một số thông tin ngữ nghĩa như thời gian hoặc các thông tin từ vựng liên quan đến chuyên ngành y tế. Sau đây chúng tôi trình bày các công cụ được sử dụng và mục đích cụ thể của chúng.

\subsection*{OpenNLP}
Thư viện Apache OpenNLP bao gồm các công cụ dựa trên học máy dùng để xử lý, phân tích các chuỗi văn bản được viết dưới dạng ngôn ngữ tự nhiên. OpenNLP hỗ trợ rất nhiều tác vụ cơ bản trong xử lý ngôn ngữ tự nhiên, trong số các tác vụ tách từ (tokenize), xác định từ loại (part-of-speech tagging), phân định cụm từ (chunking) và xác định danh từ trung tâm (head nouns) được chúng tôi sử dụng để hỗ trợ quá trình rút trích đặc trưng trong hệ thống. 

Trong việc phân tích từ vựng (lexical analysis), tác vụ tách từ là tác vụ chia nhỏ câu văn thành các thành phần nhỏ hơn như từ, cụm từ, ký hiệu, hoặc các yếu tố có ý nghĩa khác được gọi chung là token. Ví dụ câu văn \texttt{"The female patient is 65 years old ."} sau khi được phân giải sẽ gồm các token là \texttt{"The"}, \texttt{"Female"}, \texttt{"patient"}, \texttt{"is"}, \texttt{"65"}, \texttt{"years"}, \texttt{"old"} và \texttt{"."}. Ở đây các dấu câu hoặc các ký tự đặc biệt cũng được xem như là token. Trong hệ thống của chúng tôi, tác vụ tách từ được sử dụng như một tác vụ hỗ trợ cho các tiến trình khác như xác định từ loại hay phân định cụm từ.

Xác định từ loại là tác vụ phân loại các từ dựa vào vai trò ngữ pháp của chúng trong câu. Danh sách các vai trò ngữ pháp trong tiếng Anh được định nghĩa sẵn dựa theo dự án Penn TreeBank \cite{Santorini1990}. Bảng \ref{tab:POSTag} miêu tả một số nhãn từ loại thông dụng. Tác vụ xác định từ loại được thư viện OpenNLP hỗ trợ bằng phương thức nhận vào một mảng các Token và trả về một mảng các vai trò ngữ pháp tương ứng với các token đó. Ví dụ câu văn \texttt{"They refuse to permit us"} được tách từ thành \texttt{[They, refuse, to, permit, us]} và được xác định từ loại như sau \texttt{[They/PRP, refuse/VBP, to/TO, permit/VB]}.

Trong ngôn ngữ học, trung tâm của một cụm từ là từ giúp xác định vai trò ngữ pháp cụm từ đó. Ví dụ từ trung tâm của cụm danh từ \texttt{"boiling hot water"} là danh từ \texttt{"water"}. Thư viện OpenNLP hỗ trợ tác vụ xác định danh từ trung tâm bằng phương thức nhận vào một cụm từ theo định dạng IBO và trả về từ trung tâm của cụm từ đó.

\begin{table}[ht]
\centering\ra{1.2}
\caption{Các nhãn từ loại thông dụng \label{tab:POSTag}}
\footnotesize\sffamily

\begin{tabularx}{0.8\textwidth}{@{}l *5{>{\arraybackslash}X}@{}}
\toprule 
\textbf{Nhãn} & \textbf{Ý nghĩa} & \textbf{Ví dụ}\\
\midrule
CC & Liên từ (conjunction) & and, or, but\\
DT & Mạo từ xác định (determiner) & the, a, an, these\\
JJ & Tính từ (adjective) & nice, easy \\
NN & Danh từ số ít (singular noun) & tiger, chair \\
NNS & Danh từ số nhiều (plural noun) & tigers, chairs \\
PRP & Đại từ chỉ người (personal pronoun) & me, you, I \\
RB & Phó từ (adverb) & extremely, loudly \\
VB & Động từ nguyên mẫu & think  \\
VBZ & Động từ ngôi thứ ba & thinks \\
VBD & Động từ quá khứ & thought \\
WDT & Nghi vấn từ xác định & which, whatever, whichever\\
WP & Nghi vấn từ chỉ người & what, who, whom \\
WRB & Nghi vấn từ chỉ phó từ & where, when \\
WP\$ & Nghi vấn từ sở hữu & whose, whosever \\
\bottomrule
\end{tabularx}
\end{table}

\subsection*{MetaMap}
UMLS là kho từ vựng y sinh do Thư viện Y học Quốc gia Hoa Kì tiến hành xây dựng từ năm 1986 nhằm phục vụ cho nhu cầu tra cứu thông tin đã được chuẩn hóa về các loại bệnh, thuốc, nguyên nhân hoặc các thuật ngữ trong y tế. Bộ từ điển UMLS tích hợp hơn 2 triệu tên gọi cho khoảng 900.000 khái niệm y sinh và khoảng 12 triệu tên gọi cho quan hệ giữa các khái niệm này \cite{Olivier2004}. Hiện nay, bộ từ điển UMLS vẫn đang được tiếp tục cập nhật và cho phép sử dụng miễn phí phục vụ mục đích nghiên cứu khoa học.

MetaMap là công cụ hỗ trợ nhận dạng các khái niệm trong bộ từ điển UMLS từ các văn bản y sinh. Đầu vào của MetaMap là văn bản thuần không có cấu trúc, từ văn bản này MetaMap xuất ra các chuỗi định dạng XML hoặc định dạng gần gũi với con người chứa các khái niệm trong từ điển UMLS nhận dạng được. Các thông tin từ UMLS rút trích được bao gồm: mã khái niệm trong bộ từ điển UMLS, các từ đồng nghĩa và lớp ngữ nghĩa của khái niệm trong UMLS. Hình \ref{metamapstructure} miêu tả kiến trúc tổng quát của MetaMap. Quá trình xử lý dữ liệu của MetaMap có thể được tóm tắt qua hai bước. Bước đầu, dữ liệu dạng văn bản được áp dụng các tác vụ xử lý ngôn ngữ tự nhiên cơ bản như: tách câu, tách từ, xác định từ loại, tra cứu từ vựng và phân tích cú pháp. Kết quả của bước đầu tiên là các cụm từ (phrase) được xác định từ văn bản đầu vào. Ở bước thứ hai, MetaMap tiến hành sinh các biến thể (variant) cho các cụm từ đã tìm được, xác định ứng cử viên (candidate) từ các khái niệm trong UMLS khớp với các biến thể được sinh ra và tính độ tin cậy cho các ứng cử viên đó. Đầu ra của bước thứ hai là các cụm từ được phân giải từ bước đầu, kèm theo các ứng cử viên trong bộ từ điển UMLS ứng với mỗi cụm. Trong phạm vi luận án, chúng tôi sử dụng tên khái niệm và lớp ngữ nghĩa của khái niệm từ đầu ra của MetaMap cho quá trình rút trích đặc trưng.

\begin{figure}
\centering
\resizebox{\textwidth}{!}{
\begin{tikzpicture}[%
	>=angle 60,
	start chain=going right,
	node distance=3cm and 0.8cm,
	every join/.style={->, draw},
	font=\scriptsize\sffamily]
	\tikzset{
		bproc/.style = {draw, rectangle, on chain, on grid, text width=2.8cm, minimum height=1.4cm, inner sep=0.5mm, align=center}
	};
	
	\node[bproc](tkn){Nhận diện câu văn và tách từ};
	\node[bproc,join](pos){Đánh dấu vai trò ngữ pháp (POS)};
	\node[bproc,join](lkw){Tra cứu từ vựng};
	\node[bproc,join](sta){Phân tích cú pháp câu};
	
	\node[bproc,below=of tkn](vgp){Bộ sinh các biến thể từ cụm từ cho trước};
	\node[bproc,join](cdd){Nhận dạng các ứng cử viên (candidate)};
	\node[bproc,join](map){Xây dựng các ánh xạ};
	\node[bproc,join](ars){Phân giải nhập nhằng};
	\node[cylinder, minimum width=1.3cm, minimum height=1.3cm, draw, shape border rotate=90, below=1cm of map,shape aspect=.4](umls){UMLS};
	
	\draw[->] (sta.south) -- ++(down:0.8cm) -| (vgp.north);	
	\draw[dashed] ($(tkn.north west)+(-0.4,0.4)$) rectangle ($(sta.south east)+(0.4,-0.4)$);
	\coordinate[left=2.2cm of tkn.west](left-tkn);
	\draw[->] (left-tkn) -- (tkn) node[above,midway]{dữ liệu đầu vào};
	\draw[->] (umls.top) -- (map);
\end{tikzpicture}
}
\caption{Kiến trúc tổng quát của MetaMap\label{metamapstructure}}
\end{figure}

Ví dụ khi đưa câu văn \texttt{"The patient has a right upper lobe mass"} vào hệ thống MetaMap để phân tích thì ở kết quả của bước một, câu văn được phân tách thành ba cụm từ \texttt{"The patient}, \texttt{"has"} và \texttt{"a right upper lobe mass"}. Sau đó MetaMap thực hiện bước thứ hai lên các cụm từ này để cho ra kết quả cuối cùng, chúng được trình bày trên Hình \ref{metamapoutput}. Sau đây là các giải thích cho các kết quả này:

\begin{itemize}
\item Dòng 1 và 2 mô tả cụm từ \texttt{"The patient"} với một ứng cử viên có độ tin cậy 1000 ứng với khái niệm \texttt{"patient"} mang mã số khái niệm là C0030705. Khái niệm \texttt{"patient"} thuộc lớp ngữ nghĩa bệnh nhân hoặc nhóm khuyết tật (Patient or Disabled Group) trong bộ từ điển UMLS.
\item Dòng 5 mô tả cụm từ \texttt{"is"} không được tìm thấy bất kì ứng cử viên nào.
\item Các dòng từ 7 trở đi mô tả kết quả cho cụm từ \texttt{"HIV positive for two years"}, trong đó có bốn ứng cử viên có đồng số điểm là 916. Ứng cử viên đầu tiên bao gồm hai khái niệm mang mã số C1261074 và C0577573 (dòng 9 và 10). Ngoài ra các ứng cử viên khác được trình bày trên Hình \ref{metamapoutput}. 
\end{itemize}

\begin{figure}[ht]
\centering
\lstset{
	keywords={},
	tabsize=3,
	%frame=lines,
	frame=single,
	xleftmargin=20pt,
	framexleftmargin=15pt,
	numbers=left,
	numberstyle=\tiny,
	numbersep=5pt,
	breaklines=true,
	showstringspaces=false,
	columns=fullflexible,
	basicstyle=\footnotesize\ttfamily}
\lstinputlisting{sample_code/metamap_output.txt}
\caption{Mẫu kết quả của công cụ MetaMap\label{metamapoutput}}
\end{figure}

\subsection*{cTakes và các công cụ dựa trên cTakes}
Apache cTakes (clinical Text Analysis and Knowledge Extraction System) là hệ thống xử lý ngôn ngữ tự nhiên giúp rút trích thông tin từ các bệnh án điện tử dưới dạng văn bản thuần không có cấu trúc. Hệ thống cTakes có khả năng nhận diện các thực thể trong y tế như tên thuốc, các bất thường sức khỏe hay các triệu chứng/rối loạn, v.v... Hệ thống cTakes bao gồm nhiều bộ phận nhỏ như phát hiện câu văn, đánh dấu vai trò ngữ pháp, tra cứu dựa trên các bộ từ điển y tế hay chuẩn hóa thông tin y tế, v.v... Mỗi bộ phận của cTakes được huấn luyện chuyên biệt cho lĩnh vực y tế, các thông tin rút trích có thể được sử dụng làm đầu vào cho các hệ thống hỗ trợ ra quyết định y tế hoặc dùng để nghiên cứu.

cTakes được xây dựng dựa trên framework UIMA cho phép ứng dụng được cấu tách thành từ các hệ thống con (subsystem). Mỗi hệ thống con hiện thực một nhiệm vụ nhất định và được cung cấp một tập tin XML đóng vai trò là siêu dữ liệu mô tả hệ thống con đó. Hình \ref{ctakesdesc} trình bày tập tin XML mô tả hệ thống xác định từ loại của cTakes, đầu vào của hệ thống này là đường dẫn tới tập tin mô hình đã được huấn luyện cho việc xác định từ loại. Ngoài ra framework UIMA còn hỗ trợ cTakes gói các hệ thống con lại như một mạng lưới các dịch vụ, trong đó các hệ thống con của cùng mạng lưới có thể giao tiếp với nhau thông quá các miêu tả trong tập tin XML, tương tự như mô hình đường ống (pipeline). Trong phạm vi luận án, chúng tôi sử dụng công cụ MedEx \cite{HuaXu2009} và MedTime \cite{HuaXu2009} được xây dựng dựa trên cTakes và được cung cấp bởi hội đồng Open Health Natural Language Processing hỗ trợ cho quá trình rút trích đặc trưng.

MedEx là công cụ hỗ trợ trích xuất các thông tin về thuốc y tế như tên thuốc (drug name), liều lượng (strength), đường hấp thu (route) hay tần số (frequency), v.v... Đầu vào của công cụ là văn bản thuần không có cấu trúc, kết quả là một tập tin chứa các thông tin thuốc y tế nhận dạng được. Mỗi dòng trong tập tin kết quả là một thông tin thuốc nhận dạng được theo định dạng sau, trong đó bao gồm các thông tin được trình bày trên Bảng \ref{tab:med-info}:

\begin{center}
\texttt{<chỉ số câu> <nội dung câu>|<thông tin thuốc>|<tên thuốc được chuẩn hóa>}
\end{center}
%
%\noindent trong đó thông tin về thuốc bao gồm:
%
%\begin{itemize}
%\item Tên thuốc (Drug name) như kháng sinh, Vancomycin , ...
%\item Nhà sản xuất (Brand name) như Zocor, ...
%\item Dạng thuốc (Drug form) như dạng viên nén, dạng viên nhộng, ...
%\item Mức độ (Strength) như 10mg, 5ml, ...
%\item Liều lượng (Dose amount) như 2 viên, ...
%\item Đường hấp thụ (Route) như qua đường miệng, qua đường tiêm, ...
%\item Tần số (Frequency) như 2 lần mỗi ngày, mỗi sáng, ...
%\item Thời gian (Duration) như trong 10 ngày, trong 1 tháng, ...
%\end{itemize}

\begin{table}[ht]
\centering\ra{1.2}
\caption{Các loại thông tin thuốc được trích xuất bởi MedEx\label{tab:med-info}}
\footnotesize\sffamily

\begin{tabularx}{0.8\textwidth}{@{}XX@{}}
\toprule
\textbf{Thông tin} & \textbf{Ví dụ}\\
\midrule
Tên thuốc (drug name) & Vancomycin\\
Nhà sản xuất (brand name) & Zocor\\
Dạng thuốc (form) & \\
Mức độ (strength) & 10mg, 5ml,...\\
Liều lượng (dosage) & \\
Đường hấp thụ (route) & \\
Tần số (frequency) & 2 times a day, each morning,...\\
Thời gian sử dụng (duration) & for 10 days, in a month,...\\
\bottomrule
\end{tabularx}
\end{table}

MedTime là công cụ hỗ trợ trích xuất các thông tin về thời gian trong văn bản thuần không có cấu trúc, bao gồm hai loại là thời gian tường minh và thời gian suy diễn. Thời gian tường minh là các dạng thời gian được ghi một cách cụ thể trong văn bản bao gồm ngày, tháng, năm, có hoặc không có thời điểm, ví dụ như 03/06/2015. Thời gian suy diễn là các mốc thời gian được trích xuất từ một số từ khóa đứng gần khái niệm đang xét hoặc các khái niệm chỉ về sự kiện trong y tế, ví dụ như \emph{ngày nhập viện} (admission date) hay \emph{ngày thứ 2 sau khi phẫu thuật} (post-op day 2). Giống như công cụ MedEx, MedTime nhận đầu vào là các văn bản thuần không có cấu trúc và xuất ra kết quả là một tập tin chứa các thông tin thời gian được nhận dạng. Mỗi dòng trong tập tin kết quả là một thông tin về thời gian được trình bày theo định dạng:

\begin{center}
\texttt{<TIMEX3 id=<chỉ số> start=<vị trí bắt đầu> end=<vị trí kết thúc> text=<chuỗi giá trị nhận dạng> type=<loại> val=<giá trị chuẩn hóa>/>}\\
\end{center}

\begin{figure}[ht]
\centering
\lstset{
	language=xml,
	keywords={},
	tabsize=3,
	%frame=lines,
	frame=single,
	xleftmargin=20pt,
	framexleftmargin=15pt,
	numbers=left,
	numberstyle=\tiny,
	numbersep=5pt,
	breaklines=true,
	showstringspaces=false,
	columns=fullflexible,
	basicstyle=\footnotesize\ttfamily}
\lstinputlisting{sample_code/ctakes_desc.xml}
\caption{Tập tin XML mô tả bộ phận con cTakes\label{ctakesdesc}}
\end{figure}

\chapter{Hiện thực hệ thống}
\section{Nội dung bài toán}
Bài toán mà chúng tôi giải quyết là bài toán: ``Phân giải đồng tham chiếu cho các hồ sơ xuất viện tiếng Anh với các khái niệm đã được trích xuất và gán nhãn''. Đầu vào của bài toán bao gồm hai phần:

\begin{enumerate}[leftmargin=\the\parindent]
\item \emph{Tập các hồ sơ xuất viện: }Đây là những văn bản lâm
sàng được viết tay bằng ngôn ngữ tự nhiên bởi các bác sĩ, y tá. Chúng
mô tả lại toàn bộ thông tin của bệnh nhân trong một lần điều trị,
bao gồm các thông tin về tên bệnh mà bệnh nhân mắc phải, các thủ tục
y tế được thực hiện và các phương pháp điều trị được áp dụng lên bệnh
nhân.
\item \emph{Tập các khái niệm đã được trích xuất và gán nhãn:}
Mỗi hồ sơ xuất viện đi kèm với một văn bản chứa toàn bộ các khái niệm
được đề cập trong hồ sơ đó. Các khái niệm này đã được gán nhãn cho
phù hợp với loại thực thể mà nó đề cập tới. Có tất cả năm nhãn là
Problem, Treatment, Test, Person và Pronoun được i2b2 định nghĩa.
Bảng \ref{tab:EntityLabels} mô tả chi tiết ý nghĩa của năm nhãn này.
\end{enumerate}

Mục tiêu của chúng tôi là phân giải đồng tham chiếu cho các khái niệm trong tập các khái niệm ứng với mỗi hồ sơ xuất viện. Cụ thể kết quả đầu ra là danh sách các chuỗi đồng tham chiếu của khái niệm đó, ví dụ \emph{c=``the patient'' 13:0 13:1||c=``he'' 14:0 14:0||c=``his'' 14:7 14:7||t=``coref person''} mô tả một chuỗi đồng tham chiếu bao gồm các khái niệm ``the patient'' (xuất hiện ở dòng thứ 13, từ vị trí 0 đến 1), ``he'' và ``his''. Các khái niệm này đồng tham chiếu tới cùng một người (\emph{t=``coref person''}).

\begin{table}[th]
\centering\ra{1.3}
\caption{Ý nghĩa các lớp thực thể được đề xuất bởi i2b2\label{tab:EntityLabels}}
\footnotesize\sffamily

\begin{tabularx}{\textwidth}{@{}lLP{\raggedright}{0.3}@{}}
\toprule
\textbf{Lớp} & \textbf{Định nghĩa} & \textbf{Ví dụ}\\
\midrule
\emph{Person} & Những chủ thể người hoặc một nhóm người được đề cập trong bệnh án và các đại từ nhân xưng & Dr.Lightman, the patient, cardiology, he, she, ...\\
\emph{Problem} & Những bất thường về sức khỏe thân thể hoặc tinh thần của bệnh nhân, được mô tả bởi bệnh nhân hoặc quan sát của bác sĩ & Heart attack, blood pressure, cancer, ...\\
\emph{Test} & Những thủ tục y tế như xét nghiệm, đo đạc, kiểm tra trên cơ thể bệnh nhân để cung cấp thêm thông tin cho “Problem” & CT scan, Temperature, ...\\
\emph{Treatment} & Những thủ tục y tế hoặc quy trình áp dụng để chữa trị cho "Problem", bao gồm thuốc, phẫu thuật hoặc phương pháp điều trị & Surgery, ice pack, Tylenol, ...\\
\emph{Pronoun} & Những đại từ có thể tham chiếu đến bất kì lớp nào trong bốn lớp kể trên nhưng không phải là đại từ nhân xưng & Which, it, that, ...\\
\bottomrule
\end{tabularx}
\end{table}

\section{Ý tưởng hiện thực\label{ytuonghienthuc}}
Dựa vào hệ thống có hiệu năng tốt nhất của thử thách i2b2 năm 2011 (hệ thống I), mô hình phân giải đồng tham chiếu mà chúng tôi sử dụng để hiện thực hệ thống là mô hình cặp thực thể. Tư tưởng cơ bản của mô hình này là xác định xem hai khái niệm bất kì có đồng tham chiếu với nhau hay không, sau đó gom nhóm các cặp đồng tham chiếu có một khái niệm chung lại để tạo thành các chuỗi đồng tham chiếu. Như vậy kiến trúc tổng quát của hệ thống chúng tôi hiện thực gồm 2 quy trình: \emph{quy trình huấn luyện hệ thống phân loại} và \emph{quy trình phân giải đồng tham chiếu}. Trong đó quy trình huấn luyện là bước huấn luyện các model phân loại dựa trên dữ liệu mẫu đã được phân giải đồng tham chiếu. Quy trình phân giải sử dụng các model phân loại đã được huấn luyện để xác định tính đồng tham chiếu của các cặp khái niệm, từ đó sử dụng một giải thuật gom nhóm các cặp đồng tham chiếu lại để tạo thành các chuỗi đồng tham chiếu.

\subsection*{Quy trình huấn luyện}
Để xác định tính đồng tham chiếu giữa hai khái niệm bất kì, ta cần huấn luyện một model phân loại dựa trên dữ liệu mẫu. Vì đầu vào của quy trình là các văn bản BAĐT và danh sách các khái niệm đã được gán nhãn, hệ thống cần trích xuất đặc trưng của các dữ liệu thô này rồi mới có thể đưa vào để huấn luyện. Bên cạnh đó, các khái niệm đã được phân loại vào 4 nhóm chính là Person, Problem, Test và Treatment, còn các đại từ được phân vào nhóm Pronoun nên để giảm bớt số cặp khái niệm được sinh ra, chúng tôi huấn luyện 4 model để xác định tính đồng tham chiếu của riêng các cặp Person-Person, Problem-Problem, Test-Test và Treatment-Treatment (vì hai khái niệm thuộc hai lớp khác nhau thì nghiễm nhiên không đồng tham chiếu với nhau). Đối với các đại từ thì thường chỉ tới một khái niệm ở trước đó, nên việc xác định xem một đại từ thực chất mang ý nghĩa của lớp nào trong 4 lớp chính Person, Problem, Test, Treatment là một việc quan trọng. Sau khi xác định được lớp chính của đại từ, chúng tôi chọn khái niệm thuộc lớp tương ứng ở gần nhất trước đó làm tiền đề cho nó. Các ý này đều là của các tác giả hệ thống I.

Ngoài ra cũng theo các tác giả này, thông tin một khái niệm lớp Person có chỉ về bệnh nhân hay không góp một phần quan trọng trong việc phân loại đúng tính đồng tham chiếu của các cặp khái niệm lớp này. Trong miền văn bản BAĐT, các khái niệm chỉ người thường chỉ đề cập đến một trong ba loại: bệnh nhân, người thân của bệnh nhân và nhân sự của bệnh viện. Do một BAĐT, mà cụ thể là hồ sơ xuất viện, thông thường chỉ đề cập đến một bệnh nhân nên những khái niệm nào chỉ về bệnh nhân thì thường chắc chắn nằm trong cùng một chuỗi đồng tham chiếu lớn nhất và duy nhất chỉ về bệnh nhân đó. Từ nhận định này, nhóm tác giả của hệ thống I đã thêm vào đặc trưng lớp Patient (Patient-class) cho cặp hai khái niệm lớp Person, nó mang giá trị 1 khi hai khái niệm đều chỉ về bệnh nhân và 0  trong các trường hợp khác. Ở bước huấn luyện, thông tin "một khái niệm Person có chỉ về bệnh nhân hay không" được lấy từ tập chuỗi kết quả (ground truth), còn ở bước phân giải đồng tham chiếu thông tin này được xác định nhờ một model phân loại đã được huấn luyện.

Như vậy mục đích của quy trình huấn luyện là xây dựng tổng cộng 6 SVM model, trong đó 4 SVM model nhằm mục đích phân loại và đánh giá độ tin cậy đồng tham chiếu của các cặp khái niệm Person-Person, Problem-Problem, Test-Test và Treatment-Treatment; 1 SVM model để xác định các khái niệm Person có là bệnh nhân hay không (Patient-class) và 1 SVM model để phân loại các đại từ (các khái niệm lớp Pronoun) vào một trong bốn lớp Person, Problem, Test và Treatment. Đầu vào của quy trình này là toàn bộ các văn bản BAĐT với các khái niệm đã được trích xuất và gán nhãn. Sau khi tiền xử lý, hệ thống xây dựng các mẫu huấn luyện
bao gồm: Person, Person-Person, Problem-Problem, Test-Test, Treatment-Treatment và Pronoun từ danh sách các khái niệm. Sáu tập mẫu này được trích xuất thuộc tính và đưa vào để huấn luyện 6 SVM model (Hình \ref{fig:SDHL}). Thư viện SVM được nhóm sử dụng là LibSVM. 

\begin{figure}[th]
\centering
\begin{tikzpicture}[%
	>=angle 60,
	start chain=going below,
	node distance=1.5cm and 2.3cm,
	every join/.style={->, draw},
	font=\tiny\sffamily]

	\node[multidoc](emr){EMR};
	\node[multidoc, right=of emr](con){Concepts};
	\node[proc,below=1.3cm of $(emr)!0.5!(con)$](prep){Tiền xử lý};
	\node[io,join]{Các khái niệm/cặp khái niệm};
	\node[proc,join](ex){Rút trích đặc trưng};
	\node[io, right=3cm of con](perp){Đặc trưng cặp Person};
	\node[io](pati){Đặc trưng một Person};
	\node[io](prop){Đặc trưng cặp Problem};
	\node[io](tstp){Đặc trưng cặp Test};
	\node[io](trep){Đặc trưng cặp Treatment};
	\node[io](pron){Đặc trưng một Pronoun};	
	\node[proc, right=3cm of $(prop)!0.5!(tstp)$](train){Huấn luyện};
	\node[io,right=7cm of perp](perm){SVM model cặp Person};
	\node[io](patim){SVM model cho Patient-class};
	\node[io](prom){SVM model cho cặp Problem};
	\node[io](tstm){SVM model cho cặp Test};
	\node[io](trem){SVM model cho cặp Treatment};
	\node[io](pronm){SVM model cho Pronoun};

	\draw[->] (emr) -- ++(down:0.9cm) -| (prep.130);
	\draw[->] (con) -- ++(down:0.9cm) -| (prep.50);
	\draw[->] (ex) -- ++(right:2.2cm) |- (perp);
	\draw[->] (ex) -- ++(right:2.2cm) |- (pati);
	\draw[->] (ex) -- ++(right:2.2cm) |- (prop);
	\draw[->] (ex) -- ++(right:2.2cm) |- (tstp);
	\draw[->] (ex) -- ++(right:2.2cm) |- (trep);
	\draw[->] (ex) -- ++(right:2.2cm) |- (pron);
	\draw[->] (perp) -- ++(right:2cm) |- (train);
	\draw[->] (pati) -- ++(right:2cm) |- (train);
	\draw[->] (prop) -- ++(right:2cm) |- (train);
	\draw[->] (tstp) -- ++(right:2cm) |- (train);
	\draw[->] (trep) -- ++(right:2cm) |- (train);
	\draw[->] (pron) -- ++(right:2cm) |- (train);
	\draw[->] (train) -- ++(right:1.5cm) |- (perm);
	\draw[->] (train) -- ++(right:1.5cm) |- (patim);
	\draw[->] (train) -- ++(right:1.5cm) |- (prom);
	\draw[->] (train) -- ++(right:1.5cm) |- (tstm);
	\draw[->] (train) -- ++(right:1.5cm) |- (trem);
	\draw[->] (train) -- ++(right:1.5cm) |- (pronm);	
\end{tikzpicture}
\caption{Sơ đồ huấn luyện\label{fig:SDHL}}
\end{figure}

\subsection*{Quy trình phân giải}
Quy trình phân giải đồng tham chiếu sử dụng 6 SVM model đã được huấn luyện ở trên, cùng với đó là một giải thuật gom nhóm các cặp khái niệm đã được phân loại là đồng tham chiếu với nhau lại để cuối cùng tạo thành các chuỗi đồng tham chiếu. Có thể xem đây là quy trình mang đi ứng dụng thực tế để phân giải cho những văn bản BAĐT mới. Dựa vào hệ thống I, chúng tôi sử dụng giải thuật gom cụm tốt nhất trước để lựa chọn các cặp đồng tham chiếu có độ tin cậy cao nhất, sau đó xây dựng các chuỗi đồng tham chiếu bằng cách nối các cặp có một khái niệm chung. Đối với lớp Pronoun, sau khi đã xác định được lớp chính của một đại từ bất kì, chúng tôi tạo một cặp đồng tham chiếu giữa đại từ đó và khái niệm thuộc lớp chính tương ứng ở gần nhất trước đó trong văn bản. Theo nhận định của các tác giả hệ thống I, tuy cách làm này đơn giản nhưng lại tỏ ra rất hiệu quả.

Hình \ref{fig:SDPG} mô tả trực quan quy trình phân giải đồng tham chiếu. Ở bước trích xuất thuộc tính của các cặp Person, chúng tôi sử dụng model phân loại bệnh nhân để xác định giá trị cho đặc trưng lớp Patient đã được đề cập ở trên. Theo kết quả đánh giá các hệ thống dự thi thử thách i2b2 2011, ba hệ đo được sử dụng để đánh giá hiệu năng là: MUC, B-CUBED và CEAF. Chúng tôi cũng hiện thực các hệ đo này để đánh giá hệ thống của mình bằng cách so sánh với kết quả của hệ thống I.

\begin{figure}[th]
\centering
\begin{tikzpicture}[%
	>=angle 60,
	start chain=going below,
	node distance=1.5cm and 2.5cm,
	every join/.style={->, draw},
	font=\tiny\sffamily]

	\node[doc](emr){EMR};
	\node[doc, right=of emr](con){Concepts};
	\node[proc, below=1.3cm of $(emr)!0.5!(con)$](prep){Tiền xử lý};
	\node[io,join](ins){Các khái niệm/cặp khái niệm};
	\node[proc,join](ex){Rút trích đặc trưng};
	\node[io, right=of ex, join](feat){Tập vector đặc trưng};
	\node[proc, right=of feat, join](clas){Phân loại};
	\node[multidoc, above=of clas](model){6 SVM model};
	\node[io,below=of clas](conf){Độ tin cậy đồng tham chiếu};
	\node[proc,join,below=of ex](clus){Gom cụm};
	\node[io,join](sysc){Các chuỗi đồng tham chiếu};
	\node[proc,join,right=of sysc](eval){Đánh giá hiệu năng};
	\node[doc,right=of eval, text width=1.5cm](gt){Các chuỗi đồng tham chiếu kết quả (ground truth)};
	\node[io,below=of eval](res){Các số liệu độ đo};

	\draw[->] (emr) -- ++(down:0.9cm) -| (prep.130);
	\draw[->] (con) -- ++(down:0.9cm) -| (prep.50);
	\draw[->] (model) -> (clas);
	\draw[->] (clas) -> (conf);
	\draw[->] (ins) -- ++(left:1.5cm) |- (clus);
	\draw[->] (gt) -> (eval);
	\draw[->] (eval) -> (res);
	\draw[->] (model.200) -- ++(left:3cm) node[midway,above,sloped]{Model Patient-class} -> (ex.north east);
\end{tikzpicture}
\caption{Sơ đồ phân giải đồng tham chiếu\label{fig:SDPG}}
\end{figure}

\section{Tiền xử lý}
Trong quá trình rút trích đặc trưng, một số khái niệm được miêu tả cụ thể làm cho việc so trùng chuỗi hoặc tìm kiếm từ các nguồn tri thức nhân loại thiếu chính xác \cite{YanXu2012}. Ví dụ như khái niệm "her CT scan" và khái niệm "a CT scan", mặc dù hai khái niệm này cùng chỉ một thủ tục y tế nhưng không trùng chuỗi. Ngoài ra các mạo từ "her", "a" làm việc tìm kiếm tri thức nhân loại từ các nguồn như Wikipedia, WordNet không được chính xác hoặc không thể tìm được kết quả, vì vậy trước khi rút trích đặc trưng, các khái niệm cần được tiền xử lý để loại bỏ mạo từ và các thông tin ngữ cảnh. Tuy nhiên, quá trình tiền xử lý chỉ được áp dụng cho các đặc trưng liên quan so trùng chuỗi và tìm kiếm tri thức nhân loại, các đặc trưng khác không cần qua quá trình tiền xử lý mà nhận vào nguyên gốc khái niệm được xác định.

Quá trình tiền xử lý gồm hai bước: đầu tiên khái niệm sẽ được loại bỏ tất cả mạo từ, sau đó, nếu khái niệm có bao gồm giới từ thì giới từ đó và toàn bộ nội dung theo sau sẽ được lược bỏ. Ví dụ như khái niệm “an MRI of the knee” sau quá trình tiền xử lý sẽ trở thành “MRI”. Danh sách mạo từ được xây dựng từ tập dữ liệu và các mạo từ thông dụng của tiếng Anh.

Đặc biệt các khái niệm thuộc lớp Problem/Treatment/Test thường được kèm thêm thông tin về định lượng như 10mg, 5 lit và các thông tin về vị trí giải phẫu học như "upper", "left", "right". Để tăng khả năng tìm kiếm tri thức nhân loại, chúng tôi đề xuất loại bỏ các thông tin ngữ cảnh về số, định lượng và vị trí giải phẫu khỏi khái niệm. Các thông tin ngữ cảnh được loại bỏ bằng cách sử dụng biểu thức chính quy và các từ vựng được xây dựng từ tập dữ liệu. Các đặc trưng liên quan so trùng chuỗi không áp dụng bước tiền xử lý loại bỏ thông tin ngữ cảnh này.
\section{Xây dựng các cặp khái niệm}
\section{Rút trích đặc trưng}
Từ các phân tích được đề cập ở phần \ref{ytuonghienthuc}, ngoài các thuộc tính chung về mặt ngôn ngữ (như ngữ pháp hay từ vựng), từng lớp khái niệm ở BAĐT còn mang những đặc tính khác nhau. Việc này đòi hỏi chúng tôi phải thiết kế bốn hệ thống rút trích đặc trưng và phân loại tương ứng khác nhau cho lớp Person, lớp Patient, nhóm lớp Problem/Treatment/Test và lớp Pronoun. Hình \ref{fig:TongquanPhangiai} mô tả tổng quan ba hệ thống này, trong đó các khối "Đồng tham chiếu lớp X" bao hàm cả Hệ thống rút trích đặc trưng và Hệ thống phân loại cho lớp tương ứng. Đối với các lớp Person và nhóm lớp Problem/Treatment/Test đầu vào của hệ thống rút trích đặc trưng là một cặp gồm hai khái niệm thuộc các lớp tương ứng, tuy nhiên đối với lớp Pronoun và lớp Patient thì đầu vào là duy nhất một khái niệm thuộc lớp tương ứng.

\begin{figure}[ht]
\centering
\begin{tikzpicture}[%
	>=angle 60,
	start chain=going below,
	node distance=1.5cm and 4cm,
	every join/.style={->, draw},
	font=\tiny\sffamily]
	\tikzset{
		wproc/.style = {proc, text width=4.5em},
		wdoc/.style = {doc, text width=3.5em}
	};
	
	\node[io](pati){Các khái niệm Person};
	\node[io](perp){Các cặp khái niệm Person};
	\node[io](prop){Các cặp khái niệm Problem};
	\node[io](tstp){Các cặp khái niệm Test};
	\node[io](trep){Các cặp khái niệm Treatment};
	\node[io](pron){Các khái niệm Pronoun};
	
	\node[wdoc,left=2cm of $(prop)!0.5!(tstp)$](emr){EMR + Concepts};
	\node[altproc,right=2.5cm of pati](patic){Phân loại bệnh nhân};
	\node[wproc,right=4cm of perp](perr){Phân giải đồng tham chiếu lớp Person};
	\node[wproc](pror){Phân giải đồng tham chiếu lớp Problem};
	\node[wproc](tstr){Phân giải đồng tham chiếu lớp Test};
	\node[wproc](trer){Phân giải đồng tham chiếu lớp Treatment};
	\node[wproc](pronr){Phân giải đồng tham chiếu lớp Pronoun};

	\node[io,right=6cm of perr](perch){Các chuỗi Person};
	\node[io](proch){Các chuỗi Problem};
	\node[io](tstch){Các chuỗi Test};
	\node[io](trech){Các chuỗi Treatment};
	
	\coordinate[left=0.5cm of perch.west](perch-left);
	\coordinate[left=1cm of proch.west](proch-left);
	\coordinate[left=1.5cm of tstch.west](tstch-left);
	\coordinate[left=2cm of trech.west](trech-left);
	
	\draw[->] (emr) -- ++(right:1.3cm) |- (pati);
	\draw[->] (emr) -- ++(right:1.3cm) |- (perp);
	\draw[->] (emr) -- ++(right:1.3cm) |- (prop);
	\draw[->] (emr) -- ++(right:1.3cm) |- (tstp);
	\draw[->] (emr) -- ++(right:1.3cm) |- (trep);
	\draw[->] (emr) -- ++(right:1.3cm) |- (pron);
	\draw[->] (pati) -> (patic);
	\draw[->] (patic) -| (perr);
	\draw[->] (perp) -> (perr);
	\draw[->] (prop) -> (pror);
	\draw[->] (tstp) -> (tstr);
	\draw[->] (trep) -> (trer);
	\draw[->] (pron) -> (pronr);
	\draw[->] (perr) -> (perch);
	\draw[->] (pror) -> (proch);
	\draw[->] (tstr) -> (tstch);
	\draw[->] (trer) -> (trech);
	\draw[->] (pronr) -| (perch-left);
	\draw[->] (pronr) -| (proch-left);
	\draw[->] (pronr) -| (tstch-left);
	\draw[->] (pronr) -| (trech-left);
\end{tikzpicture}
\caption{Tổng quan hệ thống phân giải đồng tham chiếu \label{fig:TongquanPhangiai}}
\end{figure}

\subsection*{Nhóm Person}
Như đã đề cập trong phần \ref{ytuonghienthuc}, trong BAĐT, các khái niệm thuộc lớp Person thường được chia vào ba nhóm chính: bệnh nhân, người thân của bệnh nhân hoặc nhân sự của bệnh viện. Trong đó bệnh nhân là nhóm có số lượng khái niệm được đề cập nhiều nhất và chiếm phần lớn tổng số khái niệm lớp Person. Do vậy việc xác định một khái niệm thuộc vào nhóm nào đóng vai trò quan trọng trong việc phân giải chính xác chuỗi đồng tham chiếu cho khái niệm đó \cite{YanXu2012}. Đặc trưng lớp Patient được xác định bằng phương pháp phân loại nhị phân SVM. Hai nhóm người thân của bệnh nhân và nhân sự của bệnh viện được xác định bằng các đặc trưng từ vựng. Bảng \ref{tab:PersonFeatures} trình bày đầy đủ các đặc trưng dùng cho lớp Person.

\begin{table}[th]
\centering\ra{1.2}
\caption{Tập đặc trưng cho lớp Person \label{tab:PersonFeatures}}
\footnotesize\sffamily

\begin{tabularx}{\textwidth}{@{}P{\raggedright}{0.3}lL@{}}
\toprule 
\textbf{Đặc Trưng} & \textbf{Giá trị} & \textbf{Giải thích}\\
\midrule
Patient-class & 0, 1, 2 & Không có khái niệm nào là bệnh nhân (0), cả hai khái niệm đều là bệnh nhân (1), trường hợp khác (2)\\
Distance between sentences & 0, 1, 2, 3, ... & Số câu xuất hiện giữa hai khái niệm được xét\\
Distance between mentions & 0, 1, 2, 3, ... & Số khái niệm xuất hiện giữa hai khái niệm được xét\\
String match & 0, 1 & Trùng chuỗi hoàn toàn (1), ngược lại (0)\\
Levenshtein distance between two mentions & 0, 1, 2, 3, ... & Khoảng cách Levenshtein giữa hai khái niệm\\
Number & 0, 1, 2 & Cả hai đều là số ít hoặc nhiều (1), ngược lại (0), không xác định (2)\\
Gender & 0, 1, 2 & Cùng giới tính (1), khác giới tính (0), không xác định (2)\\
Apposition & 0, 1 & Là đồng vị ngữ (1), ngược lại (0)\\
Alias & 0, 1 & Là từ viết tắt hoặc cùng nghĩa (1), ngược lại (0)\\
Who & 0, 1 & Nếu hai khái niệm liền kề nhau và được phân cách bởi dấu ``:''\\
Name match & 0, 1 & Loại bỏ các	``stop word'' (dr, dr., mr, ...), so trùng chuỗi con, trùng (1), không trùng (0)\\
Relative match & 0, 1 & Cả hai đều cùng chỉ đến một thân nhân (1), ngược lại (0)\\
Department match & 0, 1 & Cả hai cùng chỉ đến một lĩnh vực y học (1), ngược lại (0)\\
Doctor title match & 0, 1 & Cả hai có cùng một chức vụ bác sĩ (1), ngược lại (0)\\
Doctor general match & 0, 1 & Cả hai cùng đề cập đến bác sĩ nói chung (1), ngược lại (0)\\
Twin/triplet & 0, 1 & Cả hai đều chỉ về cùng cặp sinh đôi/sinh ba (1), ngược lại (0)\\
We & 0, 1 & Cả hai đều chứa thông tin về ``chúng tôi'' (1), ngược lại (0)\\
You & 0, 1 & Cả hai đều chứa thông tin về ``tôi'' (1), ngược lại (0)\\
I & 0, 1 & Cả hai đều chứa thông tin về ``bạn'' (1), ngược lại (0)\\
Pronoun match & 0, 1 & Cả hai đều là đại từ chỉ người (1), ngược lại (0)\\
\bottomrule
\end{tabularx}
\end{table}

Đặc trưng ``Alias'' được chúng tôi hiện thực theo các bước như sau. Bước một, Loại bỏ các mạo từ hoặc đại từ trong khái niệm. Bước hai, kiểm tra các chữ cái đầu tiên của mỗi từ trong khái niệm có được viết hoa hay không. Nếu có thì ghép các chữ cái đầu tiên của mỗi từ để tạo thành từ viết tắt của khái niệm. Cuối cùng, so sánh hai từ đang xét với từ viết tắt của chính hai từ đó, nếu trùng chuỗi thì xác định đặc trưng ``Alias'' là 1, ngược lại là 0.

Đặc trưng về Giới tính được chúng tôi xác định dựa trên ba bước phân loại \cite{WeeSoon2001}. Bước thứ nhất: kiểm tra khái niệm có chứa các đại từ xác định giới tính như ``Mr'', ``Ms'', ``she'', ``he'', ... hay không. Nếu có, xác định giới tính dựa trên đại từ xuất hiện. Nếu không thực hiện bước thứ hai: kiểm tra khái niệm có xuất hiện nhiều hơn một lần hay không. Nếu xuất hiện nhiều hơn một lần thì các lần xuất hiện có chứa đại từ xác định giới tính hay không. Ví dụ khái niệm ``Peter H. Diller'' có thể xuất hiện nhiều lần, trong đó có xuất hiện dưới hình thức ``Mr. Diller''. Nếu không thể xác định giới tính qua hai bước kiểm tra, khái niệm sẽ được phân loại bằng cách sử dụng cơ sở dữ liệu về tên tiếng Anh của hệ thống Apache OpenNLP.

Với các đặc trưng ``Name match'', ``Relative match'', ``Department match'', ``Doctor title match'', ``Doctor general match'', ``Twin/Triplet'', ``We'', ``You'', ``I'', ``Pronoun match'', chúng tôi hiện thực bằng cách xây dựng tập từ điển tương ứng với từng đặc trưng dựa trên việc khảo sát tập dữ liệu và sử dụng các biểu thức chính quy. Các đặc trưng còn lại, chúng tôi hiện thực như miêu tả trong bảng \ref{tab:PersonFeatures}.

\subsection*{Nhóm Patient-class}
Từ nhận định trong việc rút trích đặc trưng của lớp Person, chúng tôi xây dựng một hệ thống SVM nhị phân để phân loại khái niệm thuộc lớp Person có phải là bệnh nhân hay không. Trong BAĐT thường chỉ có một bệnh nhân đóng vai trò là chủ thể của bệnh án.Vì vậy, các khái niệm nếu được xác định là bệnh nhân, thì sẽ được đưa vào một chuỗi đồng tham chiếu duy nhất về bệnh nhân đó. Thông qua phân tích tập dữ liệu, chúng tôi nhận thấy việc xác định một khái niệm thuộc lớp Person hay không có thể đạt được bằng cách xác định tập từ khóa chỉ về bệnh nhân như ``patient'', ``pt'', ... và tập từ khóa chỉ về nhóm người không phải bệnh nhân như ``doctor'', ``dr'', ``wife'', ...

Vì tập dữ liệu không có thông tin xác định một khái niệm thuộc lớp Person có phải là bệnh nhân hay không, dựa theo hệ thống I chúng tôi xác định bằng cách chọn chuỗi đồng tham chiếu có nhiều khái niệm nhất trong tập kết quả làm chuỗi đồng tham chiếu chỉ bệnh nhân. Các khái niệm thuộc chuỗi đồng tham chiếu này sẽ được xem là khái niệm chỉ bệnh nhân và được chọn làm mẫu dương trong quá trình huấn luyện. Các khái niệm thuộc lớp Person còn lại không thuộc vào chuỗi đồng tham chiếu này sẽ được chọn làm mẫu âm trong quá trình huấn luyện. Tuy nhiên, chúng tôi nhận thấy phương pháp xác định bệnh nhân này có một nhược điểm là các BAĐT nhỏ, có nội dung ngắn sẽ tồn tại nhiều chuỗi đồng tham chiếu lớp Person có kích thước tương tự nhau. Trong đó chuỗi đồng tham chiếu chỉ bệnh nhân không chắc chắn là chuỗi đồng tham chiếu có kích thước lớn nhất.

Bảng \ref{tab:PatientFeatures} trình bày đầy đủ các đặc trưng được sử dụng cho việc xác định khái niệm có phải là bệnh nhân hay không. Kết quả của việc phân loại này sẽ được sử dụng làm giá trị cho đặc trưng ``Patient-class'' khi rút trích đặc trưng cho lớp Person.

\begin{table}[th]
\centering\ra{1.2}
\caption{Tập đặc trưng cho lớp Patient \label{tab:PatientFeatures}}
\footnotesize\sffamily

\begin{tabularx}{\textwidth}{@{}P{\raggedright}{0.3}lL@{}}
\toprule 
\textbf{Đặc Trưng} & \textbf{Giá trị} & \textbf{Giải thích}\\
\midrule
Keyword of patient & 0, 1 & Các từ khóa về bệnh nhân (như mr., mr, ms., ms, yo-, y.o., y/o, year-old, ...)\\
Keyword of doctor & 0, 1 & Các từ khóa về bác sĩ (dr, dr., md, m.d., m.d,…)\\
Key word of doctor title & 0, 1 & Các từ khóa về chức vụ của bác sĩ (dentist, orthodontist, …)\\
Key word of department  & 0, 1 & Các từ khóa về chuyên ngành bác sĩ (electrophysiology, …)\\
Key word of general deparment & 0, 1 & Các từ khóa chung về phòng ban (team, service)\\
Key word of general doctor & 0, 1 & Các từ khóa chung về bác sĩ (doctor, dict, author, pcp, attend, provider)\\
Key word of relative & 0, 1 & Các từ khóa về người thân (wife, brother, sibling, nephew)\\
Name & 0, 1 & Có phải là tên riêng hay không\\
Last n line doctor & 0, 1 & Là tên bác sĩ ở n dòng cuối cùng\\
Twin or triplet information & 0, 1 & Thông tin về cặp sinh đôi, sinh ba (baby 1, 2, 3,…)\\
Preceded by non-patient & 0, 1 & Khái niệm đứng trước không phải là bệnh nhân.\\
Signed information  & 0, 1 & Có liên quan đến việc kí/xác nhận bệnh án\\
Previous sentence &  & Câu hoàn chỉnh liền trước khái niệm\\
Next sentence &  & Câu hoàn chỉnh liền sau khái niệm\\
Pronouns we & 0, 1 & Là đại từ chỉ chúng tôi (we, us, our, ourselves)\\
Pronouns I & 0, 1 & Là đại từ chỉ tôi (I, my, me, myself)\\
Pronouns you & 0, 1 & Là đại từ chỉ bạn (you, your, yourself)\\
Pronouns they & 0, 1 & Là đại từ chỉ họ (they, them, their, themselves)\\
Pronouns he/she most & 0, 1 & Thuộc phần đa số của đại từ chỉ cô ấy/anh ấy (he, his, her)\\
Who & 0, 1 & Là đại từ “who” hoặc liền kề với khái niệm đứng trước\\
Appositive & 0, 1 & Là đồng vị ngữ\\
\bottomrule
\end{tabularx}
\end{table}

Các đặc trưng ``Previous sentence'' và ``Next sentence'' được hiện thực bằng cách khảo sát toàn bộ các khái niệm thuộc lớp Person, sau đó xây dựng bộ từ điển các câu có thể đứng trước hoặc đứng sau khái niệm đang xét. Giá trị của đặc trưng được lấy bằng chỉ mục của câu đứng trước (hoặc đứng sau) trong bộ từ điển các câu.

Đặc trưng ``Pronouns he/she most'' mang ý nghĩa giới tính chiếm đa số trong BAĐT được xét. Việc xác định giới tính chiếm đa số trong BAĐT được hiện thực bằng cách xác định giới tính cho từng khái niệm thuộc lớp Person, sau đó chọn giới tính có số lượng khái niệm lớn hơn. Phương pháp xác định giới tính được thực hiện theo miêu tả trong đặc trưng của nhóm Person. Nếu trong BAĐT có giới tính Nam chiếm đa số thì những khái niệm là đại từ chỉ về giới tính Nam như ``he'', ``him'', ``himself'', ... sẽ có đặc trưng ``Pronouns he/she most'' mang giá trị là 1. Tương tự cho BAĐT có giới tính Nữ chiếm đa số.

Qua việc khảo sát tập dữ liệu, chúng tôi nhận thấy trong BAĐT thường được kết thúc bằng các thông tin hướng dẫn liên lạc với bác sĩ nếu bệnh nhân gặp vấn đề sau khi xuất viện, thông tin về ngày tháng và mã BAĐT, thông tin về hướng dẫn sau khi xuất viện và ký tên của bác sĩ. Để xác định phần BAĐT chứa các thông tin này, chúng tôi khảo sát từng câu trong BAĐT từ dưới lên đến khi gặp các từ khóa như ``Follow-up'', ``Dictated By'', ``Signed By'', ... Các đặc trưng ``Last n line doctor'' và ``Signed information'' được xác định bằng các thông tin trong phân đoạn BAĐT này.

Các đặc trưng liên quan đến từ khóa được chúng tôi hiện thực bằng cách khảo sát tập dữ liệu và xây dựng bộ từ điển thích hợp cho từng đặc trưng. Các đặc trưng còn lại được chúng tôi hiện thực theo như miêu tả trong bảng \ref{tab:PatientFeatures}

\subsection*{Nhóm Pronoun}
Khác với các lớp khái niệm khác trong BAĐT, lớp khái niệm Pronoun là lớp khái niệm trừu tượng, có thể chỉ về bất kì khái niệm thuộc bốn lớp Person, Problem, Treatment, Test hoặc là khái niệm độc lập không đồng tham chiếu. Để giải quyết vấn đề đồng tham chiếu cho lớp Pronoun, tác giả của hệ thống I đề xuất xây dựng hệ thống phân loại SVM nhiều lớp (multi-class SVM). Hệ thống SVM này dùng để phân loại khái niệm thuộc lớp Pronoun đang xét đồng tham chiếu đến lớp khái niệm nào trong bốn lớp Person, Problem, Treatment, Test. Sau đó ta chọn khái niệm thuộc lớp tương ứng ở gần nhất trước đó để làm tiền đề cho đại từ đang xét. Ví dụ ``\underline{Hepatitis C cirrhosis} for \textit{which} the patient was on the liver transplant list'' có ``which'' là đại từ đang xét và ``Hepatitis C cirrhosis'' là khái niệm thuộc lớp Problem. Nếu xác định được đại từ ``which'' thuộc lớp Problem, ta có thể kết luận ``which'' và ``Hepatitis C cirrhosis'' đồng tham chiếu do khái niệm ``Hepatitis C cirrhosis'' cùng thuộc lớp Problem và ở gần nhất trước đó. Bảng \ref{tab:PronounFeatures} mô tả đầy đủ các đặc trưng được sử dụng cho lớp Pronoun.

\begin{table}[th]
\centering\ra{1.2}
\caption{Tập đặc trưng cho lớp Pronoun \label{tab:PronounFeatures}}
\footnotesize\sffamily

\begin{tabularx}{\textwidth}{@{}P{\raggedright}{0.3}lL@{}}
\toprule 
\textbf{Đặc Trưng} & \textbf{Giá trị} & \textbf{Giải thích}\\
\midrule
First previous mention type & 0, 1, 2, 3, 4 & Khái niệm đứng liền trước thuộc lớp Person (0) hoặc Problem (1) hoặc Treatment (2) hoặc Test (3) hoặc không thuộc lớp nào (4)\\
Second previous mention type & 0, 1, 2, 3, 4 & Khái niệm đứng liền trước thứ 2 thuộc lớp Person (0) hoặc Problem (1) hoặc Treatment (2) hoặc Test (3) hoặc không thuộc lớp nào (4)\\
First next mention type & 0, 1, 2, 3, 4 & Khái niệm đứng liền sau thuộc lớp Person (0) hoặc Problem (1) hoặc Treatment (2) hoặc Test (3) hoặc không thuộc lớp nào (4)\\
Sentence distance & 0, 1, 2, ... & Số câu xuất hiện giữa hai khái niệm được xét\\
Pronoun & 0, 1, 2, ..., 14 & Chỉ mục của đại từ được xét trong bảng tra 15 đại từ\\
Part of speech & 0, 1, 2 & DT (0), WDT (1), PRP (2)\\
First next verb after mention & & Động từ đứng liền sau khái niệm\\
First word before mention is preposition & 0, 1 & Từ liền trước là giới từ (1), ngược lại (0)\\
First one/two/three words before mention & & 1/2/3 từ liền trước khái niệm được xét\\
First one/two/three words after mention & & 1/2/3 từ liền sau khái niệm được xét\\
An adjacent mention after pronoun + VP  & 0, 1 & Khái niệm được xét liền sau đại từ + cụm động từ\\
And, as well as, in addition to & 0, 1 & Khái niệm được xét liền sau ``and'', ``as well as'', ``in addition to''\\
\bottomrule
\end{tabularx}
\end{table}

Các đặc trưng ``First next verb after mention'', ``First one/two/three words before mention'', ``First one/two/three words after mention'' được hiện thực bằng cách khảo sát toàn bộ các khái niệm thuộc lớp Pronoun, sau đó xây dựng bộ từ điển các từ có thể đứng trước hoặc đứng sau khái niệm đang xét. Giá trị của đặc trưng được lấy bằng chỉ mục của từ đứng trước (hoặc đứng sau) trong bộ từ điển các từ.

Các đặc trưng ``Part of speech'', ``First word after mention is preposition'', ``An adjacent mention after pronoun + VP'' được hiện thực bằng cách tìm câu văn chứa khái niệm đang xét trong BAĐT. Sau đó, chúng tôi tiến hành phân đoạn câu (tokenize) và đánh dấu thông tin từ loại (POS tag) đối với câu văn chứa khái niệm đang xét. Từ danh sách thông tin từ loại có được, các đặc trưng nêu trên được tính giá trị theo mô tả. Chúng tôi sử dụng hệ thống xử lý ngôn ngữ tự nhiên Apache OpenNLP cho các tác vụ phân đoạn câu và đánh dấu thông tin từ loại. Các đặc trưng còn lại được chúng tôi hiện thực theo như mô tả trong bảng \ref{tab:PronounFeatures}.

\subsection*{Nhóm Problem/Treatment/Test}
Nhóm lớp Problem/Treatment/Test là nhóm lớp đặc biệt thuộc riêng lĩnh vực y khoa. Trong lĩnh vực này, cùng một khái niệm có thể được biểu diễn dưới nhiều hình thức khác nhau. Ví dụ để chỉ nồng độ bạch cầu trong máu, trong BAĐT có thể được diễn đạt là ``WBC'', ``white blood cell count'' hoặc ``white blood count''. Việc xác định được các cụm từ có cách diễn đạt khác nhau nhưng cùng chỉ một khái niệm có thể giúp tăng độ chính xác và giải sai sót trong quá trình phân loại. Để làm được điều này, tác giả của hệ thống I đề xuất sử dụng các nguồn tri thức nhân loại bên ngoài như Wikipedia, UMLS hoặc WordNet.

Mặt khác, đối với lớp Problem/Test/Treatment, cùng một sự kiện y khoa có thể xảy ra nhiều lần nhưng các sự kiện y khoa đó có thể không đồng tham chiếu mà mang nhiều ý nghĩa khác nhau. Nguyên nhân vì trong các sự kiện y khoa thường bị ảnh hưởng bởi ngữ cảnh mà chúng được đề cập đến. Ví dụ ``the right \textit{leg}'' và ``the left \textit{leg}'', khái niệm được đề cập đều là ``leg'' tuy nhiên ngữ cảnh trong văn bản cho thấy đó là hai khái niệm chỉ hai chân khác nhau. Vì vậy để xây dựng chính xác chuỗi đồng tham chiếu của nhóm lớp Problem/Treatment/Test, dựa theo hệ thống I, cần phải có một hệ thống trích xuất thông tin ngữ cảnh trong văn bản.

Bảng \ref{tab:ProbTreatTestFeatures} liệt kê đầy đủ các đặc trưng được chúng tôi sử dụng trong việc huấn luyện mô hình SVM cho nhóm Problem/Treatment/Test. Tuy nhiên từng lớp Problem/Treatment/Test không sử dụng toàn bộ các đặc trưng về ngữ nghĩa trong văn bản. Ví dụ đặc trưng về thuốc y tế chỉ được sử dụng cho lớp Treatment mà không sử dụng cho lớp Problem và Test. Bảng \ref{tab:SemanticFeatures} mô tả chi tiết các đặc trưng ngữ nghĩa được sử dụng cho từng lớp khái niệm. Các đặc trưng được mô tả đầy đủ trong phần Các nguồn tri thức nhân loại và phần Bộ trích xuất ngữ nghĩa trong văn bản.

\subsection*{Các nguồn tri thức nhân loại}

\subsubsection*{Wikipedia}
Wikipedia là hệ thống bách khoa toàn thư miễn phí đa ngôn ngữ, nền Web dựa trên mô hình cho phép người dùng chỉnh sửa nội dung. Các tri thức nhân loại từ Wikipedia có thể được sử dụng để xác định tên giả (alias), tên viết tắt hoặc các từ đồng nghĩa hay có liên quan với nhau. Ví dụ hai khái niệm ``head trauma'' và ``head injury'' có thể được xác định là đồng nghĩa dựa trên thông tin từ Wikipedia.

Dựa trên thiết kế của hệ thống I, chúng tôi xây dựng bộ trích xuất các thông tin về tiêu đề bài viết (title hoặc redirected link), từ in đậm (bold name) trong bài viết và các khái niệm liên quan (anchor link) được đề cập trong bài viết. Ngôn ngữ của Wikipeda được chúng tôi sử dụng là tiếng Anh . Để trích xuất được các thông tin này, chúng tôi sử dụng công cụ Wikipedia-miner được đề cập trong phần \ref{tools}. Do số lượng khái niệm cần được rút trích thông tin từ Wikipedia rất lớn, vì vậy để cải thiện hiệu năng thời gian của hệ thống, chúng tôi đề xuất trước khi thực hiện rút trích đặc trưng của nhóm lớp Problem/Treatment/Test, các khái niệm thuộc nhóm lớp này cần được rút trích riêng thông tin từ Wikipedia và ghi xuống file lưu trữ.

\subsubsection*{UMLS}
Trong quá trình phát triển của y học, nhu cầu có một bộ từ điển chung, chứa các thông tin đã được chuẩn hóa về các loại bệnh, thuốc, nguyên nhân hoặc các thuật ngữ trong y tế ngày càng cao. Vì lí do đó, bộ từ điển UMLS - Unified Medical Language System đã được Mỹ tiến hành xây dựng từ năm 1986 bởi Thư viện Y học Quốc gia Hoa Kì (US National Library of Medicine).

Trong mô tả của hệ thống I, một số đặc trưng về ngữ nghĩa trong văn bản cần khai thác thông tin từ bộ từ điển UMLS. Cụ thể bộ trích xuất ngữ nghĩa trong văn bản cần xác định một khái niệm có xuất hiện trong bộ từ điển UMLS hay không, nếu có thì khái niệm đó thuộc phân nhóm nào trong từ điển UMLS. Ví dụ như khái niệm ``lung cancer'' xuất hiện trong từ điển UMLS dưới phân nhóm Các tiến trình liên quan u, bướu (Neoplastic Process). Để trích xuất các thông tin này, chúng tôi sử dụng công cụ MetaMap được đề cập trong phần \ref{tools}. Tương tự như việc rút trích thông tin từ Wikipedia, chúng tôi đề xuất các khái niệm cần được rút trích thông tin từ UMLS và ghi xuống file lưu trữ trước khi tiến hành rút trích đặc trưng cho nhóm lớp Problem/Treatment/Test.

\subsubsection*{WordNet}
WordNet là bộ từ điển được xây dựng tiếng Anh bao gồm các từ vựng đã được phân loại vai trò ngữ pháp như danh từ, tính từ, trạng từ, động từ. Các từ xuất hiện trong WordNet đã được nhóm lại với nhau thành các nhóm từ có cùng ý nghĩa. WordNet có vai trò giống như Wikipedia trong việc xác định tên giả, viết tắt và từ đồng nghĩa.

\subsection*{Các bộ trích xuất ngữ nghĩa trong văn bản}

\subsubsection*{Trích xuất thông tin cơ quan trên cơ thể (Anatomy)}
Trong BAĐT, hai triệu chứng bệnh giống nhau có thể không đồng tham chiếu nếu chúng xuất hiện tại các cơ quan cơ thể khác nhau. Ví dụ trong hai khái niệm sau ``a \textit{thrombosis} of the left subclavian vein'' và ``\textit{thrombosis} of the left internal jugular vein''. Khái niệm đang được xét cùng là ``thrombosis'' (chứng huyết khối) tuy nhiên, triệu chứng này lại xuất hiện tại hai cơ quan khác nhau là ``subclavian vein'' (tĩnh mạch dưới đòn) và ``jugular vein'' (tĩnh mạch cổ). Vì vậy dựa theo tập kết quả, hai khái niệm trên không đồng tham chiếu với nhau.

Để phân biệt các khái niệm như ví dụ trên, chúng tôi hiện thực bộ rút trích thông tin cơ quan trên cơ thể dựa trên bộ từ điển UMLS. Trong UMLS, các khái niệm được chia vào nhiều phân nhóm khác nhau dựa theo ý nghĩa của chúng. Cụ thể, chúng tôi sử dụng MetaMap để giới hạn việc nhận dạng các khái niệm thuộc các phân nhóm sau ``Anatomical Structure'', ``Body Location or Region'', ``Body Part, Organ, or Organ Component'', ``Body Space or Junction'' và ``Body System''. Các phân nhóm trên đều là các phân nhóm được định nghĩa sẵn trong từ điển UMLS. Sau khi nhận dạng được các thông tin cơ quan trên cơ thể, chúng tôi so trùng chuỗi của cơ quan nhận dạng được.

\subsubsection*{Trích xuất thông tin vị trí (Position)}
Một số cơ quan trên cơ thể tuy giống nhau về mặt ngữ nghĩa nhưng lại được phân biệt bằng các tính từ chỉ vị trí đi kèm. Ví dụ như ``\textit{burning sensation} in the \underline{upper right leg}'' và ``\textit{burning sensation} in the \underline{lower right leg}''. Cùng một triệu chứng là cảm giác bỏng rát (burning sensation) ở chân phải (right leg), nhưng lại xuất hiện ở hai vị trí khác nhau là phía trên của chân (upper) và phía dưới của chân (lower). Vì vậy hai khái niệm không đồng tham chiếu với nhau.

Từ việc khảo sát tập dữ liệu, chúng tôi xây dựng tập từ điển gồm các từ chỉ vị trí như ``left'', ``right'', ``upper'', ``lower'', ``back'', ``front'', ... Các thông tin vị trí được rút trích thông qua việc tìm kiếm sự xuất hiện của từ trong bộ từ điển. Tuy nhiên, chúng tôi nhận thấy rằng việc so trùng thông tin vị trí chỉ mang tính chất tương đối, một số khái niệm như ``left upper leg'' và ``left leg'' vẫn có thể đồng tham chiếu với nhau mặc dù vị trí ``left upper'' và ``left'' là khác nhau.

\subsubsection*{Trích xuất thông tin thuốc y tế (Medical information)}
Các thông tin về thuốc y tế là đặc trưng quan trọng đối với lớp Treatment. Thông tin về thuốc y tế bao gồm tên thuốc (drug name, ví dụ như Morphine, Anti-biotic,...), đường hấp thụ (mode, ví dụ như qua đường miệng, tiêm, qua hậu môn,...), liều lượng (dosage, ví dụ như 10mg, 2 viên,...), tần số sử dụng (frequency, ví dụ như 2 lần một ngày, mỗi ngày,...), thời gian sử dụng (duration, ví dụ như 3 tuần, 1 tháng,...). Khi các thông tin trên có giá trị khác nhau, hai khái niệm sẽ không đồng tham chiếu với nhau. Để rút trích thông tin thuốc y tế, chúng tôi sử dụng công cụ MedEx được đề cập trong phần \ref{tools}.

\subsubsection*{Trích xuất thông tin chỉ định của thủ tục y tế (Indicator)}
Một thủ tục y tế (Test) thường đi kèm với một hoặc một vài thông tin chỉ định mô tả chi tiết thủ tục đó, ví dụ nồng độ bạch cầu (WBC), nồng độ hồng cầu (RBC), nồng độ sắt trong máu (HCT),... Khi hai khái niệm lớp Test chứa thông tin chỉ định khác nhau sẽ không đồng tham chiếu với nhau. Ví dụ khái niệm ``100 WBC'' và ``130 WBC'' tuy cùng thông tin chỉ định là đo nồng độ bạch cầu trong máu, nhưng có kết quả xét nghiệm khác nhau nên hai khái niệm trên không đồng đám chiếu. Một ví dụ khác là ``100 WBC'' và ``100 RBC'', tuy hai khái niệm có cùng kết quả xét nghiệm nhưng thông tin chỉ định là khác nhau thì hai khái niệm trên cũng không đồng tham chiếu. Như vậy cần thiết phải xác định rõ thông tin chỉ định cũng như giá trị kết quả khi phân giải đồng tham chiếu cho các khái niệm thuộc lớp Test.

Qua việc khảo sát tập dữ liệu, chúng tôi nhận thấy các thông tin chỉ định đa số là xác định nồng độ các hợp chất, tế bào, thành phần cơ thể,... hoặc các giá đo lường. Để xác định các thông tin chỉ định này, chúng tôi sử dụng MetaMap để nhận dạng khái niệm thuộc các phân nhóm liên quan như ``Quantitative Concept'', ``Pharmacologic Substance'', ``Element, Ion, or Isotope'',... Sau xác định được thông tin chỉ định, chúng tôi kết hợp với biểu thức chính quy để tìm kết quả xét nghiệm.

\subsubsection*{Trích xuất thông tin không gian (Spatial)}
Cũng như thời gian, không gian là một ngữ cảnh quan trọng mà ta cần phải xét tới khi phân giải đồng tham chiếu các khái niệm thuộc lớp Problem/Treatment/Test. Ví dụ như cùng một loại thuốc nhưng một khái niệm diễn ra ở phòng phẫu thuật và một khái niệm diễn ra ở phòng hồi sức thì hai khái niệm này không đồng tham chiếu với nhau. Các thông tin về không gian được nhóm thành 4 nhóm: tại nhà bệnh nhân, tại bệnh viện chuyển tiếp (transfer hospital), tại bệnh viện điều trị (inpatient hospital) và tại chuyên khoa riêng biệt (individual department). Hệ thống I đề xuất sử dụng phân đoạn thông tin và tập từ khóa để trích xuất thông tin về không gian. Tuy nhiên, chúng tôi không tìm được công cụ hoặc cách hiện thực cụ thể trong giai đoạn luận văn tốt nghiệp. Vì vậy trong quá trình hiện thực hệ thống, chúng tôi chưa đưa đặc trưng này vào sử dụng.

\subsubsection*{Trích xuất thông tin thời gian (Temporal)}
Thông tin thời gian là một thông tin mạnh cho phân giải đồng tham chiếu ở ba lớp Problem/Treatment/Test. Một thủ tục y tế được tiến hành ở hai thời điểm khác nhau thì không đồng tham chiếu hoặc cùng một loại thuốc nhưng được kê khai ở các thời điểm khác nhau thì độc lập. Thông tin thời gian này được chia làm hai loại. Thời gian tường minh, ví dụ 03/06/2015. Thời gian suy diễn, là các mốc thời gian được trích xuất từ một số từ khóa đứng gần khái niệm đang xét, như ``Admission date'' (ngày nhập viện) hay ``Post-op day 2'' (ngày thứ 2 sau khi phẫu thuật).

Để hiện thực, chúng tôi thực hiện các bước sau. Xác định câu văn chứa khái niệm đang được xét. Sử dụng công cụ MedTime được đề cập trong phần \ref{tools} để nhận dạng thời gian xuất hiện trong câu văn. Công cụ MedTime có thể nhận dạng của thời gian tường minh và chuẩn hóa thời gian suy diễn thành giá trị ngày tháng cụ thể. Nếu trong câu văn xuất hiện hai giá trị thời gian khác nhau, chúng tôi sẽ sử dụng giá trị thời gian nằm gần khái niệm đang xét nhất. Ngoài ra, chúng tôi nhận thấy một số thông tin thời gian được tách riêng và nằm trong câu văn độc lập. Ví dụ trong ngày ``03/06/2015'' có ba thủ tục y tế được tiến hành, khái niệm thời gian sẽ được miêu tả ở dòng đầu tiên, theo sau là mô tả ba thủ tục y tế trong ba câu văn riêng biệt. Để giải quyết vấn đề ngày, chúng tôi đề xuất nếu câu văn chứa khái niệm không có thông tin về thời gian, hệ thống sẽ xem xét 3 câu văn liền trước để cố gắng tìm kiếm thông tin thời gian liên quan.

\subsubsection*{Trích xuất thông tin phân đoạn bệnh án (Section)}
Một hồ sơ xuất viện thường được chia làm các phân đoạn (section) như: tiền sử bệnh, tiền sử dùng thuốc, tiền sử nhập viện,... Hai khái niệm xuất hiện ở hai phần khác nhau thì thường không đồng tham chiếu với nhau cho dù chúng có cùng cách viết. Ví dụ cụm “CT scan” xuất hiện ở phần tiền sử điều trị và phần xét nghiệm thể chất thì hai khái niệm “CT scan” này là độc lập với nhau.

Mỗi BAĐT thường được viết dựa trên cấu trúc của từng khoa trong bệnh viện. Tùy vào chức năng, đặc trưng của từng khoa mà BAĐT của các khoa sẽ bao gồm các phân đoạn khác nhau. Vì vậy tập hợp các phân đoạn trong BAĐT rất đa dạng và không có sẵn đầy đủ. Để chọn ra danh sách tên các phân đoạn có trong tập dữ liệu, chúng tôi dựa trên một số luật rút ra từ việc khảo sát dữ liệu \cite{RandolphMiller2008}. Những luật đó bao gồm: các đầu mục phải kết thúc bằng dấu hai chấm (:), các tên phân đoạn thường được viết hoa toàn bộ hoặc viết hoa các chữ cái đầu tiên. Từ các luật được nêu ra, chúng tôi tạo danh sách toàn bộ các cụm từ có khả năng là tên của phân đoạn trong BAĐT. Sau đó, chúng tôi chọn 15 tên phân đoạn có tần số xuất hiện nhiều nhất.

\subsubsection*{Trích xuất thông tin bổ từ (Modifier)}
Tính đồng tham chiếu của một số khái niệm thuộc lớp Test có thể được xác định bởi các từ bổ nghĩa đi kèm theo chúng. Ví dụ các từ “recent”, “prior” hay “initial”. Tập từ khóa của thông tin bổ từ được chúng tôi xây dựng thông qua việc khảo sát dữ liệu.

\subsubsection*{Trích xuất thông tin thiết bị y tế (Equipment)}
Thiết bị y tế được sử dụng cũng là một gợi ý ngữ nghĩa cho việc xác định phân giải đồng tham chiếu. Lí do vì các thủ tục y tế thường được đặt tên theo thiết bị sử dụng. Để hiện thực đặc trưng này, chúng tôi sử dụng MetaMap để nhận diện khái niệm có xuất hiện trong UMLS hay không. Nếu có xuất hiện thì khái niệm có kết thúc bằng các tiếp vĩ ngữ như ``-graphy'', ``-gram'', ``-metry'' hoặc ``-scopy'' hay không. Nếu hai điều kiện trên được thỏa thì khái niệm đang xét được coi là một thiết bị y tế.

\subsubsection*{Trích xuất thông tin phẫu thuật y tế (Operation)}
Phần lớn các khái niệm thuộc lớp Treatment được đề cập là các phẫu thuật y tế. Vì vậy đặc trưng thông tin phẫu thuật y tế có thể xem là một gợi ý cho việc phân giải đồng tham chiếu lớp Treatment. Các phẫu thuật y tế được xác định bằng cách: khái niệm có tồn tại trong bộ từ điển UMLS và kết thúc bằng các tiếp vĩ ngữ ``-tomy'', ``-plasty''.

\subsubsection*{Trích xuất thông tin giả định (Assertion)}
Thông tin giả định cho các khái niệm lớp Problem là bài toán từ thách thức I2B2 2010. Các thông tin giả định bao gồm: không liên quan đến bệnh nhân (not associated with the patient), có thể (possible), thiếu vắng (absent), ở hiện tại (present), có điều kiện (conditional) và giả thuyết (hypothetical). Ví dụ ``no \textit{fever}'', khái niệm cơn sốt (fever) được giả định rằng không có (absent). Một ví dụ khác là ``Doctors suspect \textit{an infection of the lungs}''. Khái niệm sự lây nhiễm của phổi (an infection of the lungs) được giả định rằng có thể tồn tại. Tuy nhiên do cách hiện thực được đề xuất cho đặc trưng này là xây dựng hệ thống học máy dựa trên các giải pháp cho Thách thức I2B2 2010. Trong giai đoạn luận văn, chúng tôi không có đủ thời gian hiện thực đặc trưng này.

\section{Gom cụm và xây dựng chuỗi đồng tham chiếu}
Ở mô hình cặp thực thể, hệ thống phân loại không có khả năng xây dựng chuỗi đồng tham chiếu mà nó chỉ có thể xác định một cặp khái niệm là có đồng tham chiếu hay không. Mặt khác, đối với một văn bản HSXV, số cặp khái niệm được sinh ra rất nhiều và trong số đó có nhiều cặp có chung khái niệm đứng sau, ví dụ hai cặp (“Dr. John”, “his”) và (“Mr. Brown”, “his”) có chung khái niệm đứng sau là “his” mà hai cặp này đều được hệ thống phân loại xác định là đồng tham chiếu, tuy nhiên chỉ một trong hai khái niệm “Dr. John” và “Mr. Brown” được chọn làm tiền đề cho khái niệm “his” này. Như vậy cần thiết phải có một giải thuật lựa chọn các cặp đồng tham chiếu và xây dựng các chuỗi đồng tham chiếu từ chúng.

Như đã được đề cập ở mục, có hai giải thuật được đề xuất là: \emph{gom cụm gần nhất trước} và \emph{gom cụm tốt nhất trước}. Chúng tôi lựa chọn thực hiện giải thuật gom cụm tốt nhất trước cho hệ thống của mình vì hai lý do:
\begin{enumerate}
\item Theo các tác giả của giải thuật \cite{VincentNg2002}, giải thuật gom cụm tốt nhất trước cho kết quả tốt hơn giải thuật gom cụm gần nhất trước.
\item Các tác giả hệ thống \cite{YanXu2012} cũng hiện thực giải thuật này cho hệ thống của họ.
\end{enumerate}

Về cơ bản, giải thuật gom cụm tốt nhất trước lựa chọn các cặp khái niệm được xác định là đồng tham chiếu và có độ tin cậy cao nhất ứng với mỗi hồi chỉ; đối với đại từ, giải thuật sử dụng module phân loại xác định lớp chính của đại từ đó và tạo một cặp đồng tham chiếu giữa nó với khái niệm thuộc lớp tương ứng gần nhất trước đó (theo thứ tự xuất hiện) trong văn bản HSXV. Sau khi có được tập các cặp đồng tham chiếu, giải thuật nối các cặp có một khái niệm chung lại để tạo thành các chuỗi. Đây chính là kết quả cuối cùng của hệ thống phân giải.

Như vậy giải thuật gom cụm cụm tốt nhất trước bao gồm hai bước chính: chọn lọc các cặp đồng tham chiếu tốt nhất và ghép nối các cặp được chọn để tạo thành danh sách các chuỗi đồng tham chiếu. Bước đầu tiên làm việc như sau:
\begin{enumerate}
\item Nhận đầu vào là văn bản $E$ và danh sách các khái niệm $C$ được sắp xếp theo thứ tự xuất hiện. Gọi $M$ là số khái niệm trong $C$. $P$ là danh sách các cặp khái niệm, khởi tạo $P=\emptyset$.
\item Duyệt toàn bộ danh sách $C$ từ đầu đến cuối theo thứ tự xuất hiện ($j=1\rightarrow M$), với mỗi khái niệm $C_j$, nếu $C_j$ thuộc một trong 4 lớp Person, Problem, Test hoặc Treatment thì tới bước 3, nếu $C_j$ thuộc lớp Pronoun thì tới bước 4. Nếu đã duyệt xong, đi tới bước 6.
\item Với mỗi cặp $(C_i,\,C_j)$ mà $C_i$ đứng trước $C_j$ ($i<j$), chọn ra cặp $(C_k,\,C_j)$ mà hệ thống phân loại xác định là đồng tham chiếu và có độ tin cậy cao nhất, đưa cặp $(C_k,\,C_j)$ tới bước 5.
\item Sử dụng hệ thống phân loại xác định lớp chính $T$ của $C_j$. Duyệt từ khái niệm ngay trước $C_j$ về đầu ($i=j-1\rightarrow 1$), nếu $C_i$ thuộc lớp $T$, ngừng duyệt và đưa cặp $(C_i,\,C_j$) tới bước 5.
\item Nhận cặp $(C_i,\,C_j)$ được chọn ở một trong hai bước trên, đưa cặp vào $P$ và quay trở lại bước 2.
\item Trả $P$ về làm kết quả.
\end{enumerate} 

Sau khi đã có được danh sách các cặp đồng tham chiếu tốt nhất từ bước trên, hệ thống ghép nối các cặp trong danh sách lại để tạo thành các chuỗi đồng tham chiếu:
\begin{enumerate}
\item Nhận đầu vào là danh sách cặp khái niệm $P$.
\item Khởi tạo danh sách chuỗi $H=\emptyset$.
\item Với mỗi cặp $(C_i,\,C_j)$ trong $P$, duyệt tất cả các chuỗi $K$ trong $H$, nếu tìm được một chuỗi $K^{\prime}$ mà chứa một trong hai khái niệm $C_i$ hoặc $C_j$ của cặp thì đưa $C_i$, $C_j$ vào $K^{\prime}$ và quay lại bước 3. Nếu không tìm được chuỗi nào thỏa điều kiện trên, đưa cặp $(C_i,\,C_j)$ tới bước 4. Nếu đã duyệt xong, tới bước 5.
\item Khởi tạo một chuỗi $K^{\prime}$ mới gồm hai khái niệm của cặp, $K^{\prime}=\{C_i,\,C_j\}$, đưa chuỗi vào $H$ và quay lại bước 3.
\item Trả $H$ về làm kết quả.
\end{enumerate}

\begin{table}[ht]
\centering\ra{1.2}
\caption{Phân chia đặc trưng được sử dụng trong ba lớp Problem/Treatment/Test \label{tab:SemanticFeatures}}
\footnotesize\sffamily

\begin{tabularx}{\textwidth}{@{}l *5{>{\centering\arraybackslash}X}@{}}
\toprule 
& \textbf{Problem} & \textbf{Treatment} & \textbf{Test}\\
\midrule
Anatomy & + & & +\\
Position & + & + & +\\
Medication & & + & \\
Indicator & & & +\\
Temporal & & + & +\\
Spatial & & + & \\
Section & + & + & +\\
Modifier & & & +\\
Equipment & & & +\\
Operation & & + & \\
Assertion & + & & \\
\bottomrule
\end{tabularx}
\end{table}

\begin{table}[th]
\centering
\caption{Tập đặc trưng cho lớp Problem/Treatment/Test \label{tab:ProbTreatTestFeatures}}
\footnotesize\sffamily

\begin{tabularx}{\textwidth}{@{}P{\raggedright}{0.3}lL@{}}
\toprule 
\textbf{Đặc Trưng} & \textbf{Giá trị} & \textbf{Giải thích}\\
\midrule
\textit{Tri thức nhân loại}\\
Wiki page match & 0, 1 & Hai khái niệm cùng dẫn đến một trang Wiki\\
Wiki anchor match & 0, 1 & Trang Wiki của một khái niệm chứa liên kết đến trang wiki của khái niệm còn lại\\
Wiki bold name match & 0, 1 & Trang Wiki của một khái niệm chứa từ in đậm chỉ khái niệm còn lại\\
WordNet match & 0, 1 & Hai khái niệm được xác định đồng nghĩa trên WordNet\\
\textit{Trích xuất thông tin ngữ nghĩa trong văn bản}\\
Anatomy & 0, 1, 2 & Không cùng cơ quan cơ thể (0), cùng cơ quan (1), không xác định (2)\\
Position & 0, 1, 2 & Không cùng vị trí (0), cùng vị trí (1), không xác định (2)\\
Indicator & 0, 1, 2 & Không cùng thông tin chỉ định (0), cùng thông tin chỉ định (1), không xác định (2)\\
Temporal & 0, 1, 2 & Không cùng thời gian (0), cùng thời gian (1), không xác định (2)\\
Spatial & 0, 1, 2 & Không cùng không gian (0), cùng không gian(0), không xác định (2)\\
Section & 1, 2,..., $n^{2}$ & Hai khái niệm thuộc về phân đoạn i và j của văn bản\\
Equipment & 0, 1, 2 & Không cùng thiết bị (0), cùng thiết bị (1), không xác định (2)\\
Operation & 0, 1, 2 & Không cùng phẫu thuật (0), cùng phẫu thuật (1), không xác định (2)\\
Assertion & 0, 1, 2,..., $6^{2}$ & Trạng thái lớp khẳng định của hai khái niệm\\
\textit{Trích xuất thông tin thuốc y tế}\\
Drug & 0, 1 & Cùng tên thuốc (1), ngược lại (0)\\
Mode & 0, 1, 2,..., 29 & Chỉ mục của 29 đường hấp thụ hoặc không xác định (29)\\
Dosage & 0, 1 & Cùng liều lượng (1), ngược lại (1)\\
Duration & 0, 1 & Cùng khoảng thời gian sử dụng (1), ngược lại (0)\\
Frequency & 0, 1 & Cùng tần suất sử dụng (1), ngược lại (0)\\
``List'' or ``Narrative'' & 0, 1 & Cùng là dạng liệt kê hoặc tường thuật (1), ngược lại (0)\\
Time of first mention & 0, 1, 2, 3 & Thời gian của lần đề cập đầu tiên: Quá khứ (0), hiện tại (1), tương lai (2), không xác định (3)\\
Time of second mention & 0, 1, 2, 3 & Thời gian của lần đề cập đầu tiên: Quá khứ (0), hiện tại (1), tương lai (2), không xác định (3)\\
Episode of first mention & 0, 1, 2, 3 & Chương của lần đề cập đầu tiên: bắt đầu (0), tiếp diễn (1), tạm ngưng (2), không xác định (3)\\
Episode of second mention & 0, 1, 2, 3 & Chương của lần đề cập đầu tiên: bắt đầu (0), tiếp diễn (1), tạm ngưng (2), không xác định (3)\\
Condition of first mention & 0, 1, 2, 3 & Tình trạng của lần đề cập đầu tiên: khẳng định (0), gợi ý (1), có điều kiện (2), không xác định (3)\\
Condition of second mention & 0, 1, 2, 3 & Tình trạng của lần đề cập đầu tiên: khẳng định (0), gợi ý (1), có điều kiện (2), không xác định (3)\\
\textit{Khoảng cách}\\
Sentence distance & 0, 1, 2,... & Số câu xuất hiện giữa hai khái niệm được xét\\
\textit{Ngữ pháp}\\
Article & 1, 2,..., $3^{2}$ & Trạng thái các từ hạn định (determiner) đứng trước hai khái niệm, bao gồm 3 loại: (a|an), (the|his|her…) hoặc không có (NULL)\\
\textit{So trùng chuỗi}\\
Head noun match & 0, 1 & Cùng danh từ trung tâm (1), ngược lại (0)\\
Contains & 0, 1 & Một khái niệm chứa toàn bộ chuỗi của khái niệm còn lại\\
Capital match & 0, 1 & Các kí tự đầu tiên trùng nhau (1), ngược lại (0)\\
Substring match & 0, 1 & Có cùng chuỗi con (1), ngược lại (0)\\
Cos distance & (0, 1) & Khoảng cách cos (góc) giữa hai khái niệm
\textit{Ngữ nghĩa}\\
Word match & 0, 1 & Hai khái niệm trùng chuỗi hoàn toàn\\
Procedure match & 0, 1 & Có chứa từ ``Procedure'' (1), ngược lại (0)\\
\bottomrule
\end{tabularx}
\end{table}

\chapter{Thí nghiệm đánh giá}
\section{Tập dữ liệu}
Tập dữ liệu chúng tôi sử dụng để đánh giá hiệu năng hệ thống được cung cấp bởi Partners Healthcare và Beth Israel Deaconess Medical Center. Đây cũng chính là tập được sử dụng ở tác vụ 1C thách thức i2b2 2011 để đánh giá các hệ thống tham gia. Như vậy để có được tập dữ liệu này, chúng tôi đã liên hệ với tổ chức i2b2 và cam kết về các điều khoản sử dụng dữ liệu (Data Usage Agreement) bao gồm việc chỉ sử dụng cho mục đích nghiên cứu. Bản cam kết cần được kí và gửi lại cho i2b2 qua email hoặc fax.

Tập dữ liệu chúng tôi nhận được bao gồm \emph{251 hồ sơ xuất viện} dùng cho huấn luyện và \emph{173 hồ sơ} dùng cho kiểm tra đánh giá. Trong đó, ngoài các hồ sơ xuất viện là các văn bản thuần được ghi chép lại bởi các bác sĩ, y tá thì mỗi hồ sơ còn đi kèm với một tập tin chứa danh sách các khái niệm được đề cập trong hồ sơ đó và đã được gán nhãn bởi các chuyên gia y tế theo mẫu: \emph{c=``<khái niệm>'' <bắt đầu> <kết thúc>||t=``<lớp>''}. Ví dụ \emph{c=``the patient'' 20:5 20:6||t=``person''} mô tả khái niệm ``the patient'' xuất hiện bắt đầu từ dòng 20 kí tự thứ 5, kết thúc ở dòng 20 kí tự thứ 6 và thuộc lớp Person.

Ngoài danh sách khái niệm, mỗi hồ sơ còn đi kèm với một tập tin chứa danh sách các chuỗi đồng tham chiếu đã được phân giải bởi các chuyên gia (\emph{các chuỗi kết quả}) nhằm huấn luyện hệ thống phân loại có giám sát cũng như để đánh giá hiệu năng của hệ thống phân giải. Các chuỗi đồng tham chiếu có định dạng: \emph{c=``<khái niệm 1>'' <bắt đầu> <kết thúc>||c=``khái niệm 2'' <bắt đầu> <kết thúc>||...||t=``coref <lớp>''}, trong đó \emph{t=``coref <lớp>''} mô tả lớp chính mà các chuỗi đồng tham chiếu chỉ tới bao gồm Person, Problem, Test và Treatment.

\section{Các phương pháp đánh giá}
Hiệu năng hệ thống được đánh giá qua ba độ đo: \emph{độ đúng đắn} (precision), \emph{độ đầy đủ} (recall) và \emph{độ F} (F-measure). Bài báo đánh giá các hệ thống dự thi thử thách i2b2 sử dụng ba phương pháp khác nhau để tính toán các độ này, bao gồm: MUC, B-CUBED và CEAF. Mỗi hệ đo nêu trên đánh giá các chuỗi đồng tham chiếu xuất ra bởi hệ thống cho từng văn bản HSXV, vì vậy ứng với mỗi hệ đo, trung bình không trọng số của các độ đo tính trên toàn tập dữ liệu được lấy làm kết quả cuối cùng. Ngoài ra, các kết quả của chúng tôi còn được lấy để so sánh với hệ thống \cite{YanXu2012} nhằm xác định tính khả thi của phương pháp mà chúng tôi hiện thực.

\subsection*{Hệ đo MUC}
Hệ đo MUC \cite{MarcVilain1995} xem chuỗi đồng tham chiếu là một danh sách các liên kết giữa các cặp khái niệm tạo nên chuỗi, từ đó đánh giá hệ thống dựa trên số lượng ít nhất các liên kết cần được thêm vào và loại bỏ để các chuỗi đồng tham chiếu xuất ra bởi hệ thống trùng với các chuỗi kết quả. Có thể hiểu số liên kết cần được loại bỏ là độ thiếu chính xác (precision \emph{error}) và số liên kết cần được thêm vào là độ thiếu đầy đủ (recall \emph{error}). 

Với mỗi văn bản $d$, gọi $G$ là tập các chuỗi kết quả, $S$ là tập các chuỗi hệ thống. Các độ đúng đắn ($P$) và độ đầy đủ ($R$) được tính như sau:
\[P_d=\frac{\sum_{s\in S} \left(|s| - m(s, G)\right)}{\sum_{s\in S}\left(|s| - 1\right)}\]
\[R_d=\frac{\sum_{g\in G}(|g|-m(g,S))}{\sum_{g\in G}(|g|-1)}\]

\noindent trong đó, $|s|$ là số khái niệm tạo thành chuỗi $s$, $m(s,G)$ được tính là tổng số chuỗi trong $G$ có giao nhau với $s$ cộng với số khái niệm trong $s$ không xuất hiện trong tất cả các chuỗi trong $G$. Độ $F$ của hệ MUC là trung bình điều hòa của độ chính xác và độ đầy đủ:
\[F_d=\frac{2\times P_d\times R_d}{P_d + R_d}\]

\subsection*{Hệ đo B-CUBED}
Khác với hệ đo MUC đánh giá dựa trên sự thiếu và thừa các liên kết trong chuỗi hệ thống, hệ đo B-CUBED đánh giá hiệu năng trên mỗi khái niệm trong văn bản. Theo nhận định của các tác giả hệ đo B-CUBED, cách đánh giá dựa trên các liên kết có hai nhược điểm \cite{AmitBagga1998}:

\begin{enumerate}[leftmargin=\parindent]
\item Không xét tới các \emph{khái niệm duy nhất} vì các liên kết chỉ tồn tại khi có ít nhất hai khái niệm. Mặt khác, theo quy ước, các chuỗi chỉ chứa một khái niệm không được đưa vào tập kết quả và các hệ thống tuân theo quy ước cũng không xuất những chuỗi như vậy ra.

\item Các lỗi của hệ thống được xem là như nhau, tức hệ đo MUC đánh giá cùng một mức phạt như nhau cho tất cả các sai sót của hệ thống. Tuy nhiên có thể nhận thấy là có vài loại sai sót làm cho chuỗi hệ thống bị sai lệch nhiều hơn so với các loại khác.
\end{enumerate}

Hệ đo B-CUBED được thiết kể để khắc phục hai nhược điểm trên bằng cách tính toán độ chính xác và độ đầy đủ cho từng khái niệm trong văn bản, sau đó kết hợp chúng lại để ra kết quả cuối cùng. Như vậy theo \cite{AmitBagga1998}, với khái niệm thứ $i$, độ chính xác và độ đầy đủ của nó được định nghĩa như sau:
\[P_i=\frac{\text{số khái niệm đúng trong chuỗi hệ thống chứa khái niệm thứ $i$}}{\text{số khái niệm trong chuỗi hệ thống chứa khái niệm thứ $i$}}\]
\[R_i=\frac{\text{số khái niệm đúng trong chuỗi hệ thống chứa khái niệm thứ $i$}}{\text{số khái niệm trong chuỗi kết quả chứa khái niệm thứ $i$}}\]

Độ chính xác và đầy đủ trên toàn văn bản được tính theo công thức:
\[P=\sum_{i=1}^{M} w_i \times P_i,\quad R=\sum_{i=1}^{M} w_i\times R_i\]

\noindent trong đó, $w_i$ là trọng số của khái niệm thứ $i$, $M$ là số khái niệm của văn bản đang xét. Bài báo đánh giá các hệ thông dự thi thử thách i2b2 năm 2011 sử dụng chung một trọng số cho tất cả các khái niệm, $w_i=1/M$. Để tiện cho việc so sánh với hệ thống \cite{YanXu2012}, chúng tôi cũng dùng cách gán trọng số như trên.

Như vậy để thuận tiện trong tính toán, nếu gọi $m$ là một khái niệm trong văn bản $d$ có chứa $M$ khái niệm, $G_m$ là chuỗi kết quả có chứa $m$, $S_m$ là chuỗi hệ thống có chứa $m$, $O_m$ là chuỗi giao nhau giữa $G_m$ và $S_m$ (tức $O_m=G_m\cap S_m$) thì độ đúng đắn và độ đầy đủ của hệ B-CUBED cho $d$ được tính như sau:
\[P_d=\frac{1}{M}\sum_{m\in d}\frac{|O_m|}{|S_m|},\quad R_d=\frac{1}{M}\sum_{m\in d}\frac{|O_m|}{|G_m|}\]

Độ F của hệ B-CUBED được tính tương tự như hệ MUC.

\subsection*{Hệ đo CEAF}
Hệ đo CEAF được tác giả Xiaoqiang Lou giới thiệu vào năm 2005 như một cách đánh giá khác để khắc phục các nhược điểm của hệ MUC \cite{XiaoquangLuo2005}. Thay vì tính toán các độ đo trên từng khái niệm như hệ B-CUBED, hệ CEAF đánh giá hệ thống bằng cách tối ưu hóa sự \emph{sắp xếp đầy đủ} giữa các chuỗi kết quả và chuỗi hệ thống dựa trên tổng các độ tương tự của từng cặp chuỗi, với điều kiện một chuỗi kết quả chỉ được ghép cặp với nhiều nhất một chuỗi hệ thống. Theo nhận định của tác giả hệ đo CEAF, sắp xếp tối ưu giúp ngăn ngừa việc ``ăn gian'' của các hệ thống phân giải: một hệ thống xuất ra quá nhiều chuỗi đồng tham chiếu sẽ bị đánh giá độ chính xác thấp, trong khi xuất ra quá ít chuỗi thì sẽ bị đánh giá độ đầy đủ thấp.

Với mỗi văn bản $d$, gọi $G=\{g_i:i=1,2,\dots,|G|\}$ là tập các chuỗi kết quả của $d$, $S=\{s_i:i=1,2,\dots,|S|\}$ là tập các chuỗi hệ thống xuất ra cho $d$. Vì vai trò như nhau của $S$ và $G$ trong hệ đo này, chúng tôi giả định rằng $|S|<|G|$. Theo định nghĩa của \cite{XiaoquangLuo2005}, một sự sắp xếp đầy đủ có thể được xem là một ánh xạ một-một đi từ $S$ vào $G$, $H=\{h:S\mapsto G\}$, thỏa hai điều kiện:
\begin{enumerate}
\item $\forall s \in S,\forall s^{\prime} \in S: s\neq s^{\prime} \Leftrightarrow h(s)\neq h(s^{\prime})$
\item $|H|=|S|$
\end{enumerate}

%\vspace{6pt}
Đặt $m=|S|,\, M=|G|$, $\mathcal{H}$ là toàn bộ những sắp xếp đầy đủ có thể giữa $S$ và $G$, dễ dàng tính được $|\mathcal{H}|=\binom{M}{m}m!$. Gọi $\phi(s,g)$ là độ tương tự giữa hai chuỗi $s$ và $g$ bất kì. Độ tương tự cho một sự sắp xếp đầy đủ $H\in\mathcal{H}$ giữa $S$ và $G$ được định nghĩa là tổng các độ tương tự giữa mỗi cặp $(s,h(s))$ trong $H$: $\Phi(H)=\sum_{s\in S} \phi(s,h(s))$. Như vậy một sự sắp xếp đầy đủ tối ưu giữa $S$ và $G$ chính là sự sắp xếp $H^*$ thỏa:
\begin{align*}
\Phi(H^*)&=\max_{H\in\mathcal{H}} \Phi(H)\\
&=\max_{H\in\mathcal{H}} \sum_{s\in S} \phi(s,h(s))
\end{align*}

Trong trường hợp $|S|>|G|$, vai trò của $S$ và $G$ được hoán đổi cho nhau ở các định nghĩa trên. Để tính độ tương tự giữa hai chuỗi đồng tham chiếu, tác giả hệ CEAF đề xuất bốn cách tính khác nhau:
\begin{align*}
\phi_1(s,g)=
\begin{cases}
	1 & \text{nếu } s = g\\
	0 & \text{nếu } s \neq g
\end{cases}&,\quad
\phi_2(s,g)=
\begin{cases}
	1 & \text{nếu } s\cap g\neq\emptyset\\
	0 & \text{các trường hợp khác}
\end{cases},\\
\phi_3(s,g)=|s\cap g|&,\quad\phi_4(s,g)=\frac{2|s\cap g|}{|s|+|g|}
\end{align*}

Theo nhận định của chính tác giả, $\phi_3$ và $\phi_4$ tỏ ra hiệu quả hơn trong việc đánh giá các chuỗi đồng tham chiếu. Mặt khác, thử thách i2b2 năm 2011 sử dụng $\phi_4$ để đánh giá các hệ thống dự thi nên để cho tiện trong việc so sánh với hệ thống \cite{YanXu2012}, chúng tôi cũng sử dụng $\phi_4$ để đánh giá hệ thống của mình.

Ngoài ra, việc tìm kiếm một sự sắp xếp tối ưu giữa hai tập chuỗi không thể được thực hiện bằng cách duyệt toàn bộ các sự sắp xếp đầy đủ có thể vì số các cách sắp xếp đầy đủ là rất lớn, $\binom{M}{m}m!$. Về mặt giải thuật, cách làm này được xem là có độ phức tạp hàm giai thừa. Tuy nhiên theo nhận định của tác giả CEAF, bài toán sắp xếp tối ưu này chính là bài toán giao công việc (\emph{assignment problem}) hay bài toán vận chuyển (\emph{transportation problem}) đã được giải quyết bởi Harold W. Kuhn vào năm 1955 với giải thuật có tên là \emph{phương pháp Hungary} \cite{HungarianMethod} có độ phức tạp là $\BigO(n^4)$.

Từ đây, hệ CEAF tính độ đúng đắn và độ đầy đủ cho một văn bản $d$ theo công thức:
\[P_d=\frac{\Phi(H^*)}{\sum_{s\in S} \phi(s,s)},\quad R_d=\frac{\Phi(H^*)}{\sum_{g\in G} \phi(g,g)}\]

Độ F của hệ CEAF được tính tương tự như hệ MUC.

\section{Kết quả}

\chapter{Tổng kết}

\section{Kết quả đạt được}
Sau quá trình nghiên cứu và tìm hiểu các phương pháp đồng tham chiếu cho các văn bản nói chung và bệnh án điện tử nói riêng, chúng tôi đã hoàn thành hệ thống phân giải đồng tham chiếu cho các HSXV tiếng Anh với độ F là 81.5\%. Các kiến thức chúng tôi thu thập được trong quá trình hiện thực luận án có thể làm nền tảng cho những phát triển sâu hơn của việc khai thác tri thức trong bệnh án điện tử trong tương lai, cụ thể là:
\begin{enumerate}
\item Đối với các văn bản nói chung, có ba mô hình được đề xuất để giải quyết bài toán phân giải đồng tham chiếu: mô hình cặp khái niệm, mô hình đề cập thực thể và mô hình xếp hạng. Mô hình cặp khái niệm có tư tương đơn giản nhất nhưng vì thế nó cũng có nhiều nhược điểm, vì vậy một số vấn đề cần phải được cân nhắc trong quá trình hiện thực như sự mất cân bằng lớp, giải thuật học máy, các đặc trưng được rút trích và các phương pháp gom cụm. Mô hình đề cập thực thể được đề xuất nhằm giải quyết vấn đề về sự độc lập của việc xác định tính đồng tham chiếu của các cặp khái niệm, bằng cách đưa vào các đặc trưng ở mức cụm. Trong khi đó mô hình xếp hạng lại nhắm tới vấn đề lựa chọn tiền đề, bằng cách xếp hạng các ứng cử viên để chọn ra tiền đề phù hợp nhất. 
\item Về vấn đề mất cân bằng lớp, có hai cách khác nhau để giải quyết việc này: chọn lọc mẫu (undersampling) hoặc tạo thêm mẫu (oversampling). Việc tạo thêm mẫu giúp cho các mẫu dương ngang bằng với các mẫu âm làm cho số lượng mẫu tăng lên, trong một bài toán có nhiều mẫu được sinh ra thì cách làm này khiến cho thời gian huấn luyện mô hình phân loại tăng lên rất nhiều, nhất là đối với SVM. Trong khi đó việc chọn lọc mẫu một cách tự động lại khiến cho nhiều mẫu chứa các thông tin quan trọng bị loại bỏ, dẫn tới việc mô hình phân loại có xu hướng phân loại về dương nhiều hơn mà bỏ quá sự chính xác của chúng.
\item Đối với bệnh án điện tử, nhiều đặc điểm chuyên biệt trong miền văn bản này giúp cho việc phân giải đồng tham chiếu được dễ dàng. Điển hình là thông tin về sự đề cập đến bệnh nhân, sự giống nhau về chuỗi kí tự của các khái niệm đồng tham chiếu chỉ về bệnh, thủ tục y tế và phương pháp điều trị hay các thông tin về ngữ cảnh y tế trong bệnh án. Ngoài ra đặc điểm một bệnh án chỉ đề cập tới một bệnh nhân cũng là nhân tố quan trọng trong việc phân giải đồng tham chiếu cho các khái niệm chỉ người ở miền văn bản này.
\end{enumerate}

Về mặt thực tiễn, tuy hệ thống của chúng tôi không được hiện thực cho các văn bản tiếng Việt, chúng tôi hy vọng rằng những kết quả đạt được ở luận án này sẽ đóng góp tích cực vào việc nghiên cứu và phát triển BAĐT ở Việt Nam.

\section{Hạn chế và hướng phát triển}
\subsection*{Hạn chế}
Ngoài những kết quả đạt được, hệ thống của chúng tôi vẫn còn một số hạn chế nhất định. Thứ nhất là thời gian chạy thử nghiệm tốn nhiều thời gian. Khó khăn này dẫn đến việc chúng tôi không có đủ thời gian để điều chỉnh hệ thống và thực hiện các thí nghiệm nhiều lần. Để cải thiện vấn đề này, chúng tôi đề xuất sử dụng các máy tính có khả năng tính toán mạnh trong quá trình huấn luyện hoặc áp dụng các kĩ thuật xử lý song song và hệ phân bố cho việc tìm kiếm lưới.

Thứ hai là hạn chế trong việc trích xuất các thông tin từ điển UMLS, Wikipedia và các thông tin ngữ cảnh cho các cặp khái niệm lớp Problem, Test và Treatment. Để rút ngắn thời gian huấn luyện cũng như chạy thử nghiệm trên tập kiểm tra, chúng tôi đã tiến hành rút trích các đặc trưng được nêu thành các tập tin được lưu trữ và sử dụng lại nhiều lần. Tuy nhiên cách hiện thực này làm giảm sự linh động của quá trình phân giải vì đối với các HSXV mới không có sẵn các tập tin này, chúng cần được trích xuất trước khi tiến hành phân giải.

\subsection*{Hướng phát triển}
Vì giới hạn thời gian hiện thực Luận Văn cùng với những khó khăn nêu trên, nhiều ý tưởng và hướng phát triển không được chúng tôi hiện thực trong phạm vi đề tài. Một trong số đó là phát triển để hệ thống hoạt động được với HSXV bằng tiếng Việt. Để làm được điều này chúng tôi cần thay thế các công cụ xử lý ngôn ngữ tự nhiên từ tiếng Anh qua tiếng Việt, đồng thời xây dựng lại các bộ từ điển cho các đặc trưng. Thông qua tìm hiểu, chúng tôi biết được hiện nay đã có một số công cụ hỗ trợ xử lý ngôn ngữ tự nhiên cho tiếng Việt như: công cụ JVnTextPro và Framework VNLP. Vì vậy chúng tôi tin rằng nếu tìm được tập dữ liệu tiếng Việt phù hợp, việc phát triển để hệ thống hoạt động được cho HSXV tiếng Việt là hoàn toàn khả thi.

Ngoài ra, chúng tôi cũng dự định hệ thống hiện tại với các hệ thống nhận diện thực thể để tạo thành một công cụ hoàn chỉnh giúp phân giải HSXV mà không cần danh sách các thực thể cho trước. Hiện nay đã tồn tại nhiều công cụ giúp thực hiện bước nhận diện thực thể như: \textit{CliNER} của nhóm Text Machine Lab cho các văn bản tiếng Anh hoặc kết quả từ luận văn \textit{Nhận dạng tên thực thể trong bệnh án điện tử}, được thực hiện bởi Đào Trọng Điệp và Lê Khắc Sinh, cho các văn bản tiếng Việt. Việc kết hợp yêu cầu thời gian tìm hiểu mã nguồn và cách sử dụng các công cụ nêu trên.


\clearpage
\phantomsection
\addcontentsline{toc}{chapter}{\bibname}
\bibliographystyle{ieeetr}
\bibliography{referrence}

\end{document}