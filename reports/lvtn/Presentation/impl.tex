\section{Hiện thực hệ thống}
%the problem
%\subsection{Nội dung bài toán}
%\begin{frame}{Nội dung bài toán}
%\putlogo
%Phân giải đồng tham chiếu cho các hồ sơ xuất viện tiếng Anh với các khái niệm đã được trích xuất và gán nhãn
%
%Đầu vào:
%\begin{itemize}
%	\item Tập các hồ sơ xuất viện (HSXV)
%	\begin{itemize}
%		\item Những văn bản lâm sàng mô tả thông tin điều trị bệnh nhân
%	\end{itemize}
%	\item Tập các khái niệm đã được trích xuất và gán nhãn
%	\begin{itemize}
%		\item Person
%		\item Problem/Test/Treatment
%		\item Pronoun
%	\end{itemize}
%\end{itemize}
%
%Đầu ra: 
%\begin{itemize}
%\item Danh sách các chuỗi đồng tham chiếu cho mỗi HSXV
%\end{itemize}
%\end{frame}

%system architecture
\subsection*{Ý tưởng hiện thực}
\begin{frame}{Ý tưởng hiện thực}
\putlogo
\begin{itemize}
	\item Mô hình \emph{cặp khái niệm}
	\begin{itemize}
		\item Xác định tính đồng tham chiếu cho từng cặp hai khái niệm
		\item Gom cụm các cặp đồng tham chiếu lại và xây dựng các chuỗi đồng tham chiếu ứng với mỗi cụm
	\end{itemize}
	\item Quy trình huấn luyện
	\begin{itemize}
		\item Huấn luyện các mô hình SVM phân loại tính đồng tham chiếu 
	\end{itemize}
	\item Quy trình phân giải
	\begin{itemize}
		\item Sử dụng các mô hình phân loại đã được huấn luyện
		\item Giải thuật \emph{gom cụm tốt nhất trước}
		\item[\boldmath$\rightarrow$] Các chuỗi đồng tham chiếu cho các HSXV mới
	\end{itemize}
\end{itemize}
\end{frame}

\begin{frame}{Quy trình huấn luyện}
\putlogo
\resizebox{\textwidth}{!}{
\begin{tikzpicture}[%
	>=angle 60,
	start chain=going below,
	node distance=1.5cm and 2cm,
	every join/.style={->, draw},
	font=\scriptsize\sffamily]

	\node[multidoc](emr){HSXV};
	\node[multidoc, right=of emr, text width=3.5em](con){Danh sách khái niệm};
	\node[proc,below=1.3cm of $(emr)!0.5!(con)$](prep){Tiền xử lý};
	\node[io,join]{Các khái niệm/cặp khái niệm};
	\node[proc,join](ex){Rút trích đặc trưng};
	\node[io, right=3cm of con, emph=<2>](perp){Đặc trưng cặp Person};
	\node[io, emph=<2>](prop){Đặc trưng cặp Problem};
	\node[io, emph=<2>](tstp){Đặc trưng cặp Test};
	\node[io, emph=<2>](trep){Đặc trưng cặp Treatment};
	\node[io, emph=<3>](pron){Đặc trưng Pronoun};	
	\node[io, emph=<4>](pati){Đặc trưng Person};
	\node[proc, right=2.3cm of $(trep)!0.5!(tstp)$](train){Huấn luyện};
	\node[io,right=6.3cm of perp, emph=<2>](perm){Mô hình SVM cặp Person};
	\node[io, emph=<2>](prom){Mô hình SVM cặp Problem};
	\node[io, emph=<2>](tstm){Mô hình SVM cặp Test};
	\node[io, emph=<2>](trem){Mô hình SVM cặp Treatment};
	\node[io, emph=<3>](pronm){Mô hình SVM Pronoun};
	\node[io, emph=<4>](patim){Mô hình SVM lớp Patient};

	\draw[->] (emr) -- ++(down:0.9cm) -| (prep.130);
	\draw[->] (con) -- ++(down:0.9cm) -| (prep.50);
	\draw[->] (ex) -- ++(right:2.3cm) |- (perp);
	\draw[->] (ex) -- ++(right:2.3cm) |- (pati);
	\draw[->] (ex) -- ++(right:2.3cm) |- (prop);
	\draw[->] (ex) -- ++(right:2.3cm) |- (tstp);
	\draw[->] (ex) -- ++(right:2.3cm) |- (trep);
	\draw[->] (ex) -- ++(right:2.3cm) |- (pron);
	\draw[->] (perp) -- ++(right:1.8cm) |- (train);
	\draw[->] (pati) -- ++(right:1.8cm) |- (train);
	\draw[->] (prop) -- ++(right:1.8cm) |- (train);
	\draw[->] (tstp) -- ++(right:1.8cm) |- (train);
	\draw[->] (trep) -- ++(right:1.8cm) |- (train);
	\draw[->] (pron) -- ++(right:1.8cm) |- (train);
	\draw[->] (train) -- ++(right:1.3cm) |- (perm);
	\draw[->] (train) -- ++(right:1.3cm) |- (patim);
	\draw[->] (train) -- ++(right:1.3cm) |- (prom);
	\draw[->] (train) -- ++(right:1.3cm) |- (tstm);
	\draw[->] (train) -- ++(right:1.3cm) |- (trem);
	\draw[->] (train) -- ++(right:1.3cm) |- (pronm);
\end{tikzpicture}
}
\end{frame}

\begin{frame}{Quy trình phân loại}
\putlogo
\centering
\resizebox{!}{0.95\textheight}{
\begin{tikzpicture}[%
	>=angle 60,
	start chain=going below,
	node distance=1.7cm and 2.7cm,
	every join/.style={->, draw},
	font=\scriptsize\sffamily]

	\node[doc](emr){HSXV};
	\node[doc, right=of emr, text width=3.5em](con){Danh sách khái niệm};
	\node[proc, below=1.3cm of $(emr)!0.5!(con)$](prep){Tiền xử lý};
	\node[io,join](ins){Các khái niệm/cặp khái niệm};
	\node[proc,join](ex){Rút trích đặc trưng};
	\node[io, right=of ex, join](feat){Tập vector đặc trưng};
	\node[proc, right=of feat, join](clas){Phân loại};
	\node[multidoc, above=of clas, text width=3.5em](model){6 mô hình SVM};
	\node[io,below=of clas](conf){Độ tin cậy đồng tham chiếu};
	\node[proc,join,below=of ex](clus){Gom cụm};
	\node[io,join](sysc){Các chuỗi đồng tham chiếu};
%	\node[proc,join,right=of sysc](eval){Đánh giá hiệu năng};
%	\node[doc,right=of eval, text width=2cm](gt){Các chuỗi đồng tham chiếu kết quả (ground truth)};
%	\node[io,below=of eval](res){Các số liệu độ đo};

	\draw[->] (emr) -- ++(down:0.9cm) -| (prep.130);
	\draw[->] (con) -- ++(down:0.9cm) -| (prep.50);
	\draw[->] (model) -> (clas);
	\draw[->] (clas) -> (conf);
	\draw[->] (ins) -- ++(left:1.8cm) |- (clus);
%	\draw[->] (gt) -> (eval);
%	\draw[->] (eval) -> (res);
	\draw[->] (model.200) -- ++(left:3cm) node[midway,above,sloped]{SVM Patient-class} -> (ex.north east);
\end{tikzpicture}
}
\end{frame}

%\subsection{Tiền xử lý}
%\begin{frame}{Tiền xử lý}
%\putlogo
%\begin{itemize}
%	
%\end{itemize}
%\end{frame}

%pair construction
\subsection*{Xây dựng các cặp khái niệm}
\begin{frame}{Xây dựng các cặp khái niệm}
\putlogo
\begin{itemize}
	\item Xây dựng các cặp mà hai khái niệm có chung lớp Person, Problem, Test hoặc Treatment
	\begin{itemize}
		\item Hai khái niệm khác lớp thì không đồng tham chiếu
		\item Giảm số cặp khái niệm được sinh ra
		\item Đặc trưng được rút trích chuyên biệt
	\end{itemize}
	\item \textbf{Vấn đề:} Số cặp \emph{không} đồng tham chiếu (mẫu âm) $\gg$ Số cặp đồng tham chiếu (mẫu dương)
	\begin{itemize}
		\item Xảy ra ở ba lớp Problem, Test và Treatment
		\item Mô hình SVM có xu hướng phân loại về âm
	\end{itemize}
	\item \textbf{Cách giải quyết:} Loại bỏ bớt mẫu âm theo phương pháp của Ng and Cardie (2002)
	\begin{itemize}
		\item Giữ lại các cặp hai \emph{khái niệm duy nhất} trùng chuỗi kí tự
		\begin{itemize}
			\item Khái niệm duy nhất: không đồng tham chiếu với bất kì khái niệm nào khác
		\end{itemize}
	\end{itemize}
\end{itemize}
\end{frame}

%\begin{frame}{Xây dựng các cặp khái niệm}
%\putlogo
%\resizebox{\textwidth}{!}{
%\begin{tikzpicture}[%
%	>=angle 60,
%	start chain=going right,
%	node distance=1.5cm and 2cm,
%	out=60, in=120,
%	font=\sffamily]	
%	\tikzset{
%		con/.style={base, rectangle, inner sep=1mm, minimum height=0.8cm, minimum width=0.8cm},
%		sgt/.style={base, circle, inner sep=1mm, minimum height=0.8cm}
%	};
%
%	\node[con, emph2=<2-4>](C1){$C_1$};
%	\node[con, emph2=<6>](D1){$D_1$};
%	\node[con, emph=<6>](D2){$D_2$};
%	\node[sgt, emph=<5>](a){$a$};
%	\node[sgt, emph=<5>](b){$b$};
%	\node[con, emph=<4>](C2){$C_2$};
%	\node[con, emph=<1-3>](C3){$C_3$};
%	
%	\path[->, looseness=0.3] (C1.north) edge (C2.north);
%	\path[->, looseness=0.7] (D1.north) edge (D2.north);
%	\path[->, looseness=0.7] (C2.north) edge (C3.north);
%	
%\end{tikzpicture}
%}
%
%\begin{minipage}[t][5cm]{0.24\textwidth}
%\uncover<3->{
%\begin{itemize}
%	\item $(C_2\,,C_3)$
%	\item $(b\,,C_3)$
%	\item $(a\,,C_3)$
%	\item $(D_2\,,C_3)$
%	\item $(D_1\,,C_3)$
%	\item $(C_1\,,C_3)$
%\end{itemize}
%}
%\end{minipage}
%\begin{minipage}[t][5cm]{0.24\textwidth}
%\uncover<4->{
%\begin{itemize}
%	\item $(b\,,C_2)$
%	\item $(a\,,C_2)$
%	\item $(D_2\,,C_2)$
%	\item $(D_1\,,C_2)$
%	\item $(C_1\,,C_2)$
%\end{itemize}
%}
%\end{minipage}
%\begin{minipage}[t][5cm]{0.24\textwidth}
%\uncover<6->{
%\begin{itemize}
%	\item $(D_1\,,D_2)$
%\end{itemize}
%}
%\end{minipage}
%\begin{minipage}[t][5cm]{0.24\textwidth}
%\uncover<7->{
%\begin{itemize}
%	\item $(a\,,b)$
%\end{itemize}
%}
%\end{minipage}
%\end{frame}

%person extraction
\subsection*{Đặc trưng lớp Person}
\begin{frame}{Đặc trưng lớp Person}{Cặp khái niệm lớp Person}
\putlogo
\begin{itemize}
	\item \emph{Đặc tính chung}: thường là tên riêng, chức danh và đại từ nhân xưng
	\begin{itemize}
		\item Khoảng cách: câu, khái niệm
		\item So trùng chuỗi
		\item Số lượng, giới tính, ...
	\end{itemize}
	\item Trong \emph{lĩnh vực y tế}: các khái niệm chỉ về bệnh nhân có số lượng nhiều nhất $\rightarrow$ đặc trưng Patient-class:
	\begin{itemize}
		\item Bằng 1 khi cả 2 khái niệm đều chỉ về bệnh nhân
		\item Bằng 0 trong các trường hợp khác
	\end{itemize}

\end{itemize}
\end{frame}

\begin{frame}{Đặc trưng lớp Person}{Xác định đặc trưng Patient-class}
\putlogo
\begin{itemize}
	\item Có thể xác định dựa vào các {\color{red} từ khóa:}
	\begin{itemize}
		\item Bệnh nhân: patient, pt, mr, ...
		\item Người thân: wife, sibling, brother, ...
		\item Nhân sự bệnh viên: doctor, dr, dentist, ...
	\end{itemize}
	\item Thuộc nhóm đại từ chỉ giới tính chiếm đa số
\end{itemize}
\end{frame}

%pronoun extraction
\subsection*{Đặc trưng lớp Pronoun}
\begin{frame}{Đặc trưng lớp Pronoun}
\putlogo
\begin{itemize}
	\item Cần được xác định đại từ chỉ vào lớp nào trong 4 lớp chính Person, Problem, Treatment hoặc Test
	\item Chứa ít thông tin ngữ nghĩa trong bản thân khái niệm
	\item Đặc trưng từ vựng, ngữ pháp:
	\begin{itemize}
		\item Vai trò ngữ pháp: đại từ, nghi vấn từ, sở hữu từ, ...
		\item Các từ liền trước và liền sau
	\end{itemize}
	\item Đặc trưng ngữ cảnh:
	\begin{itemize}
		\item Lớp của khái niệm liền trước và liền sau
	\end{itemize}
\end{itemize}
\end{frame}

%problem/treatment/test extraction
\subsection*{Đặc trưng lớp Problem/Treatment/Test}
\begin{frame}{Đặc trưng lớp Problem/Treatment/Test}
\putlogo
\begin{itemize}
	\item Hai khái niệm đồng tham chiếu thường {\color{red} trùng chuỗi}
	\item Nhiều \emph{từ đồng nghĩa}
	\begin{itemize}
		\item ``Head injury'' và ``Head trauma''
	\end{itemize}
	\item \emph{Ngữ cảnh trong văn bản} khiến cho 2 khái niệm \emph{không} đồng tham chiếu
	\begin{itemize}
		\item ``Pain in \underline{right leg}'' và ``Pain in \underline{left leg}''
	\end{itemize}
\end{itemize}
\end{frame}

\begin{frame}{Đặc trưng lớp Problem/Treatment/Test}
\putlogo
\begin{itemize}
	\item So trùng chuỗi, khoảng cách, ...
	\item Tiêu đề, từ in đậm và liên kết của {\color{red} Wikipedia}
	\item Trích xuất thông tin ngữ cảnh trong văn bản:
	\begin{itemize}
		\item Cơ quan trên cơ thể
		\item Vị trí: left, right, upper, ...
		\item Thuốc y tế: tên thuốc, liều lượng, thời gian, ...
		\item Kết quả của xét nghiệm
		\item Thời gian
		\item Mục bệnh án
	\end{itemize}
\end{itemize}
\end{frame}

\subsection*{Gom cụm}
\begin{frame}{Gom cụm và xây dựng chuỗi đồng tham chiếu}
\putlogo
\begin{itemize}
\item Sử dụng giải thuật \emph{gom cụm tốt nhất trước}
\end{itemize}
\begin{center}
\begin{tikzpicture}[%
	>=angle 60,
	start chain=going right,
	node distance=1.5cm and 4.5cm,
	every join/.style={->, draw}]
	\node[proc, text width=7em,inner sep=2mm](step1){Lựa chọn các cặp đồng tham chiếu tốt nhất};
	\node[proc,join,text width=7em,inner sep=2mm](step2){Gom các cặp có chung một khái niệm};
	\node[draw,rectangle callout,callout absolute pointer={(step1.south)},below=0.7cm of $(step1.south)!0.5!(step2.south)$, text width=0.95\textwidth, visible on=<2->](step1note){
		\vspace{-3pt}
		\begin{itemize}
			\item Đối với nhóm lớp Person/Problem/Test/Treatment
			\begin{itemize}
				\item Với mỗi khái niệm $C_j$, chọn khái niệm $C_i$ đứng trước có độ tin cậy đồng tham chiếu cao nhất
			\end{itemize}
			\item Đối với lớp Pronoun
			\begin{itemize}
				\item Xác định lớp chính của đại từ và tạo một cặp gồm đại từ đó và khái niệm thuộc cùng lớp ở gần nhất trước đó
			\end{itemize}
		\end{itemize}		
	};
\end{tikzpicture}
\end{center}
%\begin{itemize}
%\item Sử dụng giải thuật \emph{gom cụm tốt nhất trước}
%\item Với nhóm lớp Person/Problem/Test/Treatment
%	\begin{itemize}
%		\item Với mỗi khái niệm $C_j$, chọn khái niệm $C_i$ đứng trước có độ tin cậy đồng tham chiếu cao nhất
%	\end{itemize} 
%\item Với lớp Pronoun
%	\begin{itemize}
%		\item Xác định lớp chính của đại từ và tạo một cặp gồm đại từ đó và khái niệm thuộc lớp chính ở gần nhất trước đó
%	\end{itemize}
%\item Gom các cặp được chọn mà có chung một khái niệm để tạo thành các cụm
%	\begin{itemize}
%	\item[\boldmath$\rightarrow$] Mỗi cụm ứng với một chuỗi đồng tham chiếu
%	\end{itemize}
%\end{itemize}
\end{frame}
