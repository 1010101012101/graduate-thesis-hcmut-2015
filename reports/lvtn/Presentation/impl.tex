\section{Hiện thực hệ thống}
%the problem
\subsection{Nội dung bài toán}
\begin{frame}{Nội dung bài toán}
\putlogo
Phân giải đồng tham chiếu cho các hồ sơ xuất viện tiếng Anh với các khái niệm đã được trích xuất và gán nhãn

Đầu vào:
\begin{itemize}
	\item Tập các hồ sơ xuất viện (HSXV)
	\begin{itemize}
		\item Những văn bản lâm sàng mô tả thông tin điều trị bệnh nhân
	\end{itemize}
	\item Tập các khái niệm đã được trích xuất và gán nhãn
	\begin{itemize}
		\item Person
		\item Problem/Test/Treatment
		\item Pronoun
	\end{itemize}
\end{itemize}

Đầu ra: 
\begin{itemize}
\item Danh sách các chuỗi đồng tham chiếu cho mỗi HSXV
\end{itemize}
\end{frame}

%system architecture
\subsection{Kiến trúc hệ thống}
\begin{frame}{Kiến trúc hệ thống}
\putlogo
\end{frame}

%pair construction
\subsection{Xây dựng các cặp khái niệm}
\begin{frame}{Xây dựng các cặp khái niệm}
\putlogo
\end{frame}

%person extraction
\subsection{Đặc trưng lớp Person}
\begin{frame}{Đặc trưng lớp Person}
\putlogo
\begin{itemize}
	\item Thường là tên riêng, chức danh và đại từ nhân xưng
	\item Gồm 3 nhóm chính: bệnh nhân, người thân bệnh nhân và nhân sự bệnh viện
	\begin{itemize}
		\item Nhóm bệnh nhân có số lượng khái niệm nhiều nhất
		\item Xác định khái niệm thuộc nhóm nào đóng vai trò quan trọng
	\end{itemize}
	\item Các khái niệm chỉ bệnh nhân được đưa vào cùng một chuỗi đồng tham chiếu
	\item Các đặc trưng được sử dụng bao gồm:
	\begin{itemize}
		\item Đặc trưng lớp bệnh nhân
		\item Nhóm đặc trưng ngữ nghĩa
		\item Nhóm đặc trưng ngữ pháp
		\item Nhóm đặc trưng khoảng cách
		\item Nhóm đặc trưng từ vựng
	\end{itemize}
\end{itemize}
\end{frame}

%pronoun extraction
\subsection{Đặc trưng lớp Pronoun}
\begin{frame}{Đặc trưng lớp Pronoun}
\putlogo
\begin{itemize}
	\item Là các khái niệm có thể chỉ về bất kì khái niệm thuộc bốn lớp Person, Problem, Treatment và Test
	\begin{itemize}
		\item Ví dụ: which, this, that,...
	\end{itemize}
	\item Cần được xác định lớp khái niệm mà đại từ ám chỉ
	\item Đại từ đang xét được ghép chung vào chuỗi đồng tham chiếu của khái niệm gần nhất trước đó có cùng lớp khái niệm
	\item Các đặc trưng được sử dụng bao gồm:
	\begin{itemize}
		\item Nhóm đặc trưng quan hệ
		\item Nhóm đặc trưng ngữ nghĩa
		\item Nhóm đặc trưng ngữ pháp
		\item Nhóm đặc trưng khoảng cách
	\end{itemize}
\end{itemize}
\end{frame}

%problem/treatment/test extraction
\subsection{Đặc trưng lớp Problem/Treatment/Test}
\begin{frame}{Đặc trưng lớp Problem/Treatment/Test}
\putlogo
\begin{itemize}
	\item Là nhóm khái niệm chỉ bất thường sức khỏe, phương pháp chữa trị và thủ tục y tế
	\item Chịu ảnh hưởng mạnh bởi ngữ cảnh văn bản và chứa nhiều từ ngữ chuyên môn
	\begin{itemize}
		\item Các khái niệm có cùng chuỗi nhưng không đồng tham chiếu do thông tin ngữ cảnh khác nhau
		\item Cùng một thực thể nhưng có nhiều cách diễn đạt bằng nhiều hình thức khác nhau
	\end{itemize}
	\item Các đặc trưng về tri thức nhân loại:
	\begin{itemize}
		\item Wikipedia
		\item UMLS
	\end{itemize}
\end{itemize}
\end{frame}

\begin{frame}{Đặc trưng lớp Problem/Treatment/Test}
\begin{itemize}
	\item Các đặc trưng về ngữ nghĩa văn bản:
	\begin{itemize}
		\item Thông tin cơ quan trên cơ thể (Anatomy)
		\item Thông tin vị trí (Position)
		\item Thông tin thuốc y tế (Medical information)
		\item Thông tin chỉ định của thủ tục y tế (Indicator)
		\item Thông tin thời gian (Temporal)
		\item Thông tin phân đoạn bệnh án (Section)
		\item Thông tin bổ từ (Modifier)
		\item Thông tin thiết bị y tế (Equipment)
		\item Thông tin phẫu thuật y tế (Operation)
	\end{itemize}
	\item Ngoài ra các đặc trưng cơ bản cũng được sử dụng
	\begin{itemize}
		\item Ví dụ nhóm đặc trưng khoảng cách, ngữ nghĩa,...
	\end{itemize}
\end{itemize}
\end{frame}
