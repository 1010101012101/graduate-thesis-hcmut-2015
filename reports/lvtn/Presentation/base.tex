\section{Các định nghĩa và thuật ngữ}
\begin{frame}{Các định nghĩa và thuật ngữ}
\putlogo
\emph{Khái niệm}
\begin{itemize}
	\item Những đề cập trong văn bản cần phân giải đồng tham chiếu
\end{itemize}
\emph{Sự đồng tham chiếu}
\begin{itemize}
	\item Hai hay nhiều khái niệm cùng chỉ tới một thực thể trong thế giới thực
	\item Phụ thuộc vào ngữ cảnh
	\item Là một quan hệ tương đương
	\begin{itemize}
		\item Tính phản xạ
		\item Tính đối xứng 
		\item Tính bắc cầu
	\end{itemize}
\end{itemize}
\end{frame}

\begin{frame}{Các định nghĩa và thuật ngữ}
\putlogo
\emph{Cặp khái niệm}
\begin{itemize}
	\item Hai khái niệm có thể có hoặc không đồng tham chiếu với nhau
	\item Một cặp khái niệm đồng tham chiếu
	\begin{itemize}
		\item Khái niệm đứng trước: \emph{tiền đề}
		\item Khái niệm đứng sau: \emph{hồi chỉ}
	\end{itemize}
	\item Nhiều khái niệm cùng chỉ đến một thực thể: \emph{chuỗi đồng tham chiếu}
\end{itemize}
\emph{Khái niệm duy nhất}
\begin{itemize}
	\item Không đồng tham chiếu với bất kì khái niệm nào khác
\end{itemize}
\emph{Phân giải đồng tham chiếu}
\begin{itemize}
	\item Xác định những chuỗi đồng tham chiếu trong văn bản
\end{itemize}
\end{frame}
