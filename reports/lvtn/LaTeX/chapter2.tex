\chapter{Các công trình liên quan}
\section{Bệnh án điện tử}
Bệnh án là văn bản ghi chép các thông tin sức khỏe của một cá nhân trong quá trình khám và chữa bệnh. Bệnh án điện tử chính là bệnh án được số hóa bằng HTTBV. BAĐT thông thường chứa những dữ liệu cơ bản cho quản lý, các dữ liệu cận lâm sàng và lâm sàng của người bệnh trong một lần nằm viện \cite{HoTuBao2015}. Dữ liệu lâm sàng là những \textit{văn bản lâm sàng} (clinical text) do bác sĩ và y tá ghi chép hàng ngày về thông tin khám và chữa bệnh của người bệnh. Các văn bản lâm sàng trong bệnh án điện tử chủ yếu gồm ba loại \cite{HoTuBao2015}:

\begin{enumerate}
\item \emph{Phiếu điều trị} (doctor daily notes): ghi chép các chuẩn đoán, nhận định và y lệnh hàng ngày của bác sĩ về bệnh nhân.
\item \emph{Phiếu chăm sóc} (nurse narratives): là những ghi chép trong ngày của y tá trong quá trình chăm sóc và thực hiện y lệnh của bác sĩ.
\item \emph{Hồ sơ xuất viện} (discharge summary): toàn bộ dữ liệu và thông tin cơ bản của bệnh nhân trong một lần điều trị.
\end{enumerate}

Dữ liệu trong BAĐT thường tồn tại dưới dạng tường thuật, ghi chép của bác sĩ hoặc y tá, đây là dạng dữ liệu không có cấu trúc. Một số thông tin hữu ích có trong dữ liệu không cấu trúc của BAĐT là:

\begin{itemize}
\item Lý do nhập viện, lịch sử điều trị, tiền sử thuốc sử dụng.
\item Ghi chép của bác sĩ hoặc y tá trong quá trình điều trị hàng ngày.
\item Các kết quả xét nghiệm.
\item Các ghi chú về quá trình phẫu thuật.
\end{itemize}

Ngoài các văn bản lâm sàng được lưu trữ dưới dạng phi cấu trúc, một số tiêu chuẩn được đưa ra để nhằm giúp cấu trúc hóa các BAĐT như:

\begin{itemize}
\item \emph{IDC} (International Classification of Diseases): bao gồm các loại mã cũng như thông tin về bệnh như tên bệnh, mô tả, triệu chứng, dấu hiệu hay mức độ, v.v...
\item \textit{CPT} (Current Procedural Terminology): bao gồm các mã mang tính thủ tục trong bệnh viện như mã xét nghiệm, gây tê, phẫu thuật, X quang, thuốc hay cấp cứu, v.v...
\end{itemize}

Tại các nước phát triển, BAĐT đã được quan tâm phát triển trong hơn một thập kỉ qua. Năm 2004, Viện y tế Quốc gia Hoa Kì (NIH: National Institute of Health) đã kêu gọi thành lập một mạng lưới nghiên cứu cấp quốc gia về y sinh. Để đáp lại lời kêu gọi đó, bảy Trung tâm nghiên cứu công nghệ tính toán y sinh (NBCB: National Center for Biomedical Computing) đã được thành lập dưới sự tài trợ của NIH với nhiệm vụ xây dựng cơ sở hạ tầng phục vụ cho việc áp dụng khoa học máy tính vào lĩnh vực y sinh, hỗ trợ cho công việc nghiên cứu. Trong đó, i2b2 (Informatics for Integrating Biology and the Bedside), một NBCB được thành lập bởi sự hợp tác giữa hai trường đại học nổi tiếng là Havard và MIT, bắt đầu từ năm 2006 đã tổ chức các cuộc thi hàng năm nhằm tìm kiếm các phương pháp phân tích và rút trích kiến thức trên dữ liệu BAĐT, gọi tắt là các Thách thức (Challenges). Mỗi Thách thức đưa ra một vấn đề phân tích và một tập dữ liệu BAĐT được cung cấp bởi các bệnh viện trong và ngoài nước Mỹ. Hàng năm có trên dưới 100 nhóm nghiên cứu tham gia đề xuất giải pháp và gửi kết quả phân tích, trong đó những giải pháp tốt được chọn lọc để công bố ở một hội thảo quốc tế và được áp dụng rộng rãi vào các dịch vụ chăm sóc sức khỏe. 

Vào năm 2009, ngay sau khi trở thành tổng thống Hoa Kỳ, Barack Obama đã yêu cầu chuẩn hóa và số hóa mọi bệnh án của các bệnh viện trong vòng 5 năm. Ở Nhật Bản, các bệnh viện lớn và vừa cũng được chính phủ tạo điều kiện để xây dựng BAĐT. Tính đến năm 2011, khoảng 34.7\% bệnh viện lớn và vừa tại Nhật đã có hệ thống BAĐT sử dụng được \cite{HoTuBao2015}. Một số ví dụ thực tiễn trong việc ứng dụng BAĐT vào dự đoán, điều trị bệnh trên thế giới là hệ thống dự đoán nguy cơ mắc bệnh đái tháo đường loại 2 từ cấu trúc gen \cite{AbelKho2012} hoặc hệ thống cho phép nghiên cứu diện rộng bệnh tâm thần và cách điều trị chứng phiền muộn \cite{Perlis2012}.

Tại Việt Nam, các HTTBV cũng đang dần được triển khai, tiêu biểu là Bệnh viện đa khoa Vân Đồn tỉnh Quảng Ninh--cơ quan y tế đầu tiên có trang bị hệ thống bệnh án điện tử hiện đại và hoàn chỉnh với giải pháp MEDI SOLUTIONS của công ty phần mềm Hoa Sen. Cùng với việc xây dựng, tập thể nghiên cứu ``Học máy và ứng dụng'' của viện John von Neumann thuộc đại học Quốc Gia TP Hồ Chí Minh đã tiến hành phát triển các phương pháp và phần mềm phục vụ cho khai thác bệnh án điện tử tiếng Việt.

Nguyên nhân BAĐT nhận được nhiều sự quan tâm như vậy là vì BAĐT không những thuận tiện hơn bệnh án giấy trong việc lưu giữ các thông tin và tri thức thu thập được trong quá trình khám chữa bệnh mà còn cho phép chia sẻ nguồn thông tin đó giữa các bệnh viện, các thành phố hoặc giữa các quốc gia với nhau. Thông qua chia sẻ, nhiều BAĐT được đối chiếu và phân tích để phát hiện những tri thức y học mới hoặc kiểm chứng những kiến thức đã có. BAĐT đóng vai trò quan trọng trong sự phát triển của việc khám chữa bệnh cũng như nghiên cứu trong y học.

\section{Nhận dạng thực thể có tên}
Rút trích thông tin (information extraction), một trong những vấn đề của xử lý ngôn ngữ tự nhiên, là công việc tự động rút trích thông tin từ những dữ liệu không có cấu trúc hoặc dữ liệu bán cấu trúc. Dữ liệu có cấu trúc là dữ liệu máy tính hiểu hoàn toàn, thông thường nằm ở dạng bảng hoặc trong các hệ quản trị dữ liệu quan hệ. Dữ liệu không có cấu trúc là dữ liệu máy tính hoàn toàn không hiểu, như ngôn ngữ tự nhiên. Dữ liệu bán cấu trúc là dữ liệu chứa các thẻ hoặc các hình thức đánh dấu khác giúp phân tách bộ phận ngữ cảnh nền ra khỏi dữ liệu, điển hình là các ngôn ngữ đánh dấu như XML, JSON, HTML.

Tác vụ rút trích thông tin gồm hai bước con là nhận dạng thực thể và rút trích quan hệ. Trong đó, nhận dạng thực thể là bước đầu tiên của bài toán rút trích thông tin. Nhận dạng thực thể, hay nhận dạng thực thể có tên, là xác định các thực thể được đề cập trong văn bản và phân loại chúng vào các lớp khái niệm được định nghĩa sẵn, trong đó khái niệm có tên là các cụm từ có chứa tên của con người, tổ chức hay nơi chốn \cite{KimSang2003}. Ví dụ đầu ra của bước nhận dạng thực thể từ câu văn ``Duy Hưng là sinh viên đại học Bách Khoa của thành phố Hồ Chí Minh'' là:

\begin{itemize}
\item ``Duy Hưng'' - Con người
\item ``đại học Bách Khoa'' - Tổ chức
\item ``thành phố Hồ Chí Minh'' - Nơi chốn
\end{itemize}

Bài toán nhận dạng thực thể có tên thường bao gồm 2 bước: xác định thực thể và phân loại thực thể vào các nhóm ngữ nghĩa \cite{KimSang2003}. Trong đó, bước đầu tiên của bài toán thường được xem đơn giản như là một bài toán phân mảnh các từ trong câu thành các ``tên'', với ``tên'' là một chuỗi các từ liên tục có ý nghĩa và chỉ tới một thực thể có thật.

Các hệ thống nhận diện thực thể có tên nếu hoạt động tốt trong một lĩnh vực chuyên biệt (như y tế, địa chất, ký sự) thì sẽ cho kết quả không tốt nếu đem ứng dụng vào lĩnh vực khác. Việc chỉnh sửa cho một hệ thống có sẵn để hoạt động tốt trong một lĩnh vực mới thường tiêu tốn nhiều công sức.

Tùy theo mỗi lĩnh vực quan tâm cụ thể, các loại thực thể sẽ được định nghĩa khác nhau. Với những vấn đề không đặc thù, những nhóm thực thể thường được nhắc đến là: động vật, con người, tổ chức hay nơi chốn v.v... Khi nghiên cứu về nhận dạng thực thể trong bệnh án điện tử, có năm loại thực thể cần được quan tâm là: vấn đề về sức khỏe (Problem), phương pháp điều trị (Treatment), thủ tục y tế (Test), con người (Person) và đại từ (Pronoun).

Năm 2010, trung tâm i2b2 đưa ra Thách thức về vấn đề xử lý ngôn ngữ tự nhiên cho các văn bản y tế lâm sàng bao gồm ba tác vụ:

\begin{enumerate}
\item Trích xuất và nhận dạng các thực thể có tên trong y học.
\item Phân loại bệnh vào một trong các dạng: đang xảy ra ở hiện tại, không xảy ra ở hiện tại, có thể xảy ra trong tương lai, ...
\item Rút trích các quan hệ giữa các bệnh, phương pháp điều trị và thủ tục y tế.
\end{enumerate}

Đối với Thách thức này, i2b2 tập trung vào giải quyết nhóm bài toán rút trích thông tin vì đây là nhóm bài toán nền tảng, tạo tiền đề để nghiên cứu cho các hướng đi khác. Tuy Thách thức i2b2 năm 2010 có đề cập đến việc rút trích các quan hệ giữa các thực thể trong bệnh án (tác vụ thứ 3), nhưng mối quan hệ đồng tham chiếu lại không được bao gồm trong số đó. Chính vì thế, năm 2011, i2b2 tổ chức Thách thức lần thứ 5 dành riêng cho việc giải quyết vấn đề phân giải đồng tham chiếu trên dữ liệu BAĐT với đầu vào là kết quả nhận diện thực thể từ Thách thức năm 2010. Vấn đề được nêu trong Thách thức i2b2 2011 cũng chính là vấn đề được chúng tôi giải quyết trong nội dung luận án và được trình bày chi tiết trong các phần sau.