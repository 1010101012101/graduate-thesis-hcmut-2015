\chapter{Tổng quan}
\section{Giới thiệu đề tài\label{gioithieudetai}}
Trong hơn mười năm trở lại đây, với sự bùng nổ của kỉ nguyên công nghệ thông tin, việc số hóa dữ liệu trở nên phổ biến hơn bao giờ hết, và bệnh án cũng không phải là ngoại lệ. Bệnh án điện tử đã và đang dần thay thế cho phương pháp ghi chép và lưu trữ truyền thống thông tin của bệnh nhân trong quá trình khám và chữa bệnh. Hầu hết bệnh viện ở những nước phát triển đã triển khai các \emph{hệ thông tin bệnh viện} (HTTBV) để phục vụ cho việc số hóa loại tài liệu này.

%Bệnh án điện tử (BAĐT) là bệnh án được số hóa bằng các công cụ hiện đại. BAĐT chứa đầy đủ các thông tin cơ bản của một bệnh án như: dữ liệu quản lý, dữ liệu cận lâm sàng và lâm sàng. So với bệnh án được lưu trữ bằng giấy, BAĐT có nhiều ưu điểm như: lưu trữ chính xác và đầy đủ thông tin bệnh nhân, hỗ trợ quá trinh tìm kiếm và truy xuất thông tin, dữ liệu có thể được chia sẻ hoặc tích hợp. Qua nhiều năm sử dụng, BAĐT có thể thu thập được một lượng lớn dữ liệu liên quan đến bệnh, triệu chứng và cách điều trị. Lượng dữ liệu này vô cùng quý giá cho việc nghiên cứu y tế. Vì vậy, bên cạnh việc xây dựng các hệ thống BAĐT, khai thác dữ liệu chứa trong BAĐT cũng là một vấn đề quan trọng.

Bên cạnh việc xây dựng bệnh án điện tử (BAĐT) thì việc khai thác nguồn dữ liệu lớn
này cũng đang là một lĩnh vực rất được quan tâm trong những năm gần đây. Nền tảng của việc khai thác này là rút trích và cấu trúc hóa thông tin trong các văn bản thô. Một trong những vấn đề của bài toán rút trích thông tin là phân giải đồng tham chiếu. Một cách tổng quát, phân giải đồng tham chiếu cho một văn bản là xác định liệu hai hay nhiều sự đề cập trong văn bản có ám chỉ tới cùng một sự vật hoặc hiện tượng hay không, từ đó xây dựng các chuỗi đồng tham chiếu. Khi mà đa phần các văn bản được viết tay bằng ngôn ngữ tự nhiên, chứa đựng rất nhiều các khái niệm phụ thuộc vào ngữ cảnh thì việc phân giải đồng tham chiếu giúp cho máy tính có một cái nhìn mang tính cấu trúc và nhiều ngữ nghĩa hơn về văn bản, từ đó làm nền tảng cho việc rút trích các kiến thức sâu từ những hiểu biết này.

Tuy vấn đề về phân giải đồng tham chiếu trong những năm gần đây đã được quan tâm nghiên cứu rất nhiều cho các loại văn bản khác (ví dụ các bài báo) thì ở phạm vi BAĐT vấn đề này vẫn còn ít được quan tâm, đặc biệt là ở Việt Nam. Đứng trước nhu cầu đó, chúng tôi quyết định bắt tay vào phát triển một hệ thống phân giải đồng tham chiếu cho các văn bản BAĐT tiếng Anh, làm nền tảng cho các nghiên cứu trên dữ liệu tiếng Việt sau này. Đầu vào của hệ thống bao gồm: 
\begin{enumerate}
\item Các văn bản BAĐT ở dạng thô chưa qua xử lý.
\item Ứng với mỗi văn bản BAĐT là một tập tin chứa những khái niệm đã được nhận dạng và gán nhãn xuất hiện trong bệnh án.
\end{enumerate}

Trong đó việc nhận dạng và gán nhãn các khái niệm, hay gọi tắt là nhận dạng thực thể, là việc trích xuất các từ/cụm từ trong bệnh án điện tử và gán nhãn dựa theo ý nghĩa mà chúng đề cập đến, bao gồm con người, các thực thể y tế như các vấn đề về sức khỏe, các thủ tục y tế và các phương pháp điều trị. Kết quả đầu ra của hệ thống là các chuỗi đồng tham chiếu được phân giải cho một văn bản, trong đó mỗi chuỗi chứa ít nhất hai khái niệm chỉ về cùng một người hay một thực thể y tế như đã nêu.

\section{Mục tiêu và phạm vi đề tài}
Những ích lợi BADT mang lại như đề cập ở trên đã tạo động lực cho chúng tôi tiến hành hiện thực một hệ thống rút trích thông tin cho BAĐT. Cụ thể mục tiêu của chúng tôi là xây dựng thành công một hệ thống phân giải đồng tham chiếu, với hy vọng sẽ góp phần hỗ trợ cho các công trình nghiên cứu sâu hơn về BAĐT cũng như được dùng để xây dựng các công cụ thống kê hoặc các hệ thống truy xuất thông tin trong y tế.

Vì giới hạn thời gian, chúng tôi quyết định chỉ hiện thực hệ thống phân giải đồng tham chiếu cho các hồ sơ xuất viện được viết bằng tiếng Anh với danh sách các thực thể đã được xác định trước. Trung tâm i2b2\footnote{Informatics for Integrating Biology and the Bedside, \url{https://www.i2b2.org/}} cung cấp một tập dữ liệu là các báo cáo xuất viện đã được phân giải đồng tham chiếu bởi các chuyên gia y tế. Tập dữ liệu này được cung cấp miễn phí kèm theo một số cam kết sử dụng dữ liệu (Data Use Agreement), đây là tập dữ liệu vô cùng giá trị cho việc huấn luyện hệ thống của chúng tôi. 

Dựa trên kết quả của Thách thức i2b2 năm 2011 \cite{OzlemUzuner2012}, vấn đề phân giải đồng tham chiếu trong hồ sơ xuất viện có hai hướng giải quyết đạt kết quả tốt là: hướng tiếp cận \emph{dựa trên luật} (rule-based, hay còn gọi là hướng tiếp cận về ngôn ngữ học) và hướng tiếp cận dựa trên \emph{học máy} (machine learning). Chúng tôi quyết định hiện thực hệ thống dựa trên hệ thống sử dụng học máy có kết quả tốt nhất trong Thách thức 2011 của các tác giả Yan Xu et al. \cite{YanXu2012}. Trong đó, giải thuật học máy được áp dụng là Support Vector Machine và mô hình phân giải đồng tham chiếu được áp dụng là mô hình \emph{cặp khái niệm}.

\section{Cấu trúc luận văn}
Toàn bộ nội dung luận văn được chúng tôi trình bày thành sáu chương. Các chương này nêu lên những kiến thức cần thiết, chi tiết cách thức hiện thực để xây dựng và hoàn thiện hệ thống phân giải đồng tham chiếu trong BAĐT cũng như các kết quả đánh giá hệ thống này. Ở chương cuối, chúng tôi đưa ra các tổng kết về kết quả đạt được của luận án, các hạn chế và các hướng phát triển trong tương lai. Sau đây là nội dung chính của mỗi chương:

\begin{description}[style=nextline,leftmargin=0cm]
\item[Chương 1: Tổng quan] Chương đầu tiên chúng tôi nêu lên mục tiêu, động cơ và phạm vi của luận án. Toàn bộ chương này giúp người đọc có được cái nhìn toàn cảnh về lí do chúng tôi tiến hành thực hiện đề tài, vai trò và vị trí của đề tài trong việc phát triển BAĐT ở Việt Nam cũng như phạm vi hiện thực của đề tài.
\item[Chương 2: Các công trình liên quan] Chương này nêu lên một số công trình liên quan đến phân giải đồng tham chiếu trong bệnh án điện tử, bao gồm giới thiệu chung về bệnh án, các loại văn bản bệnh án, lịch sử số hóa bệnh án và các chuẩn chung cho việc định dạng cấu trúc của bệnh án điện tử. Ngoài ra, chúng tôi cũng trình bày về vấn đề nhận dạng thực thể và một số giải pháp cho nó, vì kết quả của nhận dạng thực thể là đầu vào của hệ thống phân giải đồng tham chiếu mà chúng tôi hiện thực.
\item[Chương 3: Kiến thức nền tảng] Chương này cung cấp các cơ sở lý thuyết về các mô hình, giải thuật và các công cụ được chúng tôi sử dụng để hiện thực hệ thống, bao gồm các mô hình phân giải đồng tham chiếu mà đặc biệt là mô hình cặp khái niệm, giải thuật học Support Vector Machine dùng cho phân loại tính đồng tham chiếu và các công cụ rút trích đặc trưng cho mục đích huấn luyện hay phân loại trong quá trình phân giải.
\item[Chương 4: Hiện thực hệ thống] Trong chương này, chúng tôi trình bày về các chi tiết hiện thực của hệ thống, bao gồm quy trình huấn luyện hệ thống phân loại và quy trình phân giải đồng tham chiếu. Ngoài mô hình phân giải đồng tham chiếu, nội dung của chương còn tập trung vào vấn đề rút trích đặc trưng cho các thực thể trong văn bản thô, bao gồm những đặc trưng cần thiết và cách thức rút trích chúng.
\item[Chương 5: Thí nghiệm đánh giá] Các kết quả so sánh thực nghiệm của hệ thống được chúng tôi trình bày trong chương này. Tập dữ liệu được sử dụng để đánh giá là tập các bệnh án điện tử được cung cấp bởi i2b2 và các phương pháp đánh giá bao gồm MUC, B-CUBED và CEAF, tất cả được trình bày chi tiết trong nội dung của chương. Ngoài ra, cuối chương chúng tôi nêu lên một số giải thích và nhận xét được rút ra từ kết quả đánh giá hệ thống.
\item[Chương 6: Tổng kết] 
Trong chương này, chúng tôi tổng kết lại những kết quả đạt được của chúng tôi sau quá trình thực hiện luận án, cũng như những hạn chế của hệ thống phân giải đồng tham chiếu mà chúng tôi hiện thực, từ đó đề xuất những hướng phát triển mở rộng trong tương lai.
\end{description}
